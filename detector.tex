\chapter{El LHC y el detector ATLAS}
\label{cap:detector}


\section{LHC}

El Gran Colisionador de Hadrones (LHC, del inglés \emph{Large Hadron Collider}) es el
acelerador de hadrones de laboratorio CERN\footnote{CERN son las
  siglas en francés de \emph{Conseil Européen pour la Recherche Nucléaire}}, ubicado en
la frontera entre Francia y Suiza. Posee una longitud de 27 km y fue construido
en el mismo túnel en el que funcionaba el acelerador de electrones LEP entre
1989 y el 2000.

El LHC está diseñado para acelerar protones a 7 \tev, alcanzando en las colisiones, energías de
centro de masa de 14 \tev, y una luminosidad de $10^{34}$ cm$^{-2}$s$^{-1}$. La luminosidad
instantánea ($L$) es uno de los parámetros mas importantes para caracterizar el funcionamiento del
acelerador, definida como el número de
partículas por unidad de tiempo por unidad de área, que puede calcularse
mediante la relación:

\begin{equation}
  L = f_\text{rev} n_b \frac{N_1 N_2}{A}
\end{equation}
%
donde $f$ es la frecuencia de revolución ($\sim 11$ kHz), $n_b$ es el número de
\emph{bunches} (paquetes de protones) por haz, $N_i$ es el número de partículas
en cada bunch y $A$ es la sección efectiva del haz, que puede expresarse en
termino de los parámetros del acelerador como:

\begin{equation}
  A = \frac{4\pi \epsilon_n \beta^{*}}{\gamma F}
\end{equation}
%
donde $\epsilon_n$ es la emitancia transversal normalizada (la dispersión
transversal media de las partículas del haz en el espacio de coordenadas e
impulsos), $\beta^{*}$ es la función de amplitud en el punto de interacción
(IP), relacionada al poder de focalización de los cuadrupolos), $\gamma$ es el
factor relativista de Lorentz y $F$ es un factor de reducción geométrico, debido
al ángulo de cruce de los haces en el punto de interacción (IP).

El diseño contempla trenes de 2808 paquetes de $\sim 10^{11}$ protones cada uno,
espaciados temporalmente en $\unit[25]{ns}$.
Para acelerar los haces de protones y mantenerlos en sus orbitas circulares el
LHC cuenta con 1232 dipolos magnéticos superconductores que generan un campo
magnético de $\unit[8.4]{T}$ enfriados a $\unit[1.9]{K}$. El sistema de focalización de los haces
consiste de 392 cuadrupolos magnéticos que generan campos magnéticos de $\unit[6.8]{T}$.
Los haces circulan en direcciones opuestas en cavidades de ultra alto vacío
separadas a presión de $\unit[10^{-10}]{torr}$.


Durante el año 2012, las colisiones se realizaron a $4\TeV$ por haz ($\sqrt{s} = 8 \TeV$)
y con una luminosidad que fue incrementándose hasta
alcanzar los $\unit[2 \cdot 10^{32}]{cm^{-2}s^{-1}}$ en Octubre. \todo{complete
  with 8 TeV run}



\section{ATLAS}

ATLAS (A Torodial LHC AparatuS) es un detector de partículas multipropósito del LHC, diseñado y construido
para estudiar las colisiones protón-protón a una energía de centro de masa de
hasta 14 TeV.

%% Los objetivos básicos utilizandos para su diseño fueron los siguientes:

%% \begin{itemize}
%% \item Un muy buen calorímetro electromagnético para la identificación y medidas
%%   de fotones y electrones, complementado con un calorímetro hadrónico para las
%%   medidas precisas de jets y energía faltante.
%% \item Medidas de alta precisión de momentos de muones, con la capacidad de
%%   garantizar mediciones precisas a la mas alta luminosidad usando sólo el
%%   espectrómetro de muones externo.
%% \item Alta eficiencia de la detección de trazas a alta luminosidad para medidas
%%   de momentos de leptones con alto \pt, identificación de electrones y fotones,
%%   identificación de $\tau$, y capacidad de reconstrucción total de eventos con
%%   luminosidad baja.
%% \item Gran aceptancia en pseudorapidez cubriendo casi todo el ángulo
%%   azimutal en todo el detector.
%% \item Selección y medidas de partículas con bajo \pt, proveyendo alta eficiencia
%%   para la mayor parte de los procesos físicos de interés en el LHC.
%% \end{itemize}


\begin{figure}[!htbp]
  \centering

  \includegraphics[width=\textwidth]{atlas}
  \caption{Esquema general del detector de ATLAS. Las dimensiones del detector
  son $\unit[25]{m}$ de altura y $\unit[44]{m}$ de largo. El peso promedio es de aproximadamente 7000 toneladas.}\label{fig:atlas}

\end{figure}

El esquema general del detector se muestra en la \cref{fig:atlas}, donde se
señalan los componentes principales. ATLAS está diseñado en capas de
sub-detectores que cumplen diferentes roles en la identificación de las
partículas producidas durante las colisiones (ver \cref{fig:how_atlas_works}).
Desde el punto de la colisión
hacia afuera ATLAS se compone de un detector interno de trazas (ID) compuesto de
un detector de píxeles, un detector de bandas de silicio (SCT) y un detector de
radiación de transición (TRT).
Por sobre el detector interno se encuentra un
solenoide superconductor que genera un campo magnético de $\sim 2$ Tesla, a fin
de curvar la trayectoria de las partículas cargadas.
A continuación están ubicados los calorímetros. En primer lugar el calorímetro
electromagnético para medir la energía cinética de electrones y fotones, y
posteriormente el calorímetro hadrónico para medir la energía de los jets.

En la capa más externa se encuentra el espectrómetro de muones que le da a ATLAS
el tamaño total de aproximadamente $\unit[45]{m}$ de largo y más de
$\unit[25]{m}$ de alto. Intercalado con éste se encuentra el sistema de toroides
que genera el campo magnético de $\sim$ 4 Tesla para curvar la trayectoria de
los muones hacia el final de su pasaje por el detector ATLAS.

El detector ATLAS se divide geométricamente en dos regiones, la parte central
denominada \emph{barrel}, y la región de las tapas llamadas \emph{end-cap}.
En cada una de estas regiones la ubicación de los
sub-detectores es distinta. En la región \emph{barrel}, los sub-detectores están
ubicadas como cilindros concéntricos, mientras que en la región \emph{end-cap}
están ubicados como discos consecutivos perpendiculares a la dirección del haz.


\begin{figure}[!htbp]
  \centering

  \includegraphics[width=0.9\textwidth]{figura} %%how_atlas_works}

  \caption{Esquema del corte transversal del detector de ATLAS, ilustrando los distintos
  sub-detectores y el pasaje de las distintas particulas y en que detector depositan su energia.}
  \label{fig:how_atlas_works}

\end{figure}

\subsection{Sistema de coordenadas}

%% El sistema de coordenadas de ATLAS corresponde a un sistema cartesiano, cuyo
%% origen coincide con el punto de interacción nominal. El eje $z$ es escogido,
%% naturalmente dada la concepción cilíndrica del detector, a lo largo del eje del
%% haz, en sentido antihorario.

El sistema de coordenadas y la nomenclatura utilizada para describir el detector
ATLAS y las trayectorias de las partículas que emergen de las colisiones son
resumidas en esta Sección, ya que se utilizaran a lo largo de la Tesis. Se
define como origen de coordenadas al punto de interacción nominal, mientras que
la dirección del haz define el eje $z$ y el plano $x-y$ es el transversal a la
dirección del haz. El eje $x$ se define desde el punto de interacción, apuntando
al centro del anillo del LHC, y el eje $y$ se define apuntando hacia arriba.

%% El plano transversal $x-y$ es definido con valores positivos de $x$ e $y$
%% desde el origen en dirección hacia el centro del anillo del LHC y hacia la
%% superficie, respectivamente.
Para describir la posición de los distintos
sub-detectores y la trayectoria de las partículas dentro de ATLAS se utilizan
frecuentemente sistemas de coordenadas cilíndricas o polares. El radio $R$ se
define como la distancia perpendicular al eje del haz. El ángulo azimutal $\phi
$ es medido alrededor del eje del haz, mientras que el angulo polar $\theta$
es el angulo respecto al eje del haz.

Una cantidad muy importante utilizada en física de altas energías es la
llamada rapidez:

\begin{equation}
  y = \frac{1}{2} \ln \left( \frac{E+p_z}{E-p_z} \right)
\end{equation}
%
donde $E$ es la energía total de la partícula y $p_z$ es la componente
longitudinal de su impulso. En el límite de altas energías esta cantidad se
aproxima (en forma exacta para objetos no masivos) por la llamada
\emph{pseudorapidez}, $\eta$, relacionada con el ángulo polar $\theta$ como:

\begin{equation}
  \eta = - \ln \tan \left( \frac{\theta}{2} \right)
\end{equation}

La razón detrás de esta transformación de coordenadas es el hecho que la
multiplicidad de partículas producidas es aproximadamente constante como función
de $\eta$, y que la diferencia de pseudo-rapidez entre dos partículas es
invariante frente a transformaciones (boosts) de Lorentz a lo largo de la
dirección del haz. En el caso de colisiones hadronicas, la fracción del impulso
del protón adquirida por cada uno de las partones interactuantes es desconocida;
parte de este impulso es transferido en la interacción dura, mientras cierta
fracción remanente escapa el detector a lo largo del haz. Así, no es posible
reconstruir el movimiento longitudinal del centro de masa en la interacción, y
aplicar leyes de conservación sobre la cinemática de cada evento. Sin embargo,
dado que los protones inciden a lo largo de la dirección del haz, el impulso
total transverso es conservado durante la colisión. Por esta razón, solo las
componentes transversales son utilizadas en la descripción de la cinemática del
evento, por ejemplo $\et (= E \sin \theta)$ y $\pt (= p \sin \theta)$. En términos de
la pseudo-rapidez, la energía transversa de una partícula resulta:


\begin{equation}
  \et = \frac{E}{\cosh \eta}
\end{equation}
%
donde $E$ es su energía total.


\section{Los sub-detectores de ATLAS}

%% A continuación se describen brevemente cada uno de los sub-detectores,
%% particularmente aquellos subsistemas utilizados para la identificación de
%% electrones y fotones, pertinentes al análisis presentado en esta Tesis.

\subsection{El detector interno}

El esquema del detector interno se muestra en la
\cref{fig:detector_interno}. Este sistema combina detectores de muy alta
resolución para distancias cortas al punto de interacción con detectores
continuos de trazas a distancias más lejanas. El detector interno está contenido
dentro del solenoide que provee un campo magnético nominal de $\unit[2]{T}$.

\begin{figure}[!htbp]
  \centering

  \includegraphics[width=0.6\textwidth]{detector_interno}
  \caption{Esquema del detector interno mostrando la traza de una partícula
    cargada de $\pt = 10\gev$ atravezandolo. La trayectoria atraviesa el
    tubo del haz de berilio (\emph{beam-pipe}), las tres capaz del detector de píxeles de silicio (\emph{Pixel}),
    las cuatro capaz doble de sensores semiconductores (SCT), y
    aproximadamente 36 tubos contenidos en los modelos del detector por radiación
  de transición (TRT).}\label{fig:detector_interno}

\end{figure}

\subsubsection{Pixel}

Más cerca del punto de interacción se encuentra el detector de píxeles[], cuyo
objetivo principal consiste en la medicion de la posicion de trazas de
particulas cargadas con la mas alta precision posible y es de vital importancia
para la reconstruccion de los vertices primarios y secundarios.

%% Se compone de
%% tres capas en la region \emph{barrel} (a 4, 10 y 13 cm del tubo del haz) y tres
%% discos en cada \emph{end-cap}.

%% Proveen mediciones de altísima precisión y
%% granularidad tan cerca del punto de interacción como es posible. El sistema
%% contiene en total 80 millones de elementos de $\unit[14 \times 115]{\mu m}$ en (R$\phi$,z),
%% capaces de resolver la posición de las partículas mejor que 14$\mu$m.

\subsubsection{SCT}
Por fuera del detector de píxeles se encuentra el detector semiconductor de
trazas (SCT) que consta de ocho capas de detectores de micro bandas de silicio
que provee puntos de alta precisión en las coordenadas (R$\phi$,z). La
resolución espacial es de 16 $\mu$m en R$\phi$ y de 580 $\mu$m en z y tiene 6.2
millones de canales. Las trazas pueden distinguirse si están separadas más de
$\sim$200 $ \mu$m. El SCT cubre el rango de pseudorapidez de $|\eta|<$2.5.


\subsubsection{TRT}
La parte más externa del detector de trazas es el detector de radiación de
transición (TRT). Este detector está basado en el uso de detectores tubos que
pueden operar a alta frecuencia de eventos gracias a su pequeño diámetro (4mm) y
la aislación de sus hilos centrales en volúmenes de gas individuales.

El TRT además de detectar el pasaje de partículas cargadas, detecta la radiación
de transición que permite distinguir entre partículas cargadas pesadas y
livianas. La separación entre señales de trazas y de radiación por transición se
hace analizando tubo por tubo impactos de alto umbral e impactos de baja señal.
El largo de los tubos varía segun la zona del detector, llegando hasta los 144
cm en la zona del barril. El Barril contiene 50000 tubos y las tapas contienen
320000 tubos orientados radialmente. El número total de canales es de 420000 y
la resolución espacial es de 0.17mm.


\subsection{Calorímetros}

Una vista de los calorímetros de ATLAS puede verse en la \cref{fig:calorimetros}.
Consiste en un calorímetro electromagnético cubriendo la región de pseudorapidez
$|\eta| < 3.2$, un calorímetro hadrónico en la sección \emph{barrel} cubriendo
la región $|\eta| < 3.2$, calorímetro hadrónicos en las \emph{end-cap} cubriendo
la región $1.5 < |\eta| < 3.2$, y calorímetros \emph{forward} cubriendo $3.1 <
|\eta| < 4.9$.

%% \begin{figure}[H]
%%   \centering \includegraphics[width=0.5\textwidth]{figures/figura}
%%   \caption{Esquema general del calorímetro del detector de
%%     ATLAS}\label{fig:calo}
%% \end{figure}


\subsubsection{Calorímetro electromagnético}
El calorímetro electromagnético \cite{caloemTDR} se divide en una parte central
(\emph{barrel}): $|\eta|<$1.475) y los extremos (\emph{end-caps}):
1.375$<|\eta|<$3.2. El barril está compuesto por dos mitades, separadas por una
distancia pequeña (6 mm) a $z = 0$. Las tapas del calorímetro están divididas en
dos ruedas coaxiales: una rueda externa cubriendo la región 1.375$<|\eta|<$2.5 y
una parte interna que cubre la región 2.5$<|\eta|<$3.2.

El calorímetro electromagnético es un detector de muestreo de Argón Líquido
(LAr) con electrodos de kaptón en forma de acordeón y planchas absorbentes de
plomo. El espesor total del calorimetro electromagnetico\ es $>$24 $X_0$ en el
barril y $>$26 $X_0$ en las tapas, ($X_0$ = longitud de radiación).

En la región dedicada a los estudios de física de precisión ($|\eta|<$2.5) el
calorímetro electromagnético está segmentado en tres secciones longitudinales.%
como se esquematiza en la figura \ref{fig:caloem}.

\begin{figure}[!htbp]
  \centering

  \includegraphics[width=0.9\textwidth]{calorimetros}

  \caption{Sistema de calorímetros del detector de ATLAS}
  \label{fig:calorimetros}

\end{figure}


La sección de las bandas (\emph{strips}) que tiene un espesor constante de
$\sim$6 $X_0$ en función de $\eta$, está equipado con bandas finas de 4 mm de
largo en la dirección $\eta$. Esta sección actúa como un detector de pre-cascada
(\emph{pre-shower}) aumentando la capacidad de identificación de partículas,
(como por ejemplo la distinción entre $\gamma$ y $\pi_0$ o entre electrón y
$\pi^\pm$) y dando una precisa medición de la posición en $\eta$.

La sección del medio está segmentada transversalmente en torres cuadradas de
$\Delta \phi \times \Delta \eta=$0.025 $\times$ 0.025 (4 $\times$ 4 cm$^2$ en
$\eta=0$). El espesor total del detector hasta el final de la sección del medio
es $\sim$24$X_0$.

La sección mas externa tiene una granularidad de
$\Delta\phi\times\Delta\eta=$0.025 $\times$ 0.05 y su espesor varía entre 2 y 12
$X_0$.

\subsubsection{Calorímetro hadrónico}
El calorímetro hadrónico de ATLAS \cite{calohadTDR} cubre el rango $|\eta|<$4.9
usando diferentes materiales.

La parte del barril de este sistema consiste en un calorímetro de muestreo que
utiliza acero como absorbente y tejas centelladoras como material activo. Las
tejas están ubicadas radialmente y apiladas en profundidad.

%% \begin{figure}[!htbp]
%%   \centering

%%   \includegraphics[width=0.85\textwidth]{calorimetro_had}

%%   \caption{Calorímetro Hadrónico del detector de ATLAS}
%%   \label{fig:calohad}
%% \end{figure}

La estructura es periódica en z. Las tejas tienen un espesor de 3 mm y el
espesor de las placas de acero en un período es de 14 mm.

El calorímetro de tejas se extiende radialmente desde un radio interno de 2.28 m
hasta un radio externo de 4.25 m.

En la región de las tapas, el calorímetro hadrónico consiste en dos ruedas de
2.3 m de radio, perpendiculares al tubo del haz, hechas con placas de cobre y
tungsteno como material absorbente y argón líquido como material activo. Estos
detectores extienden la aceptancia del calorímetro de ATLAS hasta prácticamente
cubrir el ángulo sólido del punto de colisión.


\subsection{Espectrómetro de muones}

Los muones de alto {\pt} generados en el punto de interacción tienen un altísimo
poder de penetración y son poco interactuantes. Por ello el espectrómetro de
muones \cite{muonTDR} se encuentra situado en la parte más exterior del detector
ATLAS, alrededor del sistema de imanes de toroides, y está diseñado para obtener
mediciones de alta precisión de posición e impulso de muones de alto \pt.

%% \begin{figure}[H]
%%   \centering \includegraphics[width=0.7\textwidth]{figures/figura}
%%   \caption{Espectrómetro de muones del detector de ATLAS}\label{fig:especmuones}
%% \end{figure}

La figura \ref{fig:especmuones} muestra un esquema del espectrómetro de muones
de ATLAS. Es el subdetector más grande y el que le da a ATLAS su tamaño.

La región del barril está compuesta por tres capas concéntricas de cámaras de
trigger y de cámaras de precisión posicionadas a 5m, 7.5m y 10m del tubo del
LHC, cubriendo la región $|\eta|<1$. Las regiones de las tapas están compuestas
por cuatro capas de cámaras de trigger y cámaras de precisión a $|z|$= 7.4m,
10.8m, 14m y 21.5m cubriendo el rango de 1.0$<|\eta|<$2.7. Hay una pequeña
brecha en $|z|=0$ que permite el acceso de los servicios al ID.

El espectrometro de muones es el sub-detector mas grande de ATLAS, construido
dentro y alrededor de los imanes toroidales. Los muones son altamente penetrantes
y son las unicas particulas (excepto las invisibles que interacuan) que lelgan
a el sistema de muones. Los muones pierden parte de su energia mientras penetran
las capas internas de ATLAS antes de llegar al espectrometro de muones. La perdida
de energia tiene que ser tenida en cuenta utilizando los depositos de energia
en los calorimetros

%% The muon system is exposed to challenging background conditions. Final state
%% particles undergoing secondary interactions in the detector, shielding or surround-
%% ing machine material, result in a large number of particles, mainly photons and
%% neutrons with energy of order ∼ 1 MeV, penetrating the muon system. The muon
%% spectrometer is designed to cope with these high rates of particle flux. Cosmic ray
%% event data was used to measure the muon reconstruction efficiency and the momen-
%% tum resolution of the muon spectrometer [59]. For muons with transverse momenta
%% of ∼ 100 GeV the momentum resolution is at its best, with an expected fractional
%% resolution of ∼ 2 %. At p T ∼ 1 TeV this resolution decreases to ∼ 10 %, and below
%% 100 GeV the resolution also drops off as an increasing fraction of the muon energy
%% is lost traversing the detector material downstream.

El sistema de muones incluye
%% The muon system includes a separate trigger designed to identify events contain-
%% ing highly energetic muons, which are interesting for many new physics searches.
%% The muon trigger has coverage up to |η| < 2.4, with the full range of the muon
%% system being |η| < 2.7. Three concentric cylinders surrounding the calorimeters
%% at radii of 5 m, 7.5 m and 10 m comprise the muon system in the barrel. Wheels
%% are placed at distances of approximately 7.4 m, 10.8 m, 14 m and 21.5 m from the
%% interaction point, constituting the muon system in the end-caps. A total of four
%% different technologies are incorporated into the muon system to fulfil the triggering
%% and precision physics requirements. The Monitored Drift Tubes (MDTs) and Cath-
%% ode Strip Chambers (CSCs) provide precision energy measurements and tracking,
%% with the Resistive Plate Chambers (RPCs) and Thin Gap Chambers (TGCs) be-
%% ing responsible for triggering. A particular challenge for the muon trigger system
%% is the ability to maintain a stable p T resolution for the full |η| < 2.4, since the
%% momentum, p, of a muon of a given p T increases as a function of η. This demands
%% increased granularity in the more forward regions of the muon spectrometer so as to
%% ensure a momentum resolution consistent with the barrel. To achieve this different
%% technologies are used in the barrel and end-cap regions of the muon system.
%% The MDTs are gas filled (Ar, CO 2 ) aluminium tubes of diameter ∼ 30 mm with
%% tungsten wires immersed within. When a muon traverses these tubes the gas is
%% ionised and the resulting electrons collect on the tungsten. The MDTs are found in
%% both the barrel and the end-caps, and provide precision position and momentum
%% measurements over the full |η| < 2.7 range of the muon system. A typical MDT
%% chamber consists of two multi-layers of drift tubes, which are separated by spacer
%% bars made of aluminium. Each of these multi-layers contains four layers of tubes
%% in the barrel, or three in the outer regions of the muon system. This means that a
%% muon penetrating the MDTs will on average pass through 20 individual tubes.
%% The safe operational counting rate per unit area for the MDTs is exceeded in
%% the forward regions of the muon system. Cathode strip chambers, with a quicker
%% response time and double the resolution, are used in this region. The CSCs cover the
%% range 2.0 < |η| < 2.7 and are used in the innermost part of the muon system in the
%% end-caps. They are multi-wire proportional chambers containing multiple closely
%% spaced anode wires surrounded by gas (Ar, CO 2 , CF 4 ), with cathode strips running
%% perpendicular to the wires. This orthogonal layout allows the charge distribution
%% to be measured in both directions normal to the beam axis. The CSC system itself
%% is segmented in φ, resulting in eight chambers in each of the two disks. With four
%% CSC layers in each chamber, the average number of measurements per track is
%% considerably less than in the MDTs.
%% The RPCs make up the barrel part (|η| < 1.05) of the dedicated trigger system.
%% Two resistive plates are separated by 2 mm of gas (C 2 H 2 F 4 , Iso-C 4 H 10 , SF 6 ), which
%% may be ionised by a muon passing through, causing a cascade of electrons towards
%% the anode. These chambers are made simpler in construction due to the absence
%% of any wires within, and this feature also makes them less sensitive to any small
%% deviations in positioning (e.g. wire sag). There are three cylindrical trigger layers
%% in the barrel, and each layer contains two RPCs. As such, a muon in the barrel will
%% deliver six hits in the RPCs. The RPCs are able to provide adequate triggering for
%% the barrel region.
%% In addition to the increased demands on granularity, the forward region suffers
%% from radiation levels up to 10 times those in central regions. This further compli-
%% cates the already challenging environment in the end-caps. The TGCs are used to
%% trigger on muon tracks in the end-caps (1.05 < |η| < 2.4). They are multi-wire
%% proportional chambers working similarly to the CSCs, but in this case in order to
%% satisfy the higher granularity requirements the gap between the wire and cathode
%% is smaller than the wire-wire spacing. They exist in two concentric rings, one inner
%% ring containing two TGC layers and one outer ring containing seven. These layers
%% are segmented radially, are tailored to provide excellent time resolution and are able
%% to cope with high particle flux rates. Both the TGCs and RPCs are designed to
%% deliver signals over a time spread of less than 25 ns. This way the bunch crossing
%% responsible for the muon triggering the chamber can be identified with an efficiency
%% of > 99 %.





%---------
% Trigger
%---------
\section{El sistema de Trigger}

Bajo las condiciones nominales de diseño del LHC, la tasa de interacción
protón-protón en el LHC es de $\mathcal{O}(1)$ GHz, considerando una
frecuencia de cruce de haces de $\unit[40]{MHz}$ y $\sim$ 23 interacciones por cruce. Dado
que la mayoría de los eventos son de baja energía y no son de interés para los
análisis mas relevantes en ATLAS, y también debido a las limitaciones de
almacenamiento y del poder de computo, el flujo de datos incidente debe ser
reducido al máximo permitido para su almacenamiento permanente ($\sim 200$ Hz).
El tamaño típico por evento es de $\sim 1.4$ MB, lo que resulta en un ancho de
banda requerido de $\sim 300$ MB/$s$. Esta reducción se logra mediante una
rápida y eficiente preselección de eventos, conocida como \emph{trigger}.

Esencialmente, el sistema de \emph{trigger} de ATLAS[ref] consiste en una seleccion
de eventos basada en tres niveles:
Nivel 1 (L1), Nivel 2 (L2) y Filtro de eventos (EF), donde
los dos últimos conforman el \emph{High Level Trigger} (HLT). Cada nivel permite
analizar los eventos con mayor detalle, aumentando la precisión de los criterios
de selección y la complejidad de los algoritmos utilizados. El sistema de
adquisición de datos (DAQ) transfiere y almacena los datos seleccionados por el
\emph{trigger}.

El primer nivel del \emph{trigger} se encarga de la selección inicial, reduciendo la
frecuencia de eventos que pasaran al siguiente nivel a $\sim 75$ kHz. Debido al
tamaño limitado de las memorias temporales donde se guardan los datos
de cada sub-detector y al considerable tiempo de vuelo de las partículas hasta el
espectrómetro de muones, la decisión debe tomarse en una escala de tiempo muy
limitada ($2.5 \mu s$). El Nivel 1 esta basado en hardware y selecciona objetos
de alto {\pt} construidos a partir de la información de varios subdetectores.
Los muones son identificados en las cámaras de trigger descriptas en la sec \XXX,
mientras que la información de los calorímetros, con una resolución reducida, se
utiliza para identificar candidatos a electrones, fotones, jets y taus decayendo
hadrónicamente. La posición de cada objeto encontrado define una \emph{region de
  interés} (RoI) en un evento potencialmente interesante, que se extiende como
un cono desde el punto de interacción a lo largo del detector.

En el calorímetro, el L1 se basa en las senales analógicas obtenidas en cada
torre del trigger, es decir en la suma de celdas en una ventana $\Delta \eta \times \Delta
\phi = 0.1 \times 0.1$, definida separadamente para el calorimetro electromagnetico y
hadrónico.
%% El \emph{trigger} de muones en el L1 utiliza las medidas de las trayectorias en
%% las diferentes estaciones de las cámaras de trigger: las RPCs en la región del
%% barrel y las TGCs en los endcaps.
La aceptancia geométrica del L1 esta ligada al diseno del detector, donde las medidas de precisión en los
calorímetros y la cobertura del detector interno están limitadas a la región
$\abseta < 2.5$. El trigger de fotones, electrones, muones y taus debe asegurar
la cobertura en esta región. En el caso del trigger de jets, las torres del \emph{trigger}
se extienden hasta $\abseta < 3.2$, mientras que para el cálculo de la energía
transversa total (faltante) se utiliza todo el sistema calorimétrico (i.e.
$\abseta < 4.9$). Los resultados de los subsistemas del trigger son procesados
en el \emph{Central Trigger Processor} (CTP), en donde se aplica una serie de
selecciones (menu) definidas como una combinación de criterios
individuales, que pueden ser ajustados según la luminosidad y los requerimientos
físicos particulares de cada toma de datos. Un total de 256 configuraciones
(\emph{items}) estan disponibles en el L1, donde se programa el tipo de RoI (EM,
TAU, JET, etc.) y los umbrales de energía total y de aislamiento requeridos en
cada caso. Por ejemplo, el item L1EM14 acepta eventos donde al menos un (dos)
cluster(s) en el calorímetro electromagnético posee(n) $\et \geq 14 \GeV$.

El segundo nivel del trigger (L2) se centra unicamente en las RoIs donde el L1
encontro actividad, combinando informacion de todos los subdetectores dentro de
cada una ($\sim 2$ % de la cobertura total del detector). El L2 consiste de una
serie de algoritmos de reconstruccion y seleccion especializados, dise\~nados
para reducir la frecuencia de eventos hasta aproximadamente 1 kHz. Estos
algoritmos estan implementados en clusters de procesamiento dedicados (PC farms)
que analizan cada evento dentro de un tiempo de latencia medio de $\sim
\unit[40]{ms}$. El menor flujo de informacion en este nivel del trigger permite
calcular las variables calorimetricas con mayor precision y hacer uso de la
informacion de las trazas reconstruidas, haciendo posible la distincion entre
fotones y electrones, y el rechazo de fondo proveniente en su mayoria de jets.
13 En general, si bien la seleccion se basa en las mismas variables que la
identificacion offline descripta en la seccion 5.2 (sobre las caracteristicas de
las lluvias electromagneticas), los valores de corte en cada variable son
relajados (o a lo sumo igualados) respecto a la seleccion offline, para evitar
el rechazo prematuro de candidatos que satisfacen los criterios identificacion
durante el analisis final. La ultima etapa de la seleccion del trigger se lleva
a cabo en el Event Filter (EF), que reduce la frecuencia de eventos a $\sim
\unit[200]{Hz}$ ($\sim \unit[300]{MB/s}$). En este nivel se tiene acceso a toda
la informacion del evento en los distintos subdetectores de ATLAS, con la maxima
granularidad e incluyendo detalles sobre la calibracion de energia de los
calormetros, la alineacion de los subdetectores y el mapa de campo magnetico. El
tiempo de latencia relativamente largo disponible para tomar la decision final
sobre el evento ($\avg{t} \sim \unit[4]{s}$) permite la reconstruccion completa
del mismo, y el refinamiento de las variables y criterios de seleccion al nivel
de aquellos implementados en el analisis offline. Los eventos aceptados por el
EF son finalmente grabados a disco y distribuidos, accesibles offline para todos
los analisis subsecuentes.

Cabe mencionar que el sistema de TDAQ permite, en principio, una tasa de
procesamiento/almacenamiento por encima de estos parámetros de dise\~no, por
periodos cortos de tiempo. Por ejemplo, durante el periodo de mas baja
luminosidad instantánea en el 2010 ($L\sim 10^{32} \text{cm}^{-2} s^{-1}$) se
alcanzaron frecuencias de lectura a la salida del EF de $\sim 600$ Hz, para
beneficiar los primeros análisis físicos de ATLAS. Asimismo, el tiempo medio de
procesamiento por evento del EF fue de $\sim \unit[400]{ms}$, muy por debajo del
esperado. Al igual que en el Level 1, en cada nivel del HLT se configuran
ciertos criterios (\emph{signatures}) según el tipo y multiplicidad de la
partícula que se busca en el evento, y el conjunto de cortes de identificación
aplicados. La nomenclatura adoptada como convención en el trigger de ATLAS tiene
la forma general L ipX Y, donde L es el nivel del trigger (L2,EF), i la
multiplicidad, p la partícula de interés (e.g. g=fotón, e=electrón), X el {\pt}
mínimo requerido e Y el tipo de identificación aplicada (loose, tight, etc.
según se describe en la Sec. 5.2). 14 Las signatures del L2/EF y su item
asociado en el L1 (i.e. el que pasa las RoI al L2) definen en conjunto una de
las \emph{cadenas} del trigger, que toman el nombre de la signature del HLT
(i.e. ipX Y segun la convención anterior) y conforman el \emph{menu} final del
trigger.

Para cada item del trigger a Nivel 1 (L2/EF) se puede asignar además
un factor de escala o prescale (PS), que define la frecuencia con la que un dado
item/signature es evaluado por el trigger (i.e. solo en uno de cada PS eventos).
Se habla de una cadenade trigger \emph{unprescaled} si su factor de escala es
PS = 1 en cada nivel, e decir, es evaluada evento a evento. La asignacion de estos
factores se hace incluso dinamicamente durante una toma de datos, para tener en
cuenta el descenso de la luminosidad instantanea con el tiempo y mantener la
tasa de procesamiento aproximadamente constante.



\section{Modelo computacional y distribución de datos}

El modelo computacional de ATLAS esta diseñado para permitir a todos los
miembros de la colaboración un acceso ágil, directo y distribuido a la gran
cantidad de datos colectados por el detector ($\sim \text{PB}/\text{a\~no}$),
así como a las diversas simulaciones MC. El modelo se basa en la tecnología
GRID, compartiendo el poder de procesamiento y la capacidad de almacenamiento
disponibles en distintos centros de computo asociados alrededor del mundo.



El
software de ATLAS se desarrolla dentro un entorno C++ común llamado
\textsc{Athena} [104–106], basado en el projecto GAUDI [107]. Todo el
procesamiento de los datos en ATLAS se realiza dentro de este entorno,
incluyendo la implementación y configuración del HLT, la simulación de la
respuesta del detector, la generación de las muestras MC de los distintos
procesos físicos, y la reconstrucción y análisis de los datos. Los eventos
aceptados por el trigger deben ser procesados para reducir su tamaño y ser
utilizados para los análisis offline. A la salida del EF, los eventos son
almacenados como Raw Data Objects (RDOs). Luego de aplicar los algoritmos de
reconstrucción y calibración, las colecciones de los distintos objetos físicos
obtenidas (fotones, electrones, etc.) son almacenadas en formato ESD (Event
Summary Data) y AOD (Analysis Data Object), una versión reducida del primero
($\sim 100$ kB/evento). A partir de las ESDs/AODs, se ha definido un formato de
datos significativamente mas pequeño (10-15 kB/evento) conocido como D3PD
(Derived Physics Data), sobre el que se realiza el análisis final. Las D3PDs son
archivos (\emph{ntuples}), accesibles vía el entorno de análisis de datos ROOT
[108], que contienen un conjunto de variables para diferentes objetos físicos,
según las necesidades de cada grupo de análisis dentro de ATLAS. Para el
análisis de esta tesis, se utilizaron las D3PDs definidas y producidas en forma
centralizada por el SM Direct Photon group. La misma cadena de reconstrucción y
distribución se aplica a las simulaciones Monte Carlo, a fin de conservar un
modelo de análisis único y garantizar la consistencia en la comparación de estas
con los datos experimentales.



\section{Datos de colisiones $pp$ a $\sqrt{s} = 8$ \tev}

El presente análisis utiliza el conjunto de eventos colectados de las colisiones
$pp$ a una energía de centro de masa $\sqrt{s} = 8\tev$ con el detector ATLAS
durante el ano 2012. Estos eventos colectados corresponden a una luminosidad
total integrada de $21.7 \ifb$. Como el análisis utilizan fotones, electrones,
muones, jets y energía faltante, es impresindible que todos los sub-sistemas
del detector ATLAS hayan estado operando en condiciones normales durante la toma
de datos. Este requerimiento adicional resulta en una reducción de los datos de
$\sim 6\%$, dejando una luminosidad total de $\int L\, dt = 20.3 \pm 0.6\, (2.8
\%) \, \ifb$\cite{lumi2012} para análisis físicos.





%% after the application of beam, detector and data quality requirements \footnote{GRL \texttt{data12\_8TeV.periodAllYear\_DetStatus-v61\-pro14\-02\_DQDefects\-00\-01\-00\_PHYS\_StandardGRL\_All\_Good.xml}}.
%% The collected integrated luminosities split up by run period is shown in {\tab} \ref{tab:data_periods}.
%% The data sample is collected by a single photon trigger (\trigchain) with a transverse momentum threshold of 120 \gev, which selects events with at least one photon passing the loose identification criteria. This trigger has been kept unprescaled through the whole data taking, and is fully efficient selecting photons with $p_{T}>125 \gev$ accepted by the signal selection cuts described in {\sec} \ref{sec:event_selection}. The trigger efficiency extraction and the uncertainty evaluation is treated in {\sec} \ref{sec:trigger_eff}.

\begin{table}[ht]
  \centering
  \caption{Luminosidad integrada de cada periodo de toma de datos de colisiones
    $pp$ a una energía de centro de masa de $8 \tev$ utilizada en este
    análisis.}
  \label{tab:data_periods}

  \begin{tabular}{ccr}
    \hline
    Periodo & Runs & Luminosidad $[\ipb]$ \\
    \hline
    A & 200804 - 201556 & 795.91 \\
    B & 202660 - 205113 & 5113.61 \\
    C & 206248 - 207397 & 1409.06 \\
    D & 207447 - 209025 & 3297.54 \\
    E & 209074 - 210308 & 2534.11 \\
    G & 211522 - 212272 & 1279.54 \\
    H & 212619 - 213359 & 1449.04 \\
    I & 213431 - 213819 & 1018.45 \\
    J & 213900 - 215091 & 2605.48 \\
    L & 215414 - 215643 & 841.634 \\
    \hline
    Total & 200804 - 215643 & 20344.37 \\
    \hline
  \end{tabular}

\end{table}



%% 2 ATLAS Offline Software Overview
%% The ATLAS software framework, Athena [3], uses Python
%% as an object-oriented scripting and interpreter language
%% to configure and load C++ algorithms and objects.
%% Rather than develop an entirely new high-energy physics
%% data processing infrastructure, ATLAS adopted the Gaudi
%% framework [6, 7], originally developed for LHCb and written
%% in C++. Gaudi was created as a flexible framework to
%% support a variety of applications through base classes and
%% basic functionality. As much as possible, the infrastructure
%% relies on the CLHEP common libraries [8], which include
%% utility classes particularly designed for use in high-energy
%% physics software (e.g. vectors and rotations).
%% Athena releases are divided into major projects by
%% functionality [9], and all of the ATLAS simulation software
%% (including event generation and digitization) resides
%% in a single project. The dependencies of the “simulation”
%% project are the “core” project, which includes the Athena
%% framework, the “conditions” and “detector description”
%% projects, which include all code necessary for the description
%% of the ATLAS detector, and the “event” project,
%% which includes descriptions of persistent objects. The number
%% of lines of code by software language for the simulation
%% project are summarized in Table 1, as calculated using
%% cloc [10] in Athena release 14.4. Lines of code in the upstream
%% Athena projects, excluding external dependenccies
%% like Gaudi and CLHEP, are summarized in Table 2.


%% El mecanismo de seleccion se basa en el
%% concepto de bloques de luminosidad (LB). Cada toma de datos (Run) en ATLAS se
%% divide en estos pequenos intervalos de aproximadamente 2 minutos, dentro de los
%% cuales la luminosidad instantanea es esencialmente constante.


%% Producto de la complejidad de un experimento de la magnitud de ATLAS y de las\note{Cambiar y mover a ATLAS}
%% demandantes condiciones de funcionamiento del LHC, se pueden observar
%% ocasionalmente ciertas ineficiencias en los diversos sub-detectores y/o en la
%% cadena de procesamiento de los datos colectados. Para asegurar la calidad de los
%% datos a ser considerados en los análisis físicos de ATLAS, los grupos
%% responsables de cada sub-detector definen un conjunto de criterios de
%% calidad\note{Ref?}, con los cuales se construyen listas, llamadas GRL, de las
%% Runs y los rangos de LB dentro de ellas que son apropiadas para cada tipo de
%% análisis. Se producen de forma centralizada para brindar listas oficiales
%% comunes para los distintos grupos dentro de ATLAS, distribuidas en un formato
%% \textsc{XML} para luego ser utilizadas durante el análisis final.



\begin{figure}[!htbp]
  \centering

  \includegraphics[width=0.49\textwidth]{intlumivstime2012DQ}
  \includegraphics[width=0.49\textwidth]{mu_2012-dec}

  \caption{Izquierda: Luminosidad acumulada en función del tiempo, entregada por el LHC (verde),
    guardada por ATLAS (amarillo), y que pasa los criterios de calidad (azul),
    durante haces estables en colisiones $pp$ a $\sqrt{s}=8\tev$ durante el ano 2012\cite{lumiplots}.
    Derecha: Distribución del valor medio del número de interacción  por cruce
    de haz durante la toma de datos en el a\~no 2012 pesado con la luminosidad.
    La luminosidad integrada y el valor medio de $\mu$ están detallados en la figura.
  }
  \label{fig:lumi}

\end{figure}
