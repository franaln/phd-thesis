\chapter{LHC y el detector ATLAS}


\section{LHC}

El Gran Colisionador de Hadrones (LHC, del ingles Large Hadron Collider) es el acelerador
de hadrones ubicado en el laboratorio CERN\footnote{CERN son las siglas en francés de
  \emph{Conseil Européen pour la Recherche Nucléaire}}, en la frontera entre Francia y Suiza.
Posee una longitud de 27 km y fue construido en el mismo túnel en el que funcionaba el acelerador
de electrones LEP entre 1989 y el 2000.

El LHC está diseñado para acelerar protones a 7 \tev, alcanzando energías de centro de masa de 14 \tev,
y una luminosidad de $10^{34}$ cm$^{-2}$s$^{-1}$.
Uno de los parámetros mas importantes para caracterizar el funcionamiento del acelerador es la
luminosidad instantánea (L), definida como el numero de partículas por unidad de tiempo por unidad de
área, que puede calcularse mediante la relación:

\begin{equation}
  L = f_\text{rev} n_b \frac{N_1 N_2}{A}
\end{equation}
%
donde $f$ es la frecuencia de revolucion (sim 11 kHz), $n_b$ es el numero de \emph{bunches}
(paquetes de protones) por haz, $N_i$ es el numero de partículas en cada bunch y A es la
sección efectiva del haz, que puede expresarse en termino de los parámetros del acelerador
como:

\begin{equation}
  A = \frac{4\pi \epsilon_n \beta^{*}}{\gamma F}
\end{equation}
%
donde $\epsilon_n$ es la emitancia transversal normalizada (la dispersión transversal media de las
partículas del haz en el espacio de coordenadas e impulsos), $\beta^{*}$ es la función de amplitud
en el punto de interacción (IP), relacionada al poder de focalización de los cuadrupolos), $\gamma$
es el factor relativista de Lorentz y $F$ es un factor de reducción geométrico, debido al ángulo
de cruce de los haces en el IP.


Durante el año 2010, las colisiones se realizaron a 3.5 TeV por haz (7 TeV de energía de centro de masa)
y con una luminosidad que fue incrementándose hasta alcanzar los $2 \times 10^{32}$ cm$^{-2}$ s$^{-1}$
en Octubre. %%\todo{complete with 8 TeV run}

El diseño contempla trenes de 2808 paquetes de $\sim 10^{11}$ protones cada uno,
espaciados temporalmente en 25 ns.

Para acelerar los haces de protones y mantenerlos en sus orbitas circulares el LHC cuenta con 1232
dipolos magnéticos superconductores que generan un campo magnético de 8.4 T enfriados a 1.9 K.
El sistema de focalización de los haces consiste de 392 cuadrupolos magnéticos que generan campos
magnéticos de 6.8 T. Los haces circulan en direcciones opuestas en cavidades de ultra alto vacío
separadas a presión de 10$^{-10}$ torr.


\section{ATLAS}

ATLAS es un detector de partículas multipropósito del LHC diseñado y construido para estudiar
las colisiones protón-protón a una energía de centro de masa de hasta 14 TeV.
El nombre significa \textbf{A T}orodial \textbf{L}HC aparatu\textbf{S}.

%% Los criterios básicos para el diseño del detector de ATLAS fueron:

%% \begin{itemize}
%% \item Un muy buen calorímetro electromagnético para la identificación y medidas de fotones y electrones, complementado con un calorímetro hadrónico para las medidas precisas de \emph{jets} y energía transversa perdida.
%% \item Medidas de alta precisión de momentos de muones, con la capacidad de garantizar mediciones precisas a la mas alta luminosidad usando sólo el espectrómetro de muones externo.
%% \item Alta eficiencia de la detección de trazas a alta luminosidad para medidas de momentos de leptones con alto \pt, identificación de electrones y fotones, identificación de $\tau$,  y capacidad de reconstrucción total de eventos con luminosidad baja.
%% \item Gran aceptancia en pseudorapidez ($\eta$) cubriendo casi todo el ángulo azimutal ($\phi$) en todo el detector.
%% \item Selección y medidas de partículas con bajo \pt, proveyendo alta eficiencia para la mayor parte de los procesos físicos de interés en el LHC.
%% \end{itemize}


El esquema general del detector se muestra en la figura \ref{fig:atlas}, donde se señalan los
componentes principales.
ATLAS está diseñado en capas de subdetectores que cumplen diferentes roles en la identificación
de las partículas producidas en las colisiones pp del LHC.
Desde el punto de colision hacia afuera ATLAS se compone de un ID subdividido a su vez en un
detector de píxeles (o capa B), un detector de bandas de silicio (SCT) y un detector de radiación
de transición (TRT).

\begin{figure}[H]
  \centering
  \includegraphics[width=0.85\textwidth]{figures/figura}
  \caption{Esquema general del detector de ATLAS}\label{fig:atlas}
\end{figure}


Envolviendo este detector interno se encuentra un solenoide superconductor que genera un campo
magnético de $\sim$2 Tesla para que las partículas cargadas curven su trayectoria.
A continuación están ubicados los calorímetros: el calorímetro electromagnético para medir la
energía cinética de electrones y fotones, y posteriormente el calorímetro hadrónico para medir
la energía de los \emph{jets} de hadrones.

En la capa más externa se encuentra el espectrómetro de muones que le da a ATLAS el tamaño total
de aproximadamente 45m de largo y más de 25m de alto.
Intercalado con éste se encuentra el sistema de toroides que genera el campo magnético de $\sim$4
Tesla para curvar la trayectoria de los muones hacia el final de su pasaje por el detector ATLAS.

El detector ATLAS se divide geométricamente en dos regiones, la región del barril
(la parte central, \emph{barrel}) y la región de las tapas (ambos extremos, \emph{end-cap}).
En cada una de estas regiones la ubicación de los subdetectores es distinta. En la región del barril,
los subdetectores están ubicadas como cilindros concéntricos, mientras que en la región de las tapas
están ubicados como discos consecutivos.

\subsection{Sistema de coordenadas}

El sistema de coordenadas de ATLAS corresponde a un sistema cartesiano, cuyo origen
coincide con el punto de interacción nominal. El eje z es escogido, naturalmente dada la
concepción cilíndrica del detector, a lo largo del eje del haz, en sentido antihorario. El eje
$z$ positivo (negativo) define el lado A (C) del detector, en vista de la simetría nominal del

mismo. El plano transversal $x-y$ es definido con valores positivos de $x$ e $y$ desde el origen
en dirección hacia el centro del anillo del LHC y hacia la superficie, respectivamente.
Para describir la posición de los distintos subdetectores y la trayectoria de las partículas
dentro de ATLAS se utilizan frecuentemente sistemas de coordenadas cilíndricas o polares.
El radio R se define como la distancia perpendicular al eje del haz.  El ángulo azimutal
$\phi = 0$ corresponde al eje $x$ positivo y crece en sentido horario entorno al eje $z$ positivo,
mientras que el angulo $\theta$ se mide con respecto a este ultimo. Una cantidad muy importante
utilizada en física de altas energías es la llamada rapidez:

\begin{equation}
  y = \frac{1}{2} \ln \frac{E+p_z}{E-p_z}
\end{equation}
%
donde E es la energia total de la particula y $p_z$ es la componente longitudinal de su impulso.
En el limite de altas energias esta cantidad se aproxima (en forma exacta para objetos no masivos)
por la llamada \emph{pseudorapidez}, $\eta$, relacionada con el angulo polar $\theta$ como:

\begin{equation}
  \eta = - \ln \tan \frac{\theta}{2}
\end{equation}

La razon detras de esta transformacion de coordenadas es el hecho que la multiplicidad de particulas
producidas es aproximadamente constante como funcion de $\eta$, y que la
diferencia de pseudorapidez entre dos particulas es invariante frente a transformaciones
(boosts) de Lorentz a lo largo de la direccion del haz. En el caso de colisiones hadronicas,
la fraccion del impulso del proton adquirida por cada uno de las partones interactuantes
es desconocida; parte de este impulso es transferido en la interaccion dura, mientras cierta
fraccion remanente escapa el detector a lo largo del haz. Asi, no es posible reconstruir el
movimiento longitudinal del centro de masa en la interaccion, y aplicar leyes de conservacion
sobre la cinematica de cada evento. Sin embargo, dado que los protones inciden a lo
largo de la direccion del haz, el impulso total transverso es conservado durante la colision.
Por esta razon, solo las componentes transversales son utilizadas en la descripcion de la
cinematica del evento, e.g. $\et (= E \sin \theta)$ y $\pt (= p \sin \theta)$.
En terminos de la pseudorapidez, la energia transversa de una particula resulta:


\begin{equation}
  \et = \frac{E}{\cosh \eta}
\end{equation}
%
donde $E$ es su energia total.


\section{Los subdetectores de ATLAS}

A continuacion se describen brevemente cada uno de los subdetectores, particularmente
aquellos subsistemas utilizados para la identificacion de electrones y fotones, pertinentes
al analisis presentado en esta tesis.

\subsection{El detector interno}

El esquema del detector interno se muestra en la figura \ref{fig:innerdetector}\cite{IDTDR}.
Este sistema combina detectores de muy alta resolución para distancias cortas al punto de interacción con detectores continuos de trazas a distancias más lejanas. El detector interno está contenido dentro del solenoide que provee un campo magnético nominal de 2T.

\begin{figure}[H]
  \centering
  \includegraphics[width=0.55\textwidth]{figures/figura}
  \caption{Esquema del detector interno}\label{fig:innerdetector}
\end{figure}

\subsubsection{Pixel}
Más cerca del punto de interacción se encuentra el detector de píxeles que se compone de tres capas en el barril (a 4cm, a 10cm y a 13 cm del tubo del haz de protones) y tres discos en cada tapa.  Proveen mediciones de altísima precisión y granularidad tan cerca del punto de interacción como es posible. El sistema contiene en total 80 millones de elementos de 14x115 $\mu$m en (R$\phi$,z), capaces de resolver la posición de las partículas mejor que 14$\mu$m.

\subsubsection{SCT}
Por fuera del detector de píxeles se encuentra el detector semiconductor de trazas (SCT) que consta de ocho capas de detectores de micro bandas de silicio que provee puntos de alta precisión en las coordenadas (R$\phi$,z).
La resolución espacial es de 16 $\mu$m en R$\phi$ y de 580 $\mu$m en z y tiene 6.2 millones de canales.
Las trazas pueden distinguirse si están separadas más de $\sim$200 $ \mu$m.
El SCT cubre el rango de pseudorapidez de $|\eta|<$2.5.


\subsubsection{TRT}
La parte más externa del detector de trazas es el detector de radiación de transición (TRT).
Este detector está basado en el uso de detectores tubos que pueden operar a alta frecuencia de eventos gracias a su pequeño diámetro (4mm) y la aislación de sus hilos centrales en volúmenes de gas individuales.

El TRT además de detectar el pasaje de partículas cargadas, detecta la radiación de transición que permite distinguir entre partículas cargadas pesadas y livianas.
La separación entre señales de trazas y de radiación por transición se hace analizando tubo por tubo impactos de alto umbral e impactos de baja señal.
El largo de los tubos varía segun la zona del detector, llegando hasta los 144 cm en la zona del barril.
El Barril contiene 50000 tubos y las tapas contienen 320000 tubos orientados radialmente. El número total de canales es de 420000 y la resolución espacial es de 0.17mm.


\subsection{Calorímetros}

Una vista de los calorímetros de ATLAS puede verse en la figura \ref{fig:calo}. Consiste en un calorímetro electromagnético cubriendo la región de pseudorapidez $|\eta| < 3.2$, un calorímetro hadrónico en la sección \emph{barrel} cubriendo la región $|\eta| < 3.2$, calorímetro hadrónicos en las \emph{end-cap} cubriendo la región $1.5 < |\eta| < 3.2$, y calorímetros \emph{forward} cubriendo $3.1 < |\eta| < 4.9$.

\begin{figure}[H]
  \centering
  \includegraphics[width=0.5\textwidth]{figures/figura}
  \caption{Esquema general del calorímetro del detector de ATLAS}\label{fig:calo}
\end{figure}


\subsubsection{Calorímetro electromagnético}
El calorímetro electromagnético \cite{caloemTDR} se divide en una parte central
(\emph{barrel}): $|\eta|<$1.475) y los extremos (\emph{end-caps}): 1.375$<|\eta|<$3.2.
El barril está compuesto por dos mitades, separadas por una distancia pequeña (6 mm) a $z = 0$.
Las tapas del calorímetro están divididas en dos ruedas coaxiales: una rueda
externa cubriendo la región 1.375$<|\eta|<$2.5 y una parte interna que cubre la
región 2.5$<|\eta|<$3.2.

El calorímetro electromagnético es un detector de muestreo de Argón Líquido (LAr)
con electrodos de kaptón en forma de acordeón y planchas absorbentes de plomo.
El espesor total del calorimetro electromagnetico\ es $>$24 $X_0$ en el barril
y $>$26 $X_0$ en las tapas, ($X_0$ = longitud de radiación).

En la región dedicada a los estudios de física de precisión ($|\eta|<$2.5) el
calorímetro electromagnético está segmentado en tres secciones longitudinales.% como se esquematiza en la figura \ref{fig:caloem}.

%\begin{figure}[H]
%  \centering
%  \includegraphics[width=0.85\textwidth]{./chapters/atlas/images/caloem}
%  \caption{Calorímetro Electromagnético del detector de ATLAS}\label{fig:caloem}
%\end{figure}


La sección de las bandas (\emph{strips}) que tiene un espesor constante de $\sim$6 $X_0$ en función de $\eta$, está equipado con bandas finas de 4 mm de largo en la dirección $\eta$.
Esta sección actúa como un detector de pre-cascada (\emph{pre-shower}) aumentando la capacidad de identificación de partículas, (como por ejemplo la distinción entre $\gamma$ y $\pi_0$ o entre electrón y $\pi^\pm$) y dando una precisa medición de la posición en $\eta$.

La sección del medio está segmentada transversalmente en torres cuadradas de $\Delta \phi \times \Delta \eta=$0.025 $\times$ 0.025 (4 $\times$ 4 cm$^2$ en $\eta=0$).
El espesor total del detector hasta el final de la sección del medio es $\sim$24$X_0$.

La sección mas externa tiene una granularidad de $\Delta\phi\times\Delta\eta=$0.025 $\times$ 0.05 y su espesor varía entre 2 y 12 $X_0$.

\subsubsection{Calorímetro hadrónico}
El calorímetro hadrónico de ATLAS \cite{calohadTDR} cubre el rango $|\eta|<$4.9 usando diferentes materiales.

La parte del barril de este sistema consiste en un calorímetro de muestreo que utiliza
acero como absorbente y tejas centelladoras como material activo.
Las tejas están ubicadas radialmente y apiladas en profundidad.

%\begin{figure}[H]
%  \centering
%  \includegraphics[width=0.85\textwidth]{./chapters/atlas/images/calohad}
% \caption{Calorímetro Hadrónico del detector de ATLAS}\label{fig:calohad}
% \end{figure}

La estructura es periódica en z. Las tejas tienen un espesor de 3 mm y el
espesor de las placas de acero en un período es de 14 mm.

El calorímetro de tejas se extiende radialmente desde un radio interno de 2.28 m
hasta un radio externo de 4.25 m.

En la región de las tapas, el calorímetro hadrónico consiste en dos ruedas de 2.3 m de
radio, perpendiculares al tubo del haz, hechas con placas de cobre y tungsteno como material
absorbente y argón líquido como material activo.
Estos detectores extienden la aceptancia del calorímetro de ATLAS hasta prácticamente
cubrir el ángulo sólido del punto de colisión.


\subsection{Espectrómetro de muones}
Los muones de alto {\pt} generados en el punto de interacción tienen un altísimo poder de
penetración y son poco interactuantes.
Por ello el espectrómetro de muones \cite{muonTDR} se encuentra situado en la parte más
exterior del detector ATLAS, alrededor del sistema de imanes de toroides,
y está diseñado para obtener mediciones de alta precisión de posición e impulso de muones de alto \pt.

\begin{figure}[H]
  \centering
  \includegraphics[width=0.7\textwidth]{figures/figura}
  \caption{Espectrómetro de muones del detector de ATLAS}\label{fig:especmuones}
\end{figure}

La figura \ref{fig:especmuones} muestra un esquema del espectrómetro de muones de ATLAS.
Es el subdetector más grande y el que le da a ATLAS su tamaño.

La región del barril está compuesta por tres capas concéntricas de cámaras de trigger
y de cámaras de precisión posicionadas a  5m, 7.5m y 10m del tubo del LHC, cubriendo
la región $|\eta|<1$.
Las regiones de las tapas están compuestas por cuatro capas de cámaras de trigger y
cámaras de precisión a $|z|$= 7.4m, 10.8m, 14m y 21.5m cubriendo el rango de 1.0$<|\eta|<$2.7.
Hay una pequeña brecha en $|z|=0$ que permite el acceso de los servicios al ID.

%---------
% Trigger
%---------
\section{El sistema de Trigger}

Bajo las condiciones nominales de dise\~no del LHC, la tasa de interaccion proton-proton
en el LHC sera de $\mathcal{O}(1)$ GHz, considerando una frecuencia de bunch crossing
de 40 MHz y $\sim$ 23 interacciones por cruce. Dado que la mayoria de los eventos son
de baja energia y no son de interes para los analisis mas relevantes en ATLAS, y tambien
debido a las limitaciones de almacenamiento y del poder de computo, el flujo de datos
incidente debe ser reducido al maximo permitido para su almacenamiento permanente
($\sim 200$ Hz). El tamano tipico por evento es de $\sim 1.4$ MB, lo que resulta en un
ancho de banda requerido de $\sim 300$ MB/$s$. Esta reduccion se logra mediante una rapida
y eficiente preseleccion de eventso, conocida como \emph{trigger}. El sistema de trigger
de ATLAS [ref] esta organizado en tres niveles jerarquicos: \emph{Nivel 1} (L1),
\emph{Nivel 2} (L2) y \emph{Filtro de eventos} (EF), dondoe los dos ultimos conforman
el \emph{High Level Trigger} (HLT). Cada nivel permite analizar los eventos con mayor detalle,
aumentando la precision de los criterios de seleccion y la complejidad
de los algortimos utilizados. El sistema de adquisicion de datos (DAQ) transfiere y almacena
los datos seleccionados por el trigger. La {\fig} {\XXX} muestra un esquema del sistema de
Trigger-DAQ de ATLAS.

El primer nivel del trigger se encarga de la seleccion inicial, reduciendo la frecuencia
de eventos que pasaran al siguiente nivel a $\sim 75$ kHz. Debido al tamano limitado
de las memorias temporales (buffers) donde se guardan los datos de cada subdetector y al
considerable tiempo de vuelo de las particulas hasta el espctrometro de muones, la decision
debe tomarse en una escala de tiempo muy limitada ($2.5 \mu s$). El nivel 1 esta basado
en hardware y selecciona objetso de alto {\pt} construidos a partir de la informacion
de varios subdetectores. Los muones son identificados en las camaras de trigger descriptas
en la sec XXX, mientras que la informacion de los calorimetros, con una resolucion reducida,
se utiliza para identificar candidatos a electrones, fotones, jets y taus decayendo
hadronicamente. La posicion de cada objeto encontrado define una \emph{region de interes}
(RoI) en un evento potencialmente interesante, que se extiende como un cono desde el punto de
interaccion a lo largo del detector.

En el calorímetro, el L1 se basa en las se\~nales analógicas obtenidas en cada trigger
tower (i.e. suma de celdas en una ventana $\Delta \eta \times \Delta \phi = 0.1 \times 0.1$),
definida separadamente para el ECAL y el HCAL (Fig. 3.7). El trigger de muones en el L1 utiliza las medidas de
las trayectorias en las diferentes estaciones de las cámaras de trigger: las RPCs en la región
del barrel y las TGCs en los endcaps. La aceptancia geométrica del L1 trigger esta a ligada
al dise\~no del detector, donde las medidas de precisión en los calorímetros y la cobertura
del detector interno están limitadas a la región $\abseta < 2.5$. El trigger de fotones, electrones,
muones y taus debe asegurar la cobertura en esta región. En el caso del trigger de jets,
las trigger towers se extienden hasta $\abseta < 3.2$, mientras que para el calculo de la energía
transversa total (perdida) se utiliza todo el sistema calorimétrico (i.e. $\abseta < 4.9$).
Los resultados de los subsistemas del trigger son procesados en el Central Trigger Processor
(CTP), en donde se aplica una serie de selecciones (\emph{menu}) definidas como una
combinación de criterios individuales, que pueden ser ajustados según la luminosidad y los
requerimientos físicos particulares de cada toma de datos. Un total de 256 configuraciones
(\emph{items}) estan disponibles en el L1, donde se programa el tipo de RoI (EM, TAU, JET, etc.)
y los umbrales de energía total y de aislamiento requeridos en cada caso. Por ejemplo,
el  item L1EM14 acepta eventos donde al menos un (dos) cluster(s) en el calorímetro
electromagnético posee(n) $\et \geq 14 \GeV$.

El segundo nivel del trigger (L2) se centra unicamente en las RoIs donde el L1
encontro actividad, combinando informacion de todos los subdetectores dentro de cada una
($\sim 2$ % de la cobertura total del detector). El L2 consiste de una serie de algoritmos de
reconstruccion y seleccion especializados, dise\~nados para reducir la frecuencia de
eventos hasta aproximadamente 1 kHz. Estos algoritmos estan implementados en clusters de
procesamiento dedicados (PC farms) que analizan cada evento dentro de un tiempo de
latencia medio de $\sim \unit[40]{ms}$. El menor flujo de informacion en este nivel del trigger permite
calcular las variables calorimetricas con mayor precision y hacer uso de la informacion
de las trazas reconstruidas, haciendo posible la distincion entre fotones y electrones,
y el rechazo de fondo proveniente en su mayoria de jets. 13 En general, si bien la
seleccion se basa en las mismas variables que la identificacion offline descripta en la
seccion 5.2 (sobre las caracteristicas de las lluvias electromagneticas), los valores
de corte en cada variable son relajados (o a lo sumo igualados) respecto a la seleccion
offline, para evitar el rechazo prematuro de candidatos que satisfacen los criterios
identificacion durante el analisis final.
La ultima etapa de la seleccion del trigger se lleva a cabo en el Event Filter (EF), que
reduce la frecuencia de eventos a $\sim \unit[200]{Hz}$ ($\sim \unit[300]{MB/s}$).
En este nivel se tiene acceso a toda la informacion del evento en los distintos subdetectores
de ATLAS, con la
maxima granularidad e incluyendo detalles sobre la calibracion de energia de los calormetros,
la alineacion de los subdetectores y el mapa de campo magnetico. El tiempo de latencia
relativamente largo disponible para tomar la decision final sobre el evento ($\avg{t} \sim \unit[4]{s}$)
permite la reconstruccion completa del mismo, y el refinamiento de las variables y criterios
de seleccion al nivel de aquellos implementados en el analisis offline. Los eventos aceptados
por el EF son finalmente grabados a disco y distribuidos, accesibles offline para todos los
analisis subsecuentes.

Cabe mencionar que el sistema de TDAQ permite, en principio, una tasa de procesamiento/almacenamiento
por encima de estos parametros de dise\~no, por periodos cortos de tiempo. Por ejemplo, durante el
periodo de mas baja luminosidad instantanea en el 2010 ($L\sim 10^{32} \text{cm}^{-2} s^{-1}$) se alcanzaron
frecuencias de lectura a la salida del EF de $\sim 600$ Hz, para beneficiar los primeros analisis
fisicos de ATLAS. Asimismo, el tiempo medio de procesamiento por evento del EF fue de $\sim \unit[400]{ms}$,
muy por debajo del esperado.
Al igual que en el Level 1, en cada nivel del HLT se configuran ciertos criterios
(\emph{signatures}) segun el tipo y multiplicidad de la particula que se busca en el evento,
y el conjunto de cortes de identificacion aplicados. La nomenclatura adoptada como convencion
en el trigger de ATLAS tiene la forma general L ipX Y, donde L es el nivel del
trigger (L2,EF), i la multiplicidad, p la particula de interes (e.g. g=foton, e=electron),
X el {\pt} minimo requerido e Y el tipo de identificacion aplicada (loose, tight, etc. segun
se describe en la Sec. 5.2). 14 Las signatures del L2/EF y su  item asociado en el L1 (i.e. el
que pasa las RoI al L2) definen en conjunto una de las \emph{cadenas} del trigger, que toman
el nombre de la signature del HLT (i.e. ipX Y segun la convencion anterior) y conforman
el \emph{menu} final del trigger.


Para cada item (signature) del trigger a Level 1 (L2/EF) se puede asignar ademas un
factor de escala o prescale (PS), que define la frecuencia con la que un dado item/signature
es evaluado por el trigger (i.e. solo en uno de cada PS eventos). Se habla de una
cadenade trigger \emph{unprescaled} si su factor de escala es PS=1 en cada nivel (i.e. si es evaluada
evento a evento). La asignacion de estos factores se hace incluso dinamicamente durante
una toma de datos, para tener en cuenta el descenso de la luminosidad instantanea con el
tiempo y mantener la tasa de procesamiento aproximadamente constante.


\section{Modelo computacional y distribucion de datos}

El modelo computacional de ATLAS esta disenado para permitir a todos los miembros
de la colaboracion un acceso agil, directo y distribuido a la gran cantidad de datos
colectados por el detector ($\sim \text{PB}/\text{a\~no}$), asi como a las diversas
simulaciones MC. El modelo se basa en la tecnologia GRID, compartiendo el poder de
procesamiento y la capacidad de almacenamiento disponibles en distintos centros de
computo asociados alrededor del mundo.
El software de ATLAS se desarrolla dentro un entorno C++ comun llamado \texttt{ATHENA}
[104–106], basado en el projecto GAUDI [107]. Todo el procesamiento de los datos en
ATLAS se realiza dentro de este entorno, incluyendo la implementacion y configuracion
del HLT, la simulacion de la respuesta del detector, la generacion de las muestras MC de
los distintos procesos fisicos, y la reconstruccion y analisis de los datos.
Los eventos aceptados por el trigger deben ser procesados para reducir su tamano y
ser utilizados para los analisis offline. A la salida del EF, los eventos son almacenados
como Raw Data Objects (RDOs). Luego de aplicar los algoritmos de reconstruccion y
calibracion, las colecciones de los distintos objetos fisicos obtenidas (fotones, electrones,
etc.) son almacenadas en formato ESD (Event Summary Data) y AOD (Analysis Data
Object), una version reducida del primero ($\sim 100$ kB/evento). A partir de las ESDs/AODs,
se ha definido un formato de datos significativamente mas pequeno (10-15 kB/evento)
conocido como D3PD (Derived Physics Data), sobre el que se realiza el analisis final. Las
D3PDs son archivos (\emph{ntuples}), accesibles via el entorno de analisis de datos ROOT [108],
que contienen un conjunto de variables para diferentes objetos fisicos, segun las necesidades
de cada grupo de analisis dentro de ATLAS. Para el analisis de esta tesis, se utilizaron las
D3PDs definidas y producidas en forma centralizada por el SM Direct Photon group.
La misma cadena de reconstruccion y distribucion se aplica a las simulaciones Monte
Carlo, a fin de conservar un modelo de analisis unico y garantizar la consistencia en la
comparacion de estas con los datos experimentales.


\section{Datos}

El presente analisis se basa en el conjunto de events colectados de las colisiones $pp$
a una energia de centro de masa  $\sqrt{s} = 8\tev$ con el detector ATLAS en 2012.

Estos corresponden a una luminosidad integrada $\int L dt = 20.3 \pm 0.7 \ifb$ \cite{lumi2012}
%% after the application of beam, detector and data quality requirements \footnote{GRL \texttt{data12\_8TeV.periodAllYear\_DetStatus-v61\-pro14\-02\_DQDefects\-00\-01\-00\_PHYS\_StandardGRL\_All\_Good.xml}}.
%% The collected integrated luminosities split up by run period is shown in {\tab} \ref{tab:data_periods}.
%% The data sample is collected by a single photon trigger (\trigchain) with a transverse momentum threshold of 120 \gev, which selects events with at least one photon passing the loose identification criteria. This trigger has been kept unprescaled through the whole data taking, and is fully efficient selecting photons with $p_{T}>125 \gev$ accepted by the signal selection cuts described in {\sec} \ref{sec:event_selection}. The trigger efficiency extraction and the uncertainty evaluation is treated in {\sec} \ref{sec:trigger_eff}.

\begin{table}[ht]
  \centering
  \caption{Integrated luminosity used in this analysis. For each data taking period the run range and the integrated luminosity are given.}
  \begin{tabular}{c|c|r}
    \hline
    \hline
    Period & Run range & Luminosity $[\ipb]$ \\
    \hline
    \hline
    A & 200804--201556 &  795.91 \\
    B & 202660--205113 &  5113.61 \\
    C & 206248--207397 &  1409.06 \\
    D & 207447--209025 &  3297.54 \\
    E & 209074--210308 &  2534.11 \\
    G & 211522--212272 &  1279.54 \\
    H & 212619--213359 &  1449.04 \\
    I  & 213431--213819 &  1018.45 \\
    J & 213900--215091 &  2605.48 \\
    L & 215414--215643 &  841.634 \\
    \hline
    \hline
    Total & 200804--215643 & 20344.37 \tabularnewline
    \hline
    \hline
  \end{tabular}
  \label{tab:data_periods}
\end{table}
