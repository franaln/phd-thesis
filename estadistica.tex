\chapter{Métodos estadísticos para la búsqueda de nueva física}

En este capítulo se introducen los conceptos básicos necesarios
para entender el tratamiento estadístico de los datos. Esta enfocado
en la búsqueda de nueva física en altas energías: descubrimiento y
establecimiento de limites de exclusión.


\section{Introducción} %Probabilidad y axiomas de Kolmogorov, teorema de Bayes}


Dada la naturaleza probabilística de las colisiones en el LHC y el bajo numero
de eventos esperados en las búsquedas de nueva física, es necesario contar
con un marco estadístico para interpretar sus resultados, especialmente
para identificar una señal sobre las posibles fluctuaciones de los fondos
del {\SM}.

Un observable $x$ es por naturaleza \emph{frecuentista}, es decir, si realizamos el experimento
muchas veces, vamos a obtener distintos valores para $x$ y este conjunto de valores va a dar lugar
a una función densidad de probabilidad (pdf) de $x$, que llamamos $f(x)$.
%% , que tiene como
%% propiedad que esta normalizada a la unidad.

%% \begin{equation}
%%   \int f(x) \, dx = 1
%% \end{equation}

%% En el caso de que sea una cantidad discreta, como el numero $n$ que satisface una cierta
%% selección, la integral es reemplazada por una suma.

En general, uno tiene una familia de pdfs $f(x;\btheta)$ a la cual llamamos modelo.
Los parámetros del modelo representan , por ejemplo, parámetros de la teoría
física o alguna propiedad desconocida de la respuesta del detector.

%% \subsection{Distribución de Poisson}
\itodo{mezclar con lo anterior}
La distribución de Poisson es una distribución de probabilidad discreta que describe, a partir
de una frecuencia de ocurrencia media, la probabilidad de que ocurra un determinado número de
eventos durante cierto período de tiempo. Un ejemplo  típico que puede ser descripto por esta
distribución son el numero de clicks de un contador Geiger en un determinado intervalo de tiempo.


Se la puede pensar como un limite de la distribución binomial cunado el numero de experimentos
tiende a infinito y la probabilidad a cero, pero el producto es constantes $Np = \nu$. En ese
caso la función de probabilidad esta dada por:

\begin{equation}
  \Pois(n;\nu) = \frac{\nu^n}{n!} e^{-\nu} %%\equiv \Pois(n;\nu)
\end{equation}

%% El valor esperado y la varianza están dados por,

%% \begin{equation}
%%   E[n] = \nu, \qquad V[n] = \nu
%% \end{equation}


\section{Estimadores}

La estimación de parámetros de las distribuciones observadas es una de las tareas fundamentales
del análisis de datos en física de altas energías. A este proceso lo llamamos \emph{fitting}.
Esta tarea consta de dos pasos: la estimación de la mejor aproximación de los valores de los
parámetros reales y la estimación de las inciertas de esos valores estimados. Los
dos métodos mas utilizados para la estimación de parámetros son el de \emph{likelihood máximo} y el de
\emph{mínimos cuadrados}.

Supongamos que tenemos un conjunto de datos de $N$ observaciones $\bm{x} = (x_1, x_2, \ldots, x_N)$,
donde las medidas $x_i$ son estadísticamente independientes y cada una esta descripta por una
función de densidad de probabilidad  $f(x;\btheta)$ que no conocemos, donde $\btheta$ es un
conjunto de parámetros con valores desconocidos.
%Uno quiere estimar las distintas
%caracteristicas de la funcion como su media o varianza, o $f(x;\theta)$...

Un \emph{estimador} es una función de los datos observados $\bm{x}$ que provee valores numéricos,
el valor estimado $\hat{\btheta}$, para el vector de parámetros $\btheta$.

Algunas propiedades que es importante que cumplan los estimadores son:

\begin{itemize}\itemsep0.2cm\parskip0.2cm
\item[] {\bf Consistencia}

  Un estimador se dice consistente (o asintoticamente consistente) si converge
  al valor verdadero $\btheta$ con el numero de medidas $N$: $\lim_{N \to \infty} \btheta = \btheta$.

\item[] {\bf Sesgo}

  El sesgo esta definido como la diferencia entre el valor esperado del estimador y el valor
  verdadero: $E[\hat{\btheta}] - \btheta$ y un estimador es no sesgado cuando el sesgo es cero.

\item[] {\bf Eficiencia}

  Un estimador es eficiente si su varianza $V[\hat{\btheta}]$ es chica.

\end{itemize}

\subsection{Método del likelihood máximo}

Para $N$ mediciones estadísticamente independientes tenemos una pdf conjunta para los
valores observados x dada por $f(\bm{x}, \btheta) = \prod_i f(x_i, \btheta)$. La función
likelihood se define como la pdf dado que observe los datos,

\begin{equation}
  L(\btheta) = f((x_1, x_2, \ldots, x_n); \btheta) = \prod_{i=1}^{N} f(x_i; \btheta)
\end{equation}

El estimador de máximo likelihood (MLE) de los parámetros {\btheta} son los valores
$\hat{\btheta}$ para los cuales la función likelihood $L(\bm{x};\btheta)$ tiene su
máximo global. Una forma intuitiva de pensar es la siguiente: si asumimos que la pdf y
los parámetros son correctos, esperamos valores altos de la función likelihood que para
otros parámetros que no sean los verdaderos.

Los valores estimados $\hat{\btheta}$ de los parámetros se obtienen buscando el máximo
global de la función likelihood, En la practica, es conveniente trabajar con el logaritmo
de la función likelihood (log-likelihood), y buscar el mínimo del negativo de esta función:

\begin{equation}
  - \ln L(\btheta) = \sum_{i=1}^{N} \ln f(x_i; \btheta)
\end{equation}

\begin{equation}
  - \pd{\ln L(\hat{\btheta})}{\theta_j} = 0 \qquad \text{para} \qquad j = 1,\ldots,m.
\end{equation}

En el limite asintótico, cuando el numero de mediciones $N$ tiende a infinito, el MLE
es consistente, es decir, para cada parámetro $\theta$ el valor estimado $\hat{\theta}$
converge al valor verdadero $\theta$. En este limite también el MLE es no sesgado y tiene
su menor varianza. Esto significa que ningún otro estimador puede ser mas eficiente. Para
un numero finito de eventos $N$, sin embargo, el MLE tiene un sesgo proporcional a $1/N$.


Generalmente, cuando se modela un fenómeno aleatorio de interés, el modelo elegido
para ajustar a las observaciones de dicho fenómeno suele tener varios parámetros,
de los cuales solo algunos pueden llegar a ser de interés. De manera formal se los
denomina parámetros de interés y parámetros \emph{nuisance}, respectivamente.
Para esto conviene separar los parámetros como $\btheta \to (\mu, \btheta)$

En el caso de que la función likelihood depende de muchos parámetros, pero solo estamos
interesados en uno solo de esos parámetros $\mu$ y su incerteza, conviene definir la
funci\'on likelihood maximizada \todo{profile?} como,

\begin{equation}
  \lambda(\mu) = \frac{L(\mu,\dhat{\btheta})}{L(\hat{\mu},\hat{\btheta})}
\end{equation}
%
donde en el numerador, los parámetros {\btheta} son ajustados a su MLE ($\dhat{\btheta}$),
para un dado valor del parámetro $\mu$. En el denominador, los valores de $\hat{\btheta}$
y $\hat{\mu}$ son los valores estimados MLE.

En el limite de N grande, donde el la función likelihood se aproxima a una distribución
gaussiana, el cociente de likelihoods sigue una distribución $\chi^2$ con $m$ grados de
libertad (teorema de Wilk).


\subsubsection{Máximo likelihood extendido}

Cuando se repite un experimento con las mismas condiciones muchas veces, el numero
de eventos $n$ de un proceso va a fluctuar de acuerdo a una distribución de Poisson
alrededor del valor esperado $\nu$. Este termino de Poisson puede incorporarse entonces
a la función likelihood, obteniendo lo que llamamos \emph{likelihood extendido}:

\begin{equation}
  L(\nu,\btheta) = \Pois(n;\nu) \prod_{i=1}^{N} f(x_i; \btheta)
\end{equation}

\section{Estimación de intervalos}

Anteriormente vimos como estimar un parámetro, es decir, como encontrar el
valor que mejor represente un cierto parámetro de interés, dados los datos
observados. La estimación de intervalos, en vez de encontrar un solo valor,
provee dos limites numéricos y un nivel de confianza sobre como el valor
verdadero del parámetro de interés yace entre esos limites. A continuacion
se describe la construcción básica de un intervalo, metodo que se debe a
Neyman \todo{ref}.

%% Supongamos que conocemos la distribución muestral $g(\hat{\btheta}, \btheta)$.
%% Esta distribución depende del valor verdadero $\theta$ que no conocemos, pero
%% a pesar de eso, podemos conocer la distribución $g(\hat{\btheta})$ para un dado
%% valor del parámetro.

%% Para construir el intervalo, tenemos que especificar una probabilidad superior e
%% inferior $\alpha$ y $\beta$, y luego las funciones $u_\alpha(\theta)$ y $v_\beta(\theta)$
%% tal que:

%% \begin{align}
%%   \alpha &= P(\hat{\theta} \geq u_\alpha(\theta)) = \int_{u_\alpha(\theta)}^{\infty} g(\hat{\theta};\theta) \, d\hat{\theta} \\
%%   \beta  &= P(\hat{\theta} \leq v_\beta(\theta))  = \int^{v^\beta(\theta)}_{-\infty} g(\hat{\theta};\theta) \, d\hat{\theta}
%% \end{align}


%% La región dentro de $u_\alpha(\theta)$ y $v_\beta(\theta)$ es llamada cinturón de confianza (ver figura \XXX),
%% y la probabilidad del estimador de estar dentro de este cinturón (sin importar el valor de
%% $\theta$) es,

%% \begin{equation}
%%   P(v_\beta(\theta) \geq \hat{\theta} \geq u_\alpha(\theta)) = 1 -\alpha - \beta
%% \end{equation}

%% Se pueden encontrar las funciones inversas $a(\hat{\theta})$ y $b(\hat{\theta})$ de $u$ y $v$,
%% y para estas se cumple que,

%% \begin{equation}
%%   P(a(\hat{\theta}) \geq \theta \geq b(\hat{\theta})) = 1 -\alpha - \beta
%% \end{equation}

%% Si evaluamos las funciones $a$ y $b$ en el valor observado $\hat{\theta}_\text{obs}$ se
%% obtienen el intervalo $[a,b]$. Este intervalo es el intervalo de confianza a un nivel de
%% confianza (\emph{CL}) de $1-\alpha-\beta$. Esto significa que si el experimento se realiza
%% muchas veces, el intervalo $[a,b]$ va a incluir el valor verdadero de $\theta$ en una fracción
%% $1-\alpha-\beta$ de las veces.

%% El valor $a$ representa el limite inferior en el parámetro $\theta$ tal que $a\leq\theta$ con
%% probabilidad $1-\alpha$, y $b$ representa el limite superior en $\theta$ tal que
%% $P(\theta \leq b) = 1-\beta$. Por construcción, el valor de $a$ da el valor hipotético del valor
%% verdadero de $\theta$ para el cual una fracción $\alpha$ de los valores estimados $\hat{\theta}$
%% va a ser mayor al observado $\hat{\theta}_\text{obs}$, y lo mismo sucede con $b$. Tomando
%% $\hat{\theta}_\text{obs} = u_\alpha (a) = v_\beta (b)$,

%% \begin{align}
%%   \alpha &= \int_{\hat{\theta}_\text{obs}}^{\infty} g(\hat{\theta};a) \, d\hat{\theta} \\
%%   \beta &= \int^{\hat{\theta}_\text{obs}}_{-\infty} g(\hat{\theta};b) \, d\hat{\theta}
%% \end{align}

Otra forma equivalente de encontrar el intervalo de confianza para $\theta$ es definir un test para el
valor hipotético de $\theta$ y realizar este test por cada $\theta$, y obtener el {\pvalue}
$p_\theta$ tal que si $p_\theta < \gamma$ rechazamos $\theta$. El intervalo de confianza
consiste entonces en los valores de $\theta$ que no fueron rechazados en el test de tamaño $\gamma$,
con un nivel de confianza $1-\gamma$. Este método es equivalente al método de construcción
de Neyman.


\section{Contrastaci\'on de hipótesis} %%Test Estadístico o modelo estadistico de los datos}

La hipótesis nula, $H_0$, es la hipótesis sujeta a la prueba y usualmente corresponde
a la hipótesis que consideramos verdadera.
Esta hipótesis es comparada contra una hipótesis alternativa $H_1$ (o varias) distintas a $H_0$.

En el caso de búsqueda de nueva física, la hipótesis que juega el rol de hipótesis nula es en la que
s\'olo los procesos del {\SM} contribuyen a las mediciones. Esta hipótesis también suele llamarse
hipótesis de solo-fondo. Y se utiliza una hipótesis alternativa en la cual tambien contribuyen procesos
de nueva física y se denomina hipótesis de se\~nal+fondo. \todo{agregar lo de medidas auxiliares}

Una hipótesis pueden ser simple o compuesta. Las hipótesis simples están completamente determinadas,
no tienen ningún parámetro libre, mientras que las compuestas consisten en una familia de de hipótesis
simples.

Cada hipótesis queda determinada por la pdf que describe a los observables $f(\bm{x}|H)$.

Una vez definidas las hipótesis, uno quiere saber si los datos medidos son compatibles con la hipótesis
nula o si la hipótesis nula puede ser rechazada basados en estas mediciones. Con este objetivo se define
un \emph{estadístico de prueba} $t(\bm{x})$ que es función de los datos observados y en general puede tener
cualquier dimensión, pero consideremos solo estadísticos escalares $t(\bm{x}) \to \mathbb{R}$.

El estadístico de prueba  $t(\bm{x})$ contiene toda la información de discriminación entre $H_0$ y $H_1$
en un s\'olo número. %% El lema de Neyman-Pearson prueba que el cociente de likelihoods es el
%% discriminador mas poderoso \todo{buscar referencia}. El cociente de likelihoods se define como

Cada estadístico de prueba tendrá su pdf $g(t|H)$ y la decisión sobre la hipótesis estará basada
en el valor del estadístico observado $t_\text{obs}$ y la definición de una \emph{región critica} $R$.

La región critica queda definida por un valor de corte $t_c$ y para el caso de que la hipótesis
alternativa tiende a tener valores de $t$ mayores que bajo $H_0$, la región critica corresponde a
$t \geq t_c$ (ver figura).

Si el valor observado esta dentro de la región critica, la hipótesis $H_0$ es rechazada y si esta
fuera (región de aceptancia) no se la puede rechazar.

\begin{figure}[h]
  \centering
  \begin{tikzpicture}

    \colorlet{red}{red!60}
    \colorlet{blue}{blue!60}

    \draw[blue,line width=1] plot[domain=-1:6, samples=500] ({\x},{3*exp(-((\x-2)^2/1.5))});
    \fill[blue, pattern=north west lines, pattern color=blue] (3,0) -- plot[domain=3:6, smooth] ({\x},{3*exp(-((\x-2)^2/1.5))}) -- cycle;

    \draw[red, line width=1] plot[domain=0:6, samples=500] ({\x},{3*exp(-((\x-4)^2)/1)});
    \fill[red, pattern=north west lines, pattern color=red] (3,0) -- plot[domain=0:3,smooth] ({\x},{3*exp(-((\x-4)^2)/1)}) -- cycle;

    \draw[line width=0.8,->] (-1,0) -- (7,0) node[below] {$t$};
    \draw[line width=0.8,->] (-1,0) -- (-1,5) node[left] {$g(t)$};
    %% \draw[line width=0.8] (7,0) -- (7,5);
    %% \draw[line width=0.8] (-1,5) -- (7,5);

    \draw[dashed, line width=0.8] (3, 0) -- (3, 5);
    \draw node at (3,-0.3) {$t_c$};

    \draw node at (0.8, 2.5) {$g(t|H_0)$};
    \draw node at (5.2, 2.5) {$g(t|H_1)$};

    \node[text width=2cm, text centered] at (1, 4.2) {región de aceptancia};
    \node[text width=2.5cm, text centered] at (5, 4.2) {región crítica};

    \draw node (alpha) at (1.7,1) {$\beta$};
    \draw node (p0) at (2.8,0.2) {};
    \draw[->] (p0) edge[bend left] (alpha);

    \draw node (beta) at (4,1) {$\alpha$};
    \draw node (p1) at (3.4,0.2) {};
    \draw[->] (p1) edge[bend right] (beta);

\end{tikzpicture}

\end{figure}

%% Asumiendo que $H_0$ es verdadera, se puede definir una región critica $R$ tal que la probabilidad
%% de que $q$ pertenezca a $R$ sea igual o menor a un cierto valor. El hecho de que $q_\text{obs}$
%% este dentro de $R$, implica que $H_0$ sea rechazada, y de lo contrario aceptada.

%% Primero se define una \emph{región de aceptancia}, tal que si $T(\mathcal{d}) < k_\alpha$
%% uno acepta la hipótesis nula. El contorno $T(\mathcal{d}) = k_\alpha$ delimita el espacio de datos,
%% y es el borde de la región de aceptancia. Se define el tamaño del test, $\alpha$, como la
%% probabilidad de que la hipótesis nula sea rechazada en el caso de ser verdadera (lo que se conoce
%% como error de tipo I). Esto es equivalente a la probabilidad bajo la hipótesis nula de que los datos
%% no sean encontrados en la región de aceptancia, es decir, $\alpha = P(T(\mathcal{D}) \geq k_\alpha|H_0)$

%% En contraste, si uno acepta la hipótesis nula cuando la alternativa es verdadera, es llamado
%% error de tipo II. La probabilidad de cometer este errores la denota $\beta$ y esta dada por
%% $\beta = P(T(\mathcal{D}) < k_\alpha|H_1)$. Se llama \emph{poder del test} a $1-\beta$.

%% Lo que se quiere encontrar es un test estadístico que maximiza el poder del test para un valor
%% fijo de tamaño del test $\alpha$.

La probabilidad de rechazar $H_0$ siendo esta verdadera es,

\begin{equation}
  \alpha =  \int_{R} g(t|H_0)\, dt
\end{equation}
%
y determina el tama\~no del test $\alpha$ o el nivel de significancia del test a
$(100 - \alpha) \%$. El error de rechazar $H_0$ cuando es verdadera es llamado error de tipo I.
Por otro lado el error de aceptar $H_0$ cuando es falsa se llama error de tipo II, y su
probabilidad de ocurrencia ($\beta$), depende de la hipótesis alternativa $H_1$.
El poder del test se define como la probabilidad de rechazar la hipótesis cuando
es falsa:

\begin{equation}
1-\beta = \int_R g(t|H_1)\, dt
\end{equation}

En principio cualquier función de los datos puede ser utilizada como estadístico de prueba. Un buen
estadístico sera aquel para el cual sus distribuciones para la hipótesis nula y alternativa estén
claramente separadas.

Lo que uno quiere es para un dado tama\~no del test, tener el mayor poder posible.
Para el caso de hipótesis simples, el teorema de Neyman-Pearson afirma que el test
con mayor poder dado un tama\~no $\alpha$ esta dado por:

%% Para el caso de dos hipótesis simples (sin parámetros), el test estadístico con mayor
%% poder esta dado por el \emph{likelihood ratio} $T_\text{NP} = \vec{f}(\mathcal{D}|H_1)/\vec{f}(\mathcal{D}|H_0)$.
%% Este resultado se conoce como lema de Neyman-Pearson.


%% Desafortunadamente no hay un equivalente al lema de NP para modelos con muchos parámetros libres.
%% Sin embargo, hay una generelizacion natural basada en el profile likelihood ratio.

\begin{equation}
  t(\bm{x}) = \frac{f(\bm{x}|H_1)}{f(\bm{x}|H_0)}
\end{equation}


Para el caso de un conjunto de datos muy grandes ($N$ grande), la distribución de las medias
de los observables $\bm{x}$ van a seguir una distribución normal incluso si las cantidades
medidas $\bm{x}$ no siguen una distribución normal. Esto es consecuencia del \emph{teorema
  central del limite}. En este caso, la pdf del estadístico de prueba $t$ puede ser
fácilmente determinada. En el caso general, sin embargo, hay que utilizar métodos numéricos y
simulaciones Monte Carlo para determinarla. En ese caso, el conjunto de observables $\bm{x}$ es
generado usando la pdf $f(\bm{x}|H)$ y se calcula el valor del estadístico de prueba $t(\bm{x})$
para cada conjunto. Estos pasos se repiten para acumular estadística en la distribución de
$g(t|H)$.


El nivel de acuerdo entre los datos observados y una dada hipótesis es cuantificado calculando
el \emph{\pvalue}, es decir, la probabilidad, bajo la suposición de que $H$ es cierta,
de observar datos de igual o menor compatibilidad con la predicción de $H$,
respecto a los datos observados.

\begin{equation}
  p_H = P(t>t_\text{obs}|H) = \int_{t_\text{obs}}^{\infty} g(t|H) \, dt
\end{equation}
%
donde $t_\text{obs}$ es el valor del estadístico de prueba obtenido con los datos observados.

De esta forma, un {\pvalue} grande denota que la evidencia en contra de $H_0$ es
débil y un {\pvalue} chico denota que los datos contienen mucha evidencia en contra de $H_0$.
%% En este sentido de p-valores, se puede no hablar de pruebas de hip´otesis, sino de
%% pruebas de significancia, donde la cuantificaci´on del concepto abstracto de significancia
%% es el p-valor

Se dice que la hipótesis es excluida si el {\pvalue} esta por debajo de un valor
especifico dado por el tama\~no del test $\alpha$, donde $\alpha \in [0,1]$.

En física de partículas es usual convertir el {\pvalue} en la signifícancia equivalente, $Z$,
definida como,
\begin{equation}
  Z = \Phi^{-1}(1-p)
\end{equation}
%
donde $\Phi^{-1}$ es el cuantil (la función inversa de la distribución acumulativa) de la distribución
gaussiana.


%% Los modelos de probabilidad pueden construirse para describir simultáneamente varios
%% canales, es decir, varias regiones disjuntas de datos asociadas a un criterio de selección.
%% Si usamos $e$ como el índice sobre eventos y $c$ el índice sobre canales, los datos
%% van a ser una colección de datos: $\mathcal{D}_\text{sim} = {\mathcal{D}_1, \ldots, \mathcal{D}_{c_max}}$.
%% El punto clave ahora, es que va a haber múltiples términos de Poisson.

%% \begin{equation}
%%   \vec{f}_\text{sim} (\mathcal{D}_\text{sim}|\alpha) = \prod_{c \in \text{canales}} \left[ \Pois(n_c|\nu{(\alpha)}) \prod_{e=1}^{n_c} f(x_{c,e}|\alpha) \right]
%% \end{equation}






\section{Descubrimiento}

En general el parámetro de interés es la intensidad de la señal, $\mu$ definida como
$\sigma/\sigma_\text{nominal}$. En el caso de que $\mu = 0$,
tenemos la hipótesis de solo fondo, y cuando vale $\mu = 1$, la hipótesis de señal + fondo.

%% Si $\hat{\mu}$ y $\hat{\btheta}$ son los valores que maximizan $L(\mu,\btheta)$ o minimizan
%% $- \ln L(\mu, \btheta)$ y $\bdtheta$ es el valor que maximiza $L$ para un valor fijo de $\mu$
%% (the profile value of \btheta), el cociente de likelihood es,

%% \begin{equation}
%%   \lambda(\mu) = \frac{L(\mu, \hat{\hat{\bm{\theta}}})}{L(\hat{\mu}, \hat{\bm{\theta}})}
%% \end{equation}
%% %
%% y es independiente de los parámetros \btheta.

Para la búsqueda de una nueva señal solo queremos considerar que los datos
no tengan un buen acuerdo con la hipótesis de solo-fondo solo cuando $\hat{\mu}$
sea mayor a cero. En el caso de experimentos como las oscilaciones de neutrinos
por ejemplo la hipótesis de señal puede predecir un numero mayor o menor de eventos
con respecto a la hipótesis de no oscilaciones.

Aunque es cierto que un valor de $\hat{\mu}$ mucho menor a cero representa evidencia
contra el modelo de solo-fondo, pero este tipo de discrepancia no muestra que los datos
contenga eventos de señal.

Por lo tanto para esta caso de $\mu \geq 0$ conviene definir,

\begin{equation}
  \tilde{\lambda}(\mu) =
  \begin{cases}
    \frac{L(\mu, \hat{\hat{\bm{\theta}}})}{L(\hat{\mu}, \hat{\bm{\theta}})} & \hat{\mu} \geq 0 \\
    \frac{L(\mu, \hat{\hat{\bm{\theta}}})}{L(0, \hat{\bm{\theta}}(0))} & \hat{\mu} < 0
  \end{cases}
\end{equation}

Acá $\dhat{\btheta}(0)$ y $\dhat{\btheta}(\mu)$ se refieren a los estimadores ML de {\btheta} dados
los parámetros de intensidad de señal parámetro de 0 o $\mu$ respectivamente. El correspondiente
estadístico de prueba puede obtenerse como,

\begin{equation}
  \tilde{t}_\mu = -2 \ln \tilde{\lambda}(\mu) =
  \begin{cases}
    -2 \ln \frac{L(\mu, \hat{\hat{\bm{\theta}}})}{L(0, \hat{\bm{\theta}}(0))} & \hat{\mu} < 0 \\
    -2 \ln \frac{L(\mu, \hat{\hat{\bm{\theta}}})}{L(\hat{\mu}, \hat{\bm{\theta}})} & \hat{\mu} \geq 0
  \end{cases}
\end{equation}


\todo{reescribir}Un caso especial importante el estadístico $t_\mu$ es en el caso de $\mu=0$ en un modelo en el que
asumimos $\mu \geq 0$. Rechazar la hipótesis de $\mu=0$ es lo que llamamos \emph{descubrimiento} de
una nueva señal. Para este caso importante definimos $q_0=\tilde{t}_0$. como,

\begin{equation}
  \tilde{q}_0 =
  \begin{cases}
    -2 \ln \tilde{\lambda}(0) & \hat{\mu} > 0 \\
    0 & \hat{\mu} \leq 0
  \end{cases}
\end{equation}

Si los datos fluctúan de tal manera que uno tenga menos eventos que los predichos por
el fondo, entonces $\hat{\mu} = 0$ y tenemos $q_0=0$. A medida que el numero de eventos
crezca por encima del numero de fondo esperado (mayor $\hat{\mu}$) tenemos mayores valores
de $q_0$ que corresponde a un incremento en el nivel de incompatibilidad con la hipótesis
de solo-fondo. Para cuantificar el nivel de desacuerdo entre datos y la hipótesis de
$\mu=0$ usando el valor observado de $q_0$ podemos calcular el valor-$p$ como,

\begin{equation}
  p_0 = \int_{q_{0}^\text{obs}}^{\infty} f(q_0|0) \, dq_0
  \label{eq:p0}
\end{equation}

En el caso de que $p_0 \leq \epsilon$ donde $\epsilon$ es un tamaño del test
prefijado anteriormente, tenemos un descubrimiento. En general en física
de altas energías suele utilizarse $\epsilon = 2.87 \times 10^{-7}$ que corresponde
a una significancia equivalente $Z=5\sigma$.

Es importante hacer notar que el hecho de rechazar la hipotesis de solo-fondo en
un sentido estadistico es solo parte de descrubir un fenomeno nuevo.


\section{Limites de Exclusión}

Si $p_0> \epsilon$, no podemos rechaza $H_0$. Esto no significa que todos los valores de
$\mu$ bajo $H_1$ estén excluidos. Ahora tenemos que testear $H_1[\mu_0]: \mu = \mu_0$ contra
$H_0[\mu_0]: \mu < \mu_0$. Para esto definimos $\tilde{q}_\mu$,

\begin{equation}
  \tilde{q}_\mu =
  \begin{cases}
    -2 \ln \tilde{\lambda}(\mu) & \hat{\mu} \leq \mu \\
    0 & \hat{\mu} > \mu
  \end{cases} \label{eq:qmu}
\end{equation}

Conviene notar que para establecer un limite superior una fluctuación hacia arriba
de los datos no representa una mayor incompatibilidad de la hipótesis $\mu$.

La razón para poner $q_\mu = 0$ para $\hat{\mu} > \mu$ es que cuando seteamos un limite
superior, el hecho de que $\hat{\mu} > \mu$ representa menos compatibilidad con $'mu$ que
los datos obtenidos, y por lo tanto no se considera parte de la región de rechazo del test.

Acá hay que notar, que $q_0$ no es simplemente un caso especial de la {\eq} \eqref{eq:qmu},
sino que tiene una definición diferente. Es decir, $q_0$ es cero si los datos fluctúan
hacia abajo ($\hat{\mu}<0$), pero $q_\mu$ es cero si los datos fluctúan hacia arriba ($\hat{\mu}>\mu$).

Para los datos observados tenemos $\tilde{q}_\mu^\text{obs}$. Para generar la distribución de $\tilde{q}_\mu$
se pueden generar pseudo-experimentos con MC $\to$ $f(\tilde{q}_\mu|\mu, \btheta)$.

El valor-$p$ para una observación bajo una hipótesis $(\mu,\btheta)$ es la probabilidad de obtener
un resultado igual o mas extremo que los obtenidos dada H.

\begin{equation}
  p_{(\mu,\btheta)} = \int_{\tilde{q}_\mu^\text{obs}}^{\infty} f(\tilde{q}_\mu|\mu,\btheta) \, d\tilde{q}_\mu
\end{equation}


Solo nos interesa $\mu$ pero $p$ depende también de $\btheta$. Lo que queremos entonces es
que el valor-$p$ sea chico para todo valor de $\btheta$.

\begin{equation}
  p_\mu^\text{sup} = \sup_\theta p_{(\mu,\btheta)}
\end{equation}

La mayor razón para usar PLR con $t(\bm{x})$ es que en el limite de muchos eventos
la distribución del PLR $\lambda(\mu=\mu_\text{true})$ es independiente de los
valores de los parámetros nuisance $p_\mu^\text{sup} = p_{(\mu,\btheta)} \forall \theta$.
Esto es una consecuencia del teorema de Wilk.


Elegir $\btheta^\text{sup} = \dhat{\btheta}(\mu)$, es decir, el valor para el cual $p_{(\mu,\btheta)} = p_\mu^\text{sup}$
es un buen valor estimado de $\btheta^\text{sup}$.

\begin{equation}
  p_\mu = \int_{\tilde{q}_\mu^\text{obs}}^{\infty} f(\tilde{q}_\mu|\mu,\dhat{\btheta}(\mu, \text{obs})) \, d\tilde{q}_\mu \equiv \clsb
  \label{eq:pmu}
\end{equation}


Para obtener el nivel de confianza estándar de 95\%

\begin{equation}
  p_{\mu_\text{up}} = 0.05
\end{equation}

Para calcular el limite superior {\cls} definimos,\todo{explicar porque \cls}

\begin{equation}
  \cls = \frac{p_\mu}{1-p_b}
\end{equation}
%
donde $p_b$ es el valoro del mismo test bajo la hipótesis de solo-fondo,

\begin{equation}
  p_b = 1 - \int_{\tilde{q}_\mu^\text{obs}}^{\infty} f(\tilde{q}_\mu|0,\dhat{\btheta}(\mu=0)) \, d\tilde{q}_\mu \equiv \clb
\end{equation}


El limite superior {\cls} en $\mu$, $\mu_\text{up}$ se obtiene resolviendo $p_{\mu_\text{up}} = 0.05$.
Se rechazan los valores de $\mu$ si $\mu  < \mu_\text{up}$ con un nivel de confianza de 95\%.


\section{Aproximación asintótica}

Para poder calcular los valores $p$ de una hipótesis utilizando las ecuaciones
\eqref{eq:p0} y \eqref{eq:pmu} necesitamos conocer las distribuciones muestrales del
estadístico de prueba $f(t)$.

En el limite de un gran muestra de datos (limite asintótico) \cite{AsymAprox}, y utiliando el teorema
de Wilks\cite{WilksTheo} y Wald\cite{WaldTheo} es posible escribir la distribución
de $q_0$ completa como,

\begin{equation}
  f(q_0|\mu') = \left( 1 - \Phi\left(\frac{\mu'}{\sigma}\right)\right) \delta(q_0) +
  \frac{1}{2}\frac{1}{\sqrt{2\pi}}\frac{1}{\sqrt{q_0}} \exp \left[ -\frac{1}{2} \left( \sqrt{q_0} - \frac{\mu'}{\sigma} \right)\right]
\end{equation}

Para el caso especial de $\mu' = 0$ la distribución se reduce a:

\begin{equation}
  f(q_0|0) = \frac{1}{2} \delta(q_0) + \frac{1}{2}\frac{1}{\sqrt{2\pi}}\frac{1}{\sqrt{q_0}} e^{-q_0/2}
\end{equation}

En el limite de la gran muestra, $f(q_0|0)$ es independiente de los parámetros nuisance,
aunque $f(q_0|\mu')$ depende de los parámetros nuisance a través de $\sigma$.

De la pdf, la distribución acumulada de $q_0$ es,

\begin{equation}
  F(q_0|\mu') = \Phi \left( \sqrt{q_0} - \frac{\mu'}{\sigma} \right)
\end{equation}

Para el caso especial $\mu' = 0$ es $F(q_0|0) = \Phi(\sqrt{q_0})$ y el valor-$p$
de la hipótesis $\mu=0$ es

\begin{equation}
  p_0 = 1 - F(q_0|0)
\end{equation}

y la significancia de descubrimiento $Z$ es simplemente,

\begin{equation}
  Z = \Phi^{-1} (1-p_0) = \sqrt{q_0}
\end{equation}


La distribución de $q_\mu$ en este mismo limite es,

\begin{equation}
  f(q_\mu|\mu') = \Phi\left(\frac{\mu'-\mu}{\sigma}\right) \delta(q_\mu) +
  \frac{1}{2}\frac{1}{\sqrt{2\pi}}\frac{1}{\sqrt{q_0}} \exp \left[ -\frac{1}{2} \left( \sqrt{q_\mu} - \frac{\mu'-\mu}{\sigma} \right)^2\right]
\end{equation}


\section{Significancia esperada} %%Sensibilidad de descrubrimiento para un experimento de conteo}

When planning the experiment, we want to quantify how sensitive
we are to a potential discovery, e.g., by given median significance
assuming some nonzero strength parameter mu'.
%% So for p-value, need f(q0 |0), for sensitivity, will need f(q0 | mu'),


En física de partículas la cantidad $s/\sqrt{b}$ ha sido siempre utilizado como
una medida de la significancia esperada de descubrimiento. La explicación detrás
de esta formula es que una cantidad $n$ que sigue la distribución de Poisson con
una media grande $s+b$ puede ser aproximada por una variable $x$ distribuidas según
una gaussiana con media $s+b$ y desviación estándar $\sqrt{s+b}$. En este caso
el valor-$p$ de la hipótesis de solo-fondo dada una observación $x$ esta dado por,

\begin{equation}
  p = 1 - \Phi \left( \frac{x-\mu}{\sigma} \right) = 1 - \Phi \left( \frac{x-b}{b} \right)
\end{equation}
%
donde $\mu=b$ y $\sigma = \sqrt{b}$ se refieren a la media y la desviacion estandar de x
asumiendo que $s=0$. Traduciendo este valor-$p$ a significancia,

\begin{equation}
  Z = \frac{x-b}{b}
\end{equation}

La significancia esperada es entonces (ya que la media es $s+b$)

\begin{equation}
  Z_\text{esperada} = \frac{s}{\sqrt{b}}
  \label{eq:Zsimple}
\end{equation}

Si el numero esperado de eventos de fondo $b$ es conocido con una incerteza despreciable,
entonces la funcion likelihood es,

\begin{equation}
  L(s) = \frac{(s+b)^n}{n!} e^{-(s+b)}
\end{equation}

Utilizando la aproximacion asintotica descripta anteriormente, la significancia de descubrimiento
puede ser aproximada por $Z=\sqrt{q_0}$, lo que para este problema daria,

\begin{equation}
  Z = \sqrt{2\left( n \ln \frac{n}{b} +b -n \right)}
  \label{eq:Z}
\end{equation}
%
para $n>b$ y $Z=0$ sino. Tambien puede aproximarse la significancia esperada reemplazando los datos
por el conjunto de datos de Asimov. Substituyendo $s+b$ por $n$ en la {\eq} \eqref{eq:Z}, la aproximaxion de
Asimov para la significancia esperada $Z_A$ es,

\begin{equation}
  Z_A = \sqrt{2\left( (s+b) \ln \left( 1 + \frac{s}{b}\right) - s \right)}
\end{equation}

Si expandimos el logaritmo en la ecuaciona anterior en $s/b$ tenemos $Z_A = s/\sqrt{b} + \mathcal{O}(s/b)$
que es la expresion de la {\eq} \eqref{eq:Zsimple} y es valida solo en el limite $s \ll b$.

Si el numero de eventos esperado de fondo $b$ no es conocido, uno debe incluirlo como
un parametro nuisance en la funcion likelihood. Pero como $b$ puede ajustarse para cualquier
numero de evenots, es necesario introducir informacion adicional para constrain $b$. En general
se suele hacer mediante una medida auxiliar, mirando el numero de evnetos observados $m$ en una
region de control donde se asume que la senal esta ausente, y se puede considerar que $m$ sigue
una distribucion de Poisson con media $\tau b$, donde $\tau$ es un factor \fix{extrapolacion}.

La funcion likelihood total es el producto de las dos distribuciones de Poisson correspondientes
a cada region:

\begin{equation}
  L(\mu, \btheta) = \Pois_\text{SR}(n;s+b) \Pois_\text{CR}(\tau b)
\end{equation}

Utilizando la aproximacion $Z = \sqrt{q_0}$, valida en el limite de una gran muestra y
teniendo en cuenta que los valores esperados son $s+b$ y $\tau b$ para obtener la significancia
mediana, tenemos

\begin{equation}
  Z_A = \left[ 2 \left( (s+b) \ln \left[ \frac{s+(1+\tau)b}{(1+\tau)(s+b)} \right] + \tau b  \ln \left[ 1 + \frac{s}{(1+\tau)b} \right] \right) \right]^{1/2}
  \label{eq:Za1}
\end{equation}

Es util expresar la {\eq} \eqref{eq:Za1} en terminos de la incerteza que uno quiere
atribuirle al fondo basados en la medida de control $m$. El estimador para $b$ esta dado
por $\hat{b} = m/\tau$, y como la varianza de $m$ es igual a su media $\tau b$, la
varianza de $\hat{b}$ es $V[\hat{b}] = \sigma_b^2 = b/\tau$. Y usando esto para eliminar
$\tau$ de la {\eq} \eqref{eq:Za1} obtenemos:

\begin{equation}
  Z_A = \left[ 2 \left( (s+b) \ln \left[ \frac{(s+b)(b+\sigma_b^2)}{b^2+(s+b)\sigma_b^2} \right] - \frac{b^2}{\sigma_b^2} \ln \left[ 1 + \frac{\sigma_b^2 s}{b(b+\sigma_b^2)} \right] \right) \right]^{1/2}
  \label{eq:Za}
\end{equation}

%% \subsection{Psuedo-experimentos}

%% The role of intervals in Search Procedures

%% Exclusion A signal hypothesis can be excluded if the compatibility with the s C b
%% hypothesis is ‘small’. Several limit-setting methods exist and are a source of in-
%% tense discussions among physicists, to say the least (see also Section 4.6). Although
%% it might seem natural to define a signal as excluded at 95% confidence level if
%% CL sCb < 5%, there are some undesirable consequences associated with this choice.
%% Near the sensitivity limit, where the test statistic distribution for the b-only and sCb
%% hypotheses are not well separated (either because the signal is small or the analysis
%% is not powerful enough to separate signal and background), a downward fluctua-
%% tion in the data with respect to the b-only expectation will result in the exclusion
%% of a signal while the analysis has no real sensitivity. One of the common solutions

%% experiments invoke to address this issue is to correct for a downward fluctuation
%% by using the so-called CL s method [6], where a signal is called ‘excluded’ at 95%
%% confidence level if CL s < 0.05, where CL s


%% Para ilustrar el uso del likelihood ratio, consideremos un experimento en el cual por cada evento,
%% se miden los valores de ciertas variables cinematicas, entonces los datos pueden ser representados
%% por uno o mas histogramas.

%% Supongamos el histograma $\vec{n} = (n_1, n_2, \ldots, n_n)$. El valor esperado para $n_i$ puede
%% escribirse $E[n_i] = \mu s_i + b_i$, donde el valor medio de entradas del bin $i$ de se\~nal y fondo
%% son:

%% El parametro $\mu$ determina la intensidad de la nueva se\~nal, donde $\mu=0$ corresponde a la
%% hipotesis de solo-fondo, y $\mu=1$ es la hipotesis de se\~nal.
%% Las funciones $f_s$ y $f_b$ son las \pdf\ de la variable x para los eventos de se\~nal y fondo, y
%% $\theta_s$ y $\theta_b$ representan parametros que caracterizan la forma de las pdfs. Las
%% cantidades $s_\text{tot}$ y $b_\text{tot}$ son el valor medio total de se\~nal y fondo, y las
%% integrales en \ref{} y \ref{} representan la probabilidad de que un evento sea encontrado en el
%% bin $i$. En lo que sigue $\bm{\theta} = (\bm{\theta}_s, \bm{\theta}_b, b_\text{tot})$

%% Además del histograma $\bm{n}$, se utilizan otras medidas adicionales que ayuden a restringir
%% los parametros nuisance. Por ejemplo, se pueden elegir regiones de control donde uno espera
%% mayormente fondo, y construit un histograma $\bm{m}$. Los valores esperados de $m_i$ pueden
%% escribirse como $E[m_i] = u_i(\bm{\theta})$, donde los $u_i$ son cantidades calculadas a partir
%% de $\bm{\theta}$.

%% \itodo{
%%   It is often said that the language of science is mathematics. It could well be said that the language of experimental science is statistics.
%%   It is through statistical concepts that we quatify
%%   the correspondence between theoretical predicticions and experimental observations. While the statistical
%%   analysis of the dta is often treated as a final subsidiary step to an experimental physcis
%%   result, a more direct approach would be quite the opposite. In fact, thinking through the
%%   requirements for a robust statistical statement is an excellent way to organize an analysis
%%   strategy.
%%   }





%% \subsection{Probabilidad}

%% La definición matemática de probabilidad

%% Consideremos A un evento. Luego la probabilidad $P(A)$ es un numero que obedece los tres
%% axiomas de Kolmogorov\cite{Kolmogorov}.

%% \begin{align}
%%   &P(A) \geq 0, \\
%%   &P(S) = 1, \\
%%   &A \cap B = \phi \Rightarrow P(A \cup B) = P(A) + P(B)
%% \end{align}
%% %
%% donde $S$ es el conjunto de A.

%% La definición frecuentista de probabilidad esta dada por:

%% \begin{equation}
%%   P(A) = \lim_{n\to\infty} \frac{\text{resultado es} A}{n}
%% \end{equation}


%% \begin{equation}
%%   P(A|B) = \frac{P(B|A) P(A)}{P(B)}
%% \end{equation}
%% %
%% donde $P(B|A)$ es la probabilidad de B dado A, $P(A)$ es la probabilidad \emph{a priori} de A,
%% y ...
