\chapter{Resultados e interpretación}
\label{cap:resultados}

Como se describió en el \cref{cap:estrategia}, en el que se presentó la
estrategia general para la búsqueda de nueva física realizada en esta tesis, a
fin de obtener los resultados finales es necesario realizar un complejo análisis
estadístico. Los detalles de este análisis se describen en la
\cref{sec:analisis}. Este se lleva a cabo utilizando las regiones de control
para la normalización de los fondos, las regiones de validación y las regiones
donde se espera la señal de nueva física, definidas en los capítulos anteriores.
También deben ser incluidas las incertezas sistemáticas, tanto teóricas como
experimentales, las cuales se presentan en la \cref{sec:sistematicos}.

Los resultados del ajuste del \emph{likelihood} se presentan en la
\cref{sec:bkgonlyfit} para las distintas regiones, para luego, de acuerdo a los
resultados obtenidos, establecer los límites a los procesos de nueva física en
la \cref{sec:model_independent}. Estos resultados son finalmente interpretados
en el contexto de un modelo particular de supersimetría, el cual motiva el
trabajo de esta tesis, en la \cref{sec:susy_limits}.

Por último, en la \cref{sec:results13tev}, se muestra la significancia esperada
para el análisis de los datos $\sqrt{s} = 13 \tev$ recolectados por ATLAS durante
el a\~no 2015.


\section{Análisis estadístico}
\label{sec:analisis}

A partir de la función \emph{likelihood} general (\cref{eq:model}), se define la
función \emph{likelihood} utilizada en el análisis específico de esta tesis como:

\begin{equation}
  L(\bm{n}|\mu_s, \bm{\mu}_p, \bm{s}, \bm{b}, \bm{\alpha}) = \mathcal{P}_\text{SR} \, \times \, \mathcal{P}_\text{CRQ} \, \times \, \mathcal{P}_\text{CRW} \, \times \, \mathcal{P}_\text{CRT} \, \times \, \mathcal{C}_\text{syst}
  \label{eq:likelihood}
\end{equation}
%
donde $\bm{n}$ es el número de eventos observado en cada región (de control y
señal) y $\mu_s$ es la intensidad de la señal que constituye el parámetro de
interés del análisis. Los demás son parámetros \emph{nuisance}. Por
un lado, los parámetros de normalización de los fondos ($\bm{\mu}_p$), en este
caso $\mu_Q$ para el fondo de {\gjet}, $\mu_W$ para {\wgam} y $\mu_T$ para
{\ttgam} así como  el número de eventos esperado de señal y fondo en cada región
($\bm{s}$ y $\bm{b}$ respectivamente) que provienen de las muestras MC o
de aquellos calculados a partir de los datos como se describió en el
\cref{cap:fondos}. Por otro lado, los parámetros $\bm{\alpha}$ parametrizan las incertezas
sistemáticas en la señal y el fondo utilizando una distribución normal.

Con la función \emph{likelihood}, se construye el estadístico de prueba como el
\emph{profile likelihood ratio},

\begin{equation}
  q_{\mu_s} = -2 \ln \left( \frac{L(\mu_s,
    \doublehat{\btheta})}{L(\hat{\mu}_s,\hat{\btheta})} \right)
\end{equation}
%
que será utilizado en este análisis para poner a prueba la compatibilidad de los
datos observados con las predicciones del SM.

Realizando un ajuste simultáneo a los datos en las CR (o en las CR y SR), los
parámetros descriptos anteriormente pueden ser estimados teniendo en cuenta
apropiadamente sus correlaciones e incertezas. Además, como los parámetros de
normalización de los fondos ($\bm{\mu}_p$) dependen esencialmente de las
regiones de control con una alta estadística de datos (respecto a la SR), se
reduce el impacto de estas incertezas estadísticas en las SR.

El análisis estadístico se implementó utilizando
\textsc{HistFitter}\cite{HistFitter}, una herramienta desarrollada por el grupo
de SUSY de ATLAS, la cual utiliza las librerias \textsc{RooFit},
\textsc{RooStats}\cite{Moneta:2010pm} y
\textsc{HistFactory}\cite{Cranmer:1456844}. El hecho de utilizar una
misma herramienta en distintos análisis dentro de la colaboración, permite que
su combinación estadística sea más sencilla.


%--------------
% Sistematicos
%--------------
\chapter{Determinación de las incertezas sistemáticas}



%--------------
% Bkg-only fit
%--------------
\section{Ajuste simultáneo en las regiones de control}
\label{sec:bkgonlyfit}

Para validar los métodos utilizados en la estimación de los fondos, se realiza un
ajuste simultáneo utilizando únicamente las CR. El objetivo de este ajuste es
estimar el fondo total esperado en las VR y las SR, sin hacer ninguna suposición
del modelo de señal. Solo las muestras de fondo son incorporadas al modelo, y
las CR se suponen libres de contaminación de señal. Cabe destacarse que el
ajuste se realiza solamente en las CR, y los procesos de fondo dominantes son
normalizados al número de eventos observado en cada una de las regiones. La
función \emph{likelihood} utilizada es la de la \cref{eq:likelihood} sin el término de
la SR. Como los parámetros de la pdf correspondientes al fondo son compartidos
entre las diferentes regiones, el resultado del ajuste puede ser usado
luego, para predecir el número de eventos en las VR y SR.

Las predicciones a partir del ajuste en las CR son independientes del número de
eventos observado en cada SR y VR, lo que permite una comparación no sesgada
entre el número de eventos predicho y observado en cada región. Los resultados
de este ajuste extrapolados a las SR, también son importantes para que grupos
externos al experimento puedan hacer una prueba de hipótesis en un modelo de
nueva física no considerado por ninguno de los experimentos del LHC.

En las siguientes secciones se muestran los resultados obtenidos a partir
de este ajuste en las distintas regiones del análisis.


% Control Regions
\subsection{Resultados en las regiones de control}

El número de eventos predicho para cada proceso de fondo antes y después del ajuste en
las regiones de control puede verse en \cref{tab:fit_result_crl,tab:fit_result_crh},
comparado con el número de eventos observado.
Estas regiones no solo determinan la normalización de los procesos de fondo dominante
en cada una, sino que también restringen las incertezas sistemáticas aplicadas a todos
los fondos. Es importante observar que las variables relevantes del análisis se encuentren
bien modeladas en las regiones de control, lo cual puede observarse en las \cref{fig:bkgfit_crl_after,fig:bkgfit_crh_after}.

En la \cref{tab:bkgonly_mus} se presentan los parámetros de normalización de
los fondos {\wgam} ($\mu_W$), {\ttgam} ($\mu_T$) y {\gjet}
($\mu_Q$), obtenidos del ajuste realizado en las regiones de control.
En su mayoría, los factores resultan consistentes con la unidad dentro de sus incertezas.

En la \cref{fig:fit_unc_nuisance} se encuentran resumidos todos los parámetros
del ajuste simultáneo (las incertezas sistemáticas y los factores de
normalización de los fondos). Los valores de los puntos representan los
\emph{pulls} en los parámetros \emph{nuisance}, lo que indica el apartamiento de los valores
centrales luego del ajuste. Las líneas de error representan el
 grado de disminución de las incertezas por el
ajuste. Si este es muy grande puede resultar en un aumento artificial
en la sensibilidad a nueva física, reduciendo las incertezas sistemáticas.
Por este motivo resulta importante
chequear el comportamiento de los parámetros \emph{nuisance} después del ajuste.
En este caso se observa que todos los parámetros dan resultados razonables
y el valor central es practicamente cero.

Los puntos de la derecha de la \cref{fig:fit_unc_nuisance}, después de la linea
punteada, representan los parámetros de normalización de los distintos fondos. Se
puede ver que todos los valores centrales estan contenidos dentro de una desviación estándar.
%% La correlación de los distintos parámetros se muestran en la \cref{fig:fit_corrmatrix}.

\begin{table}[!htb]
  \centering

  \caption{Resultados del ajuste en las CR correspondientes a {\SRL}.
    El número de eventos observado es comparado con el número de eventos esperado de fondo, después de la correspondiente
    normalización en las CR. En la parte inferior de la tabla se muestran también lo valores nominales del fondo antes de
    la correspondiente normalización. Las incertezas incluyen la incerteza estadística y sistemática.}
  \label{tab:fit_result_crl}

  {\small\begin{tabularx}{\textwidth}{|z{3mm}|z{5cm}RRR}
\hline
\multicolumn{2}{l}{\hspace{6mm} \bf Regiones de control}           & {\CRQL}            & {\CRWL}            & {\CRTL}              \\
\hline
\multicolumn{2}{l}{\hspace{6mm} Eventos observados}                             & $1348$              & $8$              & $18$                    \\
\hline
\multirow{11}{*}{\rotatebox[origin=c]{90}{Después}} & Eventos esperados SM                           & $1347.21 \pm 36.51$          & $8.00 \pm 2.75$          & $18.00 \pm 4.16$              \\
\cline{2-5}
 & {\wgam}                                 & $0.28 \pm 0.27$          & $5.24 \pm 3.07$          & $3.15 \pm 1.94$              \\
 & {\ttgam}                                & $30.68 \pm 1.31$          & $1.33 \pm 0.73$          & $10.66 \pm 4.92$              \\
 %%& {\ttgam} (had)                          & $28.46 \pm 16.90$          & $0.00 \pm 0.00$          & $0.00 \pm 0.00$              \\
 & {\vqqgam}                    & $39.30 \pm 24.90$          & $0.00 \pm 0.00$          & $0.00 \pm 0.00$              \\
 & {\tgam}                   & $0.24 \pm 0.04$          & $0.22 \pm 0.06$          & $1.24 \pm 0.24$              \\
 & {\zllgam}     & $0.08_{-0.08}^{+0.10}$          & $0.14_{-0.14}^{+0.14}$          & $0.11_{-0.11}^{+0.11}$              \\
 & {\znngam}       & $0.00 \pm 0.00$          & $0.00 \pm 0.00$          & $0.00 \pm 0.00$              \\
 & {\gjet}                          & $1155.79 \pm 68.55$          & $0.13 \pm 0.04$          & $0.01 \pm 0.01$              \\
 & $e\rightarrow\gamma$              & $5.34 \pm 0.89$          & $0.26 \pm 0.05$          & $1.29 \pm 0.21$              \\
 & $j\rightarrow\gamma$              & $115.48 \pm 54.49$          & $0.69 \pm 0.32$          & $1.54 \pm 0.72$              \\
\hline
\multirow{11}{*}{\rotatebox[origin=c]{90}{Antes}} & Eventos esperados SM                            & $1397.30$          & $6.29$          & $14.14$              \\
\cline{2-5}
& {\wgam}                                & $0.21$          & $3.90$          & $2.34$              \\
& {\ttgam}                               & $21.90$          & $0.95$          & $7.61$              \\
& {\vqqgam}                   & $29.25$          & $0.00$          & $0.00$              \\
& {\tgam}                  & $0.24$          & $0.22$          & $1.24$              \\
& {\zllgam}    & $0.08$          & $0.14$          & $0.11$              \\
& {\znngam}      & $0.00$          & $0.00$          & $0.00$              \\
& {\gjet}                                & $1224.79$          & $0.14$          & $0.01$              \\
& $e\rightarrow\gamma$             & $5.34$          & $0.26$          & $1.29$              \\
& $j\rightarrow\gamma$             & $115.48$          & $0.69$          & $1.54$              \\
\hline
\end{tabularx}
}

\end{table}


\begin{table}[!htb]

  \caption{Resultados del ajuste en las CR correspondientes a {\SRH}.
    El número de eventos observado es comparado con el número de eventos esperado de fondo, después de la correspondiente
    normalización en las CR. En la parte inferior de la tabla se muestran también lo valores nominales del fondo antes de
    la correspondiente normalización. Las incertezas incluyen la incerteza estadística y sistemática.}
  \label{tab:fit_result_crh}

  \begin{tabularx}{\textwidth}{z{5cm}RRR}
\hline
{\bf Regiones de control}           & {\CRQH}     & {\CRWH}            & {\CRTH}              \\
\hline
Eventos observados          & $216$              & $25$              & $17$                    \\
\hline
Eventos esperados SM        & $216.03 \pm 14.53$          & $25.00 \pm 4.99$          & $17.00 \pm 4.08$              \\
\hline
{\wgam}         & $0.10 \pm 0.08$          & $19.32 \pm 5.62$          & $4.66 \pm 1.43$              \\
{\ttgam}          & $0.04 \pm 0.04$          & $1.91 \pm 1.34$          & $6.88 \pm 4.63$              \\
{\ttgam} (had)          & $0.31 \pm 0.24$          & $0.00 \pm 0.00$          & $0.00 \pm 0.00$              \\
V($\to$ qq)$+\gamma$          & $3.46 \pm 1.53$          & $0.00 \pm 0.00$          & $0.00 \pm 0.00$              \\
single-$t$ + $\gamma$          & $0.01_{-0.01}^{+0.02}$          & $0.54 \pm 0.06$          & $1.51 \pm 0.18$              \\
Z($\rightarrow\ell\ell$) + $\gamma$          & $0.00 \pm 0.00$          & $0.57 \pm 0.57$          & $0.19_{-0.19}^{+0.19}$              \\
Z($\rightarrow\nu\nu$) + $\gamma$          & $0.00 \pm 0.00$          & $0.00 \pm 0.00$          & $0.00 \pm 0.00$              \\
$\gamma$ + jet          & $193.31 \pm 16.91$          & $0.02_{-0.02}^{+0.94}$          & $0.00 \pm 0.00$              \\
$e\rightarrow\gamma$ fakes          & $0.27 \pm 0.04$          & $0.49 \pm 0.08$          & $2.31 \pm 0.37$              \\
$j\rightarrow\gamma$ fakes          & $18.52 \pm 8.70$          & $2.14 \pm 0.99$          & $1.46 \pm 0.68$              \\
\hline
Eventos esperado SM (MC/DD)              & $179.24$          & $22.94$          & $22.04$              \\
\hline
MC {\wgam}               & $0.08$          & $15.62$         & $3.76$              \\
MC {\ttgam}              & $0.08$          & $3.57$          & $12.82$              \\
MC {\ttgam} (had)        & $0.58$          & $0.00$          & $0.00$              \\
MC {\vqqgam}             & $2.80$          & $0.00$          & $0.00$              \\
MC {\tgam}               & $0.01$          & $0.54$          & $1.51$              \\
MC {\zllgam}             & $0.00$          & $0.57$          & $0.19$              \\
MC {\znngam}             & $0.00$          & $0.00$          & $0.00$              \\
MC {\gjet}               & $156.91$        & $0.02$          & $0.00$              \\
DD $e\rightarrow\gamma$  & $0.27$          & $0.49$          & $2.31$              \\
DD $j\rightarrow\gamma$  & $18.50$         & $2.14$          & $1.46$              \\
\hline
\end{tabularx}


\end{table}


\begin{table}[!h]
  \centering

  \caption{Factores de normalización para los fondos
    {\wgam} ($\mu_{W}$), {\ttgam} ($\mu_{T}$) y {\gjet} ($\mu_{Q}$), obtenidos
    del ajuste combinado en las CR. Las incertezas mostradas son solo estadísticas.}
  \label{tab:bkgonly_mus}

  \begin{tabularx}{0.6\textwidth}{LCCC}
    \hline
    &       $\mu_{Q}$ &       $\mu_{W}$ &       $\mu_{T}$ \\
    \hline
    \SRL & $0.94 \pm 0.44$ & $1.34 \pm 0.89$ & $1.40 \pm 0.77$ \\
    \SRH & $1.22 \pm 0.58$ & $1.24 \pm 0.39$ & $0.54 \pm 0.37$ \\
    \hline
  \end{tabularx}

\end{table}


\begin{figure}[!h]
  \centering

  \includegraphics[width=0.9\textwidth]{syst_pull_afterFit_srl} \\
  \includegraphics[width=0.9\textwidth]{syst_pull_afterFit_srh}

  \caption{Resumen de los parámetros después del ajuste para {\SRL} (arriba) y {\SRH} (abajo).}
  \label{fig:fit_unc_nuisance}

\end{figure}


%% \begin{figure}[!htb]
%%   \centering

%%   \includegraphics[width=0.49\textwidth]{corr_matrix_srl}
%%   \includegraphics[width=0.49\textwidth]{corr_matrix_srh}

%%   \caption{Matriz de correlación de los parámetros del ajuste, correspondientes a {\SRL} (izquierda) y {\SRH} (derecha).}
%%   \label{fig:fit_corrmatrix}

%% \end{figure}


\begin{figure}[!h]
  \centering

  \includegraphics[width=0.45\textwidth]{plot_crw_l_ph_pt_after} \hspace{1cm}
  \includegraphics[width=0.45\textwidth]{plot_crw_l_rt4_after} \\

  \includegraphics[width=0.45\textwidth]{plot_crt_l_ph_pt_after} \hspace{1cm}
  \includegraphics[width=0.45\textwidth]{plot_crt_l_rt4_after} \\

  \includegraphics[width=0.45\textwidth]{plot_crq_l_ph_pt_after} \hspace{1cm}
  \includegraphics[width=0.45\textwidth]{plot_crq_l_rt4_after} \\

   \caption{Distribuciones observadas de $\pt^\gamma$ (izquierda) y {\rt} (derecha) en las
     regiones de control {\CRWL} (arriba), {\CRTL} (medio) y {\CRQL} (abajo),
     después del ajuste.}
   \label{fig:bkgfit_crl_after}

\end{figure}


\begin{figure}[!htb]
  \centering

  \includegraphics[width=0.45\textwidth]{plot_crw_h_ph_pt_after} \hspace{1cm}
  \includegraphics[width=0.45\textwidth]{plot_crw_h_ht_after} \\

  \includegraphics[width=0.45\textwidth]{plot_crt_h_ph_pt_after} \hspace{1cm}
  \includegraphics[width=0.45\textwidth]{plot_crt_h_ht_after} \\

  \includegraphics[width=0.45\textwidth]{plot_crq_h_ph_pt_after} \hspace{1cm}
  \includegraphics[width=0.45\textwidth]{plot_crq_h_ht_after} \\

  \caption{Distribuciones observadas de $\pt^\gamma$ (izquierda) y {\HT} (derecha) en las
    regiones de control {\CRWH} (arriba), {\CRTH} (medio) y {\CRQH} (abajo),
    después del ajuste.}
  \label{fig:bkgfit_crh_after}

\end{figure}





%% Validation Regions
\subsection{Resultados en las regiones de validación}

Los factores de normalización calculados a partir del ajuste simultáneo en las
CR, y detallados en la \cref{tab:bkgonly_mus}, son utilizados para estimar los
fondos en las VR, donde se compara el fondo esperado con los datos observados, a
fin de validar la extrapolación que luego se hará a la SR.

En las \cref{tab:fit_result_vrl,tab:fit_result_vrh} se presentan los resultados
para las distintas regiones de validación asociadas a {\SRL} y {\SRH},
respectivamente, y en la \cref{fig:bkgfit_vr} pueden verse las distribuciones
de algunos observables en algunas de estas regiones. En general, el número de
eventos de fondo predicho está en buen acuerdo con el número de eventos
observado dentro de las incertezas, lo que permite validar el método empleado y
la viabilidad de utilizar la estimación del número de eventos de fondo en las
SR. Este resultado puede visualizarse más fácilmente en la
\cref{fig:fit_region_composition} donde se presentan los \emph{pull} para las
distintas regiones, definido como $\frac{n_\text{obs} - n_\text{exp}}{\sigma_\text{tot}}$,
donde $\sigma_\text{tot}$ es la suma de la incerteza estadística y sistemática.
El mismo acuerdo puede verse de las distribuciones de los
observables en las VR, que confirman que no existen problemas en el
modelado de los distintos observables.

Habiendo validado el método de obtención del fondo contaminante en la región
donde se espera la señal de nueva física, se procedió a la utilización de los
datos experimentales en las denominadas regiones de señal, cuyos resultados
pueden verse en la siguiente sección.


\begin{table}[!htb]

  \caption{Resultados del ajuste en las VR correspondientes a {\SRL}.
    El número de eventos observado es comparado con el número de eventos
    esperado de fondo, después de la correspondiente normalización en las CR.
    Las incertezas incluyen la incerteza estadística y sistemática.}
  \label{tab:fit_result_vrl}

  \begin{tabularx}{\textwidth}{z{4cm}RRRR}
\hline
{\bf Regiones de validación}                  & VRQ            & VRM75            & VRM100            & VRR              \\
\hline
Eventos observados        & $0$            & $54$              & $7$              & $24$                    \\
\hline
Eventos esperados SM    & $0.39 \pm 0.23$  & $51.17 \pm 47.04$          & $13.24 \pm 5.51$          & $24.32 \pm 6.26$              \\
\hline
{\wgam}          & $0.04 \pm 0.04$    & $0.89 \pm 0.79$          & $0.45 \pm 0.40$          & $6.82 \pm 5.74$              \\
{\ttgam}          & $0.30 \pm 0.20$             & $2.55 \pm 1.65$          & $1.54 \pm 1.01$          & $4.74 \pm 2.80$              \\
{\ttgam}  (had)    & $0.02 \pm 0.02$         & $1.69 \pm 1.08$          & $0.38 \pm 0.25$          & $0.00 \pm 0.00$              \\
{\vqqgam}          & $0.00 \pm 0.00$              & $2.45 \pm 2.17$          & $0.82 \pm 0.82$          & $0.00 \pm 0.00$              \\
{\tgam}          & $0.01 \pm 0.01$             & $0.25 \pm 0.10$          & $0.15 \pm 0.06$          & $0.45 \pm 0.08$              \\
{\zllgam}         & $0.00 \pm 0.00$          & $0.08_{-0.08}^{+0.08}$          & $0.04_{-0.04}^{+0.04}$          & $0.08_{-0.08}^{+0.08}$              \\
{\znngam}         & $0.00 \pm 0.00$           & $0.10_{-0.10}^{+0.12}$          & $0.07_{-0.07}^{+0.07}$          & $1.71_{-1.71}^{+1.75}$              \\
{\gjet}        & $0.01_{-0.01}^{+0.07}$            & $35.97_{-35.97}^{+44.72}$          & $7.87 \pm 4.47$          & $2.37 \pm 1.17$              \\
$e\rightarrow\gamma$           & $0.01 \pm 0.00$        & $2.56 \pm 0.72$          & $1.32 \pm 0.43$          & $6.09 \pm 1.14$              \\
$j\rightarrow\gamma$           & $0.00 \pm 0.00$          & $4.63 \pm 2.41$          & $0.60 \pm 0.33$          & $2.06 \pm 0.98$              \\
\hline
%% Eventos esperados SM (MC/DD)              & $0.29$           & $51.24$          & $12.83$          & $21.36$              \\
%% \hline
%% MC W + $\gamma$          & $0.03$                & $0.66$          & $0.33$          & $5.08$              \\
%% MC $t\bar{t}$ + $\gamma$          & $0.21$         & $1.82$          & $1.10$          & $3.39$              \\
%% MC $t\bar{t}$ + $\gamma$ (had)          & $0.01$            & $1.20$          & $0.27$          & $0.00$              \\
%% MC V($\to$ qq)$+\gamma$          & $0.00$             & $1.82$          & $0.61$          & $0.00$              \\
%% MC single-$t$ + $\gamma$          & $0.01$               & $0.25$          & $0.15$          & $0.45$              \\
%% MC Z($\rightarrow\ell\ell$) + $\gamma$          & $0.00$              & $0.08$          & $0.04$          & $0.08$              \\
%% MC Z($\rightarrow\nu\nu$) + $\gamma$          & $0.00$                & $0.10$          & $0.07$          & $1.71$              \\
%% MC $\gamma$ + jet          & $0.01$              & $38.11$          & $8.34$          & $2.51$              \\
%% DD $e\rightarrow\gamma$         & $0.01$             & $2.56$          & $1.32$          & $6.09$              \\
%% DD $j\rightarrow\gamma$         & $0.00$              & $4.63$          & $0.60$          & $2.06$              \\
%% \hline
\end{tabularx}


  \bigskip

  \begin{tabularx}{\textwidth}{z{4cm}RRRR}
\hline
{\bf Regiones de validación}                   & VRWR            & VRWM            & VRTR            & VRTM              \\
\hline
Eventos observados                             & $0$             & $5$             & $1$             & $3$                    \\
\hline
Eventos esperados SM                           & $0.20 \pm 0.15$          & $3.41 \pm 1.41$          & $0.76 \pm 0.30$          & $3.52 \pm 0.84$              \\
\hline
{\wgam}               & $0.13 \pm 0.10$          & $2.39 \pm 1.47$          & $0.04 \pm 0.03$          & $1.04 \pm 0.65$              \\
{\ttgam}              & $0.04_{-0.04}^{+0.07}$   & $0.41 \pm 0.21$          & $0.46 \pm 0.28$          & $1.86 \pm 0.92$              \\
{\ttgam} (had)        & $0.00 \pm 0.00$          & $0.00 \pm 0.00$          & $0.00 \pm 0.00$          & $0.00 \pm 0.00$              \\
{\vqqgam}             & $0.00 \pm 0.00$          & $0.00 \pm 0.00$          & $0.00 \pm 0.00$          & $0.00 \pm 0.00$              \\
{\tgam}               & $0.03 \pm 0.02$          & $0.04 \pm 0.01$          & $0.08 \pm 0.03$          & $0.19 \pm 0.04$              \\
{\zllgam}             & $0.00 \pm 0.00$          & $0.06_{-0.06}^{+0.06}$   & $0.02_{-0.02}^{+0.02}$   & $0.00 \pm 0.00$              \\
{\znngam}             & $0.00 \pm 0.00$          & $0.00 \pm 0.00$          & $0.00 \pm 0.00$          & $0.00 \pm 0.00$              \\
{\gjet}               & $0.00 \pm 0.00$          & $0.00 \pm 0.00$          & $0.00 \pm 0.00$          & $0.00 \pm 0.00$              \\
$e\rightarrow\gamma$  & $0.00 \pm 0.00$          & $0.07 \pm 0.02$          & $0.08 \pm 0.02$          & $0.17 \pm 0.03$              \\
$j\rightarrow\gamma$  & $0.00 \pm 0.00$          & $0.43 \pm 0.21$          & $0.09 \pm 0.04$          & $0.26 \pm 0.12$              \\
\hline
%% MC exp. SM events              & $0.16$          & $2.68$          & $0.62$          & $2.72$              \\
%% \noalign{\smallskip}\hline\noalign{\smallskip}
%%         MC exp. W + $\gamma$ events         & $0.09$          & $1.78$          & $0.03$          & $0.77$              \\
%%         MC exp. $t\bar{t}$ + $\gamma$ events         & $0.03$          & $0.29$          & $0.33$          & $1.33$              \\
%%         MC exp. $t\bar{t}$ + $\gamma$ (had) events         & $0.00$          & $0.00$          & $0.00$          & $0.00$              \\
%%         MC exp. V($\to$ qq)$+\gamma$ events         & $0.00$          & $0.00$          & $0.00$          & $0.00$              \\
%%         MC exp. single-$t$ + $\gamma$ events         & $0.03$          & $0.04$          & $0.08$          & $0.19$              \\
%%         MC exp. Z($\rightarrow\ell\ell$) + $\gamma$ events         & $0.00$          & $0.06$          & $0.02$          & $0.00$              \\
%%         MC exp. Z($\rightarrow\nu\nu$) + $\gamma$ events         & $0.00$          & $0.00$          & $0.00$          & $0.00$              \\
%%         MC exp. $\gamma$ + jet events         & $0.00$          & $0.00$          & $0.00$          & $0.00$              \\
%%         DD exp. $e\rightarrow\gamma$ fakes events         & $0.00$          & $0.07$          & $0.08$          & $0.17$              \\
%%         DD exp. $j\rightarrow\gamma$ fakes events         & $0.00$          & $0.43$          & $0.09$          & $0.26$              \\

%% \noalign{\smallskip}\hline\noalign{\smallskip}
\end{tabularx}


\end{table}


\begin{table}[!htb]

  \caption{Resultados del ajuste en las VR correspondientes a {\SRH}.
    El número de eventos observado es comparado con el número de eventos
    esperado de fondo, después de la correspondiente normalización en las CR.
    Las incertezas incluyen la incerteza estadística y sistemática.}
  \label{tab:fit_result_vrh}

  \begin{tabularx}{\textwidth}{LRR}
\hline
{\bf Regiones de validación}           & VRQ                      & VRH                          \\
\hline
Eventos observados                     & $2$                      & $4$                          \\
\hline
Eventos esperados SM                   & $0.89 \pm 0.32$          & $3.39 \pm 1.69$              \\
\hline
{\wgam}                                & $0.34 \pm 0.19$          & $1.18 \pm 0.55$              \\
{\ttgam}                               & $0.07 \pm 0.05$          & $0.11 \pm 0.08$              \\
{\ttgam} (had)                         & $0.01 \pm 0.01$          & $0.00 \pm 0.00$              \\
{\vqqgam}                              & $0.00 \pm 0.00$          & $0.00 \pm 0.00$              \\
{\tgam}                                & $0.03 \pm 0.01$          & $0.01 \pm 0.00$              \\
{\zllgam}                              & $0.02_{-0.02}^{+0.02}$   & $0.00 \pm 0.00$              \\
{\znngam}                              & $0.08_{-0.08}^{+0.08}$   & $1.60_{-1.60}^{+1.61}$       \\
{\gjet}                                & $0.19 \pm 0.11$          & $0.00 \pm 0.00$              \\
$e\rightarrow\gamma$                   & $0.00 \pm 0.00$          & $0.15 \pm 0.03$              \\
$j\rightarrow\gamma$                   & $0.17 \pm 0.09$          & $0.34 \pm 0.16$              \\
\hline
%% MC exp. SM               & $0.86$          & $3.25$              \\
%% \hline
%% MC W + $\gamma$          & $0.27$          & $0.96$              \\
%% MC $t\bar{t}$ + $\gamma$          & $0.12$          & $0.20$              \\
%% MC $t\bar{t}$ + $\gamma$ (had)          & $0.02$          & $0.00$              \\
%% MC V($\to$ qq)$+\gamma$          & $0.00$          & $0.00$              \\
%% MC single-$t$ + $\gamma$          & $0.03$          & $0.01$              \\
%% MC Z($\rightarrow\ell\ell$) + $\gamma$          & $0.02$          & $0.00$              \\
%% MC Z($\rightarrow\nu\nu$) + $\gamma$          & $0.08$          & $1.60$              \\
%% MC $\gamma$ + jet          & $0.15$          & $0.00$              \\
%% DD $e\rightarrow\gamma$          & $0.00$          & $0.15$              \\
%% DD $j\rightarrow\gamma$          & $0.17$          & $0.34$              \\
%% \hline
\end{tabularx}


  \bigskip

  \begin{tabularx}{\textwidth}{z{4cm}RRRR}
\hline
{\bf Regiones de validación}                      & VRWH            & VRWM            & VRTH            & VRTM              \\
\hline
Eventos observados                                & $1$              & $5$              & $2$              & $2$                    \\
\hline
Eventos esperados SM                              & $1.25 \pm 0.38$          & $4.38 \pm 1.14$          & $0.83 \pm 0.26$          & $1.76 \pm 0.38$              \\
\hline
{\wgam}                      & $1.08 \pm 0.38$          & $3.75 \pm 1.17$          & $0.28 \pm 0.17$          & $1.08 \pm 0.35$              \\
{\ttgam}                     & $0.04 \pm 0.04$          & $0.13 \pm 0.09$          & $0.17 \pm 0.13$          & $0.38 \pm 0.27$              \\
{\ttgam} (had)               & $0.00 \pm 0.00$          & $0.00 \pm 0.00$          & $0.00 \pm 0.00$          & $0.00 \pm 0.00$              \\
{\vqqgam}                    & $0.00 \pm 0.00$          & $0.00 \pm 0.00$          & $0.00 \pm 0.00$          & $0.00 \pm 0.00$              \\
{\tgam}                      & $0.02 \pm 0.01$          & $0.02 \pm 0.01$          & $0.15 \pm 0.04$          & $0.08 \pm 0.02$              \\
{\zllgam}                    & $0.00 \pm 0.00$          & $0.00 \pm 0.00$          & $0.02_{-0.02}^{+0.02}$   & $0.00 \pm 0.00$              \\
{\znngam}                    & $0.00 \pm 0.00$          & $0.00 \pm 0.00$          & $0.00 \pm 0.00$          & $0.00 \pm 0.00$              \\
{\gjet}                      & $0.02 \pm 0.01$          & $0.00 \pm 0.00$          & $0.00 \pm 0.00$          & $0.00 \pm 0.00$              \\
$e\rightarrow\gamma$         & $0.00 \pm 0.00$          & $0.04 \pm 0.01$          & $0.05 \pm 0.01$          & $0.05 \pm 0.01$              \\
$j\rightarrow\gamma$         & $0.09 \pm 0.04$          & $0.43 \pm 0.20$          & $0.17 \pm 0.09$          & $0.17 \pm 0.08$              \\
\hline
%% %%
%% MC exp. SM events              & $1.07$          & $3.78$          & $0.92$          & $1.88$              \\
%% \noalign{\smallskip}\hline\noalign{\smallskip}
%% %%
%%         MC exp. W + $\gamma$ events         & $0.87$          & $3.04$          & $0.23$          & $0.87$              \\
%% %%
%%         MC exp. $t\bar{t}$ + $\gamma$ events         & $0.08$          & $0.25$          & $0.31$          & $0.71$              \\
%% %%
%%         MC exp. $t\bar{t}$ + $\gamma$ (had) events         & $0.00$          & $0.00$          & $0.00$          & $0.00$              \\
%% %%
%%         MC exp. V($\to$ qq)$+\gamma$ events         & $0.00$          & $0.00$          & $0.00$          & $0.00$              \\
%% %%
%%         MC exp. single-$t$ + $\gamma$ events         & $0.02$          & $0.02$          & $0.15$          & $0.08$              \\
%% %%
%%         MC exp. Z($\rightarrow\ell\ell$) + $\gamma$ events         & $0.00$          & $0.00$          & $0.02$          & $0.00$              \\
%% %%
%%         MC exp. Z($\rightarrow\nu\nu$) + $\gamma$ events         & $0.00$          & $0.00$          & $0.00$          & $0.00$              \\
%% %%
%%         MC exp. $\gamma$ + jet events         & $0.02$          & $0.00$          & $0.00$          & $0.00$              \\
%% %%
%%         DD exp. $e\rightarrow\gamma$ fakes events         & $0.00$          & $0.04$          & $0.05$          & $0.05$              \\
%% %%
%%         DD exp. $j\rightarrow\gamma$ fakes events         & $0.09$          & $0.43$          & $0.17$          & $0.17$              \\
%% %%     \\
%% \noalign{\smallskip}\hline\noalign{\smallskip}
\end{tabularx}


\end{table}


\begin{figure}[!htb]
  \centering

  \includegraphics[width=0.45\textwidth]{plot_vrm75_l_met_et_after} \hspace{1cm}
  \includegraphics[width=0.45\textwidth]{plot_vrm75_l_rt4_after} \\

  \includegraphics[width=0.45\textwidth]{plot_vrr_l_met_et_after} \hspace{1cm}
  \includegraphics[width=0.45\textwidth]{plot_vrr_l_rt4_after} \\

  \caption{Distribuciones observadas de {\met} (izquierda) y {\rt} (derecha) para algunas de las regiones de validación
    correspondientes a {\SRL}: VRM75 (arriba) y VRR (abajo). }
  \label{fig:bkgfit_vr}

\end{figure}



\subsection{Resultados en las regiones de señal}

Las predicciones del fondo luego del ajuste simultáneo en las CR son presentadas
en la \cref{tab:fit_result_sr} para las regiones de señal, {\SRL} y {\SRH},
donde se espera señal de nueva física, conjuntamente con el número de eventos de
datos observados. De la tabla se desprende que los datos están en
buen acuerdo con el fondo esperado y no se observa un exceso estadísticamente
significativo de datos por sobre los predichos en ninguna de las SR. El número
de eventos esperado a partir de procesos del {\SM} en las regiones {\SRL} y
{\SRH} es de $1.27\pm0.43$ y $0.84\pm0.38$, respectivamente, mientras que el
número de eventos observado en ambas SR es 2.

\begin{table}[!htbp]
  \centering

  \caption{Resultados del ajuste en las SR. El número de eventos observado es comparado con el número de
    eventos esperado de fondo, después de la correspondiente normalización en
    las CR. Las incertezas incluyen la incerteza estadística y sistemática.}
  \label{tab:fit_result_sr}

  \begin{tabularx}{\textwidth}{LRR}
\hline
{\bf Regiones de señal} & {\SRL} & {\SRH}  \\
\hline
Eventos observados                         & $2$                        &  $2$                          \\
\hline
Eventos esperados SM                       & $1.27 \pm 0.43$            &  $0.84 \pm 0.38$              \\
\hline
{\wgam}                                    & $0.13 \pm 0.12$            &  $0.54 \pm 0.28$              \\
{\ttgam}                                    & $0.64 \pm 0.40$            &  $0.05 \pm 0.05$              \\
%% {\vqqgam}                                  & $0.00 \pm 0.00$            &  $0.00 \pm 0.00$              \\
{\tgam}                                    & $0.06 \pm 0.02$            &  $0.03 \pm 0.01$              \\
%% {\zllgam}                                  & $0.00 \pm 0.00$            &  $0.00 \pm 0.00$              \\
{\znngam}                                  & $0.03_{-0.03}^{+0.05}$     &  $0.21_{-0.21}^{+0.23}$       \\
{\gjet}                                    & $0.00_{-0.00}^{+0.06}$     &  $0.00 \pm 0.00$              \\
$e\to\gamma$                       & $0.38 \pm 0.10$            &  $0.00 \pm 0.00$              \\
$j\to\gamma$                       & $0.02_{-0.02}^{+0.08}$     &  $0.00_{-0.00}^{+0.08}$       \\
\hline
\end{tabularx}


\end{table}


Las distribuciones de {\met} para {\SRL} y {\SRH} se muestran en la
\cref{fig:met_sr} para datos y también para el fondo esperado después del ajuste
en las CR, y puede verse el buen acuerdo en la zona de bajo {\met}. Las flechas
indican el corte que define las regiones de señal utilizadas para obtener los
resultados finales del presente análisis.


\begin{figure}[!htbp]

  \centering

  \includegraphics[width=0.45\textwidth]{plot_after_SRL_met_et} \hspace{1cm}
  \includegraphics[width=0.45\textwidth]{plot_after_SRH_met_et}

  \caption{Distribución de {\met} comparando datos y el fondo esperado para la
    selección de la región {\SRL} (izquierda) y {\SRH} (derecha), sin el corte
    en {\met}. La distribución esperada para dos muestras de señal es también
    graficada para su comparación. Las líneas punteadas y las flechas negras indican
    la región de señal.}
  \label{fig:met_sr}

\end{figure}

Una comparación entre el número de eventos observado y esperado en cada una de las
regiones se presenta en la \cref{fig:fit_region_composition} a modo de resumen, donde
el panel inferior muestra el \emph{pull} de cada región, dando una idea de la significancia
estadística de los resultados.
%% Se puede ver el buen acuerdo
%% tanto en las regiones de validación como en las de señal.


\begin{figure}[!htbp]
  \centering

  \includegraphics[width=0.95\textwidth]{histpull_SRL}
  \includegraphics[width=0.95\textwidth]{histpull_SRH}

  \caption{Comparación entre el número de eventos observado y esperado en cada
    una de las regiones correspondientes a {\SRL} (arriba) y {\SRH} (abajo). El panel inferior
    muestra el \emph{pull} definido como $\frac{n_\text{obs} - n_\text{exp}}{\sigma_\text{tot}}$,
    donde $\sigma_\text{tot}$ es la suma de la incerteza estadística y sistemática.}

  \label{fig:fit_region_composition}

\end{figure}


Las incertezas sistemáticas dominantes en el número total predicho de fondo en
cada SR, se presentan en la \cref{tab:syst}.
Debido a que las incertezas individuales pueden estar correlacionadas y a las
correlaciones negativas entre normalizaciones (representando el impacto de la
estadística en las CR en la incerteza total), la incerteza total no es
necesariamente la suma en cuadratura de las mismas. La incerteza estadística se
ilustra con el valor de la raíz cuadrada del número de eventos esperado con la
finalidad de mostrar el impacto relativo de la incerteza estadística con
respecto a la incerteza total del fondo. Las incertezas sistemáticas relativas
con respecto a las predicciones del fondo son del orden de 54\% (58\%) para
{\SRL} (\SRH). En {\SRL} las incertezas dominantes provienen de la normalización
y de las incertezas teóricas del fondo de {\ttgam}, seguidas por la incerteza en
la escala de energía de los jets. En {\SRH}, las
incertezas están dominadas por las incertezas teóricas del fondo {\zgam} y
{\gjet}, y la incerteza en la normalización del fondo {\wgam}.


\begin{table}[!htb]
  \centering

  \caption{Resumen de las incertezas sistemáticas dominantes en la estimación del fondo total
    en {\SRL} y {\SRH}. Notar que las incertezas individuales pueden estar correlacionadas, y la incerteza
    total no es necesariamente la suma en cuadratura de estas. Los porcentajes muestran el tamaño
    de la incerteza relativo al fondo esperado total.}
  \label{tab:syst}

  \begin{tabularx}{\textwidth}{z{5cm}RR}
\hline
{\bf Incertezas}                                    & {\SRL}   & {\SRH}          \\
\hline
Eventos esperados SM             &  $1.27$       & $0.84$ \\
\hline
Incerteza estadística total $(\sqrt{N})$              & $\pm 1.13$     & $\pm 0.92$  \\
Incerteza sistemática total               & $\pm 0.43\ [34.18\%] $             & $\pm 0.38\ [45.27\%] $ \\
\hline
{\tgam}         & $\pm 0.36\ [28.0\%] $      & $\pm 0.02\ [1.8\%] $  \\
$\mu_{T}$         & $\pm 0.35\ [27.7\%] $       & $\pm 0.04\ [4.4\%] $ \\
%%$\gamma_\mathrm{SR}$        & $\pm 0.22\ [17.4\%] $       & $\pm 0.20\ [24.3\%] $ \\
JES         & $\pm 0.21\ [16.5\%] $       & $\pm 0.03\ [4.0\%] $ \\
Teoría {\wgam}         & $\pm 0.09\ [7.3\%] $       & $\pm 0.21\ [25.1\%] $ \\
$\mu_{W}$        & $\pm 0.08\ [6.7\%] $       & $\pm 0.17\ [20.5\%] $\\
Estimación ${j\to\gamma}$         & $\pm 0.08\ [6.3\%] $       & $\pm 0.08\ [9.5\%] $ \\
Estimación ${e\to\gamma}$         & $\pm 0.07\ [5.7\%] $       & - \\
JER         & $\pm 0.06\ [4.3\%] $      & $\pm 0.04\ [4.8\%] $ \\
Teoría ${Z\gamma}$         & $\pm 0.03\ [2.7\%] $       & $\pm 0.21\ [25.4\%] $ \\
EGMAT         & $\pm 0.03\ [2.2\%] $       & $\pm 0.04\ [4.5\%] $ \\
EGRES         & $\pm 0.03\ [2.0\%] $       & $\pm 0.05\ [5.8\%] $\\
Término \emph{soft} de  {\met}         & $\pm 0.02\ [1.7\%] $       & - \\ %% SCALEST
Corrección \emph{pile-up}         & $\pm 0.01\ [0.71\%] $       & $\pm 0.10\ [11.9\%] $ \\
RESOST         & $\pm 0.00\ [0.35\%] $      & $\pm 0.04\ [4.5\%] $ \\
${t\gamma}$         & $\pm 0.00\ [0.34\%] $      & $0.00\ [0.22\%] $  \\
Eficiencia fotones         & $\pm 0.00\ [0.06\%] $       & $\pm 0.00\ [0.40\%] $\\ % PHEFF
$\mu_{Q}$         & $\pm 0.00\ [0.02\%] $       & - \\
Teoría ${\gamma j}$         & $\pm 0.00\ [0.02\%] $ & -      \\
%% MMS      &   - & $\pm 0.01\ [1.3\%] $       \\ %MMS
%% MID      &   - & $\pm 0.01\ [1.3\%] $       \\ %MID
\hline
\end{tabularx}

\end{table}



\subsection{Eventos en las regiones de señal}

Solo dos eventos pasan la selección de cada una de las regiones de señal y resulta
instructivo una mirada detallada de los mismos. Un
esquema del detector y las se\~nales dejadas por los distintos objetos en los diferentes
subdetectores puede verse en la \cref{fig:evdisplay_srl} para los dos eventos
observados en la {\SRL} y en la \cref{fig:evdisplay_srh} para los dos eventos
en la {\SRH}.

Las visualizaciones de los eventos es generada con
\textsc{Atlantis}\cite{atlantis}. En la parte superior izquierda se puede ver el
corte transversal del detector junto con el corte longitudinal en el panel
inferior. En estas vistas se presentan los depósitos de energía en los
calorímetros en color amarillo. Los conos de distintos colores representan los
distintos jets reconstruidos en el evento, y su tama\~no es proporcional a la
energía del mismo. La línea gruesa de color amarillo representa un fotón
reconstruido y la línea roja punteada la energía faltante.



\begin{figure}[!htbp]
  \begin{center}

    \includegraphics[height=0.45\textheight]{EvtDisplay_212103_73661974}

    \vspace{1cm}

    \includegraphics[height=0.45\textheight]{EvtDisplay_212144_201737678}

    \caption{Visualización de los dos eventos que sobreviven la selección de {\SRL}. Ver detalles en el texto.}
    \label{fig:evdisplay_srl}
  \end{center}
\end{figure}


\begin{figure}[!htbp]
  \begin{center}

    \includegraphics[height=0.45\textheight]{EvtDisplay_203875_14229699}

    \vspace{1cm}

    \includegraphics[height=0.45\textheight]{EvtDisplay_212967_66359411}

  \caption{Visualización de los dos eventos que sobreviven la selección de {\SRH}. Ver detalles en el texto.}
  \label{fig:evdisplay_srh}
  \end{center}
\end{figure}


\clearpage


%-------------------------
% Model independent limit
%-------------------------
\section{Límites a procesos de nueva física} \label{sec:model_independent}

Una vez obtenidos los resultados finales en las regiones de señal en términos
del número de eventos observado en relación al fondo esperado como se discutió
en las secciones anteriores, resulta útil proveer el límite superior en el número de eventos de
nueva física impuesto por dichos resultados. Este límite, obtenido de manera independiente de cualquier modelo,
es provisto para cada región de señal, haciendo posible su aplicación a cualquier modelo de
nueva física. Para esto debe calcularse
el número de eventos de señal que se espera
en la región en cuestión y verificar que dicho número no exceda el límite
superior del presente análisis. Por supuesto que los valores de eficiencia y aceptancia deben ser
tenidos en cuenta.
Para que puedan realizarse este tipo de interpretaciones en base a otros modelos, todos
los valores y datos relevantes de este análisis son publicados en \textsc{HepData}
para su utilización \cite{hepdata}.

Con el objetivo de calcular los límites independientes del modelo se realiza un
ajuste combinado teniendo en cuenta las regiones de control y cada una de las
regiones de señal. Se supone que la contaminación de señal en las CR es nula,
pero no se realiza ninguna otra suposición acerca del modelo de nueva física. El
número de eventos de señal en la SR se agrega como el parámetro de interés y se
deja libre en el ajuste. Se realiza un muestreo de este parámetro, y el
límite superior en el número de eventos de nueva física se encuentra para el
valor en el cual el {\cls} cae debajo del 5\%. El límite superior en el número
de eventos puede transformarse en el límite superior de la sección eficaz
visible\footnote{La sección eficaz visible tiene en cuenta
  la aceptancia y eficiencia de selección del análisis, es decir, $\sigma_{\text{vis}} = \epsilon\sigma$}
utilizando la luminosidad considerada en el análisis
(20.3 \ifb).

Los resultados se detallan en la \cref{tab:upperlimits} para las dos regiones de
señal. Debido a que el número esperado de eventos de fondo es bajo y no es
posible utilizar la aproximación asintótica que fuera descripta en la \cref{sec:aprox},
se utilizaron $\sim 3000$
pseudo-experimentos generados con el método Monte Carlo. El límite en el número
de eventos de nueva física resulta en 5.5 y 5.6 para {\SRL} y {\SRH}
respectivamente, lo cual se traduce en un límite en la sección eficaz visible de
0.27 y 0.28 fb.


\begin{table}[!htbp]
  \centering

  \caption{Límite independiente del modelo de señal a 95\% de CL en la
    sección eficaz visible observada ($\langle\epsilon{\rm \sigma}\rangle_{\rm obs}$),
    y el límite en el número de eventos de nueva física observado
    $S_\text{obs}$ para las dos SR.
    La última línea ($p_0$) indica el {\pvalue} de la hipótesis de solo-fondo.}
  \label{tab:upperlimits}

  \begin{tabularx}{\textwidth}{LRR}
  \hline
  {\bf Region de señal}   &            \SRL &              \SRH \\
  \hline
  Eventos observados      &             $2$ &              $2$  \\
  Eventos esperados SM    & $1.27 \pm 0.43$ &  $0.84 \pm 0.38$  \\
  \hline
  $\avg{\epsilon{\sigma}}_\text{obs} \, [\mathrm{fb}]$  & 0.27  & 0.28 \\
  $S_\text{obs}$  & 5.5 & 5.6 \\
  %%$S_\text{exp}$ & ${4.0}^{+1.6}_{-0.1}$ & ${4.1}^{+1.7}_{-0.1}$ \\
  %% {\clb} & 0.86 & 0.83 \\
  $p_0$  & 0.16 &  0.19 \\
  \hline
\end{tabularx}


\end{table}

%% \hll{Poner el plot del scan para este límite}
%% \hll{Límite 1D para EWK?}

\clearpage


%-------------------------
% SUSY Exclusion limit
%-------------------------
\section{Límites de exclusión en el modelo de SUSY considerado}
\label{sec:susy_limits}

Debido a que no se observó un exceso significativo por encima del fondo esperado
del {\SM} en las regiones de señal, se obtuvieron los límites de exclusión en el
modelo de SUSY considerado. Los límites superiores son determinados con el
método {\cls} descripto en la \cref{sec:limits}. Para ello se utiliza la función
likelihood, \cref{eq:likelihood}, incluyendo cada región de señal y las muestras
de señal del modelo de SUSY que motivó este estudio.
Estos límites son
calculados para la sección eficaz nominal del modelo y también con una sección
eficaz de $\pm 1 \sigma$ en la incerteza teórica, siguiendo la recomendación de
ATLAS y el acuerdo de los experimentos del LHC para presentar los resultados.

La \cref{fig:limit_srs} muestra los límites de exclusión esperados y
observados para las dos SR de forma separada. Los límites fueron obtenidos
utilizando $\sim 20000$ pseudo-experimentos. La línea punteada azul muestra el
límite esperado a 95\% CL, con las bandas amarillas indicando la desviación a
1$\sigma$ teniendo en cuenta las incertezas teóricas y experimentales. Los
límites observados están indicados por las líneas en rojo oscuro, donde la línea
sólida representa el límite nominal y las líneas punteadas representan el límite
para las variaciones de la sección eficaz de la señal debido a la incerteza
teórica en la misma.

Por diseño, la {\SRL} es importante en la zona de neutralinos livianos, mientras que
la {\SRH} cubre la región más comprimida del espacio de fases. En la
\cref{fig:limit_combined}, se muestra el límite combinando ambas
regiones de señal, eligiendo en cada punto la que provee el mejor límite.


\begin{figure}[!htb]
  \centering

  \includegraphics[width=0.45\textwidth]{limitplot_srl}
  \includegraphics[width=0.45\textwidth]{limitplot_srh}

  \caption{Límites de exclusión a 95\% CL para {\SRL}  (izquierda) y {\SRH} (derecha).}
  \label{fig:limit_srs}
\end{figure}


\begin{figure}[!htb]
  \centering

  \includegraphics[width=0.9\textwidth]{limitplot_combined_rectangle}

  \caption{Límites de exclusión para {\SRL} y {\SRH} combinados. Los límites
    son obtenidos usando la región de señal con mejor sensibilidad en cada punto.
    La línea punteada azul muestra el límite esperado a 95\% CL, con las bandas
    amarillas indicando la desviación a $1\sigma$ que tiene en cuenta las incertezas
    teóricas y experimentales. Los
    límites observados están indicados por las líneas en rojo oscuro, donde la línea
    sólida representa el límite nominal y las líneas punteadas representan el límite
    para las variaciones de la sección eficaz de la señal debido a la incerteza
    teórica en la misma.}
   \label{fig:limit_combined}

\end{figure}


Como se desprende de los gráficos de los límites de exclusión, considerando el
contorno de $-1\sigma$ en la sección eficaz de SUSY, la produccion de gluino es
excluida a 95\% {\cl} para una masa de hasta 1250 \gev, para
$m_{\ninoone}<{840}\gev$. Este límite disminuye para neutralinos de baja masa
debido a la caída en la aceptancia en esa región, lo cual también ocurre cuando
la masa de gluinos y neutralinos es similar.

A modo de completitud, en la \cref{fig:susy_summary} se presenta un resumen de
los límites obtenidos en las masas de las partículas supersimétricas buscadas en
una selección de los análisis más sensibles realizados con datos de ATLAS. La
figura incluye el resultado obtenido por el análisis presentado en esta tesis,
al igual que los análisis complementarios de búsqueda de modelos GGM.
Puede observarse que los limites obtenidos, incluida la contribucion de esta tesis,
alcanzan y superan la region del {\tev} para la masa de gluino.

\begin{figure}[!htb]
  \centering

  \includegraphics[width=0.9\textwidth]{atlas_susy_run1_summary}

  \caption{Resumen de los límites de exclusión en la masa de las partículas
    supersimétricas por distintos análisis realizados en ATLAS. Solo una
    selección representativa de los resultados disponibles es
    presentada\cite{susy_summary}, incluyendo los obtenidos en esta tesis.}
  \label{fig:susy_summary}

\end{figure}



\section{Búsqueda de señal de SUSY con datos a $\sqrt{s} = 13 \tev$}
\label{sec:results13tev}

Luego de la publicación de los resultados obtenidos con los datos de
$\sqrt{s}=8\tev$, presentados en las secciones anteriores, se comenzó el
análisis de los nuevos datos a partir de las colisiones del llamado \emph{Run
  2} del LHC a una energía de centro de masa sin
precedente, de $\sqrt{s} = 13 \tev$. Desde Junio a Diciembre del a\~no 2015 el LHC funcionó a
esta energía de centro de masa y el detector ATLAS recolectó datos
correspondientes a una luminosidad total integrada de 3.2 \ifb.

Además del cambio en la energía de las colisiones, se llevaron a cabo varias
mejoras en el LHC y sus detectores. También se realizaron mejoras en el trigger, y en el
software de reconstrucción y análisis.
Durante los dos a\~nos que hubo entre las tomas de datos, el LHC fue mejorado
ampliamente para poder aumentar la energía de los haces de protones a 6.5
\tev, y disminuir el espaciado entre \emph{bunches} a $\unit[25]{ns}$.
Al detector ATLAS también se le realizaron mejoras significativas durante este período.
Se renovaron el tubo del haz, el sistema de imanes y los sistemas criogénicos. También se
reemplazaron o reacondicionaron los módulos del detector interno y los
calorímetros dañados, y para mejorar la reconstrucción de trazas y de vértices a
luminosidades mayores, se agregó una cuarta capa al detector de píxeles a una
distancia de 3.3 cm del tubo del haz.

El aumento en la energía de las colisiones conlleva a un incremento en la
sección eficaz de producción, mejorando el potencial de descubrimiento para
física más allá del SM. En particular para SUSY, la sección eficaz esperada de
producción de gluinos incrementa en un factor de 30 a 130 en el rango de masas
1.5-2 {\tev}. En la \cref{fig:13tev_xs} puede verse la comparación entre la
sección eficaz para las distintas energías de colisión.

\begin{figure}[!htb]
  \centering

  \includegraphics[width=0.8\textwidth]{xs_run1grid_gg_8TeV_13TeV}

  \caption{Sección eficaz de producción de pares de gluinos como función de la
    masa del gluino para energías de centro de masa de $8 \tev$ y
    $13 \tev$.}
  \label{fig:13tev_xs}

\end{figure}


Para el nuevo análisis se dise\~nó un conjunto de puntos de se\~nal (ver
\cref{fig:13tev_grid}), utilizando el mismo modelo considerado previamente y
explicado en la \cref{sec:sig_samples}, aunque extendiendo el espacio de
parámetros. Los mismos fueron generados utilizando
{\herwigpp}. En este caso, la masa de los squarks se fijo en 5 \tev, mientras
que todos los demás parámetros se mantuvieron iguales a la \emph{grid} anterior
utilizada en esta tesis.


\begin{figure}[!p]
  \centering

  \includegraphics[width=0.6\textwidth]{grid_updated}

  \caption{Conjunto de puntos (\emph{grid}) de señal simulados para el análisis
    de \emph{Run 2}. Los cuadrados azules representan los puntos de señal
    generados con distintos valores de $m_{\gluino}$ y $m_{\ninoone}$.}

  \label{fig:13tev_grid}

\end{figure}


Con las muestras generadas de señal y de los procesos del SM, y
a partir de la selección de las dos regiones de señal {\SRL} y {\SRH} del análisis de \emph{Run 1},
se optimizaron nuevamente los cortes, considerando una luminosidad total esperada de 4 \ifb.
El resultado de los cortes de selección de eventos es el que se detalla en la \cref{tab:13tev_selection}.
El cambio más relevante respecto a la selección previa es la introducción del corte en $M_\text{eff} = \HT + \met$.
En la \cref{tab:13tev_selection} puede verse además el fondo esperado para
estas dos regiones de señal, considerando una luminosidad total integrada de 4 \ifb.

\begin{table}[!h]
  \centering
  \caption{Izquierda: Selección final para las regiones de se\~nal {\SRL} y {\SRH} optimizadas
    para el análisis de los datos de $\sqrt{s} = 13 \tev$. Derecha: Fondo esperado para región
  de se\~nal, para una luminosidad total integrada de 4 \ifb.}
  \label{tab:13tev_selection}

  \begin{tabularx}{0.45\textwidth}{lcc}
    \hline
    & \SRL & \SRH \\
    \hline
    \nphotons & $=1$ & $=1$ \\
    $\pt^{\gamma}$ & $>145 \gev$ & $>400 \gev$ \\
    \nleptons & 0 & 0 \\
    \njets & $> 4$ & $> 2$ \\
    $\Delta\phi(\text{jet}, \met)$ & $>0.4$ & $>0.4$ \\
    $\Delta\phi(\gam, \met)$ & $>0.4$ & $>0.4$ \\
           {\met} & $>200 \gev$ & $> 400 \gev$ \\
       $M_{\text{eff}}$ & $>2000 \gev$ & $>2000\gev$ \\
       $R_T^4$ & 0.90 & - \\

       \hline
  \end{tabularx}
  \hfill
  \begin{tabularx}{0.45\textwidth}{lcc}
    \hline
    & \SRL & \SRH \\
    \hline
    $\gamma$ + jet     &     0.01 $\pm$ 0.00    &     0.04 $\pm$ 0.01    \\
    multijet           &     0.00 $\pm$ 0.00    &     0.01 $\pm$ 0.00    \\
    $W\gamma$          &     0.17 $\pm$ 0.08    &     0.15 $\pm$ 0.07    \\
    $Z(ll)\gamma$      &     0.00 $\pm$ 0.00    &     0.01 $\pm$ 0.01    \\
    $Z(\nu\nu)\gamma$  &     0.01 $\pm$ 0.01    &     0.09 $\pm$ 0.05    \\
    %% $t\bar{t}$        &     0.00 $\pm$ 0.00    &     0.00 $\pm$ 0.00    \\
    $t\bar{t}\gamma$   &     0.08 $\pm$ 0.08    &     0.15 $\pm$ 0.10    \\
    $W$+jets           &     0.00 $\pm$ 0.00    &     0.03 $\pm$ 0.02    \\
    $Z$+jets           &     0.00 $\pm$ 0.00    &     0.07 $\pm$ 0.05    \\
    \hline
    Fondo total        &     0.27 $\pm$ 0.17    &     0.55 $\pm$ 0.30    \\
    \hline
  \end{tabularx}

\end{table}

En la \cref{fig:13tev_discovery} se encuentra la significancia esperada de descubrimiento,
donde las líneas representan los contornos de $3\sigma$ y $5\sigma$.
El análisis de datos se está llevando a cabo al presente y se espera que los resultados
sean presentados en las conferencias de mediados de 2016 y publicados hacia fin de a\~no.

\begin{figure}[!p]
  \centering

  \includegraphics[width=0.6\textwidth]{discovery_combined_13tev}

  \caption{Significancia de descubrimiento esperada para el análisis de datos de 13 \tev,
    para una luminosidad total integrada de 4 \ifb. Los contornos se corresponden con $3\sigma$ y $5\sigma$.}
  \label{fig:13tev_discovery}

\end{figure}
