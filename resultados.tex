\chapter{Análisis Estadístico y Resultados}

En este capítulo se introducen algunos conceptos básicos necesarios
para el análisis estadístico de los datos, focalizandolnos en la
búsqueda de nuevas se\~nales en física de altas energías.


\section{Herramientas para el análisis estadístico}

Para el analisis estadistico se utilizo el software \texttt{HistFitter} \cite{histfitter}
desarrollado dentro del grupo de SUSY en ATLAS, que es una interfaz para herramientas
como \texttt{RooFit}, \texttt{RooStats}\cite{Moneta:2010pm} y \texttt{HistFactory}
\cite{Cranmer:1456844} y ademas posee una serie de scripts que facilitan el analisis cuando este
es muy complejo.

%% En este capítulo se presenta el análisis estadístico de los datos. La compatibilidad con el Modelo Estándar,
%% los limites en la sección eficaz y los limites de exclusión son obtenidos utilizando el profile log likelihood ratio {LLR}.

%% En esta seccion se describira el procedimiento general para la busqueda de nueva fisica en el
%% contexto de un test estadistico frecuentista.

%% The statistical treatment of the results is presented in this section.
%% The compatibility with the Standard Model, limits on the visible cross sections and the exclusions are
%% assessed with a profile log likelihood ratio (LLR) approach.

%% The profile LLR is obtained from a simultaneous fit to the contributions
%% from Standard Model background and supersymmetric signal
%% models in a given signal region and its associated background
%% control regions, which are all by design statistically independent.
%% Cross-contamination of backgrounds across control region
%% boundaries and the propagation of statistical and systematic uncertainties is cleanly taken into account.


%% \section{Configuracion del fit} \label{sec:fitconfig}

%% La configuracion del fit esta basada en las SR y CR descriptas en las tablas \XXX\ y
%% considera todos las incertezas sistematicas discutidas en seccion \XXX.

%% When fitting CRs simultaneously, common normalizations are allowed in order to correctly take into
%% account the other background contamination in a given CR. Experimental systematic uncertainties are
%% correlated across the CRs and a SR. Sample-dependent theory systematics are uncorrelated.

%% La función likelihood para cada canal el producto de distribuciones de Poisson, una para la region
%% de se\~nal y otra para cada region de control que se utilizan para constraint \wgamma\ (CRLW),
%% \ttbargam\ (CRLT) and fake-\MET\ (CRM) events, labeled as $P_\text{SR}$, $P_\text{CRLW}$,
%% $P_\text{CRLT}$, $P_\text{CRM}$, respectively, and of the PDFs constraining the systematic
%% uncertainties $C_\text{syst}$:

%% \begin{equation} \label{eq:likelihood}
%%     L({\mathbf n}|\mu,{\mathbf s},{\mathbf b},\alpha) = P_\text{SR} \times P_\text{CRLW} \times P_\text{CRLT} \times P_\text{CRM} \times C_\text{syst}
%% \end{equation}

%% Cada función de Poisson $P_{i}$ refleja el numero de eventos medidos en la region $i$,
%% ${\mathbf n_{i}}$, y el n\'umero de eventos esperados para las componentes de se\~nal
%% ${\mathbf s}$ y fondo ${\mathbf b}$. El factor de normalización $\mu$ para los fondos
%% \wgamma, \ttbargam, y \gjet son parametros libres del fit.
%% Los otros fondos considerados en el ajuste son \vgamma\ (incluyendo \znunugam) y
%% \topgamma\ MC, y los estimados mediante data-driven para electrones y jet fakes
%% descriptos en el capitulo \ref{}.
%% Los parametros nuisance $\apha$ parametrizan las incertezas sistematicas en la se\~nal
%% y el fondo por medio de una funcion gausiana. El tratamiento de las incertezas
%% sistematicas se describe en la seccion \ref{}.


%% The inputs to the fit are summarized as follows:
%% \begin{itemize}
%% \item The observed number of events in the CRLW, CRLT, CRM and the number expected from initial background predictions.
%% %• A transfer factor (TF) which is a multiplicative factor that propagates the event count from each CR to each SR for tt¯, Z +jets and multi-jets, respectively.
%% \item Fixed data-driven expectations for the number of events for electron and jet fakes events (weighted as discussed in \Sec \ref{sec:efakes} and \ref{sec:jetfakes}, respectively).
%% \item Fixed MC expectations for the number of events from the remaining backgrounds (\zgam).
%% \end{itemize}

%% By the simultaneous fit to the data in CRs or to the data in CRs and in a SR, these free and nuisance
%% parameters above can  be optimized with proper correlations and errors. In addition, by constraining the
%% parameters on background components mainly by the control regions with large statistics, the impacts of
%% these uncertainties in the SRs are reduced.

%% The results presented below in \Sec \ref{sec:results} are obtained in steps, for three kinds of fit configurations, as follows:

%% \begin{itemize}
%% \item {\bf Bkg-only fit:} the normalization for each background component and the fitted parameters are obtained simultaneously according to the data in their designed CR only
%% \item {\bf Discovery fit:} the SR is added to the fit together with a generic non-SM signal component, which is neglected in the CRs. The background prediction achieved is this
%% way conservative since any signal contribution in the CRs is attributed to background and thus results in a possible overestimation of the background in the SRs. As shown above, however, this contribution is expected to be negligible.
%% \item {\bf Exclusion fit} both CRs and SRs are used in the fit. The signal contribution is taken into account as predicted by the tested SUSY model in all the regions.
%% \end{itemize}





\section{Validación del Modelo} \label{sec:bkgonlyfit}

%% In each channel, the background fit performed to the CR data by maximising the likelihood in \Eq\ \ref{eq:likelihood} but neglecting signal component, in order to determine the normalisation of background components, the fitted nuisance parameters and their correlations.

%% A comparison of the background predictions and the observed data in the CRs after the background only fit is shown in \Fig \ref{fig:bkgfit_CR2_after} and \ref{fig:bkgfit_CR3_after} for some representative variables used in SR2 and SR3. A complete comparison of the distributions before and after the fit can be found in \App\ \ref{sec:ap_before_after}. The results of the fits are tabulated in \Tab \ref{tab:fit_result_cr2} and \ref{tab:fit_result_cr3}, for SR2 and SR3, respectively.
%% As seen in \Fig \ref{fig:sig_CR}, the signal contamination in the CRs is expected to be small. For CRM, the low \MET cut kills most of the signal events, keeping its fraction below 0.5\% across the whole grid and decreasing with the gluon mass. In CRLW and CRLT, the signal fraction is $<3\%$ for most of the grid, although it can be as big as 60(35)\% for the latter for a few signal points with light gluinos and heavy neutrinos for SR2(3) selections. In all cases, the signal contamination is anyways taken into account in the final exclusion fit.

\begin{figure}[ph!]
  \begin{center}
    \includegraphics[width=0.49\textwidth]{figures/can2_CRM_met_et_afterFit}
    \includegraphics[width=0.49\textwidth]{figures/can2_CRM_jet1_pt_afterFit} \\
    \includegraphics[width=0.49\textwidth]{figures/can2_CRM_dphi_jetmet_afterFit}
    \includegraphics[width=0.49\textwidth]{figures/can2_CRM_rt4_afterFit} \\
    \includegraphics[width=0.49\textwidth]{figures/can2_CRLW_met_et_afterFit}
    \includegraphics[width=0.49\textwidth]{figures/can2_CRLW_jet1_pt_afterFit} \\
    \includegraphics[width=0.49\textwidth]{figures/can2_CRLW_dphi_jetmet_afterFit}
    \includegraphics[width=0.49\textwidth]{figures/can2_CRLW_rt4_afterFit} \\
    %\caption{Observed distributions in SR2 control regions CRM (top) and CRLW (bottom) after fit.}
    \label{fig:bkgfit_CR2_after}
  \end{center}
\end{figure}

 \begin{figure}[ph!]
  \begin{center}
    \includegraphics[width=0.49\textwidth]{figures/can3_CRM_met_et_afterFit}
    \includegraphics[width=0.49\textwidth]{figures/can3_CRM_ht_afterFit} \\
    \includegraphics[width=0.49\textwidth]{figures/can3_CRM_dphi_jetmet_afterFit}
    \includegraphics[width=0.49\textwidth]{figures/can3_CRM_rt2_afterFit} \\
    \includegraphics[width=0.49\textwidth]{figures/can3_CRLW_met_et_afterFit}
    \includegraphics[width=0.49\textwidth]{figures/can3_CRLW_ht_afterFit} \\
    \includegraphics[width=0.49\textwidth]{figures/can3_CRLW_dphi_jetmet_afterFit}
    \includegraphics[width=0.49\textwidth]{figures/can3_CRLW_rt2_afterFit} \\
    %\caption{Observed distributions in SR3 control regions CRM (top) and CRLW (bottom) after fit.}
    \label{fig:bkgfit_CR3_after}
  \end{center}
\end{figure}

%% \begin{figure}[ht!]
%%   \begin{center}
%%      \includegraphics[width=0.49\textwidth]{figures/signal_contamination_CRM_2}
%%      \includegraphics[width=0.49\textwidth]{figures/signal_contamination_CRM_3} \\
%%      \includegraphics[width=0.49\textwidth]{figures/signal_contamination_CRLW_2}
%%      \includegraphics[width=0.49\textwidth]{figures/signal_contamination_CRLW_3} \\
%%      \includegraphics[width=0.49\textwidth]{figures/signal_contamination_CRLT_2}
%%      \includegraphics[width=0.49\textwidth]{figures/signal_contamination_CRLT_3} \\
%%      \caption{Signal contamination expected in the control regions associated to SR2 (left) and SR3 (right).}
%%      \label{fig:sig_CR}
%%   \end{center}
%% \end{figure}

%% The fit results and data event counts in the several validation regions for SR2 and SR3 are presented in \Tab \ref{tab:fit_result_vr2} and \ref{tab:fit_result_vr3}, respectively. The distributions of several selection variables in these regions are shown in \App \ref{sec:ap_before_after}, before and after the fit. The fitted background predictions agree well with the data event counts within the uncertainties. The normalization factors determined for \wgamma\ ($\mu_{W}$), \ttbargam\ ($\mu_{T}$) and QCD prompt photon production ($\mu_{Q}$) are shown in \Tab \ref{tab:fit_bkgonly_mus}. %The $\mu_{W}$ value is consistent
%% %% %with unity within uncertinties. %As expected, $\mu_{Q}$ is way smaller, since the statistics for the smeared pseudo-data is artificially enhanced.
%% The systematic uncertainties in the two signal regions  after the background fit are summarized in \Fig \ref{fig:fit_unc_nuisance_SR}. No profiling of the systematics uncertainties is observed. %hopefully
%% The correlations of these systematics uncertainties and the normalizations are shown in \Fig \ref{fig:fit_corr_SR}. % What do we observe here?

%% \begin{table}[h!]
%%   \centering
%%   \caption{Normalization factors for \wgamma\ ($\mu_{W}$), \ttbargam\ ($\mu_{T}$) and \gjet\ ($\mu_{Q}$) backgrounds, as obtained from the background-only fit for region SR2 and SR3. The uncertainties shown are those from the fit only.}
%%   \vspace{1pt}
%%   \begin{tabular}{|c|ccc|}
%%     \hline
%%     \multicolumn{4}{|c|} {\bf Normalization factors} \\
%%     \hline
%%     \hline
%%         {\bf SR} & \gjet\ ($\mu_{Q}$) & \wgamma\ ($\mu_{W}$) & \ttbargam\ ($\mu_{T}$) \\
%%         \hline
%%         {\bf SR2} & $0.94 \pm 0.44$ & $1.34 \pm 0.89$ & $1.40 \pm 0.77$ \\
%%         {\bf SR3} & $1.22 \pm 0.58$ & $1.24 \pm 0.39$ & $0.54 \pm 0.37$ \\
%%         \hline
%%   \end{tabular}
%%   \label{tab:fit_bkgonly_mus}
%% \end{table}


%% %% ------
%% %%  SR2
%% %% ------
\begin{table}[ht!]
  %\caption{Background fit results for SR2 control regions with an integrated luminosity of 20.3 \ifb. Nominal MC and data-driven expectations are given for comparison. The uncertainties shown are statistical + systematic.}
  \input{tables/table_SR2_CR.tex}
  \label{tab:fit_result_cr2}
\end{table}

\begin{table}[ht!]
  %\caption{Background fit results for SR2 validation regions with an integrated luminosity of 20.3 \ifb. Nominal MC and data-driven expectations are given for comparison. The uncertainties shown are statistical + systematic.}
  \input{tables/table_SR2_VR.tex}
  \label{tab:fit_result_vr2}
\end{table}


%% ------
%%  SR3
%% ------
\begin{table}[ht!]
  %\caption{Background fit results for SR3 control regions with an integrated luminosity of 20.3 \ifb. Nominal MC and data-driven expectations are given for comparison. The uncertainties shown are statistical + systematic.}
  \input{tables/table_SR3_CR.tex}
\label{tab:fit_result_cr3}
\end{table}

\begin{table}[ht!]
  %\caption{Background fit results for SR3 validation regions with an integrated luminosity of 20.3 \ifb. Nominal MC and data-driven expectations are given for comparison. The uncertainties shown are statistical + systematic.}
  \input{tables/table_SR3_VR.tex}
  \label{tab:fit_result_vr3}
\end{table}

%% \begin{figure}[h!]
%%   \begin{center}
%%      \includegraphics[width=\textwidth]{figures/syst_pull_afterFit_SR2.eps} \\
%%      \includegraphics[width=\textwidth]{figures/syst_pull_afterFit_SR3.eps}
%%      \caption{Summary of the the systematic uncertainties after the background fit for SR2 (top) and SR3 (bottom).}
%%      \label{fig:fit_unc_nuisance_SR}
%%   \end{center}
%% \end{figure}

%% \begin{figure}[h!]
%%   \begin{center}
%%      \includegraphics[width=\textwidth]{figures/corrMatrix_SR2} \\
%%      \includegraphics[width=\textwidth]{figures/corrMatrix_SR3} \\
%%      \caption{Correlation matrix of the background fit for SR2 (top) and SR3 (bottom).}
%%      \label{fig:fit_corr_SR}
%%   \end{center}
%% \end{figure}


%% The background predictions from the fit in SR2 and SR3 are presented in \Tab \ref{tab:fit_result_sr}. Both the nominal estimates
%% before the fit and the predicted yields from the background fits are shown.
%% No significant data deviation from the predicted background yields is observed in any of the SRs. The results after fit are
%% also shown for all regions in \Fig \ref{fig:fit_region_composition}.

%% The breakdown of systematic
%% uncertainties on the total background prediction are presented in \Tab \ref{tab:fit_result_sr2_syst} and \ref{tab:fit_result_sr3_syst}.
%% Since the individual uncertainties can be correlated and the negative correlations between normalizations (representing the
%% impact of the CR statistics on the total uncertainty) and systematic uncertainties are observed as expected, they
%% do not necessarily add up by quadrature to the total background uncertainty. The square
%% root of the number of the expected events are calculated in order to illustrate the relative impact of the
%% SR Poisson statistical uncertainties with respect to the total background uncertainties. The
%%  relative systematic uncertainties with respect to the background predictions are large, around 54\% (58\%) for SR2 (SR3). In SR2, the main uncertainties after the fit come from the normalisation and the theoretical uncertainties of the \ttbargam\ background, followed by the jet energy scale and the MC statistics in the SR. In SR3, the uncertainties are dominated by the theoretical uncertainties on \zgam and \gjet backgrounds, the MC statistics and the normalisation uncertainties of the \wgamma\ production.

%% Similar breakdown tables but separated on each background prediction in each SR are presented in \Tab \ref{tab:fit_result_sr2_syst_perbkg}-\ref{tab:fit_result_sr3_syst_perbkg}.


\begin{figure}[h!]
  \begin{center}
    \includegraphics[width=0.6\textwidth]{figures/region_composition_SR2} \\
    \includegraphics[width=0.6\textwidth]{figures/region_composition_SR3}
    %%\caption{Background fit results in all SR/CR/VR regions, for SR2 (top) and SR3 (bottom) analyses.}
    \label{fig:fit_region_composition}
  \end{center}
\end{figure}



%% %The additional checks on the fit results are performed, following the recommendation on SUSYChekList twiki \cite{}.
%% % The results are shown in \App \ref{sec:ap:fitchecks} and confirmed all expected.

\begin{table}[ht!]
  \centering
  %% \caption{Background predictions in SR2 and SR3, obtained with the background fit for an integrated luminosity of 20.3 \ifb.
  %%   Nominal MC and data-driven expectations are given for comparison. The uncertainties shown are statistical + systematic.}
  \input{tables/table_SR.tex}
  \label{tab:fit_result_sr}
\end{table}


%% %% \begin{table}
%% %%   \caption{Breakdown of the dominant systematic uncertainties on total background estimates in SR2.
%% %%     Note that the individual uncertainties can be correlated, and do not necessarily add up quadratically to the total background uncertainty.
%% %%     The percentages show the size of the uncertainty relative to the total expected background.}
%% %%   \input{tables/table_SR2_syst.tex}
%% %%   \label{tab:fit_result_sr2_syst}
%% %% \end{table}

%% %% \begin{table}
%% %%   \caption{Breakdown of the dominant systematic uncertainties on total background estimates in SR3.
%% %%     Note that the individual uncertainties can be correlated, and do not necessarily add up quadratically to the total background uncertainty.
%% %%     The percentages show the size of the uncertainty relative to the total expected background.}
%% %%   \input{tables/table_SR3_syst.tex}
%% %%   \label{tab:fit_result_sr3_syst}
%% %% \end{table}

%% %% \begin{table}
%% %%   \caption{Breakdown of the dominant systematic uncertainties on each of the main backgrounds estimates in SR2.
%% %%     Note that the individual uncertainties can be correlated, and do not necessarily add up quadratically to the total background uncertainty.
%% %%     The percentages show the size of the uncertainty relative to the total expected background.}
%% %%   \input{tables/table_SR2_syst_bkgs.tex}
%% %%   \label{tab:fit_result_sr2_syst_perbkg}
%% %% \end{table}

%% %% \begin{table}
%% %%   \caption{Breakdown of the dominant systematic uncertainties on each of the main backgrounds estimates in SR3.
%% %%     Note that the individual uncertainties can be correlated, and do not necessarily add up quadratically to the total background uncertainty.
%% %%     The percentages show the size of the uncertainty relative to the total expected background.}
%% %%   \input{tables/table_SR3_syst_bkgs.tex}
%% %%   \label{tab:fit_result_sr3_syst_perbkg}
%% %% \end{table}

%% \clearpage

\section{Estudios sobre la extrapolación}
%% \label{sec:vr_extrapolation}

En la definicion de las regiones de control, algunos criterios
de seleccion fueron relajados para ganar estadistica. Por este
motivo se realizaron algunos estudios para validar la extrapolacion,
ajustando los cortes en las CR de a uno:

\begin{itemize}
\item \textbf{VCRLWrt:}  igual que CRLW pero con el corte en {\rt} como en la SR.
\item \textbf{VCRLWmet:} igual que CRLW pero con el corte en {\met} como en la SR.
\item \textbf{VCRLWht:}  igual que CRLW pero con el corte en {\HT} como en la SR.
\end{itemize}

%% Similar regions were defined for CRLT control region. The results are tabulated in \Tab\ \ref{tab:fit_result_vr_ext2} and \ref{tab:fit_result_vr_ext3}. As good agreement was found in all cases, no extra systematic was considered in this analysis.

%% %% \begin{table}[ht!]
%% %%   \caption{Background fit results for SR2 validation regions with an integrated luminosity of 20.3 \ifb. Nominal MC and data-driven expectations are given for comparison. The uncertainties shown are statistical + systematic.}
%% %%   \input{tables/table_SR2_VR_ext.tex}
%% %%   \label{tab:fit_result_vr_ext2}
%% %% \end{table}

%% %% \begin{table}[ht!]
%% %%   \caption{Background fit results for SR3 validation regions with an integrated luminosity of 20.3 \ifb. Nominal MC and data-driven expectations are given for comparison. The uncertainties shown are statistical + systematic.}
%% %%   \input{tables/table_SR3_VR_ext.tex}
%% %%   \label{tab:fit_result_vr_ext3}
%% %% \end{table}

%% \subsubsection{Check on jet energy scale uncertainty} \label{sec:JEScheck}

%% For the jet energy scale (JES) uncertainty, only one nuisance parameter is used. However, an alternative background-only fit was performed using the reduced list of JES components to check potential correlation effects on the final results.
%% The list of components provided by the JetEtmiss group is:

%% {\footnotesize
%% \begin{itemize}
%% \item[.] \texttt{EffectiveNP\_1, EffectiveNP\_2, EffectiveNP\_3, EffectiveNP\_4, EffectiveNP\_5, EffectiveNP\_6,}
%% \item[.] \texttt{EtaIntercalibration\_Modelling},
%% \item[.] \texttt{EtaIntercalibration\_StatAndMethod},
%% \item[.] \texttt{SingleParticle\_HighPt},
%% \item[.] \texttt{RelativeNonClosure\_Pythia8},
%% \item[.] \texttt{PileupOffsetTermMu},
%% \item[.] \texttt{PileupOffsetTermNPV},
%% \item[.] \texttt{PileupPtTerm},
%% \item[.] \texttt{PileupRhoTopology},
%% \item[.] \texttt{CloseBy},
%% \item[.] \texttt{FlavorCompUncert},
%% \item[.] \texttt{FlavorResponseUncert},
%% \item[.] \texttt{BJes}.
%% \end{itemize}
%% }

%% The breakdown of the dominant systematics uncertainties on the total backgrounds in this new configuration can be
%% seen in \Tab\ \ref{tab:fulljes_check_sr2} and \ref{tab:fulljes_check_sr3}, for SR2 and SR3 respectively. The difference in the final
%% systematic uncertainty, with respect to \Tab\ \ref{tab:fit_result_sr2_syst} and \ref{tab:fit_result_sr2_syst}, is in both cases within $1\%$.
%% A single nuisance parameter was therefore used for the nominal analysis.


%% %% \begin{table}[ht!]
%% %%   \caption{Breakdown of the dominant systematic uncertainties on total background estimates in SR2, using the full set of \texttt{JES} parameters.
%% %%     Note that the individual uncertainties can be correlated, and do not necessarily add up quadratically to the total background uncertainty.
%% %%     The percentages show the size of the uncertainty relative to the total expected background.}
%% %%   \input{tables/table_SR2_syst_sr_fulljes.tex}
%% %%   \label{tab:fulljes_check_sr2}
%% %% \end{table}

%% %% \begin{table}[ht!]
%% %%   \caption{Breakdown of the dominant systematic uncertainties on total background estimates in SR3, using the full set of \texttt{JES} parameters.
%% %%     Note that the individual uncertainties can be correlated, and do not necessarily add up quadratically to the total background uncertainty.
%% %%     The percentages show the size of the uncertainty relative to the total expected background.}
%% %%   \input{tables/table_SR3_syst_sr_fulljes.tex}
%% %%   \label{tab:fulljes_check_sr3}
%% %% \end{table}




%% \clearpage

\section{Distribuciones en las regiones de se\~nal}

%% The unblinded distributions of some of the discriminating variables in SR2 and SR3 are shown in \Fig \ref{fig:unblind_dist_2} and \ref{fig:unblind_dist_3},
%% respectively. The statistical prediction of the background is described in the next sections.
%% Details and event displays for the 4 events surviving the SR selections are included in \App \ref{sec:ap:atlantis}.

\begin{figure}[ht!]
  \begin{center}
   \includegraphics[width=0.49\textwidth]{figures/can2_SR_ph_pt_afterFit}
   \includegraphics[width=0.49\textwidth]{figures/can2_SR_jet1_pt_afterFit}\\
   \includegraphics[width=0.49\textwidth]{figures/can2_SR_ht_afterFit}
   \includegraphics[width=0.49\textwidth]{figures/can2_SR_met_et_afterFit}
   \caption{Comparación datos/MC de algunas variables discriminatorias
     para {\SRL}, despues del ajuste final.}
    \label{fig:unblind_dist_2}
  \end{center}
\end{figure}

\begin{figure}[h!]
  \begin{center}
   \includegraphics[width=0.49\textwidth]{figures/can3_SR_ph_pt_afterFit}
   \includegraphics[width=0.49\textwidth]{figures/can3_SR_jet1_pt_afterFit}\\
   \includegraphics[width=0.49\textwidth]{figures/can3_SR_ht_afterFit}
   \includegraphics[width=0.49\textwidth]{figures/can3_SR_met_et_afterFit}
   \caption{Comparación datos/MC de algunas variables discriminatorias
     para {\SRH}, despues del ajuste final.}
    \label{fig:unblind_dist_3}
  \end{center}
\end{figure}

\clearpage

\section{Interpretaciones} \label{sec:interpretations}

Debido a que no se observa un exceso significativo por
arriba del fondo esperado del {\SM} en las regiones de
señal, se obtendrán los limites de exclusión en el modelo
de SUSY GGM considerado. Los limites superiores son obtenidos
utilizando el método del {\cls} descripto en []. Estos limites
son recalculados con una sección eficaz de $\pm 1 \sigma$ en la
incerteza teórica siguiendo la recomendación de ATLAS y el acuerdo
para presencian de resultados en el LHC.

También se obtienen limites independientes del modelo en la siguiente
sección.

\subsection{Limites generales} \label{sec:model_independent}

%% The discovery fit is performed to the CR data and SR data by maximizing the likelihood in \Eq \ref{eq:likelihood} but
%%  with the signal component only in the SR. The discovery test is done by the background-only hypothesis
%% and quantified using pseudo-experiments by p-value $p_b = P(q \leq q_{obs}|b)$ where $q$ is the test statistics.
%% In the absence of a statistically significant excess, limits are set on contributions to the SRs from new
%% physics. Calculated p-values and model independent limits on the number of new physics contribution
%% by each SR are listed in \Tab \ref{tab:upperlimits}. Two sets of results are obtained with 3000 toy experiments and with
%% asymptotic approximation, respectively. Although the observed limits on the visible cross sections are not much different,
%% the asymptotic approximation is supposed to hold only for high statistics scenarios. Thus, we decided to keep all the
%% limit results with toys. The results with asymptotic approximation are also shown in \App \ref{sec:ap:asym_limits} for comparison.

%% %Since the limit numbers between the two sets are not much different, we expect that the model-dependent limits will not change.

\begin{table}[h!]
  \caption{Limite independiente del modelo de señal.}
    %% Signal-model-independent 95\% CL upper limits on the observed ($\langle\epsilon{\rm \sigma}\rangle_{\rm obs}^{95}$) %and expected ($\langle\epsilon{\rm \sigma}\rangle_{\rm exp}^{95}$)
    %% visible cross-section, and on the observed ($S_{\rm obs}^{95}$) and expected ($S_{\rm exp}^{95}$) number of beyond-SM
    %% events in the various SRs. The fourth line indicates the CLB value, i.e. the observed confidence level for
    %% the background-only hypothesis. The last line (p0) indicates the probability of the observation being
    %% consistent with the estimated background. Two sets of results are obtained with 3000 toy experiments
    %% (top-half) and with asymptotic approximation (lower-half).}
  \begin{center}
    \setlength{\tabcolsep}{0.0pc}
    \begin{tabular*}{\textwidth}{@{\extracolsep{\fill}}lccccc}
      \noalign{\smallskip}\hline\noalign{\smallskip}
              {\bf Region de señal}                   & {\SRL} & {\SRH} \\
              \noalign{\smallskip}\hline\noalign{\smallskip}
              Fondo esperado   &   $0.78\pm 0.42$   &  $0.84 \pm 0.49$  \\
              Eventos observados   &   2  &  2  \\
              \noalign{\smallskip}\hline\noalign{\smallskip}
              $\avg{\epsilon{\rm \sigma}}_\text{obs}^{95}$[fb]  & 0.28  & 0.28 \\
              $S_\text{obs}^{95}$  & 5.7 & 5.7 \\
              $S_\text{exp}^{95}$ & ${4.0}^{+1.6}_{-0.1}$ & ${4.1}^{+1.7}_{-0.1}$ \\
              $CL_{B}$ & 0.86 & 0.83 \\
              $p(s=0)$  & 0.16 &  0.19 \\
              \noalign{\smallskip}\hline\noalign{\smallskip}
              %
              $\avg{\epsilon{\rm \sigma}}_\text{obs}^{95}$[fb]  & 0.27  & 0.28 \\
              $S_\text{obs}^{95}$  & 5.5 & 5.6 \\
              $S_\text{exp}^{95}$ &  ${3.9}^{+2.1}_{-1.3}$ & ${4.1}^{+2.2}_{-1.1}$ \\
              $CL_{B}$ &  $0.76$  & $0.73$ \\
              $p(s=0)$  &  $0.18$ & $0.21$ \\
              \noalign{\smallskip}\hline\noalign{\smallskip}
    \end{tabular*}
  \end{center}
  \label{tab:upperlimits}
\end{table}


\subsection{Limites de exclusión sobre el modelo de SUSY}

La \cref{fig:limit_SR_toys} muestra los limites de exclusión
esperados y observados para las dos SR de forma separada. Los limites
fueron obtenidos utilizando $\sim 20000$ pseudoexperimentos. La linea
punteada azul muestra el limite esperado a 95\% CL, con las bandas
amarillas indicando la desviación a 1$\sigma$ que tiene en cuenta
las incertezas teóricas y experimentales.
Los limites observados están indicados por las lineas en rojo oscuro,
donde la linea solida representa el limite nominal y las lineas
punteadas representan el limite para las variaciones de la sección
eficaz de la señal debido a la incerteza teórica en la misma.


Por diseño, la {\SRL} es importante en la zona de gluinos pesado,
mientras que la {\SRH} cubre la región mas comprimida del espacio
de fases. En la \cref{fig:limit_SR_combined_toys}, se muestra
el limite combinando ambas regiones de señal, eligiendo en cada punto
la que provee el mejor limite.


\begin{figure}[h!]
  \begin{center}
     \includegraphics[width=0.8\textwidth]{figures/limitPlot_SR2_toys} \\
     \vspace{1cm}
     \includegraphics[width=0.8\textwidth]{figures/limitPlot_SR3_toys}
     \caption{Limites de exclusión a 95\% CL para {\SRL}  (arriba) y {\SRH} (abajo).}
     \label{fig:limit_SR_toys}
  \end{center}
\end{figure}


\begin{figure}[h!]
  \begin{center}
     \includegraphics[width=0.8\textwidth]{figures/limitPlot_SR2_SR3_toys} \\
     \caption{Limites de exclusión para {\SRL} y {\SRH} combinados.
       Los limites son obtenidos usando la region de senal con mejor sensibilidad
       en cada punto.}
     \label{fig:limit_SR_combined_toys}
  \end{center}
\end{figure}

\begin{figure}[h!]
  \begin{center}
     \includegraphics[width=0.8\textwidth]{figures/limitPlot_SR2_SR3_BestSR_toys}
     \caption{Limites de exclusión para {\SRL} y {\SRH} combinados.
       Los limites son obtenidos usando la region de senal con mejor sensibilidad
       en cada punto, la cual es indicada con el numero en gris.}
     \label{fig:limit_SR_combined_best}
  \end{center}
\end{figure}

%\clearpage
