\chapter{Análisis Estadístico y Resultados}
\label{cap:resultados}

%% A partir de los los conceptos estadísticos explicados en \cref{cap:estadistica}
%% y el modelo descripto en \cref{cap:estrategia}, en este Capítulo se presentan
%% el análisis estadístico de los datos y los resultados obtenidos.

A partir de la función likelihood general (\cref{eq:model}), se define la función
likelihood para el análisis que se presenta en esta Tesis como:

\begin{equation}
  L(\bm{n}|\mu_s, \bm{\mu}_p, \bm{s}, \bm{b}, \bm{\alpha}) = \mathcal{P}_\text{SR} \, \times \, \mathcal{P}_\text{CRQ} \, \times \, \mathcal{P}_\text{CRW} \, \times \, \mathcal{P}_\text{CRT} \, \times \, \mathcal{C}_\text{syst} \nonumber \\
   %% &= \Pois(n|\lambda(\mu_s, \bm{\mu}_p, \btheta)) %%\, \times \, \prod_{i \in \text{CR}}
  %% \Pois(n_i|\lambda_i(\mu_s, \bm{\mu}_p, \btheta)) \times C_\text{syst} (\btheta^0, \btheta) \label{eq:model}
  \label{eq:likelihood}
\end{equation}
%
donde $\bm{n}$ es el número de eventos observado en cada región, $\mu_s$ es la
intensidad de la señal y constituye el parámetro de interés del análisis.
Los demás parámetros son parámetros nuisance. Por un lado los parámetros de
normalización de los fondos ($\bm{\mu}_p$), en este caso $\mu_Q$ para el fondo
de {\gjet}, $\mu_W$ para {\wgam} y $\mu_T$ para {\ttgam}. El número de eventos
esperado de señal y fondo en cada región está dado por $\bm{s}$ y $\bm{b}$ respectivamente, y
provienen de las muestras MC o de aquellos calculados a partir de los datos como se
describe en el \cref{cap:fondos}. Los parámetros $\alpha$ parametrizan las
incertezas sistemáticas en la señal y el fondo utilizando una distribución Gausiana.

Con la función likelihood, se construye el estadístico de prueba como el \emph{profile likelihood ratio},

\begin{equation}
  q_{\mu_s} = -2 \ln \left( \frac{L(\mu_s, \doublehat{\btheta})}{L(\hat{\mu_s},\hat{\btheta})} \right)
\end{equation}
%
que será utilizado en este análisis para poner a prueba la compatibilidad de
los datos observados con las predicciones del SM.

Realizando un ajuste simultáneo a los datos en las CR (o en las CR y SR), los
parámetros descriptos anteriormente pueden ser estimados teniendo en cuenta
apropiadamente sus correlaciones e incertezas. Además, como los parámetros
de normalización de los fondos ($\bm{\mu}_p$) dependen esencialmente de las
regiones de control con una alta estadística de datos (respecto a la SR),
se reduce el impacto de estas incertezas estadísticas en las SR.


El análisis estadístico se implementó utilizando \textsc{HistFitter}\cite{HistFitter}, una herramienta
desarrollada por el grupo de SUSY de ATLAS, que es básicamente una interfaz de
\textsc{RooFit}, \textsc{RooStats}\cite{Moneta:2010pm} y
\textsc{HistFactory}\cite{Cranmer:1456844}, de forma de facilitar el análisis
estadístico. Además el hecho de utilizar una misma herramienta en distintos
análisis dentro de la colaboración, permite que su combinación estadística sea mas
sencilla.


%--------------
% Sistematicos
%--------------
\chapter{Determinación de las incertezas sistemáticas}



%--------------
% Bkg-only fit
%--------------
\section{Ajuste simultáneo en las regiones de control}
\label{sec:bkgonlyfit}

Para validar  los métodos utilizados para estimar los
fondos, se realiza un ajuste simultáneo utilizando únicamente las CR.
%% A este ajuste se lo denomina de solo-fondo.
El objetivo de este ajuste es estimar el fondo total esperado en las VR y las
SR, sin hacer ninguna suposición del modelo de señal. Solo las muestras de fondo
son utilizadas en el modelo, y las CR se suponen libres de contaminación de
señal. Cabe destacarse que el ajuste se realiza solamente en las CR, y los procesos de fondo dominantes
son normalizados al número de eventos observado en cada una de las regiones. La
función likelihood utilizada es la de la \cref{eq:likelihood} sin el término de la
SR. Como los parámetros de la pdf correspondientes al fondo son compartidos
entre las diferentes regiones, el resultado del ajuste puede ser utilizado luego, para
predecir el número de eventos en las VR y SR.

Las predicciones a partir del ajuste en las CR son independientes del número de
eventos observado en cada SR y VR, lo que permite una comparación no sesgada
entre el número de eventos predicho y observado en cada región. Los resultados
de este ajuste extrapolados a las SR, también son importantes para que grupos
externos al experimento puedan hacer una prueba de hipótesis en un modelo de
nueva física, no considerado por ninguno de los experimentos del LHC.
%% Debido a la complejidad de
%% estos ajustes resulta difícil reconstruirlos para las persones ajenas a los
%% experimentos.

En las siguientes secciones se muestran los resultados obtenidos
en las distintas regiones del análisis.



% Control Regions
\subsection{Resultados en las regiones de control}

En la \cref{tab:bkgonly_mus} se presentan los parámetros de normalización de
los fondos {\wgam} ($\mu_W$), {\ttgam} ($\mu_T$) y {\gjet}
($\mu_Q$), obtenidos del ajuste realizado en las regiones de control.
En general, los factores resultan consistentes con la unidad dentro de sus incertezas.

El número de eventos predicho para cada proceso de fondo antes y después del ajuste en
las regiones de control puede verse en \cref{tab:fit_result_crl,tab:fit_result_crh}, al
igual que el número de eventos observado.


\begin{table}[!htbp]
  \centering

  \caption{Factores de normalización para los fondos
    {\wgam} ($\mu_{W}$), {\ttgam} ($\mu_{T}$) y {\gjet} ($\mu_{Q}$), obtenidos
    del ajuste combinado en las CR. Las incertezas mostradas son solo estadísticas.}
  \label{tab:bkgonly_mus}

  \begin{tabular}{lccc}
    \hline
    &       $\mu_{Q}$ &       $\mu_{W}$ &       $\mu_{T}$ \\
    \hline
    \SRL & $0.94 \pm 0.44$ & $1.34 \pm 0.89$ & $1.40 \pm 0.77$ \\
    \SRH & $1.22 \pm 0.58$ & $1.24 \pm 0.39$ & $0.54 \pm 0.37$ \\
    \hline
  \end{tabular}

\end{table}


\begin{table}[!htbp]
  \centering

  \caption{Resultados del ajuste en las CR correspondientes a {\SRL}, con una luminosidad integrada total de 20.3 \ifb.
    El número de eventos observado es comparado con el número de eventos esperado de fondo, después de la correspondiente
    normalización en las CR. En la parte inferior de la tabla se muestran también lo valores nominales del fondo antes de
    la correspondiente normalización. Las incertezas incluyen la incerteza estadística y sistemática.}
  \label{tab:fit_result_crl}

  {\small\begin{tabularx}{\textwidth}{|z{3mm}|z{5cm}RRR}
\hline
\multicolumn{2}{l}{\hspace{6mm} \bf Regiones de control}           & {\CRQL}            & {\CRWL}            & {\CRTL}              \\
\hline
\multicolumn{2}{l}{\hspace{6mm} Eventos observados}                             & $1348$              & $8$              & $18$                    \\
\hline
\multirow{11}{*}{\rotatebox[origin=c]{90}{Después}} & Eventos esperados SM                           & $1347.21 \pm 36.51$          & $8.00 \pm 2.75$          & $18.00 \pm 4.16$              \\
\cline{2-5}
 & {\wgam}                                 & $0.28 \pm 0.27$          & $5.24 \pm 3.07$          & $3.15 \pm 1.94$              \\
 & {\ttgam}                                & $30.68 \pm 1.31$          & $1.33 \pm 0.73$          & $10.66 \pm 4.92$              \\
 %%& {\ttgam} (had)                          & $28.46 \pm 16.90$          & $0.00 \pm 0.00$          & $0.00 \pm 0.00$              \\
 & {\vqqgam}                    & $39.30 \pm 24.90$          & $0.00 \pm 0.00$          & $0.00 \pm 0.00$              \\
 & {\tgam}                   & $0.24 \pm 0.04$          & $0.22 \pm 0.06$          & $1.24 \pm 0.24$              \\
 & {\zllgam}     & $0.08_{-0.08}^{+0.10}$          & $0.14_{-0.14}^{+0.14}$          & $0.11_{-0.11}^{+0.11}$              \\
 & {\znngam}       & $0.00 \pm 0.00$          & $0.00 \pm 0.00$          & $0.00 \pm 0.00$              \\
 & {\gjet}                          & $1155.79 \pm 68.55$          & $0.13 \pm 0.04$          & $0.01 \pm 0.01$              \\
 & $e\rightarrow\gamma$              & $5.34 \pm 0.89$          & $0.26 \pm 0.05$          & $1.29 \pm 0.21$              \\
 & $j\rightarrow\gamma$              & $115.48 \pm 54.49$          & $0.69 \pm 0.32$          & $1.54 \pm 0.72$              \\
\hline
\multirow{11}{*}{\rotatebox[origin=c]{90}{Antes}} & Eventos esperados SM                            & $1397.30$          & $6.29$          & $14.14$              \\
\cline{2-5}
& {\wgam}                                & $0.21$          & $3.90$          & $2.34$              \\
& {\ttgam}                               & $21.90$          & $0.95$          & $7.61$              \\
& {\vqqgam}                   & $29.25$          & $0.00$          & $0.00$              \\
& {\tgam}                  & $0.24$          & $0.22$          & $1.24$              \\
& {\zllgam}    & $0.08$          & $0.14$          & $0.11$              \\
& {\znngam}      & $0.00$          & $0.00$          & $0.00$              \\
& {\gjet}                                & $1224.79$          & $0.14$          & $0.01$              \\
& $e\rightarrow\gamma$             & $5.34$          & $0.26$          & $1.29$              \\
& $j\rightarrow\gamma$             & $115.48$          & $0.69$          & $1.54$              \\
\hline
\end{tabularx}
}

\end{table}


\begin{table}[!htbp]

  \caption{Resultados del ajuste en las CR correspondientes a {\SRH}, con una luminosidad integrada total de 20.3 \ifb.
    El número de eventos observado es comparado con el número de eventos esperado de fondo, después de la correspondiente
    normalización en las CR. En la parte inferior de la tabla se muestran también lo valores nominales del fondo antes de
    la correspondiente normalización. Las incertezas incluyen la incerteza estadística y sistemática.}
  \label{tab:fit_result_crh}

  \begin{tabularx}{\textwidth}{z{5cm}RRR}
\hline
{\bf Regiones de control}           & {\CRQH}     & {\CRWH}            & {\CRTH}              \\
\hline
Eventos observados          & $216$              & $25$              & $17$                    \\
\hline
Eventos esperados SM        & $216.03 \pm 14.53$          & $25.00 \pm 4.99$          & $17.00 \pm 4.08$              \\
\hline
{\wgam}         & $0.10 \pm 0.08$          & $19.32 \pm 5.62$          & $4.66 \pm 1.43$              \\
{\ttgam}          & $0.04 \pm 0.04$          & $1.91 \pm 1.34$          & $6.88 \pm 4.63$              \\
{\ttgam} (had)          & $0.31 \pm 0.24$          & $0.00 \pm 0.00$          & $0.00 \pm 0.00$              \\
V($\to$ qq)$+\gamma$          & $3.46 \pm 1.53$          & $0.00 \pm 0.00$          & $0.00 \pm 0.00$              \\
single-$t$ + $\gamma$          & $0.01_{-0.01}^{+0.02}$          & $0.54 \pm 0.06$          & $1.51 \pm 0.18$              \\
Z($\rightarrow\ell\ell$) + $\gamma$          & $0.00 \pm 0.00$          & $0.57 \pm 0.57$          & $0.19_{-0.19}^{+0.19}$              \\
Z($\rightarrow\nu\nu$) + $\gamma$          & $0.00 \pm 0.00$          & $0.00 \pm 0.00$          & $0.00 \pm 0.00$              \\
$\gamma$ + jet          & $193.31 \pm 16.91$          & $0.02_{-0.02}^{+0.94}$          & $0.00 \pm 0.00$              \\
$e\rightarrow\gamma$ fakes          & $0.27 \pm 0.04$          & $0.49 \pm 0.08$          & $2.31 \pm 0.37$              \\
$j\rightarrow\gamma$ fakes          & $18.52 \pm 8.70$          & $2.14 \pm 0.99$          & $1.46 \pm 0.68$              \\
\hline
Eventos esperado SM (MC/DD)              & $179.24$          & $22.94$          & $22.04$              \\
\hline
MC {\wgam}               & $0.08$          & $15.62$         & $3.76$              \\
MC {\ttgam}              & $0.08$          & $3.57$          & $12.82$              \\
MC {\ttgam} (had)        & $0.58$          & $0.00$          & $0.00$              \\
MC {\vqqgam}             & $2.80$          & $0.00$          & $0.00$              \\
MC {\tgam}               & $0.01$          & $0.54$          & $1.51$              \\
MC {\zllgam}             & $0.00$          & $0.57$          & $0.19$              \\
MC {\znngam}             & $0.00$          & $0.00$          & $0.00$              \\
MC {\gjet}               & $156.91$        & $0.02$          & $0.00$              \\
DD $e\rightarrow\gamma$  & $0.27$          & $0.49$          & $2.31$              \\
DD $j\rightarrow\gamma$  & $18.50$         & $2.14$          & $1.46$              \\
\hline
\end{tabularx}


\end{table}


\begin{figure}[!htbp]
  \centering

  \includegraphics[width=0.49\textwidth]{can_crql_met_et_afterFit}
  \includegraphics[width=0.49\textwidth]{can_crql_rt4_afterFit} \\

  \includegraphics[width=0.49\textwidth]{can_crwl_met_et_afterFit}
  \includegraphics[width=0.49\textwidth]{can_crwl_rt4_afterFit} \\

  \includegraphics[width=0.49\textwidth]{can_crtl_met_et_afterFit}
  \includegraphics[width=0.49\textwidth]{can_crtl_rt4_afterFit} \\

   \caption{Distribuciones observadas de {\met} (izquierda) y {\rt} (derecha) en las
     regiones de control {\CRQL} (arriba), {\CRWL} (medio) y {\CRTL} (abajo),
     después del ajuste combinado.}
   \label{fig:bkgfit_crl_after}

\end{figure}


\begin{figure}[!htbp]
  \centering

  \includegraphics[width=0.49\textwidth]{can_crqh_met_et_afterFit}
  \includegraphics[width=0.49\textwidth]{can_crqh_ht_afterFit} \\

  \includegraphics[width=0.49\textwidth]{can_crwh_met_et_afterFit}
  \includegraphics[width=0.49\textwidth]{can_crwh_ht_afterFit} \\

  \includegraphics[width=0.49\textwidth]{can_crth_met_et_afterFit}
  \includegraphics[width=0.49\textwidth]{can_crth_ht_afterFit} \\

  \caption{Distribuciones observadas de {\met} (izquierda) y {\HT} (derecha) en las
    regiones de control {\CRQH} (arriba), {\CRWH} (medio) y {\CRTH} (abajo),
    después del ajuste combinado.}
  \label{fig:bkgfit_crh_after}

\end{figure}





%% Validation Regions
\subsection{Resultados en las regiones de validación}

Los factores de normalización calculados a partir del ajuste simultáneo en las
CR, detallados en la \cref{tab:bkgonly_mus}, son utilizados para estimar los fondos
en las VR, donde se compara el fondo esperado con los datos observados, para
validar la extrapolación que luego se hará a la SR.

Las distribuciones de algunos observables en estas regiones pueden verse en
\cref{fig:bkgfit_vr}. El número de eventos de fondo predicho
está en buen acuerdo con el número de eventos observado dentro de las
incertezas, lo que permite validar el método empleado y la viabilidad de
utilizar la estimación del número de eventos de fondo en las SR.

Habiendo validado el método de obtención del fondo contaminante en la región
donde se espera la señal de nueva física, se procedió a la utilización de los
datos experimentales en las denominadas regiones de señal, cuyos resultados
pueden verse en la siguiente sección.


%% %% %with unity within uncertinties. %As expected, $\mu_{Q}$ is way smaller, since the statistics for the smeared pseudo-data is artificially enhanced.
%% The systematic uncertainties in the two signal regions  after the background fit are summarized in \Fig \ref{fig:fit_unc_nuisance_SR}. No profiling of the systematics uncertainties is observed. %hopefully
%% The correlations of these systematics uncertainties and the normalizations are shown in \Fig \ref{fig:fit_corr_SR}. % What do we observe here?

%% Los resultados del ajuste y el número de eventos en los datos observados se
%% presentan en la \cref{tab:fit_result_vrl,tab:fit_result_vrh} para las distintas
%% regiones de validación.


\begin{table}[!htbp]

  \caption{Resultados del ajuste en las VR correspondientes a {\SRL}.
    El número de eventos observado es comparado con el número de eventos esperado de fondo, después de la correspondiente
    normalización en las CR. Las incertezas incluyen la incerteza estadística y sistemática.}
  \label{tab:fit_result_vrl1}

  \begin{tabularx}{\textwidth}{z{4cm}RRRR}
\hline
{\bf Regiones de validación}                  & VRQ            & VRM75            & VRM100            & VRR              \\
\hline
Eventos observados        & $0$            & $54$              & $7$              & $24$                    \\
\hline
Eventos esperados SM    & $0.39 \pm 0.23$  & $51.17 \pm 47.04$          & $13.24 \pm 5.51$          & $24.32 \pm 6.26$              \\
\hline
{\wgam}          & $0.04 \pm 0.04$    & $0.89 \pm 0.79$          & $0.45 \pm 0.40$          & $6.82 \pm 5.74$              \\
{\ttgam}          & $0.30 \pm 0.20$             & $2.55 \pm 1.65$          & $1.54 \pm 1.01$          & $4.74 \pm 2.80$              \\
{\ttgam}  (had)    & $0.02 \pm 0.02$         & $1.69 \pm 1.08$          & $0.38 \pm 0.25$          & $0.00 \pm 0.00$              \\
{\vqqgam}          & $0.00 \pm 0.00$              & $2.45 \pm 2.17$          & $0.82 \pm 0.82$          & $0.00 \pm 0.00$              \\
{\tgam}          & $0.01 \pm 0.01$             & $0.25 \pm 0.10$          & $0.15 \pm 0.06$          & $0.45 \pm 0.08$              \\
{\zllgam}         & $0.00 \pm 0.00$          & $0.08_{-0.08}^{+0.08}$          & $0.04_{-0.04}^{+0.04}$          & $0.08_{-0.08}^{+0.08}$              \\
{\znngam}         & $0.00 \pm 0.00$           & $0.10_{-0.10}^{+0.12}$          & $0.07_{-0.07}^{+0.07}$          & $1.71_{-1.71}^{+1.75}$              \\
{\gjet}        & $0.01_{-0.01}^{+0.07}$            & $35.97_{-35.97}^{+44.72}$          & $7.87 \pm 4.47$          & $2.37 \pm 1.17$              \\
$e\rightarrow\gamma$           & $0.01 \pm 0.00$        & $2.56 \pm 0.72$          & $1.32 \pm 0.43$          & $6.09 \pm 1.14$              \\
$j\rightarrow\gamma$           & $0.00 \pm 0.00$          & $4.63 \pm 2.41$          & $0.60 \pm 0.33$          & $2.06 \pm 0.98$              \\
\hline
%% Eventos esperados SM (MC/DD)              & $0.29$           & $51.24$          & $12.83$          & $21.36$              \\
%% \hline
%% MC W + $\gamma$          & $0.03$                & $0.66$          & $0.33$          & $5.08$              \\
%% MC $t\bar{t}$ + $\gamma$          & $0.21$         & $1.82$          & $1.10$          & $3.39$              \\
%% MC $t\bar{t}$ + $\gamma$ (had)          & $0.01$            & $1.20$          & $0.27$          & $0.00$              \\
%% MC V($\to$ qq)$+\gamma$          & $0.00$             & $1.82$          & $0.61$          & $0.00$              \\
%% MC single-$t$ + $\gamma$          & $0.01$               & $0.25$          & $0.15$          & $0.45$              \\
%% MC Z($\rightarrow\ell\ell$) + $\gamma$          & $0.00$              & $0.08$          & $0.04$          & $0.08$              \\
%% MC Z($\rightarrow\nu\nu$) + $\gamma$          & $0.00$                & $0.10$          & $0.07$          & $1.71$              \\
%% MC $\gamma$ + jet          & $0.01$              & $38.11$          & $8.34$          & $2.51$              \\
%% DD $e\rightarrow\gamma$         & $0.01$             & $2.56$          & $1.32$          & $6.09$              \\
%% DD $j\rightarrow\gamma$         & $0.00$              & $4.63$          & $0.60$          & $2.06$              \\
%% \hline
\end{tabularx}


\end{table}


\begin{table}[!htbp]

  \caption{Resultados del ajuste en las VR correspondientes a {\SRL}.
    El número de eventos observado es comparado con el número de eventos esperado de fondo, despues de la correspondiente
    normalizacion en las CR. Las incertezas incluyen la incerteza estadistica y sistematica.}
  \label{tab:fit_result_vrl2}

  \begin{tabularx}{\textwidth}{z{4cm}RRRR}
\hline
{\bf Regiones de validación}                   & VRWR            & VRWM            & VRTR            & VRTM              \\
\hline
Eventos observados                             & $0$             & $5$             & $1$             & $3$                    \\
\hline
Eventos esperados SM                           & $0.20 \pm 0.15$          & $3.41 \pm 1.41$          & $0.76 \pm 0.30$          & $3.52 \pm 0.84$              \\
\hline
{\wgam}               & $0.13 \pm 0.10$          & $2.39 \pm 1.47$          & $0.04 \pm 0.03$          & $1.04 \pm 0.65$              \\
{\ttgam}              & $0.04_{-0.04}^{+0.07}$   & $0.41 \pm 0.21$          & $0.46 \pm 0.28$          & $1.86 \pm 0.92$              \\
{\ttgam} (had)        & $0.00 \pm 0.00$          & $0.00 \pm 0.00$          & $0.00 \pm 0.00$          & $0.00 \pm 0.00$              \\
{\vqqgam}             & $0.00 \pm 0.00$          & $0.00 \pm 0.00$          & $0.00 \pm 0.00$          & $0.00 \pm 0.00$              \\
{\tgam}               & $0.03 \pm 0.02$          & $0.04 \pm 0.01$          & $0.08 \pm 0.03$          & $0.19 \pm 0.04$              \\
{\zllgam}             & $0.00 \pm 0.00$          & $0.06_{-0.06}^{+0.06}$   & $0.02_{-0.02}^{+0.02}$   & $0.00 \pm 0.00$              \\
{\znngam}             & $0.00 \pm 0.00$          & $0.00 \pm 0.00$          & $0.00 \pm 0.00$          & $0.00 \pm 0.00$              \\
{\gjet}               & $0.00 \pm 0.00$          & $0.00 \pm 0.00$          & $0.00 \pm 0.00$          & $0.00 \pm 0.00$              \\
$e\rightarrow\gamma$  & $0.00 \pm 0.00$          & $0.07 \pm 0.02$          & $0.08 \pm 0.02$          & $0.17 \pm 0.03$              \\
$j\rightarrow\gamma$  & $0.00 \pm 0.00$          & $0.43 \pm 0.21$          & $0.09 \pm 0.04$          & $0.26 \pm 0.12$              \\
\hline
%% MC exp. SM events              & $0.16$          & $2.68$          & $0.62$          & $2.72$              \\
%% \noalign{\smallskip}\hline\noalign{\smallskip}
%%         MC exp. W + $\gamma$ events         & $0.09$          & $1.78$          & $0.03$          & $0.77$              \\
%%         MC exp. $t\bar{t}$ + $\gamma$ events         & $0.03$          & $0.29$          & $0.33$          & $1.33$              \\
%%         MC exp. $t\bar{t}$ + $\gamma$ (had) events         & $0.00$          & $0.00$          & $0.00$          & $0.00$              \\
%%         MC exp. V($\to$ qq)$+\gamma$ events         & $0.00$          & $0.00$          & $0.00$          & $0.00$              \\
%%         MC exp. single-$t$ + $\gamma$ events         & $0.03$          & $0.04$          & $0.08$          & $0.19$              \\
%%         MC exp. Z($\rightarrow\ell\ell$) + $\gamma$ events         & $0.00$          & $0.06$          & $0.02$          & $0.00$              \\
%%         MC exp. Z($\rightarrow\nu\nu$) + $\gamma$ events         & $0.00$          & $0.00$          & $0.00$          & $0.00$              \\
%%         MC exp. $\gamma$ + jet events         & $0.00$          & $0.00$          & $0.00$          & $0.00$              \\
%%         DD exp. $e\rightarrow\gamma$ fakes events         & $0.00$          & $0.07$          & $0.08$          & $0.17$              \\
%%         DD exp. $j\rightarrow\gamma$ fakes events         & $0.00$          & $0.43$          & $0.09$          & $0.26$              \\

%% \noalign{\smallskip}\hline\noalign{\smallskip}
\end{tabularx}


\end{table}


\begin{table}[!htbp]

  \caption{Resultados del ajuste en las VR correspondientes a {\SRH}.
    El número de eventos observado es comparado con el número de eventos esperado de fondo, después de la correspondiente
    normalización en las CR. Las incertezas incluyen la incerteza estadística y sistemática.}
  \label{tab:fit_result_vrh1}

  \begin{tabularx}{\textwidth}{LRR}
\hline
{\bf Regiones de validación}           & VRQ                      & VRH                          \\
\hline
Eventos observados                     & $2$                      & $4$                          \\
\hline
Eventos esperados SM                   & $0.89 \pm 0.32$          & $3.39 \pm 1.69$              \\
\hline
{\wgam}                                & $0.34 \pm 0.19$          & $1.18 \pm 0.55$              \\
{\ttgam}                               & $0.07 \pm 0.05$          & $0.11 \pm 0.08$              \\
{\ttgam} (had)                         & $0.01 \pm 0.01$          & $0.00 \pm 0.00$              \\
{\vqqgam}                              & $0.00 \pm 0.00$          & $0.00 \pm 0.00$              \\
{\tgam}                                & $0.03 \pm 0.01$          & $0.01 \pm 0.00$              \\
{\zllgam}                              & $0.02_{-0.02}^{+0.02}$   & $0.00 \pm 0.00$              \\
{\znngam}                              & $0.08_{-0.08}^{+0.08}$   & $1.60_{-1.60}^{+1.61}$       \\
{\gjet}                                & $0.19 \pm 0.11$          & $0.00 \pm 0.00$              \\
$e\rightarrow\gamma$                   & $0.00 \pm 0.00$          & $0.15 \pm 0.03$              \\
$j\rightarrow\gamma$                   & $0.17 \pm 0.09$          & $0.34 \pm 0.16$              \\
\hline
%% MC exp. SM               & $0.86$          & $3.25$              \\
%% \hline
%% MC W + $\gamma$          & $0.27$          & $0.96$              \\
%% MC $t\bar{t}$ + $\gamma$          & $0.12$          & $0.20$              \\
%% MC $t\bar{t}$ + $\gamma$ (had)          & $0.02$          & $0.00$              \\
%% MC V($\to$ qq)$+\gamma$          & $0.00$          & $0.00$              \\
%% MC single-$t$ + $\gamma$          & $0.03$          & $0.01$              \\
%% MC Z($\rightarrow\ell\ell$) + $\gamma$          & $0.02$          & $0.00$              \\
%% MC Z($\rightarrow\nu\nu$) + $\gamma$          & $0.08$          & $1.60$              \\
%% MC $\gamma$ + jet          & $0.15$          & $0.00$              \\
%% DD $e\rightarrow\gamma$          & $0.00$          & $0.15$              \\
%% DD $j\rightarrow\gamma$          & $0.17$          & $0.34$              \\
%% \hline
\end{tabularx}


\end{table}


\begin{table}[!htbp]

  \caption{Resultados del ajuste en las VR correspondientes a {\SRH}.
    El número de eventos observado es comparado con el número de eventos esperado de fondo, después de la correspondiente
    normalización en las CR. Las incertezas incluyen la incerteza estadística y sistemática.}
  \label{tab:fit_result_vrh2}

  \begin{tabularx}{\textwidth}{z{4cm}RRRR}
\hline
{\bf Regiones de validación}                      & VRWH            & VRWM            & VRTH            & VRTM              \\
\hline
Eventos observados                                & $1$              & $5$              & $2$              & $2$                    \\
\hline
Eventos esperados SM                              & $1.25 \pm 0.38$          & $4.38 \pm 1.14$          & $0.83 \pm 0.26$          & $1.76 \pm 0.38$              \\
\hline
{\wgam}                      & $1.08 \pm 0.38$          & $3.75 \pm 1.17$          & $0.28 \pm 0.17$          & $1.08 \pm 0.35$              \\
{\ttgam}                     & $0.04 \pm 0.04$          & $0.13 \pm 0.09$          & $0.17 \pm 0.13$          & $0.38 \pm 0.27$              \\
{\ttgam} (had)               & $0.00 \pm 0.00$          & $0.00 \pm 0.00$          & $0.00 \pm 0.00$          & $0.00 \pm 0.00$              \\
{\vqqgam}                    & $0.00 \pm 0.00$          & $0.00 \pm 0.00$          & $0.00 \pm 0.00$          & $0.00 \pm 0.00$              \\
{\tgam}                      & $0.02 \pm 0.01$          & $0.02 \pm 0.01$          & $0.15 \pm 0.04$          & $0.08 \pm 0.02$              \\
{\zllgam}                    & $0.00 \pm 0.00$          & $0.00 \pm 0.00$          & $0.02_{-0.02}^{+0.02}$   & $0.00 \pm 0.00$              \\
{\znngam}                    & $0.00 \pm 0.00$          & $0.00 \pm 0.00$          & $0.00 \pm 0.00$          & $0.00 \pm 0.00$              \\
{\gjet}                      & $0.02 \pm 0.01$          & $0.00 \pm 0.00$          & $0.00 \pm 0.00$          & $0.00 \pm 0.00$              \\
$e\rightarrow\gamma$         & $0.00 \pm 0.00$          & $0.04 \pm 0.01$          & $0.05 \pm 0.01$          & $0.05 \pm 0.01$              \\
$j\rightarrow\gamma$         & $0.09 \pm 0.04$          & $0.43 \pm 0.20$          & $0.17 \pm 0.09$          & $0.17 \pm 0.08$              \\
\hline
%% %%
%% MC exp. SM events              & $1.07$          & $3.78$          & $0.92$          & $1.88$              \\
%% \noalign{\smallskip}\hline\noalign{\smallskip}
%% %%
%%         MC exp. W + $\gamma$ events         & $0.87$          & $3.04$          & $0.23$          & $0.87$              \\
%% %%
%%         MC exp. $t\bar{t}$ + $\gamma$ events         & $0.08$          & $0.25$          & $0.31$          & $0.71$              \\
%% %%
%%         MC exp. $t\bar{t}$ + $\gamma$ (had) events         & $0.00$          & $0.00$          & $0.00$          & $0.00$              \\
%% %%
%%         MC exp. V($\to$ qq)$+\gamma$ events         & $0.00$          & $0.00$          & $0.00$          & $0.00$              \\
%% %%
%%         MC exp. single-$t$ + $\gamma$ events         & $0.02$          & $0.02$          & $0.15$          & $0.08$              \\
%% %%
%%         MC exp. Z($\rightarrow\ell\ell$) + $\gamma$ events         & $0.00$          & $0.00$          & $0.02$          & $0.00$              \\
%% %%
%%         MC exp. Z($\rightarrow\nu\nu$) + $\gamma$ events         & $0.00$          & $0.00$          & $0.00$          & $0.00$              \\
%% %%
%%         MC exp. $\gamma$ + jet events         & $0.02$          & $0.00$          & $0.00$          & $0.00$              \\
%% %%
%%         DD exp. $e\rightarrow\gamma$ fakes events         & $0.00$          & $0.04$          & $0.05$          & $0.05$              \\
%% %%
%%         DD exp. $j\rightarrow\gamma$ fakes events         & $0.09$          & $0.43$          & $0.17$          & $0.17$              \\
%% %%     \\
%% \noalign{\smallskip}\hline\noalign{\smallskip}
\end{tabularx}


\end{table}


\begin{figure}[!htbp]
  \centering

  \includegraphics[width=0.49\textwidth]{can_vrm75l_met_et_afterFit}
  \includegraphics[width=0.49\textwidth]{can_vrm75l_rt4_afterFit} \\

  \includegraphics[width=0.49\textwidth]{can_vrh_met_et_afterFit}
  \includegraphics[width=0.49\textwidth]{can_vrh_ht_afterFit} \\

  \caption{Distribuciones observadas en algunas de las regiones de validación, después del ajuste combinado.}
  \label{fig:bkgfit_vr}

\end{figure}

\clearpage



\subsection{Resultados en las regiones de señal}

Las predicciones del fondo luego del ajuste simultáneo en las CR son presentadas
en la \cref{tab:fit_result_sr} para las regiones de señal {\SRL} y {\SRH},
conjuntamente con el número de eventos de datos observados. El acuerdo entre los
datos observados y el fondo esperado es bueno y no se observa un exceso
significativo de datos por sobre los predichos en ninguna de las SR. Como se
desprende de la Tabla, el número de eventos predicho en la region {\SRL} y
{\SRH} es de $1.27\pm0.43$ y $0.84\pm0.38$, respectivamente, mientras que el
número de eventos observado en ambas SR es de 2.

\begin{table}[!htbp]
  \centering

  \caption{Resultados del ajuste en las SR, con una luminosidad integrada total
    de 20.3 \ifb. El número de eventos observado es comparado con el número de
    eventos esperado de fondo, después de la correspondiente normalización en
    las CR. En la parte inferior de la tabla se muestran también lo valores
    nominales del fondo antes de la correspondiente normalización. Las
    incertezas incluyen la incerteza estadística y sistemática.}
  \label{tab:fit_result_sr}

  \begin{tabularx}{\textwidth}{LRR}
\hline
{\bf Regiones de señal} & {\SRL} & {\SRH}  \\
\hline
Eventos observados                         & $2$                        &  $2$                          \\
\hline
Eventos esperados SM                       & $1.27 \pm 0.43$            &  $0.84 \pm 0.38$              \\
\hline
{\wgam}                                    & $0.13 \pm 0.12$            &  $0.54 \pm 0.28$              \\
{\ttgam}                                    & $0.64 \pm 0.40$            &  $0.05 \pm 0.05$              \\
%% {\vqqgam}                                  & $0.00 \pm 0.00$            &  $0.00 \pm 0.00$              \\
{\tgam}                                    & $0.06 \pm 0.02$            &  $0.03 \pm 0.01$              \\
%% {\zllgam}                                  & $0.00 \pm 0.00$            &  $0.00 \pm 0.00$              \\
{\znngam}                                  & $0.03_{-0.03}^{+0.05}$     &  $0.21_{-0.21}^{+0.23}$       \\
{\gjet}                                    & $0.00_{-0.00}^{+0.06}$     &  $0.00 \pm 0.00$              \\
$e\to\gamma$                       & $0.38 \pm 0.10$            &  $0.00 \pm 0.00$              \\
$j\to\gamma$                       & $0.02_{-0.02}^{+0.08}$     &  $0.00_{-0.00}^{+0.08}$       \\
\hline
\end{tabularx}


\end{table}


Las distribuciones de {\met} para {\SRL} y {\SRH} se muestran en la
\cref{fig:met_sr}, para datos y también para el fondo esperado después del
ajuste en las CR. Se puede ver el buen acuerdo en la zona de bajo {\met} y las
flechas indican la zona de las SR.

\begin{figure}[!htbp]

  \centering

  \includegraphics[width=0.49\textwidth]{plot_after_SRL_met_et}
  \includegraphics[width=0.49\textwidth]{plot_after_SRH_met_et}

  \caption{Distribución de {\met} comparando datos y el fondo esperado para la
    selección de la región {\SRL} (izquierda) y {\SRH} (derecha), sin el corte
    en {\met}. La distribución esperada para dos muestras de señal es también
    graficada para su comparación.}
  \label{fig:met_sr}

\end{figure}


El resumen de los resultados para todas las regiones pueden verse en
\cref{fig:fit_region_composition}.

\begin{figure}[!htbp]
  \centering

  \includegraphics[width=0.9\textwidth]{histpull_SRL}
  \includegraphics[width=0.9\textwidth]{histpull_SRH}

  \caption{Comparación entre el número de eventos observado y esperado en cada
    una de las regiones correspondientes a {\SRL} (arriba) y {\SRH} (abajo).}

  \label{fig:fit_region_composition}

\end{figure}


Las incertezas sistemáticas dominantes en el número total predicho de fondo en
cada SR, se presentan en \cref{tab:syst_srl,tab:syst_srh}.
Debido a que las incertezas individuales pueden estar correlacionadas y las
correlaciones negativas entre normalizaciones (representando el impacto de la
estadística en las CR en la incerteza total), la incerteza total no es
necesariamente la suma en cuadratura de las mismas. La incerteza estadística se
ilustra con el valor de la raíz cuadrada del número de eventos esperado con la
finalidad de mostrar el impacto relativo de la incerteza estadística con
respecto a la incerteza total del fondo. Las incertezas sistemáticas relativas
con respecto a las predicciones del fondo son del orden de 54\% (58\%) para
{\SRL} (\SRH). En {\SRL} las incertezas dominantes provienen de la normalización
y de las incertezas teóricas del fondo de {\ttgam}, seguidas por la incerteza en
la escala de energía de los jets y la estadística en la SR. En {\SRH}, las
incertezas están dominadas por las incertezas teóricas del fondo {\zgam} y
{\gjet}, y la incerteza estadística y de normalización del fondo {\wgam}.

%% Since the individual uncertainties can be correlated and the negative correlations between normalizations (representing the
%% impact of the CR statistics on the total uncertainty) and systematic uncertainties are observed as expected, they
%% do not necessarily add up by quadrature to the total background uncertainty. The square
%% root of the number of the expected events are calculated in order to illustrate the relative impact of the
%% SR Poisson statistical uncertainties with respect to the total background uncertainties. The
%%  relative systematic uncertainties with respect to the background predictions are large, around 54\% (58\%) for SR2 (SR3). In SR2, the main uncertainties
%% after the fit come from the normalisation and the theoretical uncertainties of the \ttbargam\ background, followed by the jet energy scale and the MC statistics
%% in the SR. In SR3, the uncertainties are dominated by the theoretical uncertainties on \zgam and \gjet backgrounds, the MC statistics and the normalisation uncertainties of the \wgamma\ production.

%% Similar breakdown tables but separated on each background prediction in each SR are presented in \Tab \ref{tab:fit_result_sr2_syst_perbkg}-\ref{tab:fit_result_sr3_syst_perbkg}.


\begin{table}[!htbp]
  \centering

  \caption{Resumen de las incertezas sistemáticas dominantes en la estimación del fondo total
    en {\SRL}. Notar que las incertezas individuales pueden estar correlacionadas, y la incerteza
    total no es necesariamente la suma en cuadratura de estas. Los porcentajes muestran el tamaño
    de la incerteza relativo al fondo esperado total.}
  \label{tab:syst_srl}

  \begin{tabularx}{\textwidth}{LR}
\hline
{\bf Incertezas}                                    & {\SRL}            \\
\hline
Eventos esperados SM             &  $1.27$       \\
\hline
Incerteza estadística total $(\sqrt{N_\mathrm{SM}})$              & $\pm 1.13$       \\
Incerteza sistemática total               & $\pm 0.43\ [34.18\%] $             \\
\hline
$\alpha_{\tgam}$         & $\pm 0.36\ [28.0\%] $       \\
$\mu_{T}$         & $\pm 0.35\ [27.7\%] $       \\
$\gamma_\mathrm{SR}$        & $\pm 0.22\ [17.4\%] $       \\
$\alpha_\mathrm{JES}$         & $\pm 0.21\ [16.5\%] $       \\
$\alpha_{\wgam}$         & $\pm 0.09\ [7.3\%] $       \\
$\mu_{W}$        & $\pm 0.08\ [6.7\%] $       \\
$\alpha_{j\to\gamma}$         & $\pm 0.08\ [6.3\%] $       \\
$\alpha_{e\to\gamma}$         & $\pm 0.07\ [5.7\%] $       \\
$\alpha_{\mathrm{JER}}$         & $\pm 0.06\ [4.3\%] $       \\
%% $\alpha_{Z\gamma}$         & $\pm 0.03\ [2.7\%] $       \\
%% $\alpha_\mathrm{EGMAT}$         & $\pm 0.03\ [2.2\%] $       \\
%% $\alpha_\mathrm{EGRES}$         & $\pm 0.03\ [2.0\%] $       \\
%% $\alpha_\mathrm{SCALEST}$         & $\pm 0.02\ [1.7\%] $       \\
%% $\alpha_\mathrm{PRW}$         & $\pm 0.01\ [0.71\%] $       \\
%% $\alpha_\mathrm{RESOST}$         & $\pm 0.00\ [0.35\%] $       \\
%% $\alpha_{t\gamma}$         & $\pm 0.00\ [0.34\%] $       \\
%% $\alpha_\mathrm{PHEFF}$         & $\pm 0.00\ [0.06\%] $       \\
%% $\mu_{Q}$         & $\pm 0.00\ [0.02\%] $       \\
%% $\alpha_{\gamma j}$         & $\pm 0.00\ [0.02\%] $       \\
\hline
\end{tabularx}

\end{table}

\begin{table}[!htbp]
  \centering

  \caption{Resumen de las incertezas sistematicas dominantes en la estimacion del fondo total
    en {\SRH}. Notar que las incertezas individuales pueden estar correlacionados, y la incerteza
    total no es necesariamente la suma en cuadratura de estas. Los porcentajes muestran el tamano
    de la incerteza relativo al fondo esperado total.}
  \label{tab:syst_srh}

  \begin{tabularx}{\textwidth}{LR}
\hline
{\bf Incertezas}                                    & {\SRH}            \\
\hline
Eventos esperados SM             &  $0.84$       \\
\noalign{\smallskip}\hline\noalign{\smallskip}
Incerteza estadística total $(\sqrt{N_{\rm SM}})$              & $\pm 0.92$       \\
Incerteza sistemática total               & $\pm 0.38\ [45.27\%] $             \\
\hline
$\alpha_{Z\gamma}$         & $\pm 0.21\ [25.4\%] $       \\
$\alpha_{\wgam}$         & $\pm 0.21\ [25.1\%] $       \\
$\gamma_\mathrm{SR}$         & $\pm 0.20\ [24.3\%] $       \\
$\mu_{W}$         & $\pm 0.17\ [20.5\%] $       \\
$\alpha_\mathrm{PRW}$         & $\pm 0.10\ [11.9\%] $       \\
$\alpha_\mathrm{JFAKE}$         & $\pm 0.08\ [9.5\%] $       \\
$\alpha_\mathrm{EGRES}$         & $\pm 0.05\ [5.8\%] $       \\
$\alpha_\mathrm{JER}$         & $\pm 0.04\ [4.8\%] $       \\
%% $\alpha_\mathrm{RESOST}$         & $\pm 0.04\ [4.5\%] $       \\
%% $\alpha_\mathrm{EGMAT}$         & $\pm 0.04\ [4.5\%] $       \\
%% $\mu_{T}$         & $\pm 0.04\ [4.4\%] $       \\
%% $\alpha_\mathrm{JES}$         & $\pm 0.03\ [4.0\%] $       \\
%% $\alpha_{\tgam}$         & $\pm 0.02\ [1.8\%] $       \\
%% $\alpha_\mathrm{MMS}$         & $\pm 0.01\ [1.3\%] $       \\
%% $\alpha_\mathrm{MID}$         & $\pm 0.01\ [1.3\%] $       \\
%% $\alpha_\mathrm{PHEFF}$         & $\pm 0.00\ [0.40\%] $       \\
%% $\alpha_{t\gamma}$         & $\pm 0.00\ [0.22\%] $       \\
\hline
\end{tabularx}


\end{table}

\hl{Mergear tablas}


\begin{figure}[!htbp]
  \centering

  \includegraphics[width=\textwidth]{syst_pull_afterFit_SR2} \\
  \includegraphics[width=\textwidth]{syst_pull_afterFit_SR3}

  \caption{Resumen de los parametros despues del ajuste para {\SRL} (arriba) y {\SRH} (abajo).}
  \label{fig:fit_unc_nuisance_SR}

\end{figure}

\begin{figure}[!htbp]
  \centering

  \includegraphics[width=\textwidth]{corr_matrix_srl} \\
  \includegraphics[width=\textwidth]{corr_matrix_srh} \\

  \caption{Matriz de correlación de los parámetros del ajuste, correspondientes a {\SRL} (arriba) y {\SRH} (abajo).}
  \label{fig:fit_corr_SR}

\end{figure}



\subsection{Eventos en las regiones de señal}

Como se mostró en la \cref{tab:fit_result_sr}, solo dos eventos pasan la
selección de cada una de las regiones de señal. Un esquema del detector y las
se\~nales dejadas por los distintos objetos en el mismo puede verse en las
\cref{fig:evdisplay_srl} para los eventos observados en la {\SRL} y en las
\cref{fig:evdisplay_srh} para los eventos en la {\SRH}.

En la parte superior izquierda se puede ver el corte transversal del detector
junto con el corte transversal en el panel inferior. En estas vistas se puede ver
los depositos de energia en los calorimetros en color amarillo. Los conos de distintos
colores representan los jets reconstruidos

\hl{Aca explicar que significan cada cosa del plot...}

Las visualizaciones de los eventos es generada con
\textsc{Atlantis}\cite{atlantis}.


\begin{figure}[!htbp]
  \begin{center}

    \includegraphics[height=0.45\textheight]{EvtDisplay_212103_73661974}

    \vspace{1cm}

    \includegraphics[height=0.45\textheight]{EvtDisplay_212144_201737678}

    \caption{Visualización de uno de los eventos que sobreviven la selección de {\SRL}. Ver detalles en el texto.} %%. Run 212103, Evento 73661974.}
    \label{fig:evdisplay_srl}
  \end{center}
\end{figure}

%% \begin{figure}[!htbp]
%%   \begin{center}
%%   \caption{Visualización de uno de los eventos que sobreviven la selección de {\SRL}. Run 212144, Evento 201737678.}
%%   \label{fig:evdisplay_sr2_2}
%%   \end{center}
%% \end{figure}

\begin{figure}[!htbp]
  \begin{center}

    \includegraphics[height=0.45\textheight]{EvtDisplay_203875_14229699}

    \vspace{1cm}

    \includegraphics[height=0.45\textheight]{EvtDisplay_212967_66359411}

  \caption{Visualización de los eventos que sobreviven la selección de {\SRH}. Ver detalles en el texto.} %%. Run 203875, Evento 14229699.}
  \label{fig:evdisplay_srh}
  \end{center}
\end{figure}


%% \clearpage


%-------------------------
% Model independent limit
%-------------------------
\section{Límites a procesos de Nueva Física} \label{sec:model_independent}

%% Mas alla de los límites en el modelo de senal de SUSY estudiado en esta Tesis,
%% el analisis que busca nueva fisica puede imponer límites en el número de eventos
%% de nueva fisica por sobre el fondo esperado en cada SR. De esta forma, para cada
%% modelo de senal de interes, cualquiera puede estimar el número de eventos de
%% senal predichos en una SR y chequear si el modelo es excluido o no de acuerdo a
%% los datos observados en el analisis.


Además de los límites en el modelo de SUSY especifico que motivó el análisis de
esta Tesis, resulta % útil proveer el límite superior en el número de eventos de
nueva física impuesto por los resultados obtenidos. Este límite, provisto para
cada región de señal, hace posible su aplicación a cualquier otro modelo de
nueva física, simplemente calculando el número de eventos de señal que se espera
en la región en cuestión y verificando que dicho número no exceda el límite
superior. Por supuesto que los valores de eficiencia y aceptancia deben ser
tenidos en cuenta, razón por lo que también se proveen estos valores\note{citar
  hepdata}.

Con el objetivo de obtener los limites independientes del modelo se realiza un
ajuste combinado teniendo en cuenta las regiones de control y cada una de las
regiones de señal. Se supone que la contaminación de señal en las CR es nula,
pero no se realiza ninguna otra suposición acerca del modelo de nueva física. El
número de eventos de señal en la SR se agrega como el parámetro de interés y se
deja libre en el ajuste. Se realiza un \emph{scan} sobre este parámetro, y el
límite superior en el número de eventos de nueva física se encuentra para el
valor en el cual el {\cls} cae debajo del 5\%. El límite superior en el número
de eventos puede transformarse en el límite superior de la sección eficaz
visible\note{definir footnote} utilizando por la luminosidad considerada en el análisis
(20.3 \ifb).

%% Estos resultados se detallan en \cref{tab:upperlimits} para las dos SR. Debido a que el
%% número esperado de eventos de fondo es bajo, no es posible utilizar la aproximacion asintotica,
%% y se generaron pseudo-experimentos MC.

%% Los resultados se obtienen de forma independiente utilizando 3000 pseudo-experimentos, y la
%% aproximacion asintotica.
%% The discovery fit is performed to the CR data and SR data by maximizing the likelihood in \Eq \ref{eq:likelihood} but
%%  with the signal component only in the SR. The discovery test is done by the background-only hypothesis
%% and quantified using pseudo-experiments by p-value $p_b = P(q \leq q_{obs}|b)$ where $q$ is the test statistics.
%% In the absence of a statistically significant excess, limits are set on contributions to the SRs from new
%% physics. Calculated p-values and model independent limits on the number of new physics contribution
%% by each SR are listed in \Tab \ref{tab:upperlimits}. Two sets of results are obtained with 3000 toy experiments and with
%% asymptotic approximation, respectively. Although the observed limits on the visible cross sections are not much different,
%% the asymptotic approximation is supposed to hold only for high statistics scenarios. Thus, we decided to keep all the
%% limit results with toys. The results with asymptotic approximation are also shown in \App \ref{sec:ap:asym_limits} for comparison.

%% %Since the limit numbers between the two sets are not much different, we expect that the model-dependent limits will not change.

\begin{table}[!htbp]
  \centering

  \caption{Límite independiente del modelo de señal a 95\% de CL en la
    sección eficaz visible observada ($\langle\epsilon{\rm \sigma}\rangle_{\rm obs}$),
    y el límite en el número de eventos de nueva física observado
    $S_\text{obs}$ y esperado $S_\text{exp}$ para las dos SR.
    La última linea ($p_0$) indica el {\pvalue} de la hipótesis de solo-fondo.}
    %% Signal-model-independent 95\% CL upper limits on the observed ($\langle\epsilon{\rm \sigma}\rangle_{\rm obs}^{95}$) %and expected ($\langle\epsilon{\rm \sigma}\rangle_{\rm exp}^{95}$)
    %% visible cross-section, and on the observed ($S_{\rm obs}^{95}$) and expected ($S_{\rm exp}^{95}$) number of beyond-SM
    %% events in the various SRs. The fourth line indicates the CLB value, i.e. the observed confidence level for
    %% the background-only hypothesis. The last line (p0) indicates the probability of the observation being
    %% consistent with the estimated background. Two sets of results are obtained with 3000 toy experiments
    %% (top-half) and with asymptotic approximation (lower-half).}
  \label{tab:upperlimits}

  \begin{tabularx}{\textwidth}{LRR}
  \hline
  {\bf Region de señal}   &            \SRL &              \SRH \\
  \hline
  Eventos observados      &             $2$ &              $2$  \\
  Eventos esperados SM    & $1.27 \pm 0.43$ &  $0.84 \pm 0.38$  \\
  \hline
  $\avg{\epsilon{\sigma}}_\text{obs} \, [\mathrm{fb}]$  & 0.27  & 0.28 \\
  $S_\text{obs}$  & 5.5 & 5.6 \\
  %%$S_\text{exp}$ & ${4.0}^{+1.6}_{-0.1}$ & ${4.1}^{+1.7}_{-0.1}$ \\
  %% {\clb} & 0.86 & 0.83 \\
  $p_0$  & 0.16 &  0.19 \\
  \hline
\end{tabularx}


\end{table}

%% \hll{Poner el plot del scan para este límite}
%% \hll{Límite 1D para EWK?}


%-------------------------
% SUSY Exclusion limit
%-------------------------
\section{Límites de exclusión en el modelo de SUSY considerado}

Debido a que no se observa un exceso significativo por encima del fondo esperado
del {\SM} en las regiones de señal, se obtuvieron los límites de exclusión en el
modelo de SUSY considerado. Los límites superiores son obtenidos utilizando el
método del {\cls} descripto en la \cref{sec:limits}. Estos límites son
calculados para la sección eficaz nominal del modelo y también con una sección
eficaz de $\pm 1 \sigma$ en la incerteza teórica, siguiendo la recomendación de
ATLAS y el acuerdo de los experimentos del LHC para presentar los resultados.

La \cref{fig:limit_srs} muestra los límites de exclusión esperados y
observados para las dos SR de forma separada. Los límites fueron obtenidos
utilizando $\sim 20000$ pseudo-experimentos. La línea punteada azul muestra el
límite esperado a 95\% CL, con las bandas amarillas indicando la desviación a
1$\sigma$ teniendo en cuenta las incertezas teóricas y experimentales. Los
límites observados están indicados por las líneas en rojo oscuro, donde la línea
sólida representa el límite nominal y las líneas punteadas representan el límite
para las variaciones de la sección eficaz de la señal debido a la incerteza
teórica en la misma.

Por diseño, la {\SRL} es importante en la zona de neutralinos livianos, mientras que
la {\SRH} cubre la región mas comprimida del espacio de fases. En la
\cref{fig:limit_combined}, se muestra el límite combinando ambas
regiones de señal, eligiendo en cada punto la que provee el mejor límite.


\begin{figure}[!htbp]
  \centering

  \includegraphics[width=0.49\textwidth]{limitplot_srl}
  \includegraphics[width=0.49\textwidth]{limitplot_srh}

  \caption{Límites de exclusión a 95\% CL para {\SRL}  (izquierda) y {\SRH} (derecha).}
  \label{fig:limit_srs}
\end{figure}


\begin{figure}[!htbp]
  \centering

  \includegraphics[width=0.8\textwidth]{limitplot_combined}

  \caption{Límites de exclusión para {\SRL} y {\SRH} combinados.
    Los límites son obtenidos usando la región de señal con mejor sensibilidad
    en cada punto.
    La linea punteada azul muestra el límite esperado a 95\% CL, con las bandas amarillas indicando la desviación a
    $1\sigma$ que tiene en cuenta las incertezas teóricas y experimentales. Los
    límites observados están indicados por las lineas en rojo oscuro, donde la línea
    solida representa el límite nominal y las líneas punteadas representan el límite
    para las variaciones de la sección eficaz de la señal debido a la incerteza
    teórica en la misma.}
   \label{fig:limit_combined}

\end{figure}


Como se desprende de los gráficos de los límites de exclusión, la produccion de
gluino es excluida a 95\% {\cl} para una masa mínima (máxima) de 1190 (1320)
\gev, para $m_{\ninoone}<{840}\gev$. Para masas de gluino menores a $1\tev$, un
neutralino NLSP es excluido entre $150\gev$ y $m_{\gluino}-m_{\ninoone}>50\gev$.

En la \cref{fig:susy_summary} puede verse un resumen de los limites obtenidos en
las masas de las partículas supersimétricas buscadas en una selección de los
análisis realizados por ATLAS. En la Figura se encuentra el resultado obtenido
por el análisis presentado en esta Tesis, al igual que los otros análisis
complementarios de búsqueda de modelos GGM.


\begin{figure}[!htbp]
  \centering

  \includegraphics[width=\textwidth]{atlas_susy_run1_summary}

  \caption{Resumen de los límites de exclusión en la masa de las párticulas
    supersimétricas por distintos análisis realizados en ATLAS. Solo una
    selección representativa de los resultados disponibles es
    mostrada\cite{susy_summary}.}
  \label{fig:susy_summary}

\end{figure}


\section{Resultados con 13 \tev}

\hl{Agregar}
