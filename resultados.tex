\chapter{Análisis Estadístico y Resultados}

En este Capítulo se presentan los resultados, y el analisis estadistico
de los datos. Como se explica en el \cref{cap:estadistica}, se utiliza
como estadistico de prueba el \emph{profile likelihood ratio}



Para el analisis estadistico se utiliza \texttt{HistFitter}[], una herrmienta
desarrollada dentro del grupo de SUSY en ATLAS, y es basicamente una interfaz
de \texttt{RooFit}, \texttt{RooStats}\cite{Moneta:2010pm} y
\texttt{HistFactory} \cite{Cranmer:1456844}, de forma de facilitar el analisis
estadistico. Ademas el hecho de utilizar una misma herramienta en distintos
analisis dentro de ATLAS, permite que su combinacion estadistica sea mas sencilla.


En este capítulo se presenta el análisis estadístico de los datos. La
compatibilidad con el Modelo Estándar, los limites en la sección eficaz y los
limites de exclusión son obtenidos utilizando el profile log likelihood ratio
LLR.


%% The profile LLR is obtained from a simultaneous fit to the contributions
%% from Standard Model background and supersymmetric signal
%% models in a given signal region and its associated background
%% control regions, which are all by design statistically independent.
%% Cross-contamination of backgrounds across control region
%% boundaries and the propagation of statistical and systematic uncertainties is cleanly taken into account.


% Cuando se realiza el ajuste simultaneo en las CR,
%% When fitting CRs simultaneously, common normalizations are allowed in order to correctly take into
%% account the other background contamination in a given CR. Experimental systematic uncertainties are
%% correlated across the CRs and a SR. Sample-dependent theory systematics are uncorrelated.


%% Cada función de Poisson $P_{i}$ refleja el numero de eventos medidos en la region $i$,
%% ${\mathbf n_{i}}$, y el n\'umero de eventos esperados para las componentes de se\~nal
%% ${\mathbf s}$ y fondo ${\mathbf b}$. El factor de normalización $\mu$ para los fondos
%% \wgamma, \ttbargam, y \gjet son parametros libres del fit.
%% Los otros fondos considerados en el ajuste son \vgamma\ (incluyendo \znunugam) y
%% \topgamma\ MC, y los estimados mediante data-driven para electrones y jet fakes
%% descriptos en el capitulo \ref{}.
%% Los parametros nuisance $\apha$ parametrizan las incertezas sistematicas en la se\~nal
%% y el fondo por medio de una funcion gausiana. El tratamiento de las incertezas
%% sistematicas se describe en la seccion \ref{}.


%% By the simultaneous fit to the data in CRs or to the data in CRs and in a SR, these free and nuisance
%% parameters above can  be optimized with proper correlations and errors. In addition, by constraining the
%% parameters on background components mainly by the control regions with large statistics, the impacts of
%% these uncertainties in the SRs are reduced.


%% \begin{itemize}
%% \item {\bf Discovery fit:} the SR is added to the fit together with a generic non-SM signal component, which is neglected in the CRs. The background prediction achieved is this
%% way conservative since any signal contribution in the CRs is attributed to background and thus results in a possible overestimation of the background in the SRs. As shown above, however, this contribution is expected to be negligible.
%% \item {\bf Exclusion fit} both CRs and SRs are used in the fit. The signal contribution is taken into account as predicted by the tested SUSY model in all the regions.
%% \end{itemize}



\section{Validación del Modelo o Ajuste de solo-fondo} \label{sec:bkgonlyfit}


%--------------
% Bkg-only fit
%--------------


% Control Regions
\subsection{Resultados en las regiones de control}

Para validar el modelo propuesto y la metods utilizados para estimar los fondos,
se realiza un ajute simultaneo utilizando unicamente las CR. De este ajuste se
obtienen los factores de normalizacion de los fondos {\wgam} ($\mu_Q$), {\ttgam}
($\mu_T$) y {\gjet} ($\mu_Q$).


Los resultados del ajuste y el numero de eventos en los datos observados se
presnetan en la \cref{tab:fit_result_vrl,tab:fit_result_vrh} para las distintas
regiones de validacion.

% The distributions of several selection variables in these regions are shown in \App \ref{sec:ap_before_after}, before and after the fit. The fitted background predictions agree well with the data event counts within the uncertainties. The normalization factors determined for \wgamma\ ($\mu_{W}$), \ttbargam\ ($\mu_{T}$) and QCD prompt photon production ($\mu_{Q}$) are shown in \Tab \ref{tab:fit_bkgonly_mus}. %The $\mu_{W}$ value is consistent
%% %% %with unity within uncertinties. %As expected, $\mu_{Q}$ is way smaller, since the statistics for the smeared pseudo-data is artificially enhanced.
%% The systematic uncertainties in the two signal regions  after the background fit are summarized in \Fig \ref{fig:fit_unc_nuisance_SR}. No profiling of the systematics uncertainties is observed. %hopefully
%% The correlations of these systematics uncertainties and the normalizations are shown in \Fig \ref{fig:fit_corr_SR}. % What do we observe here?

\begin{table}[!htbp]
  \centering

  \caption{Factores de normalización utilizados para normalizar los fondos
    {\wgam} ($\mu_{W}$), {\ttgam} ($\mu_{T}$) y {\gjet} ($\mu_{Q}$), obtenidos
    del ajuste combinado de las CR. Las incertezas mostradas son las
    del ajuste.}
  \label{tab:fit_bkgonly_mus}

  \begin{tabularx}{0.8\textwidth}{CCCC} %cccc}
    \hline
    &       $\mu_{Q}$ &       $\mu_{W}$ &       $\mu_{T}$ \\
    \hline
    {\bf \SRL} & $0.94 \pm 0.44$ & $1.34 \pm 0.89$ & $1.40 \pm 0.77$ \\
    {\bf \SRH} & $1.22 \pm 0.58$ & $1.24 \pm 0.39$ & $0.54 \pm 0.37$ \\
    \hline
  \end{tabularx}

\end{table}


\begin{table}[!htbp]
  \centering

  \caption{Resultados del ajuste en las CR correspondientes a {\SRL}, con una luminosidad integrada total de 20.3 \ifb.
    El numero de eventos observado es comparado con el numero de eventos esperado de fondo, despues de la correspondiente
    normalizacion en las CR. En la parte inferior de la tabla se muestran tambien lo valores nominales del fondo antes de
    la correspondiente normalizacion. Las incertezas incluyen la incerteza estadistica y sistematica.}
  \label{tab:fit_result_cr2}

  %\resizebox{\textwidth}{!}
  {\small\input{tables/table_SR2_CR.tex}}

\end{table}


\begin{table}[!htbp]

  \caption{Resultados del ajuste en las CR correspondientes a {\SRH}, con una luminosidad integrada total de 20.3 \ifb.
    El numero de eventos observado es comparado con el numero de eventos esperado de fondo, despues de la correspondiente
    normalizacion en las CR. En la parte inferior de la tabla se muestran tambien lo valores nominales del fondo antes de
    la correspondiente normalizacion. Las incertezas incluyen la incerteza estadistica y sistematica.}
  \label{tab:fit_result_cr3}

  \input{tables/table_SR3_CR.tex}

\end{table}


\begin{figure}[!htbp]
  \centering

  \includegraphics[width=0.49\textwidth]{can_crql_met_et_afterFit}
  \includegraphics[width=0.49\textwidth]{can_crql_rt4_afterFit} \\

  \includegraphics[width=0.49\textwidth]{can_crwl_met_et_afterFit}
  \includegraphics[width=0.49\textwidth]{can_crwl_rt4_afterFit} \\

  \includegraphics[width=0.49\textwidth]{can_crtl_met_et_afterFit}
  \includegraphics[width=0.49\textwidth]{can_crtl_rt4_afterFit} \\

  \caption{Distribuciones observadas en las regiones de control {\CRQL}, {\CRWL}
    y {\CRTL}, después del ajuste combinado.}
  \label{fig:bkgfit_crl_after}

\end{figure}


\begin{figure}[!htbp]
  \centering

  \includegraphics[width=0.49\textwidth]{can_crqh_met_et_afterFit}
  \includegraphics[width=0.49\textwidth]{can_crqh_rt4_afterFit} \\

  \includegraphics[width=0.49\textwidth]{can_crwh_met_et_afterFit}
  \includegraphics[width=0.49\textwidth]{can_crwh_rt4_afterFit} \\

  \includegraphics[width=0.49\textwidth]{can_crth_met_et_afterFit}
  \includegraphics[width=0.49\textwidth]{can_crth_rt4_afterFit} \\

  \caption{Distribuciones observadas en las regiones de control {\CRQH}, {\CRWH}
    y {\CRTH},  después del ajuste combinado.}
  \label{fig:bkgfit_CRh_after}

\end{figure}





%% Validation Regions
\subsection{Resultados en las regiones de validación}

\begin{table}[!htbp]

  \caption{Resultados del ajuste en las VR correspondientes a {\SRL}, con una luminosidad integrada total de 20.3 \ifb.
    El numero de eventos observado es comparado con el numero de eventos esperado de fondo, despues de la correspondiente
    normalizacion en las CR. En la parte inferior de la tabla se muestran tambien lo valores nominales del fondo antes de
    la correspondiente normalizacion. Las incertezas incluyen la incerteza estadistica y sistematica.}
  \label{tab:fit_result_vr2}

  \input{tables/table_SR2_VR.tex}

\end{table}


\begin{table}[!htbp]

  \caption{Resultados del ajuste en las VR correspondientes a {\SRH}, con una luminosidad integrada total de 20.3 \ifb.
    El numero de eventos observado es comparado con el numero de eventos esperado de fondo, despues de la correspondiente
    normalizacion en las CR. En la parte inferior de la tabla se muestran tambien lo valores nominales del fondo antes de
    la correspondiente normalizacion. Las incertezas incluyen la incerteza estadistica y sistematica.}
  \label{tab:fit_result_vr3}

  \input{tables/table_SR3_VR.tex}

\end{table}


\begin{figure}[!htbp]
  \centering

  \includegraphics[width=0.49\textwidth]{figura}
  \includegraphics[width=0.49\textwidth]{figura} \\

  \includegraphics[width=0.49\textwidth]{figura}
  \includegraphics[width=0.49\textwidth]{figura} \\

  \includegraphics[width=0.49\textwidth]{figura}
  \includegraphics[width=0.49\textwidth]{figura} \\

  \caption{Distribuciones observadas en las regiones de validación, después del ajuste combinado.}
  \label{fig:bkgfit_crl_after}

\end{figure}


\begin{figure}[!htbp]
  \centering

  \includegraphics[width=0.49\textwidth]{figura}
  \includegraphics[width=0.49\textwidth]{figura} \\

  \includegraphics[width=0.49\textwidth]{figura}
  \includegraphics[width=0.49\textwidth]{figura} \\

  \includegraphics[width=0.49\textwidth]{figura}
  \includegraphics[width=0.49\textwidth]{figura} \\

  \caption{Distribuciones observadas en las regiones de validación,  después del ajuste combinado.}
  \label{fig:bkgfit_CRh_after}

\end{figure}

\clearpage


\subsection{Parámetros del ajuste}

\begin{figure}[!htbp]
  \centering

  \includegraphics[width=\textwidth]{syst_pull_afterFit_SR2} \\
  \includegraphics[width=\textwidth]{syst_pull_afterFit_SR3}

  \caption{Resumen de los parametros despues del ajuste para {\SRL} (arriba) y {\SRH} (abajo).}
  \label{fig:fit_unc_nuisance_SR}

\end{figure}

\begin{figure}[!htbp]
  \centering

  \includegraphics[width=\textwidth]{figures/corrMatrix_SR2} \\
  \includegraphics[width=\textwidth]{figures/corrMatrix_SR3} \\

  \caption{Matriz de correlacion de los parametros del ajuste, correspondientes a {\SRL} (arriba) y {\SRH} (abajo).}
  \label{fig:fit_corr_SR}

\end{figure}


%% The background predictions from the fit in SR2 and SR3 are presented in \Tab \ref{tab:fit_result_sr}. Both the nominal estimates
%% before the fit and the predicted yields from the background fits are shown.
%% No significant data deviation from the predicted background yields is observed in any of the SRs. The results after fit are
%% also shown for all regions in \Fig \ref{fig:fit_region_composition}.

%% The breakdown of systematic
%% uncertainties on the total background prediction are presented in \Tab \ref{tab:fit_result_sr2_syst} and \ref{tab:fit_result_sr3_syst}.
%% Since the individual uncertainties can be correlated and the negative correlations between normalizations (representing the
%% impact of the CR statistics on the total uncertainty) and systematic uncertainties are observed as expected, they
%% do not necessarily add up by quadrature to the total background uncertainty. The square
%% root of the number of the expected events are calculated in order to illustrate the relative impact of the
%% SR Poisson statistical uncertainties with respect to the total background uncertainties. The
%%  relative systematic uncertainties with respect to the background predictions are large, around 54\% (58\%) for SR2 (SR3). In SR2, the main uncertainties after the fit come from the normalisation and the theoretical uncertainties of the \ttbargam\ background, followed by the jet energy scale and the MC statistics in the SR. In SR3, the uncertainties are dominated by the theoretical uncertainties on \zgam and \gjet backgrounds, the MC statistics and the normalisation uncertainties of the \wgamma\ production.

%% Similar breakdown tables but separated on each background prediction in each SR are presented in \Tab \ref{tab:fit_result_sr2_syst_perbkg}-\ref{tab:fit_result_sr3_syst_perbkg}.


\subsection{Regiones de Señal}


\begin{table}[!htbp]
  \centering

  \caption{Resultados del ajuste en las SR, con una luminosidad integrada total de 20.3 \ifb.
    El numero de eventos observado es comparado con el numero de eventos esperado de fondo, despues de la correspondiente
    normalizacion en las CR. En la parte inferior de la tabla se muestran tambien lo valores nominales del fondo antes de
    la correspondiente normalizacion. Las incertezas incluyen la incerteza estadistica y sistematica.}
  \label{tab:fit_result_sr}

  \input{tables/table_SR.tex}

\end{table}


\begin{figure}[!htbp]
  \centering

  \includegraphics[width=0.6\textwidth]{region_composition_srl}
  \includegraphics[width=0.6\textwidth]{region_composition_srh}

  \caption{Background fit results in all SR/CR/VR regions, for SR2 (top) and SR3 (bottom) analyses.}
  \label{fig:fit_region_composition}

\end{figure}


\begin{figure}[!htbp]

  \centering

  \includegraphics[width=0.49\textwidth]{figura}
  \includegraphics[width=0.49\textwidth]{figura}

  \caption{Comparación datos/MC de algunas variables discriminatorias
    para {\SRL}, despues del ajuste final.}
  \label{fig:unblind_dist_2}

\end{figure}

\begin{figure}[!htbp]
  \centering

  \includegraphics[width=0.49\textwidth]{figura}
  \includegraphics[width=0.49\textwidth]{figura}

  \caption{Comparación datos/MC de algunas variables discriminatorias
    para {\SRH}, despues del ajuste final.}
  \label{fig:unblind_dist_3}

\end{figure}


%% %The additional checks on the fit results are performed, following the recommendation on SUSYChekList twiki \cite{}.
%% % The results are shown in \App \ref{sec:ap:fitchecks} and confirmed all expected.



%% %% \begin{table}
%% %%   \caption{Breakdown of the dominant systematic uncertainties on total background estimates in SR2.
%% %%     Note that the individual uncertainties can be correlated, and do not necessarily add up quadratically to the total background uncertainty.
%% %%     The percentages show the size of the uncertainty relative to the total expected background.}
%% %%   \input{tables/table_SR2_syst.tex}
%% %%   \label{tab:fit_result_sr2_syst}
%% %% \end{table}

%% %% \begin{table}
%% %%   \caption{Breakdown of the dominant systematic uncertainties on total background estimates in SR3.
%% %%     Note that the individual uncertainties can be correlated, and do not necessarily add up quadratically to the total background uncertainty.
%% %%     The percentages show the size of the uncertainty relative to the total expected background.}
%% %%   \input{tables/table_SR3_syst.tex}
%% %%   \label{tab:fit_result_sr3_syst}
%% %% \end{table}

%% %% \begin{table}
%% %%   \caption{Breakdown of the dominant systematic uncertainties on each of the main backgrounds estimates in SR2.
%% %%     Note that the individual uncertainties can be correlated, and do not necessarily add up quadratically to the total background uncertainty.
%% %%     The percentages show the size of the uncertainty relative to the total expected background.}
%% %%   \input{tables/table_SR2_syst_bkgs.tex}
%% %%   \label{tab:fit_result_sr2_syst_perbkg}
%% %% \end{table}

%% %% \begin{table}
%% %%   \caption{Breakdown of the dominant systematic uncertainties on each of the main backgrounds estimates in SR3.
%% %%     Note that the individual uncertainties can be correlated, and do not necessarily add up quadratically to the total background uncertainty.
%% %%     The percentages show the size of the uncertainty relative to the total expected background.}
%% %%   \input{tables/table_SR3_syst_bkgs.tex}
%% %%   \label{tab:fit_result_sr3_syst_perbkg}
%% %% \end{table}

%% \clearpage


%% %% \begin{table}[ht!]
%% %%   \caption{Background fit results for SR2 validation regions with an integrated luminosity of 20.3 \ifb. Nominal MC and data-driven expectations are given for comparison. The uncertainties shown are statistical + systematic.}
%% %%   \input{tables/table_SR2_VR_ext.tex}
%% %%   \label{tab:fit_result_vr_ext2}
%% %% \end{table}

%% %% \begin{table}[ht!]
%% %%   \caption{Background fit results for SR3 validation regions with an integrated luminosity of 20.3 \ifb. Nominal MC and data-driven expectations are given for comparison. The uncertainties shown are statistical + systematic.}
%% %%   \input{tables/table_SR3_VR_ext.tex}
%% %%   \label{tab:fit_result_vr_ext3}
%% %% \end{table}

%% \subsubsection{Check on jet energy scale uncertainty} \label{sec:JEScheck}

%% For the jet energy scale (JES) uncertainty, only one nuisance parameter is used. However, an alternative background-only fit was performed using the reduced list of JES components to check potential correlation effects on the final results.
%% The list of components provided by the JetEtmiss group is:


%% The breakdown of the dominant systematics uncertainties on the total backgrounds in this new configuration can be
%% seen in \Tab\ \ref{tab:fulljes_check_sr2} and \ref{tab:fulljes_check_sr3}, for SR2 and SR3 respectively. The difference in the final
%% systematic uncertainty, with respect to \Tab\ \ref{tab:fit_result_sr2_syst} and \ref{tab:fit_result_sr2_syst}, is in both cases within $1\%$.
%% A single nuisance parameter was therefore used for the nominal analysis.


%% %% \begin{table}[ht!]
%% %%   \caption{Breakdown of the dominant systematic uncertainties on total background estimates in SR2, using the full set of \texttt{JES} parameters.
%% %%     Note that the individual uncertainties can be correlated, and do not necessarily add up quadratically to the total background uncertainty.
%% %%     The percentages show the size of the uncertainty relative to the total expected background.}
%% %%   \input{tables/table_SR2_syst_sr_fulljes.tex}
%% %%   \label{tab:fulljes_check_sr2}
%% %% \end{table}

%% %% \begin{table}[ht!]
%% %%   \caption{Breakdown of the dominant systematic uncertainties on total background estimates in SR3, using the full set of \texttt{JES} parameters.
%% %%     Note that the individual uncertainties can be correlated, and do not necessarily add up quadratically to the total background uncertainty.
%% %%     The percentages show the size of the uncertainty relative to the total expected background.}
%% %%   \input{tables/table_SR3_syst_sr_fulljes.tex}
%% %%   \label{tab:fulljes_check_sr3}
%% %% \end{table}




%% \clearpage

%% \section{Distribuciones en las regiones de se\~nal}

%% The unblinded distributions of some of the discriminating variables in SR2 and SR3 are shown in \Fig \ref{fig:unblind_dist_2} and \ref{fig:unblind_dist_3},
%% respectively. The statistical prediction of the background is described in the next sections.
%% Details and event displays for the 4 events surviving the SR selections are included in \App \ref{sec:ap:atlantis}.


%% \clearpage


%-------------------------
% SUSY Exclusion limit
%-------------------------
\section{Limites de exclusión sobre el modelo de SUSY}

Debido a que no se observa un exceso significativo por arriba del fondo esperado
del {\SM} en las regiones de señal, se obtuvieron los limites de exclusión en el
modelo de SUSY considerado. Los limites superiores son obtenidos utilizando el
método del {\cls} descripto en la \cref{sec:limits}. Estos limites son
calculados para la seccion eficaz nominal del modelo y tambien con una sección
eficaz de $\pm 1 \sigma$ en la incerteza teórica, siguiendo la recomendación de
ATLAS y el acuerdo para presentar los resultados en el LHC.

La \cref{fig:limit_SR_toys} muestra los limites de exclusión
esperados y observados para las dos SR de forma separada. Los limites
fueron obtenidos utilizando $\sim 20000$ pseudoexperimentos. La linea
punteada azul muestra el limite esperado a 95\% CL, con las bandas
amarillas indicando la desviación a 1$\sigma$ que tiene en cuenta
las incertezas teóricas y experimentales.
Los limites observados están indicados por las lineas en rojo oscuro,
donde la linea solida representa el limite nominal y las lineas
punteadas representan el limite para las variaciones de la sección
eficaz de la señal debido a la incerteza teórica en la misma.


Por diseño, la {\SRL} es importante en la zona de gluinos pesado,
mientras que la {\SRH} cubre la región mas comprimida del espacio
de fases. En la \cref{fig:limit_SR_combined_toys}, se muestra
el limite combinando ambas regiones de señal, eligiendo en cada punto
la que provee el mejor limite.


\begin{figure}[!htbp]
  \centering

  \includegraphics[width=0.49\textwidth]{figura} %limitPlot_SR2toys}
  \includegraphics[width=0.49\textwidth]{figura} %limitPlot_SR3_toys}

  \caption{Limites de exclusión a 95\% CL para {\SRL}  (izquierda) y {\SRH} (derecha).}
  \label{fig:limit_SR_toys}
\end{figure}


\begin{figure}[!htbp]
  \centering

  \includegraphics[width=0.8\textwidth]{figura}

  \caption{Limites de exclusión para {\SRL} y {\SRH} combinados.
    Los limites son obtenidos usando la región de señal con mejor sensibilidad
    en cada punto.}
  \label{fig:limit_SR_combined_toys}

\end{figure}

%% \begin{figure}[!htbp]
%%   \centering

%%   \includegraphics[width=0.8\textwidth]{figures/limitPlot_SR2_SR3_BestSR_toys}

%%   \caption{Limites de exclusión para {\SRL} y {\SRH} combinados.
%%     Los limites son obtenidos usando la region de senal con mejor sensibilidad
%%     en cada punto, la cual es indicada con el numero en gris.}
%%   \label{fig:limit_SR_combined_best}

%% \end{figure}



A search for the experimental signature of an isolated high-\pt\ photon, jets
and high missing transverse momentum has been performed using 20.3 \ifb\ of pp
collision data at 8 \tev\ produced by the LHC and collected by the ATLAS
detector. No excess has been observed over the Standard Model predictions.
Limits are set in a GGM scenario with a NLSP neutralino that is a mixture of
higgsino and bino. For a neutralino mass range $m_{\neut}<{840}\gev$, gluino
production is excluded at 95\% CL at a minimum (maximum) mass of 1190 (1320)
\gev. For a gluino mass below 1TeV, a bino-higgsino NLSP is excluded at 95\% CL
between 150 \gev\ and $m_{\gluino}-m_{\neut}>50\gev$.



%-------------------------
% Model independent limit
%-------------------------
\section{Limites a eventos de Nueva Física} \label{sec:model_independent}

%% Mas alla de los limites en el modelo de senal de SUSY estudiado en esta Tesis,
%% el analisis que busca nueva fisica puede imponer limites en el numero de eventos
%% de nueva fisica por sobre el fondo esperado en cada SR. De esta forma, para cada
%% modelo de senal de interes, cualquiera puede estimar el numero de eventos de
%% senal predichos en una SR y chequear si el modelo es excluido o no de acuerdo a
%% los datos observados en el analisis.


Ademas de los limites en el modelo de SUSY considerado en esta Tesis,
resulta util proveer el limite superior en el numero de eventos de nueva fisica,
independientes del modelo. Este limite, provisto para cada region de senal,
hace posible aplicarlos a cualquier otro model, simplemente contando
el numero de eventos de senal que se espera que contribuyan a la regon en cuestion
y verdificar que el numero no exceda el limite superior.

%Este limite se obtiene del ajuste combinado
Con este objetivo se realiza un ajuste combinado teniendo en cuenta las regiones
de control y cada una de las regiones de senal. Se supone que la contaminacion
de senal en las CR es nula, pero no se realiza ninguna otra suposicion acerca
del modelo de senal. El numero de eventos de senal en la SR se agrega como un
el parametro de interes del ajuste. Se realiza un ``scan'' sobre este parametro,
y el limite superior en el numero de eventos de nueva fisica se encuentra para el
valor en el cual el {\cls} cae debajo del 5\%. El limite superior en el numero de
eventos puede transformarse en el limite superior en la seccion eficaz visible diviendolo
por la luminosidad utilizada, 20.3 \ifb.

Estos resultados se detallan en \cref{tab:upperlimits} para las dos SR. Debido a que el
numero esperado de eventos de fondo es bajo, no es posible utilizar la aproximacion asintotica,
y se generaron pseudo-experimentos MC.

%% Los resultados se obtienen de forma independiente utilizando 3000 pseudo-experimentos, y la
%% aproximacion asintotica.
%% The discovery fit is performed to the CR data and SR data by maximizing the likelihood in \Eq \ref{eq:likelihood} but
%%  with the signal component only in the SR. The discovery test is done by the background-only hypothesis
%% and quantified using pseudo-experiments by p-value $p_b = P(q \leq q_{obs}|b)$ where $q$ is the test statistics.
%% In the absence of a statistically significant excess, limits are set on contributions to the SRs from new
%% physics. Calculated p-values and model independent limits on the number of new physics contribution
%% by each SR are listed in \Tab \ref{tab:upperlimits}. Two sets of results are obtained with 3000 toy experiments and with
%% asymptotic approximation, respectively. Although the observed limits on the visible cross sections are not much different,
%% the asymptotic approximation is supposed to hold only for high statistics scenarios. Thus, we decided to keep all the
%% limit results with toys. The results with asymptotic approximation are also shown in \App \ref{sec:ap:asym_limits} for comparison.

%% %Since the limit numbers between the two sets are not much different, we expect that the model-dependent limits will not change.

\begin{table}[!htbp]
  \centering

  \caption{Limite independiente del modelo de señal a 95\% de CL en la
    seccion eficaz visible observada ($\langle\epsilon{\rm \sigma}\rangle_{\rm obs}$),
    y el límite en el número de eventos de nueva física observado
    $S_\text{obs}$ y esperado $S_\text{exp}$ para las dos SR.
    La última linea ($p_0$) indica el {\pvalue} de la hipotesis de solo-fondo.}
    %% Signal-model-independent 95\% CL upper limits on the observed ($\langle\epsilon{\rm \sigma}\rangle_{\rm obs}^{95}$) %and expected ($\langle\epsilon{\rm \sigma}\rangle_{\rm exp}^{95}$)
    %% visible cross-section, and on the observed ($S_{\rm obs}^{95}$) and expected ($S_{\rm exp}^{95}$) number of beyond-SM
    %% events in the various SRs. The fourth line indicates the CLB value, i.e. the observed confidence level for
    %% the background-only hypothesis. The last line (p0) indicates the probability of the observation being
    %% consistent with the estimated background. Two sets of results are obtained with 3000 toy experiments
    %% (top-half) and with asymptotic approximation (lower-half).}
  \label{tab:upperlimits}

  \begin{tabularx}{\textwidth}{LRR}
    \noalign{\smallskip}\hline\noalign{\smallskip}
            {\bf Region de señal}                   & {\bf \SRL} & {\bf \SRH} \\
            \noalign{\smallskip}\hline\noalign{\smallskip}
            Eventos SM esperados   &   $0.78\pm 0.42$   &  $0.84 \pm 0.49$  \\
            Eventos observados   &   2  &  2  \\
            \noalign{\smallskip}\hline\noalign{\smallskip}
            $\avg{\epsilon{\sigma}}_\text{obs} \, [\mathrm{fb}]$  & 0.28  & 0.28 \\
            $S_\text{obs}$  & 5.7 & 5.7 \\
            $S_\text{exp}$ & ${4.0}^{+1.6}_{-0.1}$ & ${4.1}^{+1.7}_{-0.1}$ \\
            %% {\clb} & 0.86 & 0.83 \\
            $p_0$  & 0.16 &  0.19 \\
            \noalign{\smallskip}\hline\noalign{\smallskip}
  \end{tabularx}

\end{table}
