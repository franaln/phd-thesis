\section{Incertezas sistemáticas}
\label{sec:sistematicos}

Uno de los aspectos más importantes de las investigaciones con datos recolectados
por ATLAS es la determinación de todas las incertezas sistemáticas que afectan a
los resultados. Las incertezas sistemáticas están asociadas a las predicciones
de todas las componentes del fondo y a los valores esperados de señal. Estas
incertezas pueden ser clasificadas en dos tipos: de origen experimental o
teórico, y pueden tener un impacto importante en el número de eventos esperado
en las CR y SR como así también en los parámetros $\mu_p$ utilizados en la
extrapolación del fondo. En esta sección se presentan todas las fuentes
posibles de incertezas sistemáticas incluidas en el presente análisis.

Cabe mencionarse que, en la determinación de los resultados finales utilizando
el ajuste con el PLR, las predicciones de los fondos se permiten variar dentro
del tamaño de las incertezas sistemáticas y por lo tanto son restringidas por
los datos.


\subsection{Incertezas Experimentales}\label{sec:expsyst}

A continuación se describen los procedimientos realizados para evaluar las
incertezas experimentales comunes para todos los procesos tanto de señal como de
fondo considerados en el análisis.


\subsubsection{Luminosidad}

Existen dos detectores dise\~nados específicamente para la medida de la
luminosidad en ATLAS: LUCID y BCM \cite{lumi2011}. La luminosidad es calculada
midiendo la tasa de interacción durante la toma de datos mediante estos
dispositivos dispuestos a cada lado de ATLAS, a pequeños ángulos respecto de la
dirección del haz. La calibración absoluta se ha determinado a partir de los
parámetros del acelerador mediante una serie de estudios dedicados, conocidos
como \emph{Van der Meer scans} \cite{vanderMeer:296752}, realizados durante el
mes de Noviembre del a\~no 2012. La incerteza sistemática relativa en la medida
de la luminosidad integrada estimada es $\pm$2.8\% \cite{lumi2012}.

Esta incerteza debe ser propagada a todas las muestras simuladas utilizadas
para la estimación de fondo y señal, las que deben ser normalizadas a la
luminosidad de datos analizada.


\subsubsection{Trigger}

La estimación de la eficiencia del trigger de fotones utilizado y su incerteza se presentó
en detalle en la \cref{sec:trigger}. La incerteza sistemática resulta $< 1\%$. %%, siendo esta igual a $100^{+0}_{-1.41}\stat \, {}^{+0}_{-0.7}\syst \%$.



\subsubsection{Pile-up}

En la \cref{sec:sim_atlas} se presentó la necesidad de realizar un repesado
evento a evento en las muestras simuladas para modelar las condiciones de
\emph{pile-up} de los datos, ya que las muestras MC se simulan con un perfil de
\emph{pile-up} fijo. Para calcular la incerteza asociada a este procedimiento se
varía el valor medio del número de interacciones por cruce de \emph{bunches} en
$\pm 10\%$.

%% %% Siguiendo las recomendaciones de ATLAS, se utiliza
%% %% un factor de $1.09 \pm 0.04$ aplicado al valor medio del número de
%% %% interacciones por cruce de haz ($\avg{\mu}$) en el repesado debido al pile-up en las muestras simuladas
%% %% para describir mejor la multiplicidad de vértices del minimum bias en los datos.
%% %% La incerteza asociada es calculada variando este factor de escala dentro
%% %% de su incerteza.


\subsubsection{Escala de energía y resolución de energía de fotones y leptones}

Como se menciona en la \cref{sec:fotones}, es necesario aplicar factores de
escala a las muestras MC para corregir la eficiencia de  identificación
de electrones y fotones. Estos factores son provistos por los grupos
de \textsc{Egamma} de ATLAS, y la incerteza en estos factores, la cual es menor al 3\%, es propagada
a través del análisis, siguiendo \cite{PhoEffTwiki,EleEffTwiki,EGScaleTwiki,MCPTwiki}.

La estimación de la incerteza en la escala y resolución de energía de los fotones y
electrones \cite{EGScaleTwiki} también es considerada y posee el mismo orden de magnitud.

Como el requerimiento de aislamiento de fotones aplicado por la herramienta
oficial para derivar los factores de corrección es distinto al utilizado en este
análisis, la eficiencia de identificación fue evaluada nuevamente para los
distintos criterios de aislamiento para fotones con $\pt>125\gev$ en muestras MC
de señal y fondo. Se encontró que el efecto es $<1\%$ y por lo tanto es
despreciado.




\subsubsection{Escala y resolución de energía de los jets}

La calibración de la escala de energía de los jets a la escala hadrónica correspondiente,
tiene una incerteza asociada debida a distintas fuentes, entre las que se encuentran
la respuesta del calorímetro y el \emph{pile-up}.
La sobre-estimación o subestimación de la energía puede resultar
en que se seleccione un conjunto distinto de jets para el análisis, debido a que la corrección
no es la misma para todos los jets y depende de sus propiedades. Para evaluar el efecto
de esta incerteza en los eventos seleccionados, la corrección es modificada
en $\pm 1\sigma$, utilizando una herramienta provista por el grupo de \textsc{Jet/Etmiss}
de ATLAS \cite{JesTwiki}.

El impacto de la incerteza asociada a la resolución de energía de los jets es
considerado comparando el número de eventos obtenido utilizando los jets
nominales seleccionados para el análisis con el obtenido modificando la energía
de los jets de acuerdo a una distribución gaussiana, de media 1 y ancho igual a
la incerteza.

%% A fin de tener en cuenta una posible subestación de la resolución de energía de los jets
%% en la simulación MC, se realiza un smearing del {\pt} de los jets en función de su \pt y su $\eta$.
%% Cada jet es smeareado de acuerdo a una distribución Gausiana, de media unidad y un ancho
%% dado por una función de resolución dependiente de {\pt} y $\eta$. Para evaluar el impacto
%% de la incerteza sistemática debida a la resolución de energía, el número de eventos obtenido utilizando
%% los jets nominales es comparado con los resultados obtenidos de los jets smeareados.
%% Este procedimiento también se realiza utilizando una herramienta provista por el grupo de Jet/Etmiss
%%de ATLAS.

%% Los jets producidos depositan su energia en los acalorimetros,
%% pero esta energia medida no se corresponde con la energia real del quark original que
%% produjo el jet. La calibracion JES corrige la energia medida por los calorimetros. La
%% incerteza de esta correccion aparece debido a distintas fuentes incluyendo la respuesta
%% del calorimtro y el pile-up.


%% Esta herramienta provee la incerteza relativa en la escala de energia de los jets,
%% que es la suma en cuadratura de tres componentes dependientes de: el {\pt} y $\eta$
%% del jet, el $\Delta R$ al jet más cercano, y la composicion media quark-gluon de la muestra.

%% There is an uncertainty associated with the jet energy scale (JES) calibration of
%% jets to the hadronic scale [114]. Jets produced at ATLAS provide energy deposits
%% in the calorimeters, and the energy measured will not correspond to the true energy
%% of the original quark producing the jet. The JES calibration offers a correction to
%% the calorimeter-measured energy, scaling it to better represent the energy of the
%% particles in the jet. The uncertainty on this correction arises from many sources
%% including calorimeter response and pile-up of events at the interaction point. An
%% over- or under-estimate of the JES can result in a different set of jets being selected
%% for analysis since the correction is not the same for all jets and depends on their
%% properties. To evaluate the effect of the JES uncertainty on the data selections
%% the corrections associated with each JES component are fluctuated up and down
%% resulting in separate distributions from the ±1σ variations.

%% A total of seven JES
%% parameters are used for these variations, with these being associated to the flavour
%% composition and response of the simulated sample, the η-calibration modelling,
%% the b-jet energy scale, the topology of pile-up interactions and those from in-situ
%% jet-balance measurements.



%% \subsubsection{Resolucion de la energia de jets ($\alpha_\mathrm{JER}$)}


%%  %% An extra \pt smearing is added to the jets based on their \pt and $\eta$ to
%%  %% account for a possible underestimate of the jet energy resolution in the MC
%%  %% simulation. This is done using the \texttt{JetResolution-02-00-03} package. The
%%  %% so found yield differences are then applied as a $\pm$ uncertainty.


\subsubsection{Eficiencia de identificación de {\bjets}}

La identificación de {\bjets} solo es utilizada en la selección de las regiones
de control CRW y CRT. En estas regiones, se evaluó la incerteza
variando los factores de escala aplicados a cada jet en las simulaciones, que
dependen de $\eta$ y {\pt}, dentro del rango que refleja la incerteza sistemática
en la eficiencia medida de identificación de {\bjets}. Esta eficiencia fue
medida en datos y para el algoritmo utilizado MV1 (y la configuración que se
corresponde a una eficiencia de 70\% para una muestra $t\bar{t}$), la
incerteza varía entre 5\% y 19\%, aumentando con el {\pt} del jet
\cite{btagging}.



\subsubsection{Energía faltante}

Al modificar la escala de energía de los objetos o la resolución de energía de los jets, el
efecto se propaga al término correspondiente en el cálculo de {\met}.
Adicionalmente, también se evalúa la incerteza en el término \emph{soft} de {\met}, variando
la escala y resolución de energía de los \emph{clusters}, utilizando una herramienta provista por
el grupo \textsc{Jet/Etmiss} de ATLAS.


\subsubsection{Estimación de los fondos de electrones y jets mal identificados}

La estimación de los fondos provenientes de la mala identificación de electrones
o jets a partir de los datos ha sido discutida en detalle en las \cref{sec:efakes,sec:jfakes}.

En el caso de electrones dicha estimación se obtiene multiplicando el número de eventos
de una muestra de electrones en cada región de señal por el factor {\feg} relacionado
con la probabilidad de que un electrón sea identificado como un fotón.
Este factor es calculado a partir de una muestra de datos {\Zee} y varía entre 1.4\% y 3.3\%
dependiendo del {\abseta} del electrón, con una incerteza sistemática del 25\%-35\%.
El efecto de la incerteza sistemática en el factor {\feg} se propaga en
el análisis comparando el número de eventos obtenido en las SR, modificando este factor
en $\pm 1\sigma$.

El fondo de jets identificados como fotones es estimado teniendo en cuenta las
diferencias en el perfil de la energía de aislamiento entre fotones reales y falsos. A partir
de estas diferencias se calcula la fracción de fotones falsos $f_{j\to\gamma} = 0.0857 \pm 0.0002 \stat \pm 0.04 \;\syst$.
La incerteza sistemática en esta fracción tiene en cuenta las posibles incertezas
del método utilizado, y se propaga al número de eventos en las SR, variando
este factor en $\pm 1\sigma$.



\subsection{Incertezas Teóricas}\label{sec:theosyst}


\subsubsection{Señal de SUSY}\label{sec:syst_signal}

La sección eficaz de producción de pares de gluinos fue calculada utilizando
el programa {\nllfast}, al igual que las incertezas asociadas debido a la elección
de la escala de renormalización y factorización, y a la elección de la PDF.
Para ello se siguen las recomendaciones PDF4LHC \cite{Botje:2011sn} como
se describe en detalle en la \cref{sec:xs_calc}.

Los valores centrales y sus incertezas para las diferentes masas de gluinos consideradas se resumen en
la \cref{tab:signal_xs_theo_unc}. Las incertezas totales varían entre 22.5\% para $m_{\gluino} = 800\GeV$
y 36.8\% para $m_{\gluino} = 1.3 \tev$.

\begin{sidewaystable}[!htbp]
  \centering
  \caption{La sección eficaz total NLO+NLL con sus incertezas en función de la masa de los gluinos. Se detallan
    las diferentes contribuciones a la incerteza total asociadas a la elección de la PDF,  la escala de renormalización
    y factorización, y el valor de $\alpha_s$. Los detalles del cálculo se encuentran en la \cref{sec:xs_calc}.}
  \label{tab:signal_xs_theo_unc}

  \begin{tabular}{l|rrrrrrrrrrr}
    \hline
    $M_3$ [\gev]                             & 800                & 850               & 900                & 950               & 1000             & 1050             & 1100             & 1150             & 1200             & 1250        & 1300 \\
    $m_{\tilde{g}}$ [\gev]                   & 885              & 932             & 978              & 1023            & 1068           & 1113           & 1157           & 1202           & 1246           & 1290      & 1333 \\
    \hline
    Sección eficaz ($\sigma$) [pb]              &           0.0690    & 0.0449            & 0.0297             & 0.01984           & 0.01341          & 0.00910          & 0.00628          & 0.00432          & 0.00301          & 0.00210     & 0.00149 \\
    Incerteza     [$\%$]                     &             22.5    & 23.8              & 25.2               & 26.5         & 27.7             & 29.0             & 30.4             & 32.0             & 33.7             & 35.2        & 36.8 \\[5pt]
    Incerteza PDF {\cteq} [$\%$]             & \unc{22.5}{15.3}    & \unc{23.5}{15.9}   & \unc{24.6}{16.5}    & \unc{25.8}{17.1}   & \unc{26.9}{17.7}  & \unc{28.2}{18.4}  & \unc{29.4}{19.0}  & \unc{30.7}{19.7}  & \unc{32.0}{20.4}  & \unc{33.4}{21.1}    & \unc{34.8}{21.8} \\[5pt]
    Incerteza PDF {\mstw} [$\%$]             &   \unc{9.5}{9.1}    & \unc{9.9}{9.4}     & \unc{10.2}{9.7}     & \unc{5.6}{5.0}     & \unc{10.9}{10.3}  & \unc{11.3}{10.6}  & \unc{11.8}{10.9}  & \unc{12.2}{11.2}  & \unc{12.6}{11.6}  & \unc{13.1}{11.9}    & \unc{13.6}{12.3} \\[5pt]
    Incerteza Escala {\cteq} [$\%$]          &   \unc{9.8}{9.8}    & \unc{9.9}{9.9}     & \unc{9.9}{10.0}     & \unc{5.0}{5.0}     & \unc{10.2}{10.1}  & \unc{10.3}{10.2}  & \unc{10.4}{10.3}  & \unc{10.5}{10.4}  & \unc{10.5}{10.5}  & \unc{10.7}{10.6}    & \unc{10.8}{10.7} \\[5pt]
    Incerteza Escala {\mstw} [$\%$]          & \unc{10.3}{10.1}    & \unc{10.4}{10.2}   & \unc{10.5}{10.2}    & \unc{5.6}{5.3}     & \unc{10.7}{10.4}  & \unc{10.8}{10.4}  & \unc{10.9}{10.5}  & \unc{11.0}{10.6}  & \unc{11.2}{10.7}  & \unc{11.3}{10.8}    & \unc{11.4}{11.0} \\[5pt]
    Incerteza $\alpha_{s}$ {\cteq} [$\%$]    &   \unc{6.9}{4.7}    & \unc{7.0}{4.8}     & \unc{7.2}{4.9}      & \unc{7.4}{5.0}     & \unc{7.5}{5.1}    & \unc{7.7}{5.2}    & \unc{7.9}{5.4}    & \unc{8.1}{5.5}    & \unc{8.3}{5.6}    & \unc{8.5}{5.7}      & \unc{8.7}{5.8} \\[5pt]
    Incerteza $\alpha_{s}$ {\mstw} [$\%$]    &   \unc{3.2}{3.1}    & \unc{3.3}{3.1}     & \unc{3.3}{3.1}      & \unc{3.4}{3.1}     & \unc{3.4}{3.1}    & \unc{3.5}{3.0}    & \unc{3.5}{3.0}    & \unc{3.6}{3.0}    & \unc{3.6}{3.0}    & \unc{3.6}{2.9}      & \unc{3.7}{2.9} \\[5pt]
    Sección eficaz NLL (\cteq) [pb]          &           0.0674       & 0.044             & 0.0292             & 0.0195            & 0.0132           & 0.00896          & 0.0062           & 0.00428          & 0.00299          & 0.00209     & 0.00148 \\
    Sección eficaz NLO (\cteq) [pb]          &           0.0618       & 0.0402            & 0.0265             & 0.0177            & 0.0119           & 0.00811          & 0.00559          & 0.00384          & 0.00266          & 0.00186     & 0.00131 \\
    Sección eficaz LO  (\cteq) [pb]          &           0.0289       & 0.0186            & 0.0121             & 0.00801           & 0.00534          & 0.00358          & 0.00244          & 0.00165          & 0.00113          & 0.000776   & 0.000539 \\
    Factor $k$ ($\sigma/\sigma_{\text{LO}}$)          & 2.4                & 2.41              & 2.46               & 2.48              & 2.51             & 2.54             & 2.57             & 2.62             & 2.67             & 2.71      & 2.76 \\
    \hline
  \end{tabular}

\end{sidewaystable}


\subsubsection{Producción de \ttgam}\label{sec:syst_ttbargamma}

La incerteza teórica asociada a la elección de la configuración
utilizada en la simulación MC es
tenida en cuenta considerando variaciones en la generación de eventos
con respecto a la configuración nominal. Las variaciones consideradas
incluyen las escalas de factorización y renormalización ($0.5\times, 2\times$),
$\alpha_{s}$, y el modelo ISR/FSR. %% (ver \cref{tab:mc_ttbar_samples}).

La \cref{tab:syst_ttbargam_truth} resume los cambios relativos en el
número de eventos esperado en cada región de señal para cada una de las
variaciones con respecto a la muestra nominal como resultado de un estudio realizado a nivel generador.
La diferencia principal entre todas las variaciones y la configuración nominal
aparece mayormente en los cortes en {\HT} y {\rt}, como es esperado ya que las
propiedades de la actividad hadrónica del
evento son las más sensibles a las variaciones de escala.

\begin{table}[ht!]
  \centering
  \caption{Resumen de las incertezas teóricas de la producción de {\ttgam}
    obtenidas en cada región de señal.}
  \label{tab:syst_ttbargam_truth}

  \begin{tabular}{z{6cm}x{2cm}x{2cm}}
    \hline
    & {\SRL} & {\SRH} \\
    \hline
    Escala de factorización/renormalización &  56 \%  & 20 \% \\
    ISR/FSR                                 &   4 \%  & 20 \% \\
    $\alpha_{s}$                            &   4 \%  &  0 \% \\
    %      total cross section                  &  20  &   20 \\
    \hline
    Total				&   56 \%    &   28 \% \\
    \hline
  \end{tabular}

\end{table}

La incerteza de cada una de las variaciones es simetrizada y luego se suman
en cuadratura para obtener la incerteza teórica
total, que resulta en 56 \% y 28 \% para {\SRL} y {\SRH}, respectivamente.


\subsubsection{Producción de {\wgam}} %% ($\alpha_{\tgam}$)}\label{sec:syst_wgamma}

De manera similar a lo realizado para estimar las incertezas teóricas de la
producción  de {\ttgam},
las incertezas en el número estimado de eventos de {\wgam} fueron determinadas
realizando el análisis a nivel generador para las muestras simuladas con variaciones
de la configuración nominal. Para las incertezas
debidas a la elección de escala de factorización/renormalización, se simularon muestras
modificando la escala al doble y la mitad. También se consideró la incerteza debida
a la escala de asociación entre  ME y PS, cuyo valor nominal es de 20 \gev, y para el caso de las variaciones se
modificó a 15 y 30 \gev.

Los valores obtenidos a partir de las diferentes fuentes de incertezas sistemáticas
se resumen en la \cref{tab:syst_wgamma_truth},
donde el valor total es de  $73$\% y $39$\% para {\SRL} y
{\SRH}, respectivamente.

\begin{table}[!ht]
  \centering

  \caption{Resumen de las incertezas teóricas de la producción de {\wgam} obtenidas en cada región de señal.}
  \label{tab:syst_wgamma_truth}

  \begin{tabular}{z{6cm}x{2cm}x{2cm}}
    \hline
    & {\SRL} & {\SRH} \\
    \hline
    Escala de factorización   & $0 \%$  & $16 \%$ \\
    Escala de renormalización & $71 \%$ & $33 \%$ \\
    Escala de asociación ME/PS (CKKW)   & $14 \%$ & $12 \%$ \\
    \hline
    Total  &   $73 \%$    &  $39 \%$     \\
    \hline
  \end{tabular}

\end{table}



\subsubsection{Producción de fotones directos}

La normalización del fondo de {\gjet} y sus incertezas son tomadas del ajuste a datos en la CR
como se describe en \cref{sec:bkg_gjet}. Una incerteza adicional es obtenida comparando el
número de eventos en las distintas regiones de señal entre la muestra nominal
(\sherpa) y los obtenidos con una muestra generada con otro generador ({\pythia}).
El estudio se realizó a nivel generador para evitar tener en cuenta nuevamente los efectos del
detector. Esta incerteza resultó del 45\%.


\subsubsection{Otros fondos MC}

La mayor contribución de procesos de fondo que se estiman directamente de las simulaciones
MC en las regiones de se\~nal proviene
del {\zgam}, para el que se utilizó una incerteza conservativa del 100\%.
