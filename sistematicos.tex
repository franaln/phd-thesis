\chapter{Determinación de las incertezas sistemáticas}

Las incertezas sistemáticas están asociadas a las predicciones de todas
las componentes de fondo y a los valores esperados de señal. Estas
incertezas pueden ser clasificadas en dos tipos: incertezas experimentales
e incertezas teóricas. Estas incertezas pueden tener impacto en el numero
de eventos esperado en las CR y SR como también en los TF usados
en la extrapolación del fondo esperado.

Este capitulo overviews las incertezas sintomáticas consideradas y su determinación.
En la determinación de los resultados finales utilizando el ajuste con el PLR,
las predicciones de los fondos se permiten variar dentro del tamaño de las
incertezas sistemáticas y por lo tanto son constraint por los datos


\section{Incertezas Experimentales}\label{sec:expsyst}

En esta sección se describe los procedimientos realizados para evaluar
las incertezas experimentales comunes para todos los procesos. Para esto
se utilizo la herramienta oficial \texttt{SUSYTools v00-03-24-02}.

Las incertezas sistemáticas de las estimaciones data-driven
de los fondos de electrones y jets miidentifacdos ya fueron discutidas
en sec \XXX


\subsection{Incerteza debida al Pileup}

Siguiendo las recomendaciones del grupo de \emph{tracking}, se utiliza
un factor de $1.09 \pm 0.04$ aplicado al numero promedio del numero de
interacciones por cruce de haz ($<\mu>$) en las simulaciones para poder
describir mejor la multiplicad de vértices del minimum bias en los datos.
La incerteza asociada es calculada variando este factor de escala dentro
de su incerteza.

\subsection{Incerteza en la luminosidad}

La incerteza en la luminosidad integrada es de $\pm$2.8\%. % \cite{lumi2012}.
Esta es derivada siguiendo la misma metodología que la que se detalla en \cite{lumi2011},
a partir de la calibración preliminar of the luminosity scale derived from beam-separation
scans performed in November 2012.


\subsection{Eficiencia del Trigger}\label{sec:trigger_eff}

La eficiencia del trigger de fotones utilizado {\trigchain} es de $100^{+0}_{-1.41}$ (stat) ${}^{+0}_{-0.7}$(syst) \%,
relativo a candidatos a fotones aislados con $\pt > 125 \gev$ y satisfaciendo los requerimientos
de la selección de la región de señal descripta en la sec \ref{sec:event_selection}.
Esta fue calculada utilizando el método de \emph{bootstrap} siguiendo la descripción en \cite{Damazio:1609629}
y se muestra en la {\fig} \ref{fig:trigger_perf} como función del {\pt} del fotón.

Mas allá de la poca estadística, es claro que la eficiencia llega a un plateau antes del threshold
en el pt que se utiliza en este análisis. La incerteza total tiene en cuenta la limitada estadística de
la muestra de referencia y correciones debido a la pureza también son aplicadas.

\begin{figure*}[ht!]
  \centering
  \includegraphics[width=0.49\textwidth]{figures/figura}
  \caption{Eficiencia del trigger para {\trigchain} como funcion del {\pt} del fotón,
    medida a partir de los datos.}
  \label{fig:trigger_perf}
\end{figure*}

\subsection{Identificación, escala de energía y resolución de fotones}\label{sec:syst_photonid}

La incerteza en los factores de escala para fotones (SFs) aplicados al MC es propagada
a través del análisis, siguiendo \cite{PhoEffTwiki}. %to obtain the number of tight identified photons and related variables.

Los factores de escala están basadas en la combinación de varias medidas data-driven.
Los valores centrales y las incertezas son provistos por el grupo Egamma mediante la herrmaienta
\texttt{PhotonEfficiencyCorrectionTool}. %and propagated here to the MC background predictions ({\bf systPhId}).
Sin embargo, el requerimiento de aislamiento aplicado para derivar los SFs
(\texttt{topoEtcone40} $< 4\gev$) no es el mismo que el utilizado en este análisis. Para tener esto
en cuenta,

%% the tight identification efficiency was evaluated after each isolation criteria
%% for photons with $\pt>125\gev$ in both signal and \gjet\ MC. The effect is found to be very small ($<1\%$) and it is neglected.

%% Uncertainties on photon energy scale and resolution are also considered following \Ref \cite{EGScaleTwiki}.


\subsection{Autointoxicación, escala de energía y revolución de leptones}\label{sec:syst_leptonid}

Similar a los fotones, la incerteza en la eficiencia de identificación, escala de
energía y resolución son consideradas siguiendo \cite{EleEffTwiki}, \cite{EGScaleTwiki}
y \cite{MCPTwiki}.

\subsection{Escala de energía de jets (JES)}
The jet energy is scaled up and down (in a fully correlated way) by the $\pm1\sigma$ total uncertainty on the jet energy scale obtained
using the {\small \texttt{JetUncertainties}} package (v00-08-07) provided by the Jet/Etmiss group \cite{JesTwiki}. The nominal analysis
uses only the total JES uncertainty, although it has been checked that the effect of using the reduced set of JES components on the fit
results is within 1\% (see \Sec \ref{sec:JEScheck}).

\subsection{Resolucion de la energia de jets (JER)}
 An extra \pt smearing is added to the jets based on their \pt and $\eta$ to account
for a possible underestimate of the jet energy resolution in the MC simulation. This is done using the
\texttt{JetResolution-02-00-03} package. The so found yield differences are then applied as a $\pm$ uncertainty.

\subsection{Termino soft de \MET}
The impact of the scales uncertainties on the soft term which enters in \MET\ is estimated by varying these
scales up and down using the \texttt{MissingETUtility-01-03-03} package. The resolution uncertainty on the cell out term is also included.

\subsection{Eficiencia de b-tagging}
Apart from the b-veto requirement in CRLW, there no explicit use of b-tagging information in this analysis.
In this particular region, the b-tagging uncertainty is evaluated by varying the $\eta-$, $\pt-$ and flavour-dependent
scale factors applied to each jet in the simulation within a range that reflects the systematic uncertainty
on the measured tagging efficiency and mistag rates. These variations are applied separately to B-jets,
C-jets and light jets, leading to three uncorrelated systematic uncertainties.

\section{Incertezas teoricas}\label{sec:theosyst}

\subsection{Senal}\label{sec:syst_signal}

La seccion eficas para la produccion de pares de gluinos  fue calculada utilizando
e; programa {\nllfast} (version 2.1)\cite{nllfast}, a traves de la herramienta
estandar usada en ATLAS, \texttt{SignalUncertaintiesUtils}.

The gluino hadroproduction cross sections are computed including
next-to-leading order supersymmetric QCD corrections and the resummation of soft gluon emission at next-to-leading-logarithmic accuracy
\cite{Beenakker:1996ch,Kulesza:2008jb,Kulesza:2009kq,Beenakker:2009ha,Beenakker:2011fu}.

Los valores centrales de la seccion eficas fueron obtenidos sigueindo las recomendaciones
PDF4LHC \cite{Botje:2011sn}.
Una \fix{envelope} de las predicciones a la seccion eficas se define usando

An envelope of cross section predictions is defined using the $68\%$
C.L. ranges of the CTEQ6.6 \cite{Nadolsky:2008zw} (including the $\alpha_s$ uncertainty) and MSTW2008 \cite{Martin:2009iq} PDF sets, together with
the variations of the scales. The nominal cross section is obtained using the midpoint of the envelope and the
uncertainty assigned is half the full width of the envelope. For more details on the procedure please refer to \cite{Kramer:2012bx}.

%% For the case of CTEQ6.6, three types of uncertainties, namely the PDF uncertainty, the scale un-
%% certainty and the αs uncertainty are taken into account. The PDF uncertainty is the 68% C.L. ranges
%% estimated with the error PDF sets for each PDF, where in case of CTEQ6.6 (MSTW2008 NLO) the varia-
%% tions of 22 (20) parameters spanning to the range of experimental uncertainties gives 44 (40) different
%% outcomes of cross sections. The αs uncertainty is defined as one half of the difference between the two
%% extreme variations in CTEQ6.6AS PDF sets.
%% The scale uncertainty is the deviations in the cross section central value by varying the factorisation
%% and the renormalisation scales by factors of two or one half. Three components are considered to be
%% independent and summed in quadrature. The same procedure is applied in case of the MSTW2008 NLO
%% except for the αs uncertainty, which is absent for MSTW2008 NLO. The nominal value is taken to be the
%% midpoint of the upper and lower envelopes. The uncertainty is assigned as the half of the full width of
%% the envelop. From definition, the uncertainty is symmetric in plus and minus sides.

For the $\gluino\gluino$ grid the squarks and all other SUSY partner masses are effectively set to
a very large scale (2.5 \TeV), so to suppress their production. Defining this large scale is arbitrary and in some cases it may
have a non-negligible impact in the production of the SUSY particles residing at the TeV scale (e.g. squarks at
high scales can still contribute to the gluino pair production process via a t-channel exchange). Thus, instead of
setting an explicit arbitrary value for the mass of the decoupled particle, the NLO+NLL calculation implemented
in {\sc NLL-Fast} assumes that this production does not interfere in any possible way with the production processes of
the rest of the particles.
%Thus, the particle is completely decoupled from the rest of the phenomenology at the TeV
%scale.

Los valores centrales y sus icnertezas para la grid de se\~nal
estan resumidos en la {\tab} \ref{tab:signal_xs_theo_unc}.
Las incertezas totales van de 22.5\% para 800 \GeV\ hasta 36.8\% para 1.3\TeV.

\begin{sidewaystable}[hp!]
  \centering
  \caption{La seccion eficaz total NLO+NLL con sus incertezas y facores $k$ para los puntos de se\~nal.}
  \begin{tabular}{c|ccccccccccc}
    \hline
    \hline
    $M_3$ [\gev]                       & 800                & 850               & 900                & 950               & 1000             & 1050             & 1100             & 1150             & 1200             & 1250        & 1300 \\
    $m_{\tilde{g}}$ [\gev]          & 885              & 932             & 978              & 1023            & 1068           & 1113           & 1157           & 1202           & 1246           & 1290      & 1333 \\
    Seccion eficaz (Best) [pb]       & 0.0690             & 0.0449            & 0.0297             & 0.01984           & 0.01341          & 0.00910          & 0.00628          & 0.00432          & 0.00301          & 0.00210     & 0.00149 \\
    Signal enc.   [$\%$]            & 22.5               & 23.8              & 25.2               & 26.5         & 27.7             & 29.0             & 30.4             & 32.0             & 33.7             & 35.2        & 36.8 \\[5pt]
    PDF enc. CTEQ [$\%$]            & \unc{22.5}{15.3}    & \unc{23.5}{15.9}   & \unc{24.6}{16.5}    & \unc{25.8}{17.1}   & \unc{26.9}{17.7}  & \unc{28.2}{18.4}  & \unc{29.4}{19.0}  & \unc{30.7}{19.7}  & \unc{32.0}{20.4}  & \unc{33.4}{21.1}    & \unc{34.8}{21.8} \\[5pt]
    PDF enc. MSTW [$\%$]            & \unc{9.5}{9.1}      & \unc{9.9}{9.4}     & \unc{10.2}{9.7}     & \unc{5.6}{5.0}     & \unc{10.9}{10.3}  & \unc{11.3}{10.6}  & \unc{11.8}{10.9}  & \unc{12.2}{11.2}  & \unc{12.6}{11.6}  & \unc{13.1}{11.9}    & \unc{13.6}{12.3} \\[5pt]
    Scale enc. CTEQ [$\%$]          & \unc{9.8}{9.8}      & \unc{9.9}{9.9}     & \unc{9.9}{10.0}     & \unc{5.0}{5.0}     & \unc{10.2}{10.1}  & \unc{10.3}{10.2}  & \unc{10.4}{10.3}  & \unc{10.5}{10.4}  & \unc{10.5}{10.5}  & \unc{10.7}{10.6}    & \unc{10.8}{10.7} \\[5pt]
    Scale enc. MSTW [$\%$]          & \unc{10.3}{10.1}    & \unc{10.4}{10.2}   & \unc{10.5}{10.2}    & \unc{5.6}{5.3}     & \unc{10.7}{10.4}  & \unc{10.8}{10.4}  & \unc{10.9}{10.5}  & \unc{11.0}{10.6}  & \unc{11.2}{10.7}  & \unc{11.3}{10.8}    & \unc{11.4}{11.0} \\[5pt]
    $\alpha_{s}$ enc. CTEQ [$\%$]    & \unc{6.9}{4.7}      & \unc{7.0}{4.8}     & \unc{7.2}{4.9}      & \unc{7.4}{5.0}     & \unc{7.5}{5.1}    & \unc{7.7}{5.2}    & \unc{7.9}{5.4}    & \unc{8.1}{5.5}    & \unc{8.3}{5.6}    & \unc{8.5}{5.7}      & \unc{8.7}{5.8} \\[5pt]
    $\alpha_{s}$ enc. MSTW [$\%$]    & \unc{3.2}{3.1}      & \unc{3.3}{3.1}     & \unc{3.3}{3.1}      & \unc{3.4}{3.1}     & \unc{3.4}{3.1}    & \unc{3.5}{3.0}    & \unc{3.5}{3.0}    & \unc{3.6}{3.0}    & \unc{3.6}{3.0}    & \unc{3.6}{2.9}      & \unc{3.7}{2.9} \\[5pt]
    Seccion eficaz NLL (cteq) [pb]   & 0.0674             & 0.044             & 0.0292             & 0.0195            & 0.0132           & 0.00896          & 0.0062           & 0.00428          & 0.00299          & 0.00209     & 0.00148 \\
    NLO cross-section (cteq) [pb]   & 0.0618             & 0.0402            & 0.0265             & 0.0177            & 0.0119           & 0.00811          & 0.00559          & 0.00384          & 0.00266          & 0.00186     & 0.00131 \\
    LO cross-section  (cteq) [pb]   & 0.0289             & 0.0186            & 0.0121             & 0.00801           & 0.00534          & 0.00358          & 0.00244          & 0.00165          & 0.00113          & 0.000776   & 0.000539 \\
    K-factor (Best CS/LO)           & 2.4                & 2.41              & 2.46               & 2.48              & 2.51             & 2.54             & 2.57             & 2.62             & 2.67             & 2.71      & 2.76 \\
    \hline
    \hline
  \end{tabular}
  \label{tab:signal_xs_theo_unc}
\end{sidewaystable}


\subsection{Producción de \ttgam}\label{sec:syst_ttbargamma}

La incerteza teorica asociada a la eleccion de la simulacion MC es
tenida en cuenta considerando variaciones en la generacion de eventos
con respecto a la configuracion nominal. Las variaciones consideradas
incluyen las escalas de factorizacion y renormalizacion ($0.5\times, 2\times$),
$\alpha_{s}$ y el modelo ISR/FSR (ver {\tab} \ref{tab:bkg_ttbar_samples}).

%are summarized below:
%
%\begin{description}
%\item {\bf Factorization and Renormalization scales}    ...
%\item $\mathbf{\alpha_s}$   ...
%\item {\bf ISR/FSR} ...
%\end{description}

La {\tab} \ref{tab:syst_ttbargam_truth} resume los cambios relativos en el
numero de eventos esperado en cada region de senal para cada una de las
variaciones con respecto a la muestra nominal at particle level.

La incertza de cada una de las tres variaciones es simetrizada, a la mayor
entre los valores por encima y debajo, y luego sumadas en cuadratura
a la incerteza debida a la seccion eficas para obtner la incerteza teorica
total. %%o get the total theoretical uncertainty (denoted as {\bf theoSysTopG} in the fit).

\begin{table}[ht!]
  \centering
  \caption{Resumen de las incertezas teoricas del {\ttgam} obtenidas en cada region de senal.
    Todos ls numeros estan en porcentage.}
  \begin{tabular}{ l | c  c  }
   \hline
    & SR2 & SR3 \\
   \hline
      Factorisation/Renormalisation scales &  56  & 20 \\
      ISR/FSR                              &  4   & 20 \\
      $\alpha_{s}$                         &  4   &  0 \\
      %      total cross section                  &  20  &   20 \\
      \hline
      \hline
      total				&   56    &   28 \\
      \hline
  \end{tabular}
  \label{tab:syst_ttbargam_truth}
\end{table}

La difernecia principal entre todas las variaciones y la configuracion
nominal aparece mayormente en la ultima parte del cutflow, para la
%%seleccion en las variables {\ht} y {\rt}. Ademas del efecto estadistico
obvio, es algo esperado ya que las propiedades de la actividad hadronica
del evento son las mas sensitivas a las variaciones de escala.

\subsection{Produccion de {\wgam}}\label{sec:syst_wgamma}

Similar a lo que se realizo para estimar las incertezas teoricas del {\ttgam}, las incertezas
en el numero estimado de eventos de {\wgam} fue determianda realizando el analisi a nivel
particula para las muestras con variaciones descriptas en {\tab} \ref{tab:bkg_wzgamma_samples}.
Ademas de las incertezas provenientes del ajuste simultaneo descript en {\Sec} \ref{sec:fitconfig},
las siguientes incertezas en la aceptancia de la seleccion fueron consideradas:

\begin{description}
\item[Escala de factorizacion y renormalizacion] La incerteza corresponde
  a la diferencia relativa entre el numero esperado de eventos predicho
  para la SR con las muestras {\sherpa} en las cuales se vario la escala
  de factorizacion y renormalizacion de forma independiente por $2\times$
  y $0.5\times$ de la escala nominal.

\item[Matching scale] The nominal matching scale between the ME and PS is 20 GeV. Alternative samples
  were generated with this scale changed to 15 and 30 \gev. The corresponding
  relative variation of yields in SR is considered as a systematic uncertainty.
\end{description}

Los valores obtenidos estan summarised en la {\tab} \ref{tab:syst_wgamma_truth},
donde la incerteza total ({\bf theoSysWG}) es de  $73$\% y $39$\% para {\SRL} y
{\SRH}, respectivamente.

\begin{table}[ht!]
  \centering
  \begin{tabular}{ l | c  c  }
    \hline
    & SR2 & SR3 \\
    \hline
    Factorisation scale   & $0$  & $16$ \\
    Renormalisation scale & $71$ & $33$ \\
    CKKW matching scale   & $14$ & $12$ \\
    \hline
    \hline
    Total  &   $73$    &  $39$     \\
    \hline
  \end{tabular}
  \caption{Resumen de las incertezas teoricas del {\wgam} obtenida en cada region de senal.
    Todos los numeros estan en porcentage.}
  \label{tab:syst_wgamma_truth}
\end{table}



\subsection{Produccion de fotones directos}

La normalización del fondo QCD {\gjet} es tomado del ajuste a datos en la CR como se describe en {\XXX}.
Una incerteza adicional es agregada comparando el numero de eventos en las distintas regiones de señal
entre la muestra nominal (\sherpa) y los obtenidos con otra muestra generada con {\pythia} (ver {\tab} \XXX).
Esto se realizo a nivel truth para no contar nuevamente los efectos del detector, y se incluyo en el ajuste
como \texttt{theoSysGJ}.

\itodo{Agregar un analisis mas detallado entre los generadores}


\subsection{Otros fondos MC}

La mayor contribucion de procesos de fondo de MC en las regiones de senial viene del
{\zgam}, para el cual se utilizo una incerteza conservativa del 100\%.


\subsection{Breakdown of the Systematic Uncertainties}\label{sec:syst_break}
\itodo{Agregar tabla resumiendo todas las incertezas sistematicas}
