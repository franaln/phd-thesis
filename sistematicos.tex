%\chapter{Determinación de las incertezas sistemáticas}
\section{Determinación de las incertezas sistemáticas}
\label{cap:sistematicos}

Uno de los aspectos más importantes de las investigaciones con datos colectados por ATLAS
es la determinacion de todas las incertezas sistematicas que afectan a los resultados.
Las incertezas sistemáticas están asociadas a las predicciones de todas las
componentes del fondo y a los valores esperados de señal. Estas incertezas
pueden ser clasificadas en dos tipos: de origen experimental o teórico.
Estas incertezas pueden tener un impacto importante en el número de eventos esperado
en las CR y SR como así también en los parámetros $\mu_p$ utilizados en la
extrapolación del fondo esperado.

En esta sección se presentan todas las fuentes posibles de incertezas sistemáticas
y la determinación de las mismas.
%% En la determinación de los resultados finales utilizando el
%% ajuste con el PLR, las predicciones de los fondos se permiten variar dentro del
%% tamaño de las incertezas sistemáticas y por lo tanto son restringidas por los
%% datos.


\subsection{Incertezas Experimentales}\label{sec:expsyst}

A continuación se describen los procedimientos realizados para evaluar
las incertezas experimentales comunes para todos los procesos tanto de señal como
de fondo considerados en el análisis.
%% \footnote{Para esto
%% se utilizó la herramienta oficial de ATLAS: \texttt{SUSYTools v00-03-24-02}}.

Las incertezas sistemáticas de las estimaciones de los fondos de electrones
($\alpha_{e\to\gamma}$) y jets ($\alpha_{j\to\gamma}$) mal identificados a partir
de los datos ya han sido discutidos en las \cref{sec:efakes,sec:jfakes}. La incerteza del
trigger se presentó en la \cref{sec:trigger} ($\alpha_\text{TRIG}$).


\subsubsection{Incerteza en la luminosidad}

La incerteza en la luminosidad integrada es de $\pm$2.8\% \cite{lumi2012}.
Esta es derivada siguiendo la misma metodología que la que se detalla en \cite{lumi2011},
a partir de la calibración preliminar  de la escala de luminosidad derivada
de los scans de\note{van der Meer?} separación de haz realizados durante Noviembre de 2012.


\subsubsection{Incerteza debida al pile-up ($\alpha_\mathrm{PRW}$)}

Como se explicó en la \cref{sec:prw} es necesario realizar
un repesado evento a evento en las muestras simuladas para modelar las condiciones
de pile-up de los datos. Para calcular la incerteza asociada a estre procedimiento
se varian los factores de escala dentro de su incerteza.

%% Siguiendo las recomendaciones de ATLAS, se utiliza
%% un factor de $1.09 \pm 0.04$ aplicado al valor medio del número de
%% interacciones por cruce de haz ($\avg{\mu}$) en el repesado debido al pile-up en las muestras simuladas
%% para describir mejor la multiplicidad de vértices del minimum bias en los datos.
%% La incerteza asociada es calculada variando este factor de escala dentro
%% de su incerteza.


\subsubsection{Incerteza en la identificación, escala de energía y resolución de fotones y leptones}

Como se menciona en la \cref{sec:fotones}, es necesario aplicar factores de
escala a los fotones en las muestras MC para corregir su eficiencia. La
incerteza en estos factores es propagada a través del análisis, siguiendo
\cite{PhoEffTwiki}. %to obtain the number of tight identified photons and
related variables. La incerteza en la escala de energia de los fotones y
electrones se tiene en cuenta de acuerdo a cuatro
componentes\cite{EGScaleTwiki}: $\alpha_\text{EGZEE}$ (incerteza debido a la
diferencia en la escala de energia proveniente del $Z$), $\alpha_\text{EGMAT}$ (incerteza debido al material),
$\alpha_\text{EGPS}$ (presampler), y $\alpha_\text{EGLOW}$ (electrones de bajo \pt).

Los factores de escala están basadas en la combinación de varias medidas data-driven.
%% Los valores centrales y las incertezas son provistos por el grupo Egamma mediante la herrmaienta
%% \texttt{PhotonEfficiencyCorrectionTool}. %and propagated here to the MC background predictions ({\bf systPhId}).
%% Sin embargo, el requerimiento de aislamiento aplicado para derivar los SFs
%% (\texttt{topoEtcone40} $< 4\gev$) no es el mismo que el utilizado en este análisis. Para tener esto
%% en cuenta,

%% the tight identification efficiency was evaluated after each isolation criteria
%% for photons with $\pt>125\gev$ in both signal and \gjet\ MC. The effect is found to be very small ($<1\%$) and it is neglected.

%% Uncertainties on photon energy scale and resolution are also considered following \Ref \cite{EGScaleTwiki}.



%% \subsection{Incerteza en la identificación, escala de energía y resolución de leptones}
%%($\alpha_\text{EGZEE}$, $\alpha_\text{EGMAT}$, $\alpha_\text{EGPS}$, $\alpha_\text{EGLOW}$, $\alpha_\text{EGRES}$)}


Similar a los fotones, la incerteza en la eficiencia de identificación ($\alpha_\text{EEFF}$), y resolución de
la enegia ($\alpha_\text{EGRES}$) son consideradas siguiendo \cite{EleEffTwiki}, \cite{EGScaleTwiki}
y \cite{MCPTwiki}.





\subsubsection{Escala de energía de jets ($\alpha_\mathrm{JES}$)}

La energia de los jets es aumentada y disminuida (de forma correlacionada) en
$\pm1\sigma$ de la incerteza total de la escala de energia de jets, utilizando
una herramienta (\textsc{JetUncertainties}) provista por el grupo de Jet/Etmiss de ATLAS \cite{JesTwiki}.
Esta herramienta provee la incerteza relativa en la escala de energia de los jets,
que es la suma en cuadratura de tres componentes dependientes de: el {\pt} y $\eta$
del jet, el $\Delta R$ al jet mas cercano, y la composicion media quark-gluon de la muestra.


%% The jet energy is scaled up and down (in a fully correlated way) by the
%% $\pm1\sigma$ total uncertainty on the jet energy scale obtained using the
%% {\small \texttt{JetUncertainties}} package (v00-08-07) provided by the
%% Jet/Etmiss group \cite{JesTwiki}. The nominal analysis uses only the total JES
%% uncertainty, although it has been checked that the effect of using the reduced
%% set of JES components on the fit results is within 1\% (see
%% \cref{sec:JEScheck}).

\subsubsection{Resolucion de la energia de jets ($\alpha_\mathrm{JER}$)}

A fin de tener en cuenta una posible subestimacion de la resolucion de energia de los jets
en la simulacion MC, se realiza un smearing del {\pt} de los jets en funcion de su \pt y su $\eta$.
Cada jet es smeareado de acuerdo a una distribucion Gausiana, de media unidad y un ancho
dado por una funcion de resolucion dependiente de {\pt} y $\eta$. Para evaluar el impacto
de la incerteza sistematica debida a la resolucion de energia, el número de eventos obtenido utilizando
los jets nominales es comparado con los resultados obtenidos de los jets smeareados.
Este procedimiento tambien se realiza utilizando una herramienta provista por el grupo de Jet/Etmiss
de ATLAS.
 %% An extra \pt smearing is added to the jets based on their \pt and $\eta$ to
 %% account for a possible underestimate of the jet energy resolution in the MC
 %% simulation. This is done using the \texttt{JetResolution-02-00-03} package. The
 %% so found yield differences are then applied as a $\pm$ uncertainty.


\subsubsection{Eficiencia de identificación de {\bjets} ($\alpha_\text{BJET}$)}

La identificacion de {\bjets} solo es utilizado en la seleccion de las regiones de control CRW y CRT.
En estas regiones, se evalua la incerteza del mismo, variando los factores de escala
aplicados a cada jet en las simulaciones, que dependen de $\eta$, {\pt}, dentro del rango
que refleja la incerteza sistematica en la eficiencia medida de identificacion de {\bjets}.


%% information in this analysis. In this particular region, the b-tagging
%% uncertainty is evaluated by varying the $\eta-$, $\pt-$ and flavour-dependent
%% scale factors applied to each jet in the simulation within a range that reflects
%% the systematic uncertainty on the measured tagging efficiency and mistag rates.
%% These variations are applied separately to B-jets, C-jets and light jets,
%% leading to three uncorrelated systematic uncertainties.


\subsubsection{Término soft de \MET ($\alpha_\text{SCALEST}$, $\alpha_\text{RESOST}$)}

Al modificar la escala de enegia de los objetos o la resolucion de energia de los jets, esto
se propaga al termino correspondiente en el calculo de {\met}.

Adicionalmente, el tambien se evalua la incerteza en el termino soft de {\met}, variando
la escala de energia de los clusters, utilizando una herramienta provista por
el grupo Jet/Etmiss de ATLAS. %% (\textsc{MissingETUtility}).
Tambien se tiene en cuenta la incerteza de la resolucion en el termino soft.

%% The impact of the scales uncertainties on the soft term which enters in \MET\ is
%% estimated by varying these scales up and down using the
%% \texttt{MissingETUtility-01-03-03} package. The resolution uncertainty on the
%% cell out term is also included.




\subsection{Incertezas Teóricas}\label{sec:theosyst}


\subsubsection{Señal}\label{sec:syst_signal}

La sección eficaz para la producción de pares de gluinos  fue calculada utilizando
el programa {\nllfast} de acuerdo a las recomendaciones explicadas en \cref{sec:xs_calc}.

%% For the case of CTEQ6.6, three types of uncertainties, namely the PDF uncertainty, the scale un-
%% certainty and the αs uncertainty are taken into account. The PDF uncertainty is the 68% C.L. ranges
%% estimated with the error PDF sets for each PDF, where in case of CTEQ6.6 (MSTW2008 NLO) the varia-
%% tions of 22 (20) parameters spanning to the range of experimental uncertainties gives 44 (40) different
%% outcomes of cross sections. The αs uncertainty is defined as one half of the difference between the two
%% extreme variations in CTEQ6.6AS PDF sets.
%% The scale uncertainty is the deviations in the cross section central value by varying the factorisation
%% and the renormalisation scales by factors of two or one half. Three components are considered to be
%% independent and summed in quadrature. The same procedure is applied in case of the MSTW2008 NLO
%% except for the αs uncertainty, which is absent for MSTW2008 NLO. The nominal value is taken to be the
%% midpoint of the upper and lower envelopes. The uncertainty is assigned as the half of the full width of
%% the envelop. From definition, the uncertainty is symmetric in plus and minus sides.

%% For the $\gluino\gluino$ grid the squarks and all other SUSY partner masses are effectively set to
%% a very large scale (2.5 \TeV), so to suppress their production. Defining this large scale is arbitrary and in some cases it may
%% have a non-negligible impact in the production of the SUSY particles residing at the TeV scale (e.g. squarks at
%% high scales can still contribute to the gluino pair production process via a t-channel exchange). Thus, instead of
%% setting an explicit arbitrary value for the mass of the decoupled particle, the NLO+NLL calculation implemented
%% in {\sc NLL-Fast} assumes that this production does not interfere in any possible way with the production processes of
%% the rest of the particles.
%Thus, the particle is completely decoupled from the rest of the phenomenology at the TeV
%scale.

Los valores centrales y sus incertezas para la grid de señal están resumidos en
la \cref{tab:signal_xs_theo_unc}. Las incertezas totales van de 22.5\% para 800
{\GeV} hasta 36.8\% para 1.3 \tev.

\begin{sidewaystable}[!htbp]
  \centering
  \caption{La sección eficaz total NLO+NLL con sus incertezas y factores $k$ para los puntos de se\~nal.}
  \label{tab:signal_xs_theo_unc}

  \begin{tabular}{l|rrrrrrrrrrr}
    \hline
    $M_3$ [\gev]                     & 800                & 850               & 900                & 950               & 1000             & 1050             & 1100             & 1150             & 1200             & 1250        & 1300 \\
    $m_{\tilde{g}}$ [\gev]           & 885              & 932             & 978              & 1023            & 1068           & 1113           & 1157           & 1202           & 1246           & 1290      & 1333 \\
    \hline
    Sección eficaz (Mejor) [pb]      & 0.0690             & 0.0449            & 0.0297             & 0.01984           & 0.01341          & 0.00910          & 0.00628          & 0.00432          & 0.00301          & 0.00210     & 0.00149 \\
    Incerteza     [$\%$]             & 22.5               & 23.8              & 25.2               & 26.5         & 27.7             & 29.0             & 30.4             & 32.0             & 33.7             & 35.2        & 36.8 \\[5pt]
    PDF enc. {\cteq} [$\%$]             & \unc{22.5}{15.3}    & \unc{23.5}{15.9}   & \unc{24.6}{16.5}    & \unc{25.8}{17.1}   & \unc{26.9}{17.7}  & \unc{28.2}{18.4}  & \unc{29.4}{19.0}  & \unc{30.7}{19.7}  & \unc{32.0}{20.4}  & \unc{33.4}{21.1}    & \unc{34.8}{21.8} \\[5pt]
    PDF enc. {\mstw} [$\%$]             & \unc{9.5}{9.1}      & \unc{9.9}{9.4}     & \unc{10.2}{9.7}     & \unc{5.6}{5.0}     & \unc{10.9}{10.3}  & \unc{11.3}{10.6}  & \unc{11.8}{10.9}  & \unc{12.2}{11.2}  & \unc{12.6}{11.6}  & \unc{13.1}{11.9}    & \unc{13.6}{12.3} \\[5pt]
    Scale enc. {\cteq} [$\%$]           & \unc{9.8}{9.8}      & \unc{9.9}{9.9}     & \unc{9.9}{10.0}     & \unc{5.0}{5.0}     & \unc{10.2}{10.1}  & \unc{10.3}{10.2}  & \unc{10.4}{10.3}  & \unc{10.5}{10.4}  & \unc{10.5}{10.5}  & \unc{10.7}{10.6}    & \unc{10.8}{10.7} \\[5pt]
    Scale enc. {\mstw} [$\%$]           & \unc{10.3}{10.1}    & \unc{10.4}{10.2}   & \unc{10.5}{10.2}    & \unc{5.6}{5.3}     & \unc{10.7}{10.4}  & \unc{10.8}{10.4}  & \unc{10.9}{10.5}  & \unc{11.0}{10.6}  & \unc{11.2}{10.7}  & \unc{11.3}{10.8}    & \unc{11.4}{11.0} \\[5pt]
    $\alpha_{s}$ enc. {\cteq} [$\%$]    & \unc{6.9}{4.7}      & \unc{7.0}{4.8}     & \unc{7.2}{4.9}      & \unc{7.4}{5.0}     & \unc{7.5}{5.1}    & \unc{7.7}{5.2}    & \unc{7.9}{5.4}    & \unc{8.1}{5.5}    & \unc{8.3}{5.6}    & \unc{8.5}{5.7}      & \unc{8.7}{5.8} \\[5pt]
    $\alpha_{s}$ enc. {\mstw} [$\%$]    & \unc{3.2}{3.1}      & \unc{3.3}{3.1}     & \unc{3.3}{3.1}      & \unc{3.4}{3.1}     & \unc{3.4}{3.1}    & \unc{3.5}{3.0}    & \unc{3.5}{3.0}    & \unc{3.6}{3.0}    & \unc{3.6}{3.0}    & \unc{3.6}{2.9}      & \unc{3.7}{2.9} \\[5pt]
    Sección eficaz NLL (\cteq) [pb]  & 0.0674             & 0.044             & 0.0292             & 0.0195            & 0.0132           & 0.00896          & 0.0062           & 0.00428          & 0.00299          & 0.00209     & 0.00148 \\
    Sección eficaz NLO (\cteq) [pb]  & 0.0618             & 0.0402            & 0.0265             & 0.0177            & 0.0119           & 0.00811          & 0.00559          & 0.00384          & 0.00266          & 0.00186     & 0.00131 \\
    Sección eficaz LO  (\cteq) [pb]  & 0.0289             & 0.0186            & 0.0121             & 0.00801           & 0.00534          & 0.00358          & 0.00244          & 0.00165          & 0.00113          & 0.000776   & 0.000539 \\
    Factor $k$ (Best CS/LO)          & 2.4                & 2.41              & 2.46               & 2.48              & 2.51             & 2.54             & 2.57             & 2.62             & 2.67             & 2.71      & 2.76 \\
    \hline
  \end{tabular}

\end{sidewaystable}


\subsubsection{Producción de \ttgam}\label{sec:syst_ttbargamma}

La incerteza teórica asociada a la elección de la configuración
utilizada en la simulación MC es
tenida en cuenta considerando variaciones en la generación de eventos
con respecto a la configuración nominal. Las variaciones consideradas
incluyen las escalas de factorización y renormalización ($0.5\times, 2\times$),
$\alpha_{s}$ y el modelo ISR/FSR (ver \cref{tab:mc_ttbar_samples}).

La \cref{tab:syst_ttbargam_truth} resume los cambios relativos en el
número de eventos esperado en cada región de señal para cada una de las
variaciones con respecto a la muestra nominal a nivel generador.

La incerteza de cada una de las tres variaciones es simetrizada, a la mayor
entre los valores por encima y debajo, y luego sumadas en cuadratura
a la incerteza debida a la sección eficaz para obtener la incerteza teórica
total. %%o get the total theoretical uncertainty (denoted as {\bf theoSysTopG} in the fit).

\begin{table}[ht!]
  \centering
  \caption{Resumen de las incertezas teóricas del {\ttgam} obtenidas en cada región de señal.
    Todos los números están en porcentaje.}
  \label{tab:syst_ttbargam_truth}

  \begin{tabular}{l|cc}
    \hline
    & {\SRL} & {\SRH} \\
    \hline
    Escala de factorización/renormalización &  56  & 20 \\
    ISR/FSR                              &  4   & 20 \\
    $\alpha_{s}$                         &  4   &  0 \\
    %      total cross section                  &  20  &   20 \\
    \hline
    Total				&   56    &   28 \\
    \hline
  \end{tabular}

\end{table}

La diferencia principal entre todas las variaciones y la configuración nominal
aparece mayormente en los cortes en {\HT} y {\rt}, como es esperado ya que las
propiedades de la actividad hadrónica del
evento son las mas sensibles a las variaciones de escala.



\subsubsection{Producción de {\wgam} ($\alpha_{\tgam}$)}\label{sec:syst_wgamma}

Similar a lo que se realizo para estimar las incertezas teóricas del {\ttgam},
las incertezas en el número estimado de eventos de {\wgam} fue determinada
realizando el análisis a nivel partícula para las muestras con variaciones
descriptas en \cref{tab:bkg_wzgamma_samples}. Además de las incertezas
provenientes del ajuste simultaneo descripta en \cref{sec:fitconfig}, las
siguientes incertezas en la aceptancia de la selección fueron consideradas:
la incerteza debida a la escala de factorización y renormalización corresponde a
la diferencia relativa entre el número esperado de eventos predicho
para la SR con las muestras {\sherpa} en las cuales se vario la escala
de factorización y renormalización de forma independiente por $2\times$
y $0.5\times$ de la escala nominal, y la escala de matching\note{arreglar}
The nominal matching scale between the ME and PS is 20 GeV. Alternative samples
were generated with this scale changed to 15 and 30 \gev. The corresponding
relative variation of yields in SR is considered as a systematic uncertainty.


Los valores obtenidos a partir de las diferentes fuentes de incertezas sistematicas
se resumen en la \cref{tab:syst_wgamma_truth},
donde el valor total es de  $73$\% y $39$\% para {\SRL} y
{\SRH}, respectivamente.

\begin{table}[ht!]
  \centering

  \caption{Resumen de las incertezas teóricas del {\wgam} obtenida en cada región de señal.
    Todos los números están en porcentaje.}
  \label{tab:syst_wgamma_truth}

  \begin{tabular}{l|cc}
    \hline
    & {\SRL} & {\SRH} \\
    \hline
    Escala de factorización   & $0$  & $16$ \\
    Escala de renormalización & $71$ & $33$ \\
    Escala de CKKW matching scale   & $14$ & $12$ \\
    \hline
    Total  &   $73$    &  $39$     \\
    \hline
  \end{tabular}
\end{table}



\subsubsection{Producción de fotones directos}

La normalización del fondo de {\gjet} es tomado del ajuste a datos en la CR
como se describe en \cref{bkg_gjet}. Una incerteza adicional es obtenida comparando el
número de eventos en las distintas regiones de señal entre la muestra nominal
(\sherpa) y los obtenidos con otra muestra generada con {\pythia}.
Este estudio se realizó a nivel generador para no contar nuevamente los efectos del
detector, y la incerteza se incluyó en el ajuste como $\alpha_{\gamma j}$.


\subsubsection{Otros fondos MC}

La mayor contribución de procesos de fondo de MC en las regiones de se\~nal proviene
del {\zgam}, para el cual se utilizó una incerteza conservativa del 100\%.


%% \section{Resumen de las incertezas sistemáticas}\label{sec:syst_break}
%% \hl{Agregar tabla resumiendo todas las incertezas sistemáticas}


%% \begin{tabular}{cc}

%%   \hline
%%   \multicolumn{2}{c}{Incertezas experimentales} \\
%%   pile-up   & \\
%%   lumi      & \\
%%   trigger   & \\
%%   foton     & \\
%%   leptones  & \\
%%   jets      & \\
%%   met       & \\
%%   b-tagging & \\

%%   \hline
%%   \multicolumn{2}{c}{Incertezas teoricas} \\
%%   senal     & \\
%%   ttgam     & \\
%%   wgam      & \\
%%   phjet     & \\
%%   others    & \\
%%   \hline
%% \end{tabular}
