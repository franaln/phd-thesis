\chapter{Conclusiones}\label{cap:conclusiones}

Supersimetría (SUSY) es una de las extensiones del {\SM} de las partículas
fundamentales y sus interacciones más atractivas que se encuentra bajo estudio en el
Gran Colisionador de Hadrones (LHC) del CERN. El descubrimiento de un nuevo
escalar compatible con el bosón de Higgs del SM por los experimentos ATLAS
\cite{Aad:2012tfa} y CMS del LHC permitió establecer límites a los modelos de SUSY y el
escenario de búsqueda ha cambiado a partir de este resultado. A fin de realizar
la interpretación de los resultados de búsquedas de Supersimetría en el LHC se
consideraron diferentes modelos para el mecanismo del rompimiento de SUSY, el
cual es desconocido, de modo de reducir el número de parámetros libres
imponiendo ciertos límites fenomenológicos.

El análisis realizado en esta tesis constituye la primera búsqueda de nueva
física en eventos con un fotón aislado de alto {\pt}, jets y energía faltante,
utilizando los datos recolectados durante colisiones $pp$ a $\sqrt{s}=8 \tev$
por el detector ATLAS en el LHC. Este estado final se corresponde con una de las se\~nales
predichas en modelos de rompimiento de SUSY con mediación por campos de gauge (GGM, \emph{General Gauge
  Mediation}), donde el neutralino NLSP es una mezcla higgsino-bino con
aproximadamente igual tasa de decaimiento a fotones y al bosón $Z$. Los datos
utilizados corresponden a una luminosidad total integrada de {\ilumi}, y los
resultados obtenidos se incluyeron en la primera publicación de ATLAS en este
tema \cite{Aad:2015hea}. La experiencia ganada en las contribuciones realizadas
en el estadio inicial del doctorado, aportando al análisis en los estados
finales $\gam\gam+\met$ \cite{Aad2012519,ATLAS-CONF-2014-001} y
$\gam+e/\mu+\etmiss$ \cite{ATLAS-CONF-2012-144}, resultó valiosa para el
desarrollo del análisis que constituyó la investigación central de esta tesis.

%% El origen del trabajo se remonta al a\~no cuando se puso en marcha el LHC,
%% comenzando con el estudio de la eficiencia de los trigger de fotones utilizados
%% en la búsqueda del bosón de Higgs y en algunas búsquedas en extensiones del SM
%% que incluian fotones en el estado final.

%% Las primeras contribuciones en un
%% analisis fueron en la estimación del fondo para la busqueda de SUSY con dos
%% fotones en el estado final.

%% Los estudios previos al an?alisis de datos realizados con simulaciones
%% permitieron establecer la selecci?on ?optima de eventos, mientras que el muy
%% buen acuerdo en la forma de los dep?ositos en el calorímetro entre los datos
%% y las simulaciones permiti?o su validacio?n.

El mayor desafío para la realización de cualquier descubrimiento de nueva física
es la definición de una región en el espacio de observables donde se espera que
domine la señal por sobre el fondo del {\SM} y la estimación de dicho fondo en
esa región. En esta tesis se definieron dos regiones de señal: La primera,
{\SRL}, motivada por el decaimiento de gluinos en neutralinos de baja/media
masa, cuyos eventos están caracterizados por una gran multiplicidad de jets,
mientras que la energía del fotón y la energía faltante quedan determinadas por
la masa del neutralino. La otra, {\SRH}, destinada a cubrir el espacio de
parámetros en el que la masa del neutralino y el gluino es similar, eventos que
se caracterizan por un fotón de alto {\pt} y una gran cantidad de {\met}.

Para determinar los fondos contaminantes fue necesario el uso de diversas
simulaciones Monte Carlo y el desarrollo de métodos específicos para estimar fondos a
partir de los propios datos de colisiones.

El tratamiento estadístico de los datos merece también una mención aparte, dado
que constituye un aspecto fundamental en el análisis de datos en experimentos de
altas energías. Muchos de los métodos estadísticos utilizados fueron
especialmente desarrollados para las búsquedas de nueva física en ATLAS, como la
presentada en esta tesis.

Otro de los ingredientes importantes del análisis de datos lo constituyen las
incertezas sistemáticas que afectan la señal y la determinación del fondo. Se
incluyeron los efectos provenientes de las limitaciones de la física dentro de
las simulaciones del detector, aquellos originados en los métodos de
reconstrucción e identificación de las partículas, como también las incertezas
provenientes de las predicciones teóricas.

Todos estos ingredientes necesarios para el análisis fueron incorporados al
marco estadístico utilizado a fin de obtener los resultados finales. En cada
región de señal se observaron 2 eventos con una predicción para el fondo de
$1.27\pm0.43$ y $0.84\pm 0.38$ para la región {\SRL} y {\SRH}, respectivamente.

%% El fondo estimado en cada
%% region de senal resultó en 1.27 y 0.84 eventos para {\SRL} y {\SRH}, respectivamente.

Dado que no se observó un exceso significativo por sobre las predicciones del
{\SM}, se establecieron límites a las posibles contribuciones de nueva física.
Estos limites a 95\% CL en el número de eventos provenientes de nueva física son
de 5.5 y 5.6 para las dos regiones de señal definidas en el análisis, los cuales
se corresponden a limites en la sección eficaz visible de 0.27 y 0.28 fb.

%% Los limites para cadaresultando en un limite en la sección eficaz visible de 0.27 y
%% 0.28 fb, para las dos regiones de señal definidas en el análisis.

Adicionalmente, se realizó una interpretación de los resultados en el contexto
de un modelo de GGM SUSY con el estado final que motivó esta búsqueda, excluyendo
a 95 \% {\cl} la producción de gluinos con una masa de hasta 1.25 \tev, dependiendo de la masa
del {\ninoone}.

%% Para
%% masas del neutralino $m_{\ninoone}< 840 \gev$, se excluye la producción de
%% gluinos con una masa minima (maxima) de 1190 (1320) \gev a 95 \% CL. Para
%% gluinos con masa menor al \tev, se excluye un neutralino NLSP con masas entre
%% 150 {\gev} y $m_{\gluino}-m_{\ninoone} > 50 \gev$ a 95\% CL.

En paralelo con la edición de esta tesis, se está llevando a cabo el análisis de
búsqueda de nueva física en el estado final de un fotón de alto {\pt}, jets y
energía faltante con el conjunto total de datos recolectados por ATLAS durante
el a\~no 2015, correspondiente a una luminosidad integrada de $\sim
\unit[3]{fb^{-1}}$ a $\sqrt{s}=13 \tev$. La operación del LHC a
mayores energías incrementa la sección eficaz de producción de partículas
supersimétricas, de modo se espera que estas nuevas investigaciones extiendan la región de
sensibilidad y permitan tener, quizas, evidencias de presencia de se\~nales en la
región de gluinos de alta masa.
