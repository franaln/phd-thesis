\thispagestyle{plain}
\begin{center}
  \Large
  \textbf{Búsqueda de Supersimetría en eventos con un fotón,
      jets y energía faltante con el detector ATLAS}

  \vspace{0.4cm}
  \large
  %% Thesis Subtitle

  \vspace{0.4cm}
  Francisco Alonso

  \vspace{2cm}
  \textbf{Resumen}
\end{center}

Supersimetría (SUSY) es una de las teorías con mayor motivación teórica para
física más allá del {\SM}, proporcionando un marco para la unificación de la
física de partículas y la gravedad (gobernada por la escala de
energía de Planck). Dado que las partículas supersimétricas no han sido
observadas, SUSY debe ser una simetría rota. La fenomenología de SUSY está
ampliamente determinada por el mecanismo de rompimiento de la supersimetría. Los
modelos GGM en los que el rompimiento está mediado por los campos de gauge usuales
del {\SM}, brindan un escenario propicio para la búsqueda de SUSY en el
LHC con espectros de masas y decaimientos característicos. En esta Tesis se
presenta la primer búsqueda de nueva física en un estado final con un fotón energético, jets y
gran cantidad de energía faltante en colisiones protón-protón a una energía de
centro de masa de 8 \tev en el LHC. El análisis fue realizado utilizando todos
los datos recolectados por el detector ATLAS del LHC durante el a\~no 2012, que
corresponden a una luminosidad total integrada de 20.3 \ifb. No se observó un
exceso de eventos por sobre los predichos por el {\SM}, por lo cual se
estableció un limite superior a 95 \% CL al número de eventos provenientes de
nueva física para este estado final. Adicionalmente, los resultados fueron
interpretados en el contexto de un modelo de GGM SUSY, considerando la
producción de un neutralino mezcla bino-higgsino, resultando en los limites más
estrictos al presente.
