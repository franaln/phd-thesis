\chapter*{Introducción}
\addcontentsline{toc}{chapter}{Introducción}

Para explorar la región de energías del TeV, en el laboratorio CERN se
construyó el Gran Colisionador de Hadrones (LHC) \cite{Evans:1129806},
diseñado para colisionar protones a una energía de centro de masa máxima de
$\sqrt{s} = 14 \tev$ y una luminosidad que excede los $\unit[10^{34}]{cm^{-2}s^{-1}}$.
El LHC se puso en funcionamiento en el a\~no 2010, marcando el inicio de
una nueva era en el entendimiento de la naturaleza más fundamental y la
búsqueda de nuevas partículas e interacciones. En el primer período de
investigación del LHC, denominado \emph{Run 1}, funcionó a $\sqrt{s}=7\tev$ durante el
primer a\~no, aumentando a
$\sqrt{s}=8\tev$ a partir del a\~no 2012 y hasta principios de 2013. Una vez
realizadas las mejoras necesarias para alcanzar la energía y luminosidad
nominales, en el 2015 se dio inicio al llamado \emph{Run 2}, con una energía de centro
de masa de $\sqrt{s} = 13 \tev$.

Uno de los detectores multipropósito del LHC es ATLAS (\emph{A Torodial LHC
  AparatuS}) \cite{atlas}, el detector de partículas de mayor volumen construido
al presente (de $\unit[25]{m}$ de diámetro, $\unit[45]{m}$ de largo y con un
peso de 7000 toneladas), fue dise\~nado para estudiar un amplio espectro de
fen\'omenos físicos en la escala de energía del {\tev}. Los haces de partículas
del LHC colisionan en el centro de ATLAS generando miles de nuevas partículas
desde el punto de interacción. Para la detección de partículas cargadas, ATLAS
posee un complejo detector de trazas, inmerso en un campo magnético de
$\unit[2]{T}$ para curvar sus trayectorias a fin de medir sus momentos con una
precisión sin precedentes. Contiene un calorímetro electromagnético con alta
granularidad que permite medir con gran resolución la posición y energía de
electrones y fotones, y un calorímetro hadrónico destinado a la detección de las
cascadas hadrónicas (\emph{jets}) y realizar medidas de la energía faltante,
ambos cubriendo herméticamente la región de pseudorapidez $|\eta|<2.5$. El
detector se complementa con un espectrómetro de muones con un un sistema
toroidal que produce un campo magnético de hasta $\unit[4]{T}$, el cual le da el
gran tamaño a ATLAS.

Uno de los grandes logros del siglo XX ha sido la revelación de teorías que
describen todos los procesos de la naturaleza en términos de principios
elegantes basados en consideraciones de simetría. El {\SM} (SM)
\cite{PhysRevLett.19.1264,PhysRev.127.965,Glashow1961579} de las partículas elementales y sus interacciones proporciona el marco
indiscutible de la física de partículas. El acuerdo entre sus predicciones y
los datos experimentales es excelente, en algunas casos con una precisión
mayor al 1\%. Una de las predicciones críticas del SM está relacionada con
el mecanismo de rompimiento espontáneo de la simetría electrodébil,
necesaria para explicar el origen de las masas de los bosones de gauge $W$ y $Z$,
mediadores de la fuerza débil, y de los fermiones \cite{PhysRevLett.13.321,PhysRevLett.13.508}.
Este mecanismo
introduce un campo escalar complejo, lo que lleva a la predicción de una
nueva partícula escalar, el bosón de Higgs. A pesar de los éxitos del SM, se
cuenta con numerosos indicios empíricos y teóricos que promueven la
existencia de nueva física a la escala del TeV y que conducen a interpretar al
SM como el límite de bajas energías de alguna teoría que incluya nuevas
interacciones. Esto sugeriría que el SM sería una teoría efectiva a bajas
energías, existiendo un aspecto formal que hace a su consistencia interna: el
potencial de Higgs es inestable ante correcciones radiativas, problema que pone
de manifiesto interrogantes respecto del origen de la masa y que se conoce
bajo el nombre de \emph{problema de la jerarquía}, cuestión aún no
resuelta.

Uno de los resultados más importantes obtenidos con datos recolectados por ATLAS ha sido el
histórico descubrimiento del bosón de Higgs\cite{Aad:2012tfa} que permitió
completar el SM con el entendimiento del mecanismo del rompimiento electrodébil.
Este logro experimental ha marcado el inicio de una nueva etapa, cambiando los
objetivos de las investigaciones centradas en el descubrimiento del bosón de
Higgs, a la fase de búsqueda de nuevas partículas e interacciones.

Supersimetría (SUSY)
\cite{Miyazawa:1966,Ramond:1971gb,Golfand:1971iw,Neveu:1971rx,Neveu:1971iv,Gervais:1971ji,Volkov:1973ix,Wess:1973kz,Wess:1974tw}
es una de las teorías con mayor motivación teórica para física más allá del
{\SM}. SUSY se presenta particularmente interesante ya que provee una
solución natural a los dilemas teóricos del SM.
En su realización mínima (el MSSM), se introduce
una nueva partícula por cada bosón y fermión en el SM. Dado que
estas partículas no han sido observadas experimentalmente, la nueva simetría debe
estar rota en la naturaleza, presumiblemente a la escala del {\tev} (a fin de solucionar el
llamado problema de \emph{fine-tuning}). El MSSM posee
105 parámetros libres, y para reducir este
número, se debe suponer algún mecanismo de ruptura de la simetría.
Los modelos de SUSY con rompimiento de supersimétria con
mediación por campos de gauge (GMSB)
\cite{Dine:1981gu,AlvarezGaume:1981wy,Nappi:1982hm,Dine:1993yw,Dine:1994vc,Dine:1995ag}
suponen un sector oculto en el cual la supersimetría se rompe y este rompimiento
de la simetría se comunica con el sector visible a través de las interacciones
usuales de bosones de gauge del SM. En estos modelos la partícula supersimétrica más liviana
(LSP) es el gravitino (\gravino), el cual es muy liviano, y
bajo ciertas condiciones resulta en un candidato viable de materia oscura. El
espectro de masas de las partículas supersimétricas, y en particular
la naturaleza de la segunda partícula más liviana (NLSP), que resulta ser el
neutralino más liviano {\ninoone} para la mayor parte del espacio de parámetros,
dictaminan las partículas observables en el detector y los posibles canales
de búsqueda.
En a\~nos recientes, el esfuerzo para formular GMSB de una
forma menos dependiente de los modelos llevó al desarrollo de lo que se conoce
como \emph{General Gauge Mediation} (GGM) \cite{GGM} y su fenomenología comprende
una gran variedad de estados finales, que motivaron el estudio realizado en esta
Tesis.

%% Supersymmetry (SUSY)
%% is a theoretically well motivated candidate for physics beyond the Standard
%% Model (SM). If strongly interacting supersymmetric particles, i.e. squarks and
%% gluinos, are present at the TeV-scale, such particles should be produced already
%% in the 7 TeV collisions at the Large Hadron Collider (LHC).
%Si las partículas supersimétricas que interactúa
%fuertemente, es decir, squarks y gluinos, están presentes en la escala del {\tev},
%deben producirse en colisiones de protones en el LHC.

%% Theories of Gauge
%% Mediated Symmetry Breaking (GMSB)
%% \cite{Dine:1981gu,AlvarezGaume:1981wy,Nappi:1982hm,Dine:1993yw,
%%   Dine:1994vc,Dine:1995ag} presume a hidden sector in which supersymmetry is
%% broken and the symmetry breaking is communicated to the visible sectors through
%% Standard Model gauge boson interactions.

%% Such theories are especially attractive
%% because the hypothesis of an intermediate hidden sector suppresses the magnitude
%% of flavor-changing neutral currents. The lightest supersymmetric particle (LSP)
%% in GMSB is the ultra-light gravitino (\gravino), which under certain
%% circumstances is a viable dark matter candidate
%% \cite{Goldberg:1983nd,Ellis:1983ew}. The \gravino\ has a derivative coupling to
%% each particle and its superpartner with an interaction strength inversely
%% proportional to $\sqrt{F}$, where $F$ is a vacuum expectation value of an
%% auxiliary field which determines the magnitude of supersymmetry breaking in the
%% vacuum state.

%% The phenomenology of GMSB models is determined by the nature of the
%% next-to-lightest supersymmetric particle (NLSP), which for a large part of the
%% GMSB parameter space is the lightest neutralino \ninoone.

%% Neutralinos are mixtures of gaugino ($\tilde{B}$, $\tilde{W}^{0}$) and higgsino
%% ($\tilde{H}^{0}_{u},\tilde{H}^{0}_{d}$) eigenstates, and therefore the lightest
%% neutralino decays to a \gravino\ and either a \gam, Z, or h. If the \ninoone\ is
%% bino-like, the main decay mode is $\ninoone\to\gam\gravino$. If the \ninoone\ is
%% higgsino-like, it decays as $\ninoone\to h\gravino$. In addition, since the
%% longitudinal polarisation component of the Z boson is also a Goldstone mode of
%% the Higgs field, a higgsino-like neutralino can also decay as $\ninoone\to
%% Z\gravino$. Consequently, a \ninoone\ pair produced in a collider can give rise
%% to the diboson final states (hh, h\gam, hZ, Z\gam, ZZ, \gam\gam) + \etmiss.
%% Several searches for these signatures have been performed at the Tevatron
%% \cite{Abazov:2007ag,Buescher:2005he} and the LHC
%% \cite{Aad:2012zza,Aad:2012jva,Aad:2011kz,Aad2012519,leptonphoton7,Chatrchyan:2011wc,Chatrchyan:2011ah,tagkey2015503}.
%% The scenarios considered in this analysis are the ones with only one \gam\ %i.e.
%% $\g Z$ and $\g h$, plus two \gravino\ in the final state, in the case
%% \ninoone\ is a bino-higgsino admixture (\Fig \ref{fig:GGM_diagrams} (left)).
%% Further details on the signal models and benchmark points considered in this
%% study are given in \Sec \ref{sec:sig_samples}.

%% In recent years, the effort to formulate GMSB in a model-independent way has led
%% to the development of general gauge mediation (GGM)
%% \cite{Meade:2008wd,Buican:2008ws}. GGM includes an observable sector with all
%% the MSSM fields, together with a hidden sector that contains the source of SUSY
%% breaking. In GGM, there need not be any hierarchy between colored and uncolored
%% states, and therefore there is no theoretical constraint on the colored-states
%% mass, thus raising the feasibility of GGM discovery even with early LHC data.


%% This work describes the search for events with an isolated high-\pt\ photon,
%% jets and large \etmiss, in the full dataset of pp collisions at $\sqrt{s}=8\tev$
%% recorded in 2012 with the ATLAS detector at the LHC, corresponding to a total
%% integrated luminosity of 20.3 \ifb. This signature is complementary to other
%% ATLAS searches in \gam\gam+\etmiss \cite{Aad2012519,ATLAS-CONF-2014-001},
%% $\gam+e/\mu+\etmiss$ \cite{ATLAS-CONF-2012-144}, $\gam+b+\etmiss$
%% \cite{Aad:2012jva}, and $Z+\etmiss$ \cite{ATLAS-CONF-2012-152} final states. CMS
%% has also performed a similar search
%% \cite{CMS-PAS-SUS-12-018,CMS-PAS-SUS-14-004}, although in the case of a pure
%% bino or wino \ninoone\ state.

Esta Tesis describe la búsqueda de SUSY en el marco de los modelos GGM con
un fotón energético aislado, jets
y gran cantidad de energía faltante en el estado final, con los datos recolectados en colisiones $pp$
a una energía de centro de masa $\sqrt{s} = 8 \tev$ por el detector
ATLAS del LHC durante el a\~no 2012 correspondientes a una luminosidad total
integrada de $20.3 \ifb$. Este estado final es complementario a otras búsquedas
realizadas en ATLAS con estados finales de $\gam\gam+\met$ \cite{Aad2012519,ATLAS-CONF-2014-001},
$\gam+e/\mu+\etmiss$ \cite{ATLAS-CONF-2012-144}, $\gam+b+\etmiss$
\cite{Aad:2012jva}, y $Z+\etmiss$ \cite{ATLAS-CONF-2012-152}.
También se han realizado búsquedas similares en CMS \cite{CMS-PAS-SUS-12-018,CMS-PAS-SUS-14-004},
aunque solo para el caso de neutralinos puramente bino o wino.


% Estructura de la tesis
La Tesis esta estructurada en tres partes. En la primera parte se describe el
marco teórico que motiva el análisis realizado. Este primer capítulo incluye en
la \cref{cap:sm} los conceptos fundamentales del {\SM}, haciendo énfasis en el
buen acuerdo con los experimentos actuales de altas energías, finalizando con
algunos de los problemas teóricos y experimentales que presenta y que motivan
las teorías de nueva física. En la \cref{cap:susy} se describen conceptos de
Supersimetría, y en especial los modelos GMSB, en cuyo contexto se realizó el
análisis presentado en este trabajo.

La segunda parte consiste en la descripción del experimento. En el
\cref{cap:detector} se describe el LHC y el detector ATLAS, mientras que en el
\cref{cap:objetos} se presentan los métodos utilizados para la reconstrucción e
identificación de las partículas con el detector.

La tercera parte conforma la parte central de la Tesis y describe el análisis
específico realizado. En el \cref{cap:estadistica} se explican los conceptos
estadísticos fundamentales necesarios para la búsqueda de nueva física y los
métodos desarrollados durante los últimos a\~nos para los experimentos del LHC.
La estrategia general del análisis se describe en el \cref{cap:estrategia} en
donde se discute el modelo estadístico utilizado y la necesidad de definir
regiones de señal, control y validación, en particular para la determinación del
fondo contaminante de la señal de nueva física buscada.

En el \cref{cap:simulaciones} se presenta el modelo de SUSY que motiva el
estudio de esta Tesis. También se presentan los detalles de las simulaciones
Monte Carlo empleadas para la generación de los procesos del SM que conforman el
fondo del análisis, y de la señal de SUSY.

La definición y optimización de las regiones de señal se describe en el
\cref{cap:seleccion}, mientras que en el \cref{cap:fondos} se presentan los
métodos utilizados para la estimación del fondo en estas regiones.

Finalmente, se discuten las incertezas sistemáticas que afectan las medidas
realizadas y los resultados obtenidos de la investigación realizada en el
\cref{cap:resultados}. Las conclusiones finales de esta Tesis se presentan en el
\cref{cap:conclusiones}.
