\chapter*{Introducción}
\addcontentsline{toc}{chapter}{Introducción}

Para explorar la región de energías del TeV, en el laboratorio CERN se construyó
el Gran Colisionador de Hadrones (LHC) \cite{Evans:1129806}, diseñado para
colisionar protones a una energía de centro de masa máxima de $\sqrt{s} = 14
\tev$ y una luminosidad que excede los $\unit[10^{34}]{cm^{-2}s^{-1}}$. El LHC
se puso en funcionamiento en el a\~no 2010, marcando el inicio de una nueva era
en el entendimiento de la naturaleza más fundamental y la búsqueda de nuevas
partículas e interacciones. En el primer período de investigación del LHC,
denominado \emph{Run} 1, funcionó a $\sqrt{s}=7\tev$ durante el primer a\~no,
aumentando a $\sqrt{s}=8\tev$ a partir del a\~no 2012 y hasta principios de
2013. Una vez realizadas las mejoras necesarias para alcanzar la energía y
luminosidad nominales, en el 2015 se dio inicio al llamado \emph{Run} 2, con una
energía de centro de masa de $\sqrt{s} = 13 \tev$.

Uno de los detectores multipropósito del LHC es ATLAS (\emph{A Torodial LHC AparatuS})
\cite{atlas}, el detector de partículas de mayor volumen construido
al presente (de $\unit[25]{m}$ de diámetro, $\unit[45]{m}$ de largo y con un
peso de 7000 toneladas), que fue dise\~nado para estudiar un amplio espectro de
fenómenos físicos en la escala de energía del {\tev}.
Los haces de partículas del LHC colisionan en el centro de ATLAS generando miles
de partículas desde el punto de interacción.
Para la reconstrucción de las trazas de las partículas cargadas y los vértices
de interacción ATLAS utiliza la información de un detector de píxeles, un
detector de silicio y un detector de radiación de transición, que conjuntamente
conforman el detector interno. El detector interno tiene una cobertura azimutal
completa en la región de pseudo-rapidez $\abseta < 2.5$, y se encuentra inmerso
en un campo magnético axial de $\unit[2]{T}$, el cual permite la reconstrucción
del momento transverso de las partículas cargadas.
La energía de los fotones y electrones es medida en el calorímetro
electromagnético (EM), un detector de muestreo de argón líquido que cubre
la región $\abseta < 3.2$. Para $\abseta < 2.5$ el calorímetro EM está dividido en tres
capas longitudinales. La primera capa tiene una segmentación muy fina para
facilitar la separación entre fotones y hadrones neutros, y para poder medir la
dirección de las lluvias electromagnéticas, mientras que la mayor parte de la
energía se deposita en la segunda capa. La energía de
los jets es medida en los calorímetros EM y hadrónico. El calorímetro
hadrónico está dividido en tres subregiones. La región central consiste en azulejos
de centelleo activos y absorbentes de acero, mientras que las otras regiones
se basan en tecnología de argón líquido.
El calorímetro se encuentra rodeado por el espectrómetro de muones, el cual está
inmerso en un campo magnético provisto por un sistema toroidal, y permite la reconstrucción
y la determinación del momento de los muones en la región $\abseta < 2.7$.

Los eventos son retenidos para su posterior análisis utilizando un sistema de
\emph{trigger} de tres niveles \cite{Aad:2012xs} que identifica eventos consistentes
con topologías de interés definidas previamente. Los algoritmos del nivel 1 están
implementados en hardware, y utilizan solo una parte de la información del detector para
reducir la tasa de eventos de $\unit[\mathcal{O}(1)]{GHz}$ a menos de $\unit[75]{kHz}$. Los dos niveles siguientes,
basados en software, utilizan toda la información del detector para refinar la selección
de eventos, reduciendo finalmente la tasa de eventos a menos de  $\unit[400]{Hz}$.

Uno de los grandes logros del siglo XX ha sido la revelación de teorías que
describen todos los procesos de la naturaleza en términos de principios
elegantes basados en consideraciones de simetría. El {\SM} (SM, del inglés \emph{Standard Model})
\cite{PhysRevLett.19.1264,PhysRev.127.965,Glashow1961579} de las partículas
elementales y sus interacciones proporciona el marco indiscutible de la física
de partículas actual. El acuerdo entre sus predicciones y los datos
experimentales es excelente, en algunas casos con una precisión mayor al 1\%.
Una de las predicciones críticas del SM está relacionada con el mecanismo de
rompimiento espontáneo de la simetría electrodébil, necesaria para explicar el
origen de las masas de los bosones de gauge $W$ y $Z$, mediadores de la fuerza
débil, y de los fermiones \cite{PhysRevLett.13.321,PhysRevLett.13.508}. Este
mecanismo, denominado mecanismo de Brout-Englert-Higgs, introduce un campo escalar complejo
que lleva a la predicción de
una nueva partícula escalar, el bosón de Higgs. Uno de los resultados más
importantes obtenidos con datos recolectados por ATLAS ha sido el histórico
descubrimiento de una partícula consistente con el bosón predicho \cite{Aad:2012tfa}.
A pesar de los éxitos del SM, se cuenta con numerosos
indicios empíricos y teóricos que promueven la existencia de nueva física a la
escala del TeV y que conducen a interpretar al SM como el límite de bajas
energías de alguna teoría que incluya nuevas interacciones.
Uno de los problemas del SM se conoce como <<problema de jerarquía>> y está
relacionado a la gran diferencia entre la escala electrodébil y la escala de Planck.
%% El
%% potencial de Higgs es inestable ante correcciones radiativas, problema que pone
%% de manifiesto interrogantes respecto del origen de la masa.

La observación del bosón de Higgs en los experimentos del LHC ha marcado el inicio de una
nueva etapa, cambiando los objetivos de las investigaciones centradas en su búsqueda, a la
fase de búsqueda de nuevas partículas e interacciones.
La supersimetría (SUSY)
\cite{Miyazawa:1966,Ramond:1971gb,Golfand:1971iw,Neveu:1971rx,Neveu:1971iv,Gervais:1971ji,Volkov:1973ix,Wess:1973kz,Wess:1974tw}
es una de las teorías con mayor motivación teórica para física más allá del
{\SM}. SUSY se presenta particularmente interesante ya que provee una
solución natural a los dilemas teóricos del SM.
En su realización mínima (el MSSM), se introduce
una nueva partícula por cada bosón y fermión en el SM. Dado que
estas partículas no han sido observadas experimentalmente, la nueva simetría debe
estar rota en la naturaleza, presumiblemente a la escala del {\tev} (a fin de solucionar el
llamado problema de \emph{fine-tuning}). El MSSM posee
105 parámetros libres que se reducen suponiendo algún mecanismo de ruptura de la simetría.
Los modelos de SUSY con rompimiento de supersimetría con
mediación por campos de gauge (GMSB)
\cite{Dine:1981gu,AlvarezGaume:1981wy,Nappi:1982hm,Dine:1993yw,Dine:1994vc,Dine:1995ag}
suponen un sector oculto en el cual se rompe la supersimetría y este rompimiento
se comunica al sector visible a través de las interacciones
usuales de bosones de gauge del SM. En estos modelos la partícula supersimétrica más liviana
(LSP) es el gravitino (\gravino), el cual es muy liviano, y
bajo ciertas condiciones resulta en un candidato viable de materia oscura. El
espectro de masas de las partículas supersimétricas, y en particular
la naturaleza de la segunda partícula más liviana (NLSP) que resulta ser el
neutralino más liviano {\ninoone} para la mayor parte del espacio de parámetros,
dictamina las partículas observables en el detector y los posibles canales
de búsqueda.
En a\~nos recientes, el esfuerzo para formular GMSB de una
forma menos dependiente de los modelos llevó al desarrollo de lo que se conoce
como \emph{General Gauge Mediation} (GGM) \cite{GGM} y su fenomenología comprende
una gran variedad de estados finales, que motivaron el estudio realizado en esta
tesis.

Esta tesis describe la búsqueda de SUSY en el marco de los modelos GGM con
un fotón energético aislado, jets
y gran cantidad de energía faltante en el estado final, con los datos recolectados en colisiones
protón-protón ($pp$)
a una energía de centro de masa $\sqrt{s} = 8 \tev$ por el detector
ATLAS del LHC durante el \emph{Run} 1, correspondientes a una luminosidad total
integrada de 20.3 \ifb. Este estado final es complementario a otras búsquedas
realizadas en ATLAS con estados finales de $\gam\gam+\met$ \cite{Aad2012519,ATLAS-CONF-2014-001},
$\gam+e/\mu+\etmiss$ \cite{ATLAS-CONF-2012-144}, $\gam+b+\etmiss$
\cite{Aad:2012jva}, y $Z+\etmiss$ \cite{ATLAS-CONF-2012-152}.
También se han realizado búsquedas similares en CMS \cite{CMS-PAS-SUS-12-018,CMS-PAS-SUS-14-004},
aunque solo para el caso de neutralinos puramente bino o wino.

La tesis está estructurada en tres partes: (I) Motivación teórica, (II) Experimento
y (III) Análisis de datos.

La primera parte contiene una descripción del marco teórico en
el que se encuadra y que motiva el análisis realizado. El \cref{cap:teoria} incluye en la
\cref{cap:sm} los conceptos básicos del {\SM}, haciendo énfasis en el buen
acuerdo con los experimentos actuales de altas energías, finalizando con algunos
de los problemas teóricos y experimentales que presenta y que motivan las
teorías de nueva física. En la \cref{cap:susy} se describen conceptos de
Supersimetría, y en especial los modelos GMSB, en cuyo contexto se realizó el
análisis presentado en este trabajo.

La segunda parte consiste en la descripción del experimento. En el
\cref{cap:detector} se describe el LHC y, en particular, el detector ATLAS,
mientras que en el \cref{cap:objetos} se presentan los métodos utilizados para
la reconstrucción e identificación de las partículas con el detector.

La tercera parte conforma la parte central de la tesis y describe el análisis
específico realizado. En el \cref{cap:estadistica} se explican los conceptos
estadísticos fundamentales necesarios para la búsqueda de nueva física y los
métodos desarrollados durante los últimos a\~nos para los experimentos del LHC.
La estrategia general del análisis se describe en el \cref{cap:estrategia} en
donde se discute el modelo estadístico utilizado y la necesidad de definir
regiones de señal, control y validación, en particular para la determinación del
fondo contaminante de la señal de nueva física buscada.
En el \cref{cap:simulaciones} se presenta el modelo de SUSY que motiva el
estudio de esta tesis. También se presentan los detalles de las simulaciones
Monte Carlo empleadas para la generación de los procesos del SM que conforman el
fondo del análisis y las simulaciones de la señal de SUSY.
La definición y optimización de las regiones de señal se describe en el
\cref{cap:seleccion}, mientras que en el \cref{cap:fondos} se presentan los
métodos utilizados para la estimación del fondo en estas regiones.
Las incertezas sistemáticas que afectan las medidas
realizadas y los resultados obtenidos de esta investigación se discuten en el
\cref{cap:resultados}. Las conclusiones finales de esta tesis se presentan en el
\cref{cap:conclusiones}.
