\chapter*{Introducción}
\addcontentsline{toc}{chapter}{Introducción}

El Gran Colisionador de Hadrones (LHC) \cite{Evans:1129806} es un colisionador
de partículas, diseñado para colisionar protones a una energía de centro de masa
máxima de $\sqrt{s} = 14 \tev$, aunque durante el a\~no 2012 funcionó a $\sqrt{s}=8\tev$.

Uno de los detectores multiproposito del LHC es ATLAS (A Torodial LHC AparatuS)
\cite{atlas}, que ha sido disenado para estudiar un amplio espectro de fenomenos
fisicos explorando la escala del {\tev}. Uno de los resultados mas importantes obtenidos
con datos de ATLAS ha sido el historico descubrimiento del boson de Higgs\cite{Aad:2012tfa}
que permitio completar el SM con el entendimiento del mecanismo del rompimeinto
electrodebil. Entre los objetivos presentes de la Colaboracion ATLAS, se puede mencionar
la busqueda de nuevas particulas e interacciones.

%% Supersymmetry (SUSY)
%% is a theoretically well motivated candidate for physics beyond the Standard
%% Model (SM). If strongly interacting supersymmetric particles, i.e. squarks and
%% gluinos, are present at the TeV-scale, such particles should be produced already
%% in the 7 TeV collisions at the Large Hadron Collider (LHC).
Supersimetría (SUSY) \cite{Miyazawa:1966,Ramond:1971gb,Golfand:1971iw,Neveu:1971rx,Neveu:1971iv,Gervais:1971ji,Volkov:1973ix,Wess:1973kz,Wess:1974tw}
es una de las teorías con mayor motivación teórica para
física mas allá del {\SM}. Si las partículas supersimétricas que interactúa
fuertemente, es decir, squarks y gluinos, están presentes en la escala del {\tev},
deben producirse en colisiones de protones en el LHC.

%% Theories of Gauge
%% Mediated Symmetry Breaking (GMSB)
%% \cite{Dine:1981gu,AlvarezGaume:1981wy,Nappi:1982hm,Dine:1993yw,
%%   Dine:1994vc,Dine:1995ag} presume a hidden sector in which supersymmetry is
%% broken and the symmetry breaking is communicated to the visible sectors through
%% Standard Model gauge boson interactions.

%% Such theories are especially attractive
%% because the hypothesis of an intermediate hidden sector suppresses the magnitude
%% of flavor-changing neutral currents. The lightest supersymmetric particle (LSP)
%% in GMSB is the ultra-light gravitino (\gravino), which under certain
%% circumstances is a viable dark matter candidate
%% \cite{Goldberg:1983nd,Ellis:1983ew}. The \gravino\ has a derivative coupling to
%% each particle and its superpartner with an interaction strength inversely
%% proportional to $\sqrt{F}$, where $F$ is a vacuum expectation value of an
%% auxiliary field which determines the magnitude of supersymmetry breaking in the
%% vacuum state.

Las teorías de SUSY con rompimiento de supersimétria con mediación por campos de gauge (GMSB) \cite{Dine:1981gu,AlvarezGaume:1981wy,Nappi:1982hm,Dine:1993yw,Dine:1994vc,Dine:1995ag}
suponen un sector oculto en el cual la supersimetría se rompe y este rompimiento
de la simetría se comunica con el sector visible a través de las interacciones
usuales de bosones de gauge del SM.
%% Estas teorías son especialmente atractivas
%% debido a que la hipótesis de un sector oculto intermedio suprime...
La particula
supersimetrica mas liviana (LSP) en las teorias GMSB es el gravitino (\gravino),
el cual es muy liviano, y bajo ciertas condiciones resulta en un candidato
viable de materia oscura.


%% The phenomenology of GMSB models is determined by the nature of the
%% next-to-lightest supersymmetric particle (NLSP), which for a large part of the
%% GMSB parameter space is the lightest neutralino \ninoone.
La fenomenología de los modelos GMSB esta determinada por la naturaleza de la segunda
partícula más liviana (NLSP), que resulta ser el neutralino mas liviano {\ninoone} para
la mayor parte del espacio de parámetros.
%% Neutralinos are mixtures of gaugino ($\tilde{B}$, $\tilde{W}^{0}$) and higgsino
%% ($\tilde{H}^{0}_{u},\tilde{H}^{0}_{d}$) eigenstates, and therefore the lightest
%% neutralino decays to a \gravino\ and either a \gam, Z, or h. If the \ninoone\ is
%% bino-like, the main decay mode is $\ninoone\to\gam\gravino$. If the \ninoone\ is
%% higgsino-like, it decays as $\ninoone\to h\gravino$. In addition, since the
%% longitudinal polarisation component of the Z boson is also a Goldstone mode of
%% the Higgs field, a higgsino-like neutralino can also decay as $\ninoone\to
%% Z\gravino$. Consequently, a \ninoone\ pair produced in a collider can give rise
%% to the diboson final states (hh, h\gam, hZ, Z\gam, ZZ, \gam\gam) + \etmiss.
%% Several searches for these signatures have been performed at the Tevatron
%% \cite{Abazov:2007ag,Buescher:2005he} and the LHC
%% \cite{Aad:2012zza,Aad:2012jva,Aad:2011kz,Aad2012519,leptonphoton7,Chatrchyan:2011wc,Chatrchyan:2011ah,tagkey2015503}.
%% The scenarios considered in this analysis are the ones with only one \gam\ %i.e.
%% $\g Z$ and $\g h$, plus two \gravino\ in the final state, in the case
%% \ninoone\ is a bino-higgsino admixture (\Fig \ref{fig:GGM_diagrams} (left)).
%% Further details on the signal models and benchmark points considered in this
%% study are given in \Sec \ref{sec:sig_samples}.
Los neutralinos son mezcla de los estados de gaugino y higgsino, y por lo tanto el
neutralino mas liviano siempre decae a un gravitino y un $\gam, Z$ o $h$.


%% In recent years, the effort to formulate GMSB in a model-independent way has led
%% to the development of general gauge mediation (GGM)
%% \cite{Meade:2008wd,Buican:2008ws}. GGM includes an observable sector with all
%% the MSSM fields, together with a hidden sector that contains the source of SUSY
%% breaking. In GGM, there need not be any hierarchy between colored and uncolored
%% states, and therefore there is no theoretical constraint on the colored-states
%% mass, thus raising the feasibility of GGM discovery even with early LHC data.
En a\~nos recientes, el esfuerzo para formular GMSB de una forma menos dependiente de
los modelos llevó al desarrollo de lo que se conoce como \emph{General Gauge Mediation} (GGM).


%% This work describes the search for events with an isolated high-\pt\ photon,
%% jets and large \etmiss, in the full dataset of pp collisions at $\sqrt{s}=8\tev$
%% recorded in 2012 with the ATLAS detector at the LHC, corresponding to a total
%% integrated luminosity of 20.3 \ifb. This signature is complementary to other
%% ATLAS searches in \gam\gam+\etmiss \cite{Aad2012519,ATLAS-CONF-2014-001},
%% $\gam+e/\mu+\etmiss$ \cite{ATLAS-CONF-2012-144}, $\gam+b+\etmiss$
%% \cite{Aad:2012jva}, and $Z+\etmiss$ \cite{ATLAS-CONF-2012-152} final states. CMS
%% has also performed a similar search
%% \cite{CMS-PAS-SUS-12-018,CMS-PAS-SUS-14-004}, although in the case of a pure
%% bino or wino \ninoone\ state.
Esta Tesis describe la búsqueda de SUSY con un fotón energético aislado, jets
y gran cantidad de energía faltante en el estado final, con los datos colectados en colisiones $pp$
a una energía de centro de masa $\sqrt{s} = 8 \tev$ por el detector
ATLAS del LHC durante el a\~no 2012 correspondientes a una luminosidad total
integrada de $20.3 \ifb$. Este estado final es complementario a otras busquedas
realizadas en ATLAS en los estados finales $\gam\gam+\met$ \cite{Aad2012519,ATLAS-CONF-2014-001},
$\gam+e/\mu+\etmiss$ \cite{ATLAS-CONF-2012-144}, $\gam+b+\etmiss$
\cite{Aad:2012jva}, y $Z+\etmiss$ \cite{ATLAS-CONF-2012-152}.
También se han realizado búsquedas similares en CMS \cite{CMS-PAS-SUS-12-018,CMS-PAS-SUS-14-004},
aunque solo para el caso de neutralinos puramente bino o wino.


% Estructura de la tesis
La Tesis esta estructurada en tres partes. En la primer parte se describe el
marco teórico que motiva el análisis realizado. En el \cref{cap:sm} se describen
los conceptos fundamentales del {\SM}, haciendo énfasis en el buen acuerdo
con los experimentos actuales de altas energías, finalizando con
algunos de los problemas teóricos y experimentales que presenta y que motivan las
teorías de nueva física. En el \cref{cap:susy} se describe brevemente la
Supersimetría, y en especial los modelos GMSB, en cuyo contexto se realizó el análisis
presentado en este trabajo.

La segunda parte consiste en la descripción del experimento. En el
\cref{cap:detector} se describe el LHC y el detector ATLAS,
mientras que en el \cref{cap:objetos} se presentan los métodos utilizados para la
reconstrucción e identificación de las partículas con el detector.

La tercera parte conforma la parte central de la Tesis y describe el análisis
especifico realizado. En el \cref{cap:estadistica} se explican los conceptos estadísticos
fundamentales necesarios para la búsqueda de nueva física y los métodos
desarrollados durante los últimos a\~nos para los experimentos del LHC. La
estrategia general del análisis se describe en el \cref{cap:estrategia} en donde
se detalla el modelo estadístico utilizado y la necesidad de definir regiones de
señal, control y validación, en particular para la determinación del fondo contaminante
de la señal de nueva física buscada.

En el \cref{cap:simulaciones} se detalla el modelo de SUSY que motiva el estudio
de esta Tesis. También se
presentan los detalles de las simulaciones Monte Carlo empleadas para la
generación de los procesos del SM que conforman el fondo del análisis, y
de la señal de SUSY.

La definición y optimización de las regiones de señal se describe en el
\cref{cap:seleccion}, mientras que en el \cref{cap:fondos} se presentan los
métodos utilizados para la estimación del fondo en estas regiones. En el
\cref{cap:sistematicos} se detallan las incertezas sistemáticas, que afectan
a las medidas realizadas.

Finalmente, se presentan los resultados obtenidos de la investigación realizada
en el \cref{cap:resultados}. Las conclusiones finales de esta Tesis se
presentan en el \cref{cap:conclusiones}.
