\chapter{Estimaci\'on de los fondos} \label{cap:fondos}


\section{Estrategia General}

La contribuci\'on dominante de fondos del Modelo Estándar se espera que sea la producción de
{\wgam} y {\ttgam}, seguido de la producción de fotones con energía faltante instrumental.

Se definen tres regiones de control para determinar la normalización del MC de {\wgam} ({\CRW}),
{\ttgam} ({\CRT}) y fotones ({\CRQ}). Cada una de estas regiones de control son dominadas por cada uno
de estos fondos.

Los cortes de selecci\'on se mantienen lo mas similares posibles a la correspondiente SR para
minimizar el efecto de la extrapolaci\'on. Los fondos debidos a la mal identificaci\'on de
electrones y jets se estiman a partir de métodos basados en los propios datos observados,
descriptos en ...

\section{Electrones identificados como fotones} \label{sec:efakes}

Una contaminación significantes de eventos provenientes de procesos del SM
como W/Z + jets y {\tt} es esperado en los casos en que un electrón de alto
{\pt} sea mal identificado como un fotón. Este fondo es estimando pesando
el numero de eventos con un electrón observados en la región CSE por la
fracción de misidentificacion de fotón a electrón.

Esta muestra de electrones es obtenida como se describe en \cref{sec:CRs},
invirtiendo el rol de fotones y electrones en la selección de la región de señal.
Un electrón aislado de alto {\pt}  es requerido, y fotones de señal son vetados.

Para estimar la taza de misidentificacion de electrón fotón se utiliza el método
de tag and probe en una muestra de eventos de datos Zee que pasan el mismo trigger
de fotones que utiliza el análisis y la misma selección base (ver \cref{sec:event_baseline}).
Además, se aplica un corte de $\met  < 40 \gev$ para reducir la posible contaminación de fotones
reales de eventos de {\wgam}.

El electrón \emph{tag} se requiere que pase el criterio de identificación \texttt{tight++}
y que tenga $20 \gev < \pt < 125 \gev$.
El segundo candidato electromagnético (\emph{probe}) puede ser un electrón \texttt{tight++}
o un fotón tight, ambos con $\pt > 125\gev$ y satisfaciendo los requerimientos de aislamiento
que se describen en sec \ref{sec:obj_selection}.

Los valores de la masa de los pares de tag y prbe son guardados separadamente para tres casos:
dos electrones, un electrón y un fotón convertido, y un electrón y fotón no convertido.
En todas esas posibilidades existe una visible presencia del bisoña Z cerca de la masa del Z de 91 \gev.
Como el bisoña Z no puede decaer directamente en un electrón y un fotón, los eventos electrón-fotón
que aparecen bajo el pico del Z corresponden a electrones mal indetificados. Sin embargo, lo mismo
aplica a otras partículas que decaen en pares de electrones y pro lo tanto es necesario aplicar alguna
técnica para la substracción del fondo. Esta deberá también tener en cuenta la contaminación por el
fondo de random combinatorics.

La taza de misidentificacion puede estimarse entonces como:

\begin{equation}\label{eq:efakerate}
  f_{e\gam} = \frac{N_{e\gam}}{N_{ee}}
\end{equation}
%
donde $N_{e\gam}$ ($N_{ee}$) es el numero de pares electrón-fotón (electrón-electrón) encontrados
bajo el pico del Z en la distribución de la masa invariante, definida en el rango [81,101] \gev.
Para obtener estos números, la distribución de la masa invariante para los dos tipos de eventos
es ajustado con un modelo de señal+fondo para fotones convertidos y no convertidos, de forma
separada. Y esto se realiza en bines del eta de los probe. La baja estadística no permite calcular
con bineado en pt, pero se supone esta estimación conservativa ya que la taza de misid decrece con el pt
del electrón \cite{Kuhl:1604846}.

Como modelo de señal se utiliza un función CB+gaussian, mientras que para el fondo se utiliza un
polinomio de grado dos. Un ejemplo del ajuste se muestra en \cref{fig:invmass_pairs}, para
la selección inclusiva.

\begin{figure}[h]
  \begin{center}
    \includegraphics[width=0.45\textwidth]{figures/Fit_mee_efakes_Data_all}  \hfill
    \includegraphics[width=0.45\textwidth]{figures/Fit_meg_efakes_Data_all}
    \caption{Distribuciones de la masa invariante de los pares de electrón-electrón (izquierda) y electrón-fotón (derecha).
    También se puede ver el ajuste.}
    \label{fig:invmass_pairs}
  \end{center}
\end{figure}

La taza de misidentificacion se muestra en la \cref{fig:efake_eta}, como función
del {\abseta} del objeto probe para ``fotones'' convertidos y no convertidos.
Este factor crece con {\abseta}, lo cual esta correlacionada con el incremento en el
material del detector atravesada por los electrones y la larger
reconstruction rate of single-track converted photons.

Como verificación, el rate es comparado con la esperada de las simulaciones de eventos {\Zee}
generados con {\sherpa} and \powheg.
Un buen acuerdo fue encontrado para todos los casos dentro de las incertezas. Estas estimaciones
pueden verse en la \cref{tab:efake_eta}, y \cref{tab:efake_uc} para fotones convertidos
y no convertidos de forma separada.

\begin{figure}[h]
  \centering
  \includegraphics[width=0.45\textwidth]{figures/fegc_feta}
  \includegraphics[width=0.45\textwidth]{figures/fegu_feta}
  \caption{Probabilidad de que un electrón real sea reconstruido como un fotón convertido (izquierda)
    y un fotón no convertido (derecha), como función de la pseudo-rapidez del objeto probe. El valor
    estimado de datos es comparado con el valor esperado de simulaciones MC de eventos de {\Zee} utilizando
    dos generadores distintos.}
  \label{fig:efake_eta}
\end{figure}


\begin{table}[!h]
  \centering
  \caption{Probabilidad de que un electrón real sea reconstruido como un fotón, como función
    de la pseudo-rapidez del objeto probe. El valor estimado de datos es comparado con el valor esperado
    de simulaciones MC de eventos de {\Zee}, utilizando dos generadores distintos.}
  \begin{tabular}{cccc}
    \hline
    \hline
    & Data              & Sherpa \Zee         & Powheg \Zee \\
    \hline
    $0 < |\eta| < 0.8$    & $0.014 \pm 0.002$ & $0.012 \pm 0.001$ & $0.014 \pm 0.002$ \\
    $0.8 < |\eta| < 1.52$ & $0.018 \pm 0.003$ & $0.014 \pm 0.001$ & $0.011 \pm 0.003$ \\
    $1.52 < |\eta| < 2.5$ & $0.033 \pm 0.006$ & $0.027 \pm 0.002$ & $0.032 \pm 0.006$ \\
    Overall               & $0.019 \pm 0.001$ & $0.016 \pm 0.001$ & $0.017 \pm 0.002$ \\
    \hline
    \hline
  \end{tabular}
  \label{tab:efake_eta}
\end{table}

\begin{table}[!h]
  \centering
  \caption{Probabilidad de que un electrón real sea reconstruido como un fotón convertido o no
    convertido. El valor estimado de datos es comparado con el valor esperado
    de simulaciones MC de eventos de {\Zee}, utilizando dos generadores distintos.}
  \begin{tabular}{cccc}
    \hline
    \hline
    Overall       & Data              & Sherpa Zee        & Powheg Zee        \\
    \hline
    $f(e\to \gamma_u)$ & $0.007 \pm 0.001$ & $0.005 \pm 0.001$ & $0.005 \pm 0.001$ \\
    $f(e\to \gamma_c)$ & $0.013 \pm 0.001$ & $0.011 \pm 0.001$ & $0.011 \pm 0.002$ \\
    $f(e\to \gamma)$   & $0.019 \pm 0.001$ & $0.016 \pm 0.001$ & $0.017 \pm 0.002$ \\
    \hline
    \hline
  \end{tabular}
  \label{tab:efake_uc}
\end{table}

Para estimar la incerteza sistemática del método utilizado, el factor de misidentificacion fue
calculado variando el tamaño de la ventana de masa del $Z$, y aplicando o no la sustracción
del fondo. Como se muestra en la \cref{tab:efake_syst}, en el caso de no realizar la
sustracción del fondo, es donde se obtiene la mayor variación y por lo tanto se utiliza ese valor
como la incerteza sistemática del método.

\begin{table}[!h]
  \centering
  \caption{Probabilidad de que un electrón real sea reconstruido como un fotón
    convertido o no-convertido, para variaciones del método original.}
  \begin{tabular}{cccc}
    \hline
    \hline
     Systematics       &  $71 < m_{ee} < 111 \GeV$ & $86 < m_{ee} < 96$ & No background subtraction  \\
    \hline
    $f(e\to \gamma_u)$ & $0.007 \pm 0.001$ & $0.007 \pm 0.001$ & $0.012 \pm 0.001$ \\
    $f(e\to \gamma_c)$ & $0.013 \pm 0.001$ & $0.012 \pm 0.001$ & $0.012 \pm 0.001$ \\
    $f(e\to \gamma)$   & $0.019 \pm 0.001$ & $0.019 \pm 0.001$ & $0.024 \pm 0.001$ \\
    \hline
    \hline
  \end{tabular}
  \label{tab:efake_syst}
\end{table}


El valor estimado del factor de misidentificacion final es entonces el que se muestra en la
{\tab} \ref{tab:efake_final}.

\begin{table}[!h]
  \centering
  \caption{Valor estimado final para el factor de {\misid} electrón-fotón, como función de $\eta$.}
  \begin{tabular}{cc}
    \hline
    \hline
     Region                &  $f(e\to \gamma)$  \\
    \hline
      $0 < |\eta| < 0.8$     & $ \quad  0.014 \pm 0.002 \stat\ \pm 0.005 \syst\ $ \\
      $0.8 < |\eta| < 1.52$  & $ \quad  0.018 \pm 0.003 \stat\ \pm 0.004 \syst\ $ \\
      $1.52 < |\eta| < 2.5$  & $ \quad  0.033 \pm 0.006 \stat\ \pm 0.008 \syst\ $ \\
    \hline
  \end{tabular}
  \label{tab:efake_final}
\end{table}

%% FINAL ESTIMATION IN SIGNAL REGIONS
\subsubsection{Estimación final en las regiones de señal} \label{sec:efakes_estimation}

Como se explico anteriormente, el numero de eventos de etogam esperado en las regiones
de señal se obtiene pesando el numero de eventos observado en una región  de control CSE,
por el factor de misid calculado en las distintas regiones de $\eta$ (ver \cref{tab:efake_final}).
%% En la region CSE$_L$, there are 28 events observed, which then correspond to an estimation for this background of $N^\text{SR2}_\text{efakes} = 0.38\, \pm\, 0.07$.
%% In CSE3, given the harder \MET\ cut, there is no observed event in the electron sample for the data analised.

%number of electron events observed by the fake factor computed in data in three $\eta$ regions, [0, 0.8], [0.8, 1.52], [1.52, 2.5].

%% In SR2, only one electron event is observed, which then correspond to an estimation for this background of $N^\text{SR2}_\text{efakes} = 0.02\, \pm\, 0.02$.
%% In SR3, given the harder \MET\ cut, there is no observed event in the electron sample for the data analised.
%% The MC predicts this background to be indeed rather small (actually no event survived the SR3 selection with
%% the available statistics), which is consistent with the observed limit of $<0.02$ events. The estimation of
%% electron fake background for SR3 is then $N^\text{SR3}_\text{efakes} = 0.0^{+0.03}_{-0.0}$, which accomodates the total fake rate uncertainty.

%As a conservative estimate, the last observed yield in the \MET distribution (one event at 200\gev) is taken. This way, it translates to the same estimation as for SR2, after the \etogam\ fake factor is applied.


%The total number of electron events, summed over all $\eta$ bins is reported in \Tab \ref{tab:efake_yields}.
%From these, the number of \etogam\ events expected in the signal region is %$3.679\pm xx$,
%$0.828\pm yy$ and $0.679\pm zz$ events for SR2 and SR3 signal regions, respectively. Only statistical uncertainties are here considered. \tosolve{uncertainties!}%%The associated systematics are discussed in sec \ref{sec:syst_efakes}.
%
%\begin{table}[h!]
%  \centering
%    \caption{Number of misidentified electron events expected in the different signal regions. The unscaled number is weighted by
%    the electron-photon fake rate to get the final background yield from electron fakes in the three
%  analysis regions.}
%  \begin{tabular}{c|cc}
%    \hline
%    \hline
%    Signal region & Unscaled & Weighted  \\
%    \hline
%%    SR1 & $231$ & $3.679$ \\ %% \pm 0.251$ \\
%    SR2 & $57$ & $0.828$ \\ %% \pm 0.110$ \\
%    SR3 & $47$ & $0.679$ \\ %% \pm 0.100$ \\
%    \hline
%    \hline
%  \end{tabular}
%  \label{tab:efake_yields}
%\end{table}

\section{Jets identificados como fotones} \label{sec:jetfakes}

Los jets pueden ser mal identificados como fotones si fluctuan a uno o dos
{\pizero} con alto \pt, resultando en un objeto electromagnetico indistiguible
de un solo foton muy energetico.
ESte fondo proviene mayoritariamente de multijets, {\wjets} y eventos {\ttbar} decayendo
semi leptonicamente, y puede ser una fuente importante de contaminacion.
Como la tasa de {\misid} se espera que no este bien modelada por el MC, se la determina
utilizando un metodo a partir de los datos. La idea es utilizar las diferencias en la
distribucion de energia de aislamiento esperada para fotones reales y falsos, como
se describe a continaucion.

\subsection{Descripcion del metodo}\label{sec:jetfake_method}

Para evitar el grand fondo de jets, para seleccionar los fotones se utiliza un
critero de seleccion \emph{tight}, como se describe en \cref{sec:pho_obj}.
Esta seleccion es inclusive para fotones reales con una moderada contaminacion
de jets. Es por definicion tighter que el trigger de fotones utilizado para
la coleccion de los datos, por lo que hay una suficiente cantidad de candidatos
a fotones de jets que van a fallar la seleccion tight pero van a satisfacer una
seleccion intermedia. Estos jets photon-like, llamados \emph{pseudo-fotones},
son utiles para modelar los jets que pasan la seleccion total y la tasa de esta
contaminacion.
%Pseudo-photons are here defined as those passing the loose identification
%but failing (at least) one of a set of tight identification cuts. %, also known as {\it loose'-non-tight}.

La muestra de fotones que pasan los criterios de seleccion \emph{tight} ($N_\text{tight}$)
contiene, en general, fotones reales ($N_{\gamma}$) y falsos ($N_{j\to\gamma}$).
Estas dos contribuciones van a tener disitintas distribuciones de energia de aislamiento,
que puede ser explotada para estimar ambas contribuciones. Para hacer esto la distribucion
total de energia de aislamiento es ajustada a una combinacion de modelos de senial y fondo,
tambien tomado de los datos como se explica en \cref{sec:jfake_sig_template,sec:jfake_bkg_template}.
El numero de eventos de fotones fake que pasa la identificacion de fotones y el corte de
aislamiento puede ser estimado integrando la componente de fondo del ajuste sobre el rango
de la region de senal ($<5\GeV$). De esta forma la tasa de jets mal identificados como fotones,
$f_{j\to \gamma}$, resulta:

\begin{equation}\label{eq:jfake_formula}
    f_{j\to \gamma} = \frac{\int_{-\infty}^{5\gev} B(x)\, dx}{\int_{-\infty}^{5\gev} \left[S(x)+B(x)\right]\, dx}
\end{equation}
%
donde $x$ es la variable de energia de aislamiento (\etiso), y $S(x)$ y $B(x)$
son las distribuciones senal y fondo en el ajuste combinado.
Esta tasa de fake es estimada en una region de control, y luego utilizada para pesar la muestra
de fotones en la SR para estimar el numero de eventos provenientes de jets mal identificados Este
resultado se discute en \cref{sec:jet_fake_results}.

\subsection{Modelo de Señal} \label{sec:jfake_sig_template}

El modelo de senal se extrajo de eventos de datos {\Zee} usando el hecho que los
electrones y fotones tienen una senal similar en el calorimetro electromagnetico.
La muestra de electrones es obtenida de eventos que satisfacen el sigueinte conjutno
de cortes, despues de haber pasao la pre-seleccion descripta en \cref{sec:event_baseline}:

\begin{itemize}\itemsep0.1cm
\item[-] Trigger: \texttt{EF\_2e12Tvhi\_loose1} $\parallel$ \texttt{EF\_e24vhi\_medium1} $\parallel$ \texttt{EF\_e60\_medium1}.
\item[-] Dos electrones \emph{medium}, aislados, y con carga opuesta, con $\pt>50 \gev$ y $\pt>25 \gev$
\item[-] \MET\ $<40\gev$
\item[-] $81\gev<m_{ee}<101\gev$
\end{itemize}

Despues de aplicar los cortes anteriores, la contaminacion de fondos no provientes
del decaimiento del $Z$ es despreciable, en particular para eventos de {\ttbar}
resulta menor al $1\%$.

La {\fig} \ref{fig:isolation_vs_pt} muestra la distribcuion de la energia de aislamiento
para los electrones seleccionados, como funcion del {\pt} del electron en eventos de datos y
simulaciones MC.
De la figura es evidente que mientras las simulaciones presentan una distribcuion flat, hay
una clara dependencia del {\etiso} con el {\pt} en los datos.
Un ajuste lineal de los datos nos permite obtener un factor de correccion que luego es
aplicado para remover esta dependencia residual con el \pt. El valor del factor de
correcion obtenido en el ajuste en el rango $50<\pt=500 \gev$ es  $0.00262 \pm 0.00008$.
Hay que notar que este factor es solo aplicado para corregir los datos.
En la \cref{fig:isolation_wandwo_correction} se muestra una comparacion entre
datos y simulaciones del perfil de {\etiso} antes (izquierda) y despues (derecha) de aplicarles
la correccion. Se puede ver que despues de aplicar la correcion el acuerdo entre datos y MC
es mucho mejor.

\begin{figure}[h]
  \centering
  \includegraphics[width=0.6\textwidth]{figures/el_iso_vs_pt_wfit}
  \caption{Distribucion de {\etiso} para electrones vs. {\pt} de eventos {\Zee} en datos y MC.}
  \label{fig:isolation_vs_pt}
\end{figure}

\begin{figure}[h]
  \centering
  \includegraphics[width=0.49\textwidth]{figures/electron_iso_Zee_raw}
  \includegraphics[width=0.49\textwidth]{figures/electron_iso_Zee_corr}
  \caption{Comparacion datos/MC de la distribucion de {\etiso} de electrones provenientes de
    eventos {\Zee} (izquierda) y la correspondiente distribucion despues de aplicar la correccion
    por el {\pt} (derecha).}
  \label{fig:isolation_wandwo_correction}
\end{figure}

Para respaldar la estrategia de derivar el modelo de aislamiento de los fotones de
electrones se realizaron varios estudis. Una validacion importante del metodo consistio
en comparar la distribucion de {\etiso} de electrones provenientes de {\Zee} con la
El decaimiento radiativo del $Z$ provee un fuente de fotones puros. Los eventos se
seleccionaron requiriendo el siguiente conjunto de cortes, despues de la pre-seleccion:

\begin{itemize}\itemsep0.1cm
\item[-] Trigger: \texttt{EF\_2e12Tvhi\_loose1} $\parallel$ \texttt{EF\_e24vhi\_medium1} $\parallel$ \texttt{EF\_e60\_medium1}.
\item[-] Un foton \emph{tight} aislado con $\pt>25 \gev$.
\item[-] Dos electrones \emph{medium}, aislados, con carga opuesta, y con $\pt>50 \gev$ y $\pt>25 \gev$
\item[-] $\Delta R(\gamma,l)>0.7$
\item[-] \MET\ $<40\gev$
\item[-] $40\gev<m_{ee}<85\gev$
\item[-] $70\gev<m_{ee\gamma}<100\gev$
\end{itemize}

En la \cref{fig:photon_electron_iso} se presenta una comparacion entre
el modelo de fotones reaeles de decaimientos radiativos del $Z$ y electrones
de {\Zee}. El grafico a la derecha es obtenido despues de remover la dependencia
con el {\pt} con el factor de correccion como se describio anteriormente.
Puede verse de la {\fig} el buen acuerdo entre las distribuciones de {\etiso}
de fotones y electrones, en particular despues de aplicar la correcion, dando
confianza en la extrapolacion de usar electrones para modelar los fotones.
%% Vale la pena nota que el espectro de {\pt} de fotones tan bajo en estos
%% procesos (ver {\fig} \ref{fig:zllg_pt}) as compared to the generally broader electron \pt\ spectrum from Z decay. %of the electrons (see {\fig} \ref{fig:zeeg_pt}). \tosolve{add this last plot}

\begin{figure}[h]
  \begin{center}
    \includegraphics[width=0.49\textwidth]{figures/ph_el_data_raw}
    \includegraphics[width=0.49\textwidth]{figures/ph_el_correction}
    \caption{Comparison of isolation distributions  for real photons from radiative
      Z decays and electrons from \Zee\ before (left) and after the \pt\ correction
      is applied (right)}
    \label{fig:photon_electron_iso}
  \end{center}
\end{figure}

\begin{figure}[h]
  \begin{center}
    \includegraphics[width=0.49\textwidth]{figures/pt_sig_templates}
    \caption{Espectro del impulso transverso de fotones provenientes de decaimientos
      radiativos del Z, en datos.}
    \label{fig:zllg_pt}
  \end{center}
\end{figure}

Para estudiar el efecto To study the effect on the electron isolation template obtained from \Zee\ when using different
event topologies and kinematical regions, we present in \cref{fig:_electron_iso_HT},
the distributions for the case of different hadronic activity using a cut in \HT\ from simulations
on the left plot and from data \Zee\ on the right plot. As expected the electron template gets harder
and broader with \HT.  Unfortunately, the lack of statistic prevents us to apply a harder cut in \HT.
In any case, this effect will be consider as a possible source of systematic uncertainty.

\begin{figure}[h]
  \begin{center}
    \includegraphics[width=0.49\textwidth]{figures/electron_iso_ZeeHMC_corr}
    \includegraphics[width=0.49\textwidth]{figures/electron_iso_ZeeH_corr}
    \caption{Electron isolation dependency on \HT\ from MC (left) and data (right) \Zee}
    \label{fig:_electron_iso_HT}
  \end{center}
\end{figure}

\subsection{Modelo de fondo} \label{sec:jfake_bkg_template}

El modelo de fondo fue derivado de los datos, en eventos
que pasan todos los criterios de identificacion salvo los
criterios de identificacion \emph{tight} de fotones. La
muestra de pseudo-fotones

%% The background template was derived from the data, in events passing all signal requirements
%% but the tight photon ID cuts. The pseudo-photon sample selected this way should present an
%% isolation profile similar to that for the signal photons, allowing a low signal contamination
%% and high statistics. A dedicated study was done in MC to find the best set of tight ID cuts
%% to be reversed, in order to best model the expected isolation distribution for true hadronic fakes in the SR.
%% This is shown in {\fig} \ref{fig:jetfake_mc_data}, compared to the pseudo-photon sample in MC and data,
%% for two different ID definitions.
%% The loose-non-tight photons (LnT) are required to pass the loose ID, but fail at least one of
%% the tight cuts afterwards \footnote{For a detailed description of the photon ID variables
%%   please refer to \cite{ATLAS-CONF-2012-123}}. Despite of the larger statistics of the sample, it is clear that it
%% fails to model the background both in data and in the MC itself. A much better agreement is obtained for the loose'-non-tight (L'nT)
%% selection, in which the photons are requested to pass all the tight cuts but (at least) one on the narrow strips
%% variables ($F_\text{side}$, $w_{s3}$, $E_\text{ratio}$, $\Delta E$).
%% These variables are constructed with just a few central strips around the photon direction, and so are expected to be more
%% uncorrelated with the photon isolation, as it excludes the central cells to account for the photon self-energy
%% and covers a much wider region in space. This way, the pseudo-photon template mimics the tight photon shape and
%% allows the extrapolation to the signal region.

\begin{figure}[h]
  \begin{center}
    \includegraphics[width=0.49\textwidth]{figures/bkg_mc_pseudo_data_SR_l_ptbin}
    \includegraphics[width=0.49\textwidth]{figures/bkg_mc_pseudo_data_SR_lp_ptbin}
    \caption{Isolation distribution for truth MC photon fakes from hadronic decays, compared to pseudo-photon
      sample obtained in MC for background, obtained in the (right) nominal and (left) loose signal regions.}
  \label{fig:jetfake_mc_data}
  \end{center}
\end{figure}

%% Evenmore, as seen in {\fig} \ref{fig:jetfake_pseudo_data_pt}, the \pt\ spectrum of the L'nT photon candidates
%% better reproduce that expected for tight photons. No further \pt-reweighting is needed in this case.
%% Similarly, {\fig} \ref{fig:jetfake_pseudo_data_BE} shows the only slight shape difference for pseudo-photons
%% in the barrel and endcap regions.

%% Particularly for fake events, the extrapolation from the region used to derive the faking probability ratio and
%% the respective signal region might suffer from different topology and kinematics of the events in each case.
%% For this reason, the control region selection to derive the templates is kept close to the SR, as listed in {\tab} \ref{tab:cr_jetfake}.
%% An orthogonality requirement is placed on \MET ($50-150\gev$) and some SR cuts are loosened to retain statistics.
%% As expected, however, the larger the jet activity around the harder the photon isolation in the event. So the
%% \HT\ requirements can not be loosened much. This can be clearly observed in {\fig} \ref{fig:jetfake_pseudo_data_LR_VR},
%% for a loose (CRFJL: $>200 \gev$) and tight (CRFJ: $>600 \gev$) \HT\ selection. The latter is therefore used to
%% perform the combined fit and to compute the jet fake factor, as discussed in the next section.

\begin{table}[h!]
  \centering
  \caption{Control region definition for jet fake estimation.} %%The final estimate in SR for this background is obtained by weighting the yields in the control region (CRJ), by the
  %%$f_{j\to\gamma}$ fake factor derived in the CRFJ region. The numeric suffix indicates (when present) the SR associated to each region (SR2 or SR3).}
  \begin{tabular}{rcc}
    \hline \hline
    %%  & CRJ2(3) & CRFJ & CRFJL \\
    %% \hline %\hline
    %% $\pt(\text{pseudo}-\gamma_1)$ (\gev) $>$ &  125         &   125 & 125 \\
    %% N leptons                                &   0          &     0 &     0     \\
    %% \met (\gev)                              & $>$150 (300) &  $50<\met<150$        & $50<\met<150$      \\
    %% N jets $\ge$                             &    4 (2)     &   2       &   2     \\
    %% $\pt(j_1,2)$  (\gev)  $>$                & 100  (80)    &   100     & 100   \\
    %% $\dphijm >$                              & 0.4          &   0.4      &  0.4    \\
    %% \HT (\gev) $>$                           & - (800)      &   600    &  200   \\


         & CRFJ & CRFJL \\
    \hline %\hline
    $\pt(\text{pseudo}-\gamma_1)$ (\gev) $>$ &    125 & 125 \\
    N leptons                                &      0 &     0     \\
    \met (\gev)                              &   $50<\met<150$        & $50<\met<150$      \\
    N jets $\ge$                             &    2       &   2     \\
    $\pt(j_1,2)$  (\gev)  $>$                &    100     & 100   \\
    $\dphijm >$                              &    0.4      &  0.4    \\
    \HT (\gev) $>$                           &    600    &  200   \\
\hline \hline
  \end{tabular}
\label{tab:cr_jetfake}
\end{table}


\begin{figure}[h]
  \begin{center}
    \includegraphics[width=0.49\textwidth]{figures/bkg_data_pseudo_tight_data_VR}
    %% \caption{Distribución del {\pt} del foton para candidadtos
    %%   \emph{tight and pseudo-photon candidates, after full CRFJ selection.}
  \label{fig:jetfake_pseudo_data_pt}
  \end{center}
\end{figure}

\begin{figure}[h]
  \begin{center}
    \includegraphics[width=0.49\textwidth]{figures/bkg_pseudo_data_SREB_l}
    \includegraphics[width=0.49\textwidth]{figures/bkg_pseudo_data_SREB_lp}
    %% \caption{Isolation templates for loose-non-tight (left) and loose'-non-tight (right) pseudo-photon data,
    %%   after full CRFJ selection, in barrel and endcap regions.}
  \label{fig:jetfake_pseudo_data_BE}
  \end{center}
\end{figure}

\begin{figure}[h]
  \begin{center}
    \includegraphics[width=0.49\textwidth]{figures/bkg_pseudo_data_SR_VR_l}
    \includegraphics[width=0.49\textwidth]{figures/bkg_pseudo_data_SR_VR_lp}
    %% \caption{Isolation templates for loose-non-tight (left) and loose'-non-tight (right) pseudo-photon data,
    %%   after signal-like selection with a  loose (CRFJL: $>200 \gev$) and tight (CRFJ: $>600 \gev$) \HT\ cut.}
  \label{fig:jetfake_pseudo_data_LR_VR}
  \end{center}
\end{figure}

Finally, the background template from loose'-non-tight photons in the CRFJ region is found
to be well described by a Crystall Ball function. The data distribution and the fitted profile
is shown in \cref{fig:jetfake_sigbkg}. The parametrized template is used in the combined
fit to tight photon data to derive the jet fake rate.

%% \begin{figure}[h]
%%   \begin{center}
%%   \includegraphics[width=0.49\textwidth]{iso_fit_bkg_wpars}
%%   \caption{Fit to the observed isolation template for  loose'-non-tight pseudo-photon data in the VR.}
%%   \end{center}
%%   \label{fig:jetfake_bkg_template_fit}
%% \end{figure}

\subsection{Ajuste combinado y estimacion del fondo} \label{sec:jet_fake_results}

A template fit to both signal and background distributions is then used to model the shape of the photon isolation
for each contribution, as shown in \cref{fig:jetfake_sigbkg}, together with the fitted parameters.
%The corresponding $\chi^2$/dof are 2.4 and 8.8 for signal and background templates with 240 points.

La distribucion de {\etiso} para todos los eventos que pasan los criterios
de seleccion de la region relajada (CRFJ) es ajustada a una combinacion
lineal de los modelos de senal y fondo. El ajuste combinado y cada componente
de la distribucion pueden verse en \cref{fig:jetfake_combfit}.
Los parametros son inicializados con los valores extraidos de los ajustes
individuales a la senal y el fondo, y se los permite variar dentro de su
incerteza
%As this fit is the source of the biggest systematic uncertainty the range of the parameters is then extend to two times the
%individual fit uncertainties to compute the systematic of the method.

Para estimar la incerteza sistematica, el ajuste combinado es realizado
sin ningun constrarin en los parametros. La incerteza obtenida es del
50\% del valor de$f_{j\to\gamma}$, lo suficientemente grande para contener
cualquier potencial mismodelado de los templates.

%The fitted parameters for the signal, background and combined fits are tabulated in \Tab \ref{tab:jetfake_fit_pars}.
Los parametros estimados del ajuste combinado estan tabulados en la
\cref{tab:jetfake_fit_pars}.

El numero de fotones total (N$_\text{tot}^\text{iso}$) y fake (N$_{j\to\gamma}^\text{iso}$)
esperados en la region de control es obtenido integrando
las componentes de fondo y total del ajuste combinado sobre todo el rango
$\etiso<5\gev$, respectivamente:

\begin{itemize}
\item[] N$_{j\to\gamma}= 24143 \pm 56$

\item[] N$_\text{tot}= 281812 \pm 533$
\end{itemize}

La fraccion $f_{j\to \gamma}$ es obtenida simplemente como el cociente
en la \cref{eq:jfake_formula}:

\begin{itemize}
\item[] $f_{j\to\gamma} = 0.0857 \pm 0.0002 \stat \pm 0.04 \;\syst$
\end{itemize}

\begin{figure}[h]
  \begin{center}
  \includegraphics[width=0.49\textwidth]{figures/iso_fit_sig_wpars}  \hfill
  \includegraphics[width=0.49\textwidth]{figures/iso_fit_bkg_wpars}
  \caption{Ajuste a las distribuciones de {\etiso} de senal
    (izquierda) y fondo (derecha) (right) en la region
    de control CRFJ.}
  \label{fig:jetfake_sigbkg}
  \end{center}
\end{figure}

\begin{figure}[h]
  \begin{center}
  \includegraphics[width=0.49\textwidth]{figures/iso_fit_sarange}
  \caption{Ajuste combinado a la distribucion de {\etiso}
    para los pseudo-fotones que pasan la selección de la región
    de control CRFJ (ver texto para detalles).}
  \label{fig:jetfake_combfit}
  \end{center}
\end{figure}

\begin{table}[h!]
  \centering
  \caption{Fit parameters results from the combined fit to the isolation distribution of pseudo-photon data in CRFJ. Both signal and background were found to be well described by a Crystall Ball function.}
  \begin{tabular}{l l}
     Parameter &  Fit result \\
     \hline
     \verb|bkg_cb_alpha| & $-0.5932 \pm 0.0003$ \\
     \verb|bkg_cb_mean|  & $11.4897 \pm 0.0009 \GeV$ \\
     \verb|bkg_cb_n|     & $9.33 \pm 0.03$ \\
     \verb|bkg_cb_sigma| & $6.2862 \pm 0.0008 \GeV$ \\
     \verb|bkg_yield|    & $216579 \pm 500$ \\
     \hline
     \verb|sig_cb_alpha| & $-1.0460 \pm 0.0001$ \\
     \verb|sig_cb_mean|  & $-0.4175 \pm 0.003 \GeV$ \\
     \verb|sig_cb_n|     & $3.6675 \pm 0.0007$ \\
     \verb|sig_cb_sigma| & $0.9819 \pm 0.0001$ \\
     \verb|sig_yield|    & $2654337 \pm 546$ \\
     \hline
   \end{tabular}
    \label{tab:jetfake_fit_pars}
\end{table}


The number of jets faking photons can be estimated now using the fraction $f_{j\to\gam}$ as,

\begin{equation}\label{eq:njfakes}
  N_\text{fakes} = f_{j\to\gam} \times N_\text{tight}
\end{equation}

To keep the analysis blinded, we parametrize the number of fakes as function of \met, as shown in \cref{fig:jetfake_nfakes_met}, using an exponential
function $N_\text{fakes} = \exp(A+B \, \met)$. The values of the $A$ and $B$ parameters obtained from the fit are shown in table \cref{tab:exppars}.

\begin{table}[h!]
  \centering
  \caption{Fit parameters results from the fit to the $N_\text{fakes}$ distribution as a function of \met with an exponential function $\exp(A+B\, \met)$.}
  \begin{tabular}{crr}
    \hline
    Parameter &  SR2 & SR3 \\
     \hline
     A & $3.87 \pm 1.25$  &  $1.80 \pm 0.98$ \\
     B &  $-0.054 \pm 0.018$  & $-0.047 \pm 0,014$ \\
     \hline
  \end{tabular}
  \label{tab:exppars}
\end{table}


The final background contamination expected from jets faking photons is obtained integrating the $N_\text{fakes}$ parametrization over the \met region
of each SR:

\begin{align}
  N_\text{fakes}^\text{SR2} &= \int_{200}^{\infty} N_\text{fakes}(\met) \, d\met = 0.01 \pm 0.02 \\
  N_\text{fakes}^\text{SR3} &= \int_{300}^{\infty} N_\text{fakes}(\met) \, d\met = 0.0001 \pm 0.0001
\end{align}


\begin{figure}[h]
  \centering
  \includegraphics[width=0.49\textwidth]{figures/nfakes_SR2}  \hfill
  \includegraphics[width=0.49\textwidth]{figures/nfakes_SR3}
  \caption{Parametrización del número de jets mal identificados como
    función de {\met}.}
  \label{fig:jetfake_nfakes_met}
\end{figure}

El fondo de jet fakes es estimdo similarmente en las regiones de control
y validacion definidas en la \cref{tab:sr2,tab:sr3}, pesando
el numero de eventos observado en cada cosa por el factor $f_{j\to\gamma}$. % factor above as in \Eq \ref{eq:njfakes}.


%% The final background contamination expected from jets faking photons is obtained by weighting the number of events observed in data,
%% after the otherwise full tight SR selections detailed in sec \ref{sec:signal_regions} (i.e. only the photon identification requirement is reversed).


%% Indeed, no pseudo-photon event was ultimately observed after all CRJ selections. This leads to a
%% j$\to\gamma$ background estimate of $<0.07$ events. Consistently, MC studies have shown this
%% background to be small in the high \MET\ and \HT\ regimes on the final SRs. As seen in \Tab \ref{tab:mc_events_sr_phtype}, the
%% MC predicts a jet fake contamination of $0.10\pm 0.04$ and $0.02 \pm 0.02$ event for SR2 and SR3, respectively.
%% This supports the findings in the data, agreeing within the uncertainties. The final jet fake background yield is estimated as $N^\text{SR}_\text{jfakes} = 0.0^{+0.1}_{-0.0}$, which accommodates an extra $+0.03$ uncertainty from the comparison of the MC-data predictions.

%\tosolve{add check with MET-stepped predictions?}.


%In absence of observed events surviving the SR selection, a conservative estimate is computed under the assumption
%of one event in the pseudo photon sample for each signal region. It translates to a final yield of


%The final background contamination expected from jets faking photons was obtained by weighting the number of pseudo-photon events observed
%in data, after the otherwise full tight SR selections detailed in sec \ref{sec:signal_regions}. i.e. only the photon identification requirement is reversed.
%The results are summarized in \Tab \ref{tab:jetfake_yields}.
%
%\begin{table}[h!]
%  \centering
%  \caption{Number of misidentified jet events expected in the different signal regions. The unscaled number of pseudo-photons
%    is weighted by the $f/p$ ratio to get the final background yield from jet fakes in the three
%    analysis regions.}
%
%  \begin{tabular}{ccc}
%    \hline
%    \hline
%    Signal region & Unscaled & Weighted  \\
%    \hline
%%    SR1 & $4$ & $7.50$ \\
%    SR2 & $0$ & $0.00$ \\
%    SR3 & $0$ & $0.00$ \\
%    \hline
%    \hline
%  \end{tabular}
%  \label{tab:jetfake_yields}
%\end{table}
