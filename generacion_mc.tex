\section{Generación de eventos Monte Carlo}

La conexión entre la teoría/fenomenología de SUSY (o cualquier otra teoría de
nueva física), y los datos observados en el detector de un colisionador se
realiza por medio de un \emph{generador de eventos Monte Carlo}
\cite{Sjostrand:2006su,Dobbs:2004qw}. Dada una teoría de nueva física, que en
general predice la existencia de nuevas partículas y/o interacciones, el
generador de eventos permite calcular cómo esa teoría se manifestará en el
experimento.

El procedimiento, una vez que se selecciona un modelo particular de SUSY,
requiere de varios pasos que se encuentran esquematizados en la
\cref{fig:mc_sketch} \cite{Baer:2009tk}. En primer lugar es necesario calcular
el espectro de masas de las partículas supersimétricas, y sus acoplamientos. A
partir de ellos se calculan los anchos y tazas de decaimiento de todas estas
partículas, y por último se utiliza el generador de eventos, que toma como entrada las
masas y decaimientos calculados previamente, y genera un conjunto de eventos a
la energía de centro de masa del colisionador. A continuación, para cada evento,
se simula el pasaje de las partículas por el detector.

Para las distintas etapas suelen
utilizarse herramientas diferentes, por lo que es fundamental contar con un
forma estandarizada de intercambio de información entre ellas. Con este objetivo
se creó un tipo de archivo llamado SLHA (SUSY Les Houches Accord)\cite{SLHA} que
permite organizar toda la información de un cierto modelo de supersimetría
(parámetros, masas, decaimientos) en un formato único, de modo que sea fácil de
compartir entre los físicos y también entre las distintas herramientas
utilizadas.

\begin{figure}[!htbp]
  \centering

  \scalebox{0.71}{\begin{tikzpicture}[node distance=1.5cm, auto,]

    \node[rec] (susy) {\sc Modelo SUSY};
    \node[rec, right=of susy] (masses) {\sc Espectro de Masas};
    \node[rec, right=of masses] (decays) {\small \sc Decaimientos};
    \node[rec, right=of decays] (gen) {\sc Generador de Eventos};
    \node[rec, right=of gen] (atlas) {\sc Simulación Detector};

    \draw[arr] (1.7,0) -- (2.7,0);
    \draw[arr] (6.2,0) -- (7.2,0);
    \draw[arr] (10.8,0) -- (11.7,0);
    \draw[arr] (15.1,0) -- (16.1,0);

\end{tikzpicture}
}

  \caption{Etapas de la simulación de eventos de un modelo particular de SUSY.}
  \label{fig:mc_sketch}
\end{figure}


\subsection{Espectro de masas y decaimientos}

Los modelos de SUSY que se estudian en el LHC suelen ser la teoría efectiva a
bajas energías que resulta de una teoría mas general, como teoría de
supercuerdas, o que involucre un mecanismo particular para el rompimiento de
SUSY. Para especificar una teoría efectiva es necesario definir: la simetría de
gauge, los campos que la componen y, el Lagrangiano. Como se describió en el
\cref{cap:susy}, los efectos del rompimiento de SUSY están codificados en los
términos de rompimiento soft del Lagrangiano. También es necesario especificar
la escala de energía a la cual la teoría efectiva y el lagrangiano son válidos.
Como los experimentos prueban física a la escala del {\tev}, mientras que los
parámetros del Lagrangiano son frecuentemente especificados a energías mucho mas
altas ($M_\text{GUT}$ o $M_P$), debe usarse el grupo de ecuaciones de
renormalización (RGE) para conectar las dos escalas del modelo. Una vez que los
parámetros del Lagrangiano son conocidos a la escala electrodébil, pueden
identificarse las masas físicas de las partículas, diagonalizando las matrices
de masa correspondientes, y luego, para tener la precisión suficiente, aplicar
las correcciones a ordenes mayores (en general a 1 loop). A partir del modelo
también es posible calcular los anchos y tazas de decaimiento de todas las
partículas supersimétricas.

Existe una gran variedad de programas que permiten realizar estos cálculos. En
este trabajo utilizamos el programa {\susyhit}\cite{Djouadi:2006bz} que combina
{\suspect}\cite{Djouadi2007426} para calcular el espectro de masas, junto con
{\sdecay}\cite{Muhlleitner:2004mka} y {\hdecay}\cite{Djouadi:1997yw} para
calcular los BRs y anchos de decaimiento. Algunos de estos modos de decaimientos
son calculados a NLO en QCD.

Una vez que el espectro de masas y los decaimientos están calculados estos
pueden usarse como entrada en los generadores de eventos.


\subsection{Generador de eventos}

Los programas descriptos anteriormente permiten pasar de un modelo específico a las
predicciones en la producción de partículas y sus anchos de decaimientos en
estados finales de quarks, leptones, fotones y gluones (y LSP en modelos donde
se conserva paridad-R).

%% Por lo tanto todavía falta un paso entre estos modelos y las partículas finales
%% que dejas sus señales en los detectores, del cual se encarga el generador de
%% eventos. A partir de la colisión hadrónica inicial, estos programas modelan la
%% producción de las partículas en el estado final a ser medidas en el detector.

El paso siguiente consiste en la generación de los eventos Monte Carlo.
Para un dado tipo de colisionador y una dada energía de centro de masa,
el generador se encarga de generar los eventos esperados para el modelo
considerado.

La colisión de dos protones a altas energías en el LHC implica el estudio de las
interacciones de los constituyentes de los protones, es decir, de los quarks y
gluones. La interacción total de los dos protones es complicada, como se muestra
en la \cref{fig:mc_event_generator} y se puede descomponer en distintos pasos.
En todos los componentes del diagrama excepto en la interacción fuerte, dominan los procesos de QCD.

Los protones acelerados por el LHC interactúan para producir las partículas
predichas por la teoría. Este proceso de interacción fuerte involucra los
partones de los hadrones intervinientes en el proceso. El cálculo se realiza, en
general, a primer orden (LO) en teoría de perturbaciones aunque algunos
programas pueden incluir algunos procesos a NLO. Esta etapa involucra la
convolución con las funciones de distribución partónicas para obtener las
secciones eficaces de producción.

Las funciones de distribución partónica se utilizan para describir la
subestructura del protón y son usadas por todos los generadores de eventos.
ATLAS utiliza la librería LHAPDF \cite{Bourilkov:2006cj} que contiene un
repositorio con una gran
cantidad de PDF. Por defecto, y a menos que se indique lo contrario las PDFs
CTEQ \cite{Nadolsky:2008zw} son las utilizadas en ATLAS.

A partir de la colisión inicial y la producción de las partículas finales (en el
caso de SUSY será un par de partículas supersimétricas) se
modela el decaimiento de las mismas (posiblemente en varios pasos) hasta el
estado final partónico, de acuerdo a los BR predichos por el modelo.

La multiplicidad de partones en el estado final depende, además de los partones producidos en la interacción
fuerte, de la radiación en el evento. Pueden producirse partones adicionales por
radiación de un gluón o quark ya sea antes de la interacción fuerte (llamada
\emph{Initial State Radiation}, ISR) o después de la interacción fuerte, en la fase de
fragmentación (llamada \emph{Final State Radiation}, FSR).
Cada uno de los partones en el estado
final comienzan a perder energía, a través de la radiación de gluones. Estos
gluones fragmentan en gluones adicionales y pares quark-antiquark. Esto continua
hasta el punto en donde la energía es lo suficientemente baja para recombinar
todas las partículas de color en mesones y bariones, proceso denominado
\emph{hadronización}. Los hadrones decaen subsecuentemente en otros hadrones,
leptones y neutrinos. Cada partón proveniente de la interacción fuerte (y
ISR/FSR) resulta en una lluvia de partículas, denominada colectivamente
\emph{jet}.


Finalmente, los remanentes de los haces iniciales tienen que ser modelados
para obtener una descripción válida de la física incluyendo no solo la
interaccion dura sino los procesoso ``soft'', lo que suele llamarse evento
subyacente (UE).

Todos estos procesos se repiten hasta generar un gran número de eventos, donde cada
evento consiste básicamente en los vectores de todas las partículas en el estado
final.

%% %% Para un dado tipo de colisionador y una dada energía de centro de masa, y a partir
%% %% de un modelo específico de SUSY,
%% %% conjunto de eventos con pares de spartículas de acuerdo a su sección eficaz.
%% %% Estas partículas supersimétricas van a decaer (posiblemente en una cascada de
%% %% varios pasos) en un estado final partónico, de acuerdo a los BR fijados por el
%% %% modelo. Este estado final partónico es convertido luego en uno con partículas
%% %% que puede ser detectadas en el detector. Generando un gran número de eventos de
%% %% SUSY, se pueden simular los posibles estados finales que se esperan a partir de
%% %% un cierto modelo.



%% \begin{itemize}

%% \item Interacción fuerte:


%% \item Cascadas partónicas:

%%   Implementación de las lluvias partónicas para las partículas producidas en la
%%   colisión (lluvia de estado final o FSR), para los partones iniciales
%%   involucrados en la colisión (lluvia de estado inicial o ISR), y también para
%%   otras partículas de color que puedan ser producidas como decaimiento de
%%   partículas mas pesadas.

  %% implementación de las lluvias de partones para los estados de partículas
  %% con color inicial y final, y para otras partículas con color que puedan ser
  %% producidas como decaimiento de otros objetos mas pesados,

%% \item Decaimientos:

%%   Las partículas supersimétricas producidas en la interacción dura decaen (en
%%   general en forma de cascada) a otras partículas.

%%   %% Las particulas fundamentales masivas, como el quark top,
%%   %%   los bosones de gauge electrodebiles, los bosones de Higgs y particulas de nueva fisica, decaen
%%   %%   en una escala de tiempo que es mas corta o comparable a las lluvias partinicas QCD.
%%   %%   Dependiendo de la naturaleza de las particulas y de si se producen particlas que interacctuen
%%   %%   fuertementen en el decaimiento, estas marticulas pueden tambien iniciar lluvias partonicas
%%   %%   antes y despues de su decaimiento.

%% \item Hadronización y producción de partículas estables:

%%   En esta etapa se implementa el modelo de hadronización que describe la
%%   formación de mesones y bariones a partir de los quarks y gluones. También las
%%   partículas inestables deben decaer a partículas (cuasi) estables que son
%%   detectadas en el detector, con tasas y distribuciones que estén de acuerdo con
%%   los valores medidos o predichos.

%% \item Remanentes:


%% \end{itemize}

%% \begin{figure}[h]
%%   \centering
%%   \includegraphics[width=0.6\textwidth]{figures/mc_event_generator}
%%   \caption{Distintas etapas implementadas en los generadores de eventos Monte Carlo.
%%     (1) Interacción dura (HS), (2) Lluvias de estado inicial y final (ISR, FSR),
%%     (3) Decaimientos en cascada, (4) Hadronización, (5) Remanentes.}
%%   \label{fig:mc_event_generator}
%% \end{figure}

\begin{figure}[!htbp]
  \centering

  \scalebox{0.9}{%% \usetikzlibrary{positioning}
%% \usetikzlibrary{arrows}
%% \usetikzlibrary{patterns}
%% \usetikzlibrary{decorations.markings}
%% \usetikzlibrary{calc}
%% \usetikzlibrary{decorations}
%% \usetikzlibrary{decorations.pathmorphing}
%% \usetikzlibrary{decorations.pathreplacing}

\begin{tikzpicture}

  \tikzset{
    line/.style={line width=0.9},
    dline/.style={dashed,line width=0.9},
  }

  \definecolor{blue1}{HTML}{3F66BD}
  \definecolor{blue2}{HTML}{2C4784}

  %\draw[step=0.5,black!20,thin] (-5,-5) grid (5,5);
  %\draw[step=1,red!30,thin] (-5,-5) grid (5,5);

  \coordinate (p1) at (-2, 3);
  \coordinate (p2) at (-2,-3);
  \coordinate (O) at (2,0);

  \node at (-4, 3) {p};
  \node at (-4,-3) {p};

  \draw (-3.75, 3) -- (p1);
  \draw (-3.75,-3) -- (p2);

  \draw (0,-0.5) -- (0,0.5);
  \draw (p2) -- (0,-0.5);

  \draw[gluon] (p1) -- (-1.2, 2);
  \draw (-1.2,2) -- (-0.74,1.4);
  \draw[gluon] (-0.74,1.4) -- (-0.05, 0.55);
  \draw (-1.2,2) -- (-0.3, 2.5);
  \draw (-0.74,1.4) -- (0, 1.8);

  \draw[dashed] (0,-0.5) -- (2,-1);
  \draw         (0, 0.5) -- (2, 1);

  \draw (2,-1) -- (2.5,-1.5);
  \draw (2,-1) -- (4, 0);

  \draw[dashed] (3, -0.5) -- (3.5,-1);
  \draw (3.5, -1) -- (3.8, -1.50);
  \draw (3.5, -1) -- (4, -0.80);

  \draw[dashed] (2, 1) -- (3,2);
  \draw (2, 1) -- (2.5,0.5);

  \draw (-2,3) -- (-0.5,3.5);
  \draw (-2,3.1) -- (-0.5,3.7);

  \draw (-2,  -3) -- (-0.5,-3.5);
  \draw (-2,-3.1) -- (-0.5,-3.7);

  \draw[gluon] (-1.2,-2) -- (0,-2.5);
  \draw[gluon] (0,-2.5) -- (0.5,-2);
  \draw[gluon] (0,-2.5) -- (0.5,-3);
  \draw[gluon] (-0.55,-1.2) -- (0,-1.5);

  \draw (3,2) -- (3.5,2.5);
  \draw (3,2) -- (3.5,1.5);

  \draw[gluon] (3.5,-0.25) -- (4,-0.5);

  \node at (1,1.1) {$\tilde{g}$};
  \node at (1,-1.1) {$\tilde{q}$};
  \node at (4,-1.8) {$\tilde{\chi}^0_1$};
  \node at (3.9,1.5) {$\tilde{\chi}^0_1$};

  \draw[dashed,blue,line width=1] (0,0) circle(0.8);

  \node[blue] at (-1.75,0) {Interacción};
  \node[blue] at (-1.75,-0.5) {dura};

  \node[blue] at (0.1,2.0) {ISR, FSR};
  \node[blue,rotate=45] at (2.5,2) {Decaimientos};


  \draw[blue, decorate,decoration={brace,amplitude=4pt}] (4.2,0.23) -- (4.2,-0.55);
  \node[blue] at (5.6, -0.1) {Hadronización};

  \draw[blue, decorate,decoration={brace,amplitude=4pt}] (-0.2,-3.2) -- (-0.2,-4);
  \node[blue] at (1,-3.6) {Remanentes};

  \draw[fill=white] (p1) circle(0.25);
  \draw[fill=white] (p2) circle(0.25);

\end{tikzpicture}
}

  \caption{Esquema de las distintos pasos implementados en los generadores de
    eventos Monte Carlo. Adaptado de \cite{Baer:2009tk}.}
  \label{fig:mc_event_generator}

\end{figure}


%% \item Lluvias de partones de estado inicial y final:
%%     Las particulas de color en el evento son perturvativamente evolucionadas desde
%%     la escala dura de la colision hasta el corte infrarojo.
%%     Esto ocurre para las particulas producidas en la colision, la \emph{lluvia de estado final},
%%     y los partones iniciales involucrados en la colision, la \emph{lluvia de estado inicial}.
%%     La coherencia de la emision de gluones soft en las lluvias de partones de las particulas en
%%     la colision dura es controlada por el flujo de color de la colision dura.

%%     %% 3. Decay of heavy objects. Massive fundamental particles such as the top quark, electroweak
%%     %% gauge bosons, Higgs bosons, and particles in many models of physics beyond the Standard
%%     %% Model, decay on time-scales that are either shorter than, or comparable to that of the QCD
%%     %% parton shower. Depending on the nature of the particles and whether or not strongly interacting
%%     %% particles are produced in the decay, these particles may also initiate parton showers
%%     %% both before and after their decay. One of the major features of the Herwig++ shower
%%     %% algorithm is the treatment of radiation from such heavy objects in both their production
%%     %% and decay. Spin correlations between the production and decay of such particles are also
%%     %% correctly treated.
%%   \item Decaimiento de objetos pesados: Las particulas fundamentales masivas, como el quark top,
%%     los bosones de gauge electrodebiles, los bosones de Higgs y particulas de nueva fisica, decaen
%%     en una escala de tiempo que es mas corta o comparable a las lluvias partinicas QCD.
%%     Dependiendo de la naturaleza de las particulas y de si se producen particlas que interacctuen
%%     fuertementen en el decaimiento, estas marticulas pueden tambien iniciar lluvias partonicas
%%     antes y despues de su decaimiento.

%%   \item

%%     4. Multiple scattering. For large centre-of-mass energies the parton densities are probed in a
%%     kinematic regime where the probability of having multiple partonic scatterings in the same
%%     hadronic collision becomes significant. For these energies, multiple scattering is the dominant
%%     component of the underlying event that accompanies the main hard scattering. These
%%     additional scatterings take place in the perturbative regime, above the infrared cut-off, and
%%     therefore give rise to additional parton showers. We use an eikonal multiple scattering
%%     model [8], which is based on the same physics as the FORTRAN JIMMY package [18], together
%%     with some minor improvements. In addition to that we included non-perturbative
%%     partonic scatters below the infrared cut-off [19], which enables us to simulate minimum
%%     bias events as well as the underlying event in hard scattering processes.

%%     5. Hadronization.
%%     After the parton showers have evolved all partons involved in hard scatterings,
%%     additional scatters and partonic decays down to low scales, the final state typically
%%     consists of coloured partons that are close in momentum space to partons with which they
%%     share a colour index, their colour ‘partner’ (in the large Nc limit this assignment is unique).
%%     Herwig++ uses the cluster hadronization model [2] to project these colour–anticolour pairs
%%     onto singlet states called clusters, which decay to hadrons and hadron resonances. The
%%     original model of Ref. [2], which described this decay as pure phase space has been progressively
%%     refined as described in Sect. 7. Clusters that are too massive or too light for decay
%%     directly to hadrons to provide a good description are treated differently, again described in
%%     Sect. 7.

%%     6. Hadron decays. The hadron decays in Herwig++ are simulated using a matrix element
%%     description of the distributions of the decay products, together with spin correlations between
%%     the different decays, wherever possible. The treatment of spin correlations is fully
%%     integrated with that used in perturbative production and decay processes so that correlations
%%     between the production and decay of particles like the tau lepton, which can be
%%     produced perturbatively but decays hadronically, can be treated consistently.



\subsection{Simulación del detector ATLAS}
\label{sec:sim_atlas}

Para algunos estudios aproximados es posible utilizar directamente las
partículas producidas por los generadores Monte Carlo. Pero en el caso general
es necesario simular el pasaje de estas partículas por el detector.

A fin de poder estudiar la respuesta del detector para un gran número de procesos
físicos y escenarios, se ha implementado una simulación detallada que lleva los
eventos de la generación a un formato que es idéntico a las señales en el
detector verdadero\cite{AtlasSim}. La simulación está integrada al software de
ATLAS (\textsc{Athena}), y utiliza el paquete de simulación
{\geant}4\cite{Geant4}.

La simulación está dividida así en tres etapas: generación del evento y los
decaimientos inmediatos, simulación del detector, y digitalización de
los depósitos de energía en las regiones sensibles del detector en los voltajes
y corrientes que se encuentran a la salida del detector. Luego, la salida de
estos procesos de simulación sirve como entrada a los algoritmos de
trigger y reconstrucción de ATLAS, que son idénticos a los que se utilizan en
los datos reales.

Existe también una simulación rápida del detector que es utilizada en los casos
donde no es necesaria un simulación tan minuciosa de las lluvias
electromagnéticas en los calorímetros. Casi 80\% del tiempo de simulación se
debe a la simulación de partículas atravesando el calorímetro, y cerca del 75\%
se emplea en la simulación de las partículas electromagnéticas. En la simulación
rápida \textsc{Atlfast}\cite{Richter-Was:683751}, se remueven estas partículas
electromagnéticas de baja energía y se las reemplaza con lluvias
electromagnéticas pre-simuladas. De esta forma se reduce el consumo de CPU
notablemente considerando las características de la física en cuestión.

Cada evento en una muestra Monte Carlo es producido bajo ciertas condiciones
particulares del detector, y bajo un valor particular de pile-up. Existen
algunas diferencias en el perfil del pile-up de los datos reales y las muestras
MC, en particular debido a que las muestras MC son simuladas previo a la toma de
datos con el pile-up previsto que no es exactamente igual al real. Para
solucionar esto, se aplica un repesado evento a evento a las muestras MC de
manera de modelar las condiciones reales de la muestra de datos bajo estudio.
