%--------
% Fondos
%--------
\section{Simulación de los fondos del SM}
\label{sec:bkg_samples}

\newcommand{\mccaption}{Se detallan la sección eficaz a LO para cada modo de decaimiento,
  los factores $k$ (para la normalización NLO) y las eficiencias del filtro
  , así como también la luminosidad integrada correspondiente
  a la estadística total de cada muestra.}

Las muestras simuladas con MC utilizadas en este análisis se describen
a continuación. Como se discutirá en la \cref{sec:background_estimation},
la contaminación de fotones mal identificados provenientes de jets y electrones
es estimado con métodos basados en datos. Sin embargo, las muestras MC también
han sido consideradas en estos casos para los estudios de optimización y la
evaluación de las incertezas sistemáticas.

\subsection{$W/Z + \gamma$}

Se espera que la producción de {\wgam} y {\zgam} sea un fondo importante
para esta búsqueda. Ambas muestras fueron generadas usando el generador
de eventos {\sherpa} v1.4.1\cite{SherpaGen}, con hasta 3 partones en el
ME+PS y usando las funciones de densidad partonica CT10.
La combinación de los elementos de matriz con las lluvias partonicas
es realizada de acuerdo a un procedimiento mejorado CKKW\cite{Catani:2001cc,Krauss:2002up}.
Un filtro a nivel generados es aplicado requiriendo al menos un fotón
con $\pt > 80(70) \gev$ en el estado final de las muestras de {\wgam} (\zgam).
Todos los decaimientos leptonicos del bosón $Z$ fueron considerados,
incluyendo el decaimiento invisible $Z\to\nu\nu$.
También se tuvo en cuenta un muestra de $V(\to qq)+\gamma  (V=W,Z/\gamma*)$
debido a que cierta energía faltante real puede  ser producida en el caso
de heavy flavour decays.

\begin{table}[!htbp]
  \centering
  \caption{Muestras de W/Z$+\gamma$.
    La sección eficaz a LO se especifica para cada modo de decaimiento,
    al igual que los factores $k$, y las eficiencias del filtro.
    La luminosidad integrada correspondiente a la estadística total
    de cada muestra esta también presente.}
  \begin{tabular}{lccccc}
    \hline
    Proceso & Generador & $\sigma~[pb]$ & $k$ & Eficiencia & $L [fb^{-1}]$ \\
    \hline
    {\wenugam}    & {\sherpa} &  0.7193  &  1.0  &  1.0  &  695.16 \\
    {\wmunugam}   & {\sherpa} &  0.7178  &  1.0  &  1.0  &  696.56 \\
    {\wtaunugam}  & {\sherpa} &  0.7199  &  1.0  &  1.0  &  694.57 \\
    {\zeegam}     & {\sherpa} &  0.1861  &  1.0  &  1.0  &  1069.53 \\
    {\zmumugam}   & {\sherpa} &  0.1858  &  1.0  &  1.0  &  1076.71 \\
    {\ztautaugam} & {\sherpa} &  0.1858  &  1.0  &  1.0  &  1076.19 \\
    {\znngam}   & {\sherpa} &  0.7625  &  1.0  &  1.0  &  655.74 \\
    {\vqqgam} & {\sherpa}  &  6.756  &  1.0  &  1.0  &  89.0 \\
    \hline
    \multicolumn{5}{c}{Variaciones sistemáticas} \\
    \hline
    {\wenugam} (fact. 0.25x)    & {\sherpa} &  0.7193  &  1.0  &  1.0  &  278.05 \\
    {\wenugam} (fact. 4x)       & {\sherpa} &  0.7193  &  1.0  &  1.0  &  278.05 \\
    {\wenugam} (renorm. 0.25x)  & {\sherpa} &  0.7193  &  1.0  &  1.0  &  278.05 \\
    {\wenugam} (renorm. 4x)     & {\sherpa} &  0.7193  &  1.0  &  1.0  &  278.05 \\
    {\wenugam} (ckkw 15)        & {\sherpa} &  0.7193  &  1.0  &  1.0  &  278.05 \\
    {\wenugam} (ckkw 30)        & {\sherpa} &  0.7193  &  1.0  &  1.0  &  278.05 \\
    {\wmunugam} (fact. 0.25x)   & {\sherpa} &  0.7193  &  1.0  &  1.0  &  278.63 \\
    {\wmunugam} (fact. 4x)      & {\sherpa} &  0.7193  &  1.0  &  1.0  &  278.63 \\
    {\wmunugam} (renorm. 0.25x) & {\sherpa} &  0.7193  &  1.0  &  1.0  &  278.63 \\
    {\wmunugam} (renorm. 4x)    & {\sherpa} &  0.7193  &  1.0  &  1.0  &  278.63 \\
    {\wmunugam} (ckkw 15)       & {\sherpa} &  0.7193  &  1.0  &  1.0  &  278.63 \\
    {\wmunugam} (ckkw 30)       & {\sherpa} &  0.7193  &  1.0  &  1.0  &  278.63 \\
    {\wtaunugam} (fact. 0.25x)  & {\sherpa} &  0.7193  &  1.0  &  1.0  &  277.81 \\
    {\wtaunugam} (fact. 4x)     & {\sherpa} &  0.7193  &  1.0  &  1.0  &  277.81 \\
    {\wtaunugam} (renorm. 0.25x)& {\sherpa} &  0.7193  &  1.0  &  1.0  &  277.81 \\
    {\wtaunugam} (renorm. 4x)   & {\sherpa} &  0.7193  &  1.0  &  1.0  &  277.81 \\
    {\wtaunugam} (ckkw 15)      & {\sherpa} &  0.7193  &  1.0  &  1.0  &  277.81 \\
    {\wtaunugam} (ckkw 30)      & {\sherpa} &  0.7193  &  1.0  &  1.0  &  277.81 \\
    \hline
  \end{tabular}
  \label{tab:bkg_wzgamma_samples}
\end{table}

\subsection{$W/Z$ + jets}
\label{mc_wzjets}

Se espera que la producción de W$^{\pm}$ y bosones $Z$ en asociación con jets
contribuya a esta búsqueda, con los fotones provenientes de electrones y jets
mal identificados. Especialmente para los segundos, esta contaminación no esta
bien descripta por el MC. Por esta razón se utilizan métodos basados en datos
para estimar su contribución en las diferentes regiones de señal y control, como
se describe en el \cref{cap:fondos}. De igual manera varias muestras de MC
fueron consideradas para validar los métodos.

Como se describe en el \cref{cap:seleccion} la selección de señal involucra
muchos jets en el estado final, es importante modelar los estados final multipartonicos
de forma adecuada. Con esto en mente, el generador de eventos {\alpgen} (v 2.14)
fue utilizado, incluyendo los efectos EWK y QCD a LO para los procesos de interacción
fuerte multipartonicos. La producción de jets fue generada for up to five-parton
matrix elements. Este generados fue interfaceado con {\herwig} v6.5.2
para la simulación de las lluvias y los procesos de fragmentación y con {\jimmy}
para la simulación de los eventos subyacentes. Las funciones de densidad partónico
utilizadas fueron las CTEQ6L1. La normalización a la luminosidad integrada acumulada
fue hecha escalenado la sección eficaz mostrada en la \cref{tab:bkg_wzjets_samples}
usando cálculos QCD a NNLO de el programa FEWZ\cite{Anastasiou:2003ds}.
En cada caso los mismos factores de normalización fueron aplicados a los elementos
de matriz de {\alpgen}.

Finalmente, se realiza la remoción de eventos para evitar el conteo doble de eventos
que ya fueron tenidos en cuenta por las muestras de Z$\gamma$ y W$\gamma$.
Para esto, los eventos de $W(Z)+\text{jets}$ con fotones con $\pt > 80(70)\gev$
y $\Delta{\rm R}(e/\mu/\tau/$light-quarks$, \gamma) > 0.1$ fueron removidos
de las muestras.

\begin{table}[!htbp]
  \centering
  \caption{Muestras de $W/Z+\text{jets}$. \mccaption.}
  \begin{tabular}{lccccc}
    \hline
    Proceso & Generador & $\sigma$ [pb] & factor-$k$ & Eficiencia & $L$ [fb$^{-1}$] \\
    \hline
    \zeenj{0} &  \alpgen+\jimmy  & 711.77 & 1.23 & 1.00 & 7.548 \\
    \zeenj{1} &  \alpgen+\jimmy  & 155.17 & 1.23 & 1.00 & 6.994 \\
    \zeenj{2} &  \alpgen+\jimmy  & 48.745 & 1.23 & 1.00 & 6.746 \\
    \zeenj{3} &  \alpgen+\jimmy  & 14.225 & 1.23 & 1.00 & 6.286 \\
    \zeenj{4} &  \alpgen+\jimmy  & 3.7595 & 1.23 & 1.00 & 6.487 \\
    \zeenj{5} &  \alpgen+\jimmy  & 1.0945 & 1.23 & 1.00 & 7.428 \\
    \zmmnj{0} &  \alpgen+\jimmy  & 712.11 & 1.23 & 1.00 & 7.557 \\
    \zmmnj{1} &  \alpgen+\jimmy  & 154.77 & 1.23 & 1.00 & 7.011 \\
    \zmmnj{2} &  \alpgen+\jimmy  & 48.912 & 1.23 & 1.00 & 6.731 \\
    \zmmnj{3} &  \alpgen+\jimmy  & 14.226 & 1.23 & 1.00 & 6.286 \\
    \zmmnj{4} &  \alpgen+\jimmy  & 3.7838 & 1.23 & 1.00 & 6.445 \\
    \zmmnj{5} &  \alpgen+\jimmy  & 1.1148 & 1.23 & 1.00 & 7.292 \\
    \zttnj{0} & \alpgen+\jimmy  & 711.81 & 1.23 & 1.00 &  7.560 \\
    \zttnj{1} & \alpgen+\jimmy  & 155.13 & 1.23 & 1.00 &  6.996 \\
    \zttnj{2} & \alpgen+\jimmy  & 48.804 & 1.23 & 1.00 &  6.746 \\
    \zttnj{3} & \alpgen+\jimmy  & 14.160 & 1.23 & 1.00 &  6.315 \\
    \zttnj{4} & \alpgen+\jimmy  & 3.7744 & 1.23 & 1.00 &  6.462 \\
    \zttnj{5} & \alpgen+\jimmy  & 1.1163 & 1.23 & 1.00 &  7.283 \\
    \hline
    \wenunj{0} & \alpgen+\jimmy  & 8037.10  & 1.186 & 1.00 & 0.362 \\
    \wenunj{1} & \alpgen+\jimmy  & 1579.20  & 1.186 & 1.00 & 1.334 \\
    \wenunj{2} & \alpgen+\jimmy  & 477.20   & 1.186 & 1.00 & 6.661 \\
    \wenunj{3} & \alpgen+\jimmy  & 133.93   & 1.186 & 1.00 & 6.358 \\
    \wenunj{4} & \alpgen+\jimmy  & 35.62    & 1.186 & 1.00 & 5.917 \\
    \wenunj{5} & \alpgen+\jimmy  & 10.55    & 1.186 & 1.00 & 5.592 \\
    \wmnunj{0} & \alpgen+\jimmy  & 8040.00  & 1.186 & 1.00 & 0.363 \\
    \wmnunj{1} & \alpgen+\jimmy  & 1580.30  & 1.186 & 1.00 & 1.333 \\
    \wmnunj{2} & \alpgen+\jimmy  & 477.50   & 1.186 & 1.00 & 6.656 \\
    \wmnunj{3} & \alpgen+\jimmy  & 133.94   & 1.186 & 1.00 & 6.357 \\
    \wmnunj{4} & \alpgen+\jimmy  & 35.64    & 1.186 & 1.00 & 6.033 \\
    \wmnunj{5} & \alpgen+\jimmy  & 10.57    & 1.186 & 1.00 & 1.595 \\
    \wtnunj{0} & \alpgen+\jimmy  & 8035.80  & 1.186 & 1.00 & 0.353 \\
    \wtnunj{1} & \alpgen+\jimmy  & 1579.80  & 1.186 & 1.00 & 1.307 \\
    \wtnunj{2} & \alpgen+\jimmy  & 477.55   & 1.186 & 1.00 & 6.567 \\
    \wtnunj{3} & \alpgen+\jimmy  & 133.79   & 1.186 & 1.00 & 6.365 \\
    \wtnunj{4} & \alpgen+\jimmy  & 35.58    & 1.186 & 1.00 & 5.921 \\
    \wtnunj{5} & \alpgen+\jimmy  & 10.54    & 1.186 & 1.00 & 5.199 \\
    \hline
  \end{tabular}
  \label{tab:bkg_wzjets_samples}
\end{table}


\subsection{Pares de tops ($+\gam$)}
\label{sec:mcttbargam}

Otro fondo importante para este análisis es el {\ttgam}. Esta muestra fue
generada utilizando {\madgraph}\cite{Alwall:2007st} y la PDF CTEQ6L1.
{\pythiasix}\cite{pythia} fue usado para la simulación de las lluvias
partonicas, fragmentación y eventos subyacentes. La radiación de fotones fue
agregadas utilizando {\photos}\cite{photos}, y los decaimientos de los leptones
tau con {\tauola}\cite{tauola}. Se requirió que los fotones a nivel generador
tengan un $\pt > 80 \gev$. Para evitar efectos cinemáticos introducidos por el
filtro, el corte en el {\pt} del fotón en la muestra reconstruida se aumento a
95 {\gev}. Se utilizo un factor-$k$\note{definir} de $1.9 \pm 0.4$\cite{Melnikov:2011ta, tth}.
Los detalles de la simulación se encuentran en la \cref{tab:bkg_ttbar_samples}.
Se utilizaron además algunas muestras a nivel generador como variaciones para calcular
las incertezas sistemáticas como se explica en la \cref{sec:syst_ttbargamma}.

La producción de {\ttbar}, donde los electrones o los jets son mal identificados
como fotones es una fuente de fondo que debe ser considerado. Aunque ambas
contaminaciones fueron estimadas a partir de los datos, se utilizaron eventos simulados
en la etapa de optimización y para realizar chequeos de los métodos utilizados.
La
muestra MC fue generada utilizando
{\powheg}\cite{Nason:2004rx,Frixione:2007vw,Alioli:2010xd} con la lluvia
partonica y fragmentación hecha por {\pythia}. Para el UE se utilizo Perugia
2011C con el
conjunto de PDFs CTEQ6L1 LO. La radiación de fotones adicional fue agregada con
{\photos}\cite{photos}. Overlap removal is performed to prevent double-counting
the phase-space covered by the {\ttgam} MC sample. Events with truth prompt
photons with $\pt > 95 \gev$ and $\Delta{\rm R}(e/\mu/\tau/g/$light-quarks$,
\gamma) > 0.1$ are removed from the \ttbar\ sample\note{Revisar!}.

\begin{table}[!htbp]
  \centering
  \caption{Muestras de {\ttgam}. {\mccaption}}
  \begin{tabular}{lccccc}
    \hline
    Proceso & Generador & $\sigma$ [pb] & factor-$k$ & Eficiencia & $L [\mathrm{fb}^{-1}]$ \\
    \hline
    {\ttbar} & \powheg+\pythia & 253.00 & 1.00 & 0.543 & 580 \\
    \hline
    %    \ttbargam noAllHad \madgraph (164439) & 0.092363 & 1.9 & 1 & 1139.7 \\
    {\ttgam} (lep) & \madgraph & 0.09873 & 1.90 & 1.00 & 1066.2 \\
    {\ttgam} (had) & \madgraph  & 0.068599 & 1.90 & 1.00 & 1534.5 \\
    \hline
    \multicolumn{6}{c}{Variaciones sistematicas} \\
    \hline
        {\ttgam} (lep) (scale$^{+}$) & \madgraph & 0.09873 & 1.9 & 1 & 1066.2 \\
        {\ttgam} (lep) (scale$^{-}$) & \madgraph & 0.09873 & 1.9 & 1 & 1066.2 \\
        {\ttgam} (lep) ($\alphas^{+}$)  & \madgraph & 0.09873 & 1.9 & 1 & 1066.2 \\
        {\ttgam} (lep) ($\alphas^{-}$)  & \madgraph & 0.09873 & 1.9 & 1 & 1066.2 \\
        {\ttgam} (lep) (FSR$^{+}$) & \madgraph & 0.09873 & 1.9 & 1 & 1066.2 \\
        {\ttgam} (lep) (FSR$^{-}$) & \madgraph & 0.09873 & 1.9 & 1 & 1066.2 \\
    \hline
  \end{tabular}
  \label{tab:mc_ttbar_samples}
\end{table}

\subsection{Top (+ $\gamma$)}

La producción de un quark top con un fotón asociado fue generado utilizando
\textsc{Whizard} 2.1.1 \cite{whizard, whizard2}.
%% , con 4-flavor/5-flavor matching provided
%% using Hoppet~\cite{hoppet}.

El fotón extra puede estar tanto en la producción del top o en los decaimientos
sucesivos. Sin embargo, la producción y el decaimiento son tratados de forma
separada, por lo que los efectos de interferencia pueden ser ignorados. Para las
lluvias partónicas y la fragmentación fue utilizado {\pythia}\cite{pythia}. La
radiación de fotones fue agregada con {\photos}\cite{photos}, y los
decaimientos de los leptones tau con {\tauola}\cite{tauola}.

La producción de tops resulta en un fondo poco importante para este análisis,
aunque de mayor importancia para las regiones de control y validación. La producción $Wt$
fue generada utilizando {\powheg}, incluyendo correcciones a NLO en QCD. La lluvia
partónica y la fragmentación fue simulada utilizando {\pythia} (con P2011).
Se utilizó el conjunto de PDFs CT10. Las muestras fueron normalizadas con
la sección eficaz calculada en \cite{Kidonakis:2010ux}.

%%Para la produccion en el canal $t$, las muestras MC
%% production, the MC samples with sample ID 110101 were used, with the
%% $W$ boson decaying leptonically. These were generated with
%% \acermc \cite{acer}, with parton showering and fragmentation performed
%% by {\pythia} with the P2011C tune and CTEQ6L1 PDF set.  The samples were
%% scaled to the cross section calculated by \cite{Kidonakis:2011wy}.
%% Single top produced by $s$-channel was not used because it was found
%% to be negligible.
%%Se removieron los  Overlap between the single top and single top $\gamma$ samples has been removed.

\begin{table}[!htbp]
  \centering
  \caption{Muestras de quark top y {\tgam}. La sección eficaz a
    NNLO, eficiencia del filtro, y luminosidad integrada correspondiente a la estadística total de cada muestra
    están detalladas en la tabla.}
  \begin{tabularx}{\textwidth}{z{4cm}CCCC}
    \hline
    Proceso & Generador & $\sigma$ [pb] & Eficiencia & $L$ [\ifb] \\
    \hline
    t-channel & \acermc   & 28.4 & 1.00 & 271 \\
    Wt        & \powheg   & 22.4 & 1.00 & 892 \\
    s-channel & \powheg   & 1.82 & 1.00 & 3299 \\
    \hline
    {\tgam} (t-channel) & \wizhard+\pythia   & 0.187298 & 0.121980 & 4810 \\
    {\tgam} (t-channel) & \wizhard+\pythia   & 0.313866 & 0.012927 & 4930 \\
    \hline
    {\twgam} (dilep.) & \wizhard+\pythia          & 0.01292  & 0.16437 & 4710 \\
    {\twgam} (dilep. tDec) & \wizhard+\pythia     & 0.01454  & 0.02875 & 12000 \\
    {\twgam} (dilep. WDec) & \wizhard+\pythia     & 0.01041  & 0.07549 & 6370 \\
    {\twgam} (tlepWhad) & \wizhard+\pythia        & 0.02583  & 0.16244 & 4770 \\
    {\twgam} (tlepWhad tDec) & \wizhard+\pythia   & 0.02908  & 0.02761 & 6230 \\
    {\twgam} (tlepWhad WDec) & \wizhard+\pythia   & 0.01159  & 0.06471 & 6660 \\
    {\twgam} (thadWlep) & \wizhard+\pythia      & 0.02582  & 0.16178 & 4790 \\
    {\twgam} (thadWlep tDec) & \wizhard+\pythia   & 0.02013  & 0.04198 & 5920 \\
    {\twgam} (thadWlep WDec) & \wizhard+\pythia   & 0.02079  & 0.07574 & 3180 \\
    \hline
  \end{tabularx}
  \label{tab:bkg_st_samples}
\end{table}


\subsection{{\gjet} y multijet}

La contaminación debido a la produccion QCD de fotones directos y multijets,
es en todos los casos el resultado de eventos patológicos
(jets mal-identificados como fotones, y jets o fotones mal reconstruidos dejando
una alta cantidad de energía faltante). Sin embargo, no se espera que sea un
fondo dominante en el espacio de fase explorado en este análisis. La
contribución de eventos con jets mal identificados como fotones se estima a
partir de los datos en la \cref{sec:jetfakes}.

Las muestras de multijets listadas en la \cref{tab:bkg_qcd_samples} fueron
utilizadas para el proceso de optimización y estudios de sensibilidad
preliminares. La producción de fotones directos fue simulada con {\sherpa}
v1.4.1\cite{SherpaGen}, con hasta cuatro partones en la ME+PS y usando el
conjunto de PDFs CT10. El espectro inclusivo se dividió en diversas muestras con
diferentes umbrales de {\pt} del fotón para optimizar la generación de eventos.

Muestras alternativas fueron utilizadas para la estimación de las incertezas
sistemáticas, generadas con {\pythiaeight} (usando CTEQ6L1) y {\alpgen} v2.14
(con la misma configuración que los eventos de $W/Z +$ jets descriptos en
\cref{mc_wzjets}). Los detalles pueden verse en la \cref{tab:bkg_qcd_samples}.

\begin{table}[ht!]
  \centering
  \caption{Muestras de QCD {\gjet} y multijet utilizadas en este análisis.
    La sección eficaz a LO para cada modo de decaimiento,
    y las eficiencias del filtro están detalladas,
    así como tabine la luminosidad integrada correspondiente a la estadística
    total de cada muestra.}

   \begin{tabular}{lcccc}
    \hline
    Proceso & Generador & $\sigma [pb]$ & Eficiencia & $L [fb^{-1}]$ \\
    \hline
    {\gjet} ($\pt>70\gev$)   & {\sherpa} &    2153.0  &  1.0  &  1.160 \\
    {\gjet} ($\pt>140\gev$)  & {\sherpa} &    137.85  &  1.0  &  10.881 \\
    {\gjet} ($\pt>280\gev$)  & {\sherpa} &     5.963  &  1.0  &  167.657 \\
    {\gjet} ($\pt>500\gev$)  & {\sherpa} &     0.276  &  1.0  &  3617.291 \\
    {\gjet} ($\pt>800\gev$)  & {\sherpa} &    0.0133  &  1.0  &  7492.807 \\
    {\gjet} ($\pt>1000\gev$) & {\sherpa} &   0.00238  &  1.0  &  41980.269 \\
    \hline
    {\gjet} ($\pt>70\gev$)   & {\pythiaeight} &   3425000  &  $0.00057$  &  1535.4  \\
    {\gjet} ($\pt>140\gev$)  & {\pythiaeight} &    122170  &  $0.00097$  &  8449.2 \\
    {\gjet} ($\pt>280\gev$)  & {\pythiaeight} &    3348.7  &  $0.00145$ &  206559.7 \\
    {\gjet} ($\pt>500\gev$)  & {\pythiaeight} &    115.63  &  $0.0018$  &  4789097.0\\

    %% \hline
    %% \gjetnj{1} ($\pt>70\gev$)   & {\alpgen}+{\jimmy} &   577.480  &  1.0  &  0.147 \\
    %% \gjetnj{1} ($\pt>140\gev$)  & {\alpgen}+{\jimmy} &   26.198   &  1.0  &  3.626 \\
    %% \gjetnj{1} ($\pt>280\gev$)  & {\alpgen}+{\jimmy} &   0.83119  &  1.0  &  30.077 \\
    %% \gjetnj{1} ($\pt>500\gev$)  & {\alpgen}+{\jimmy} &   0.02914  &  1.0  &  343.159 \\

    %% % \gjetnj{2} ($\ptgam>35\gev$) \alpgen+\jimmy  ( 156846 ) &  4515.0  &  1.0  &  0.00886 \\
    %% \gjetnj{2} ($\pt>70\gev$)    & {\alpgen}+{\jimmy} &   571.870  &  1.0  &  0.175 \\
    %% \gjetnj{2} ($\pt>140\gev$)   & {\alpgen}+{\jimmy} &   38.67100  &  1.0  &  3.879 \\
    %% \gjetnj{2} ($\pt>280\gev$)   & {\alpgen}+{\jimmy} &   1.6811  &  1.0  &  29.741 \\
    %% \gjetnj{2} ($\pt>500\gev$)   & {\alpgen}+{\jimmy} &   0.075517  &  1.0  &  264.841 \\

    %% % \gjetnj{3} ($\ptgam>35\gev$) \alpgen+\jimmy  ( 156851 ) &  1717.0  &  1.0  &  0.00874 \\
    %% \gjetnj{3} ($\pt>70\gev$)  & {\alpgen}+{\jimmy} &   306.10  &  1.0  &  0.049\\
    %% \gjetnj{3} ($\pt>140\gev$) & {\alpgen}+{\jimmy} &   28.57  &  1.0  &  5.250 \\
    %% \gjetnj{3} ($\pt>280\gev$) & {\alpgen}+{\jimmy} &   1.538  &  1.0  &  32.503 \\
    %% \gjetnj{3} ($\pt>500\gev$) & {\alpgen}+{\jimmy} &   0.07707  &  1.0  &  77.822 \\

    %% % \gjetnj{4} ($\ptgam>35\gev$) \alpgen+\jimmy  ( 156856 ) &  513.940002  &  1.0  &  0.00778 \\
    %% \gjetnj{4} ($\pt>70\gev$)    & {\alpgen}+{\jimmy} &   115.850  &  1.0  &  0.216 \\
    %% \gjetnj{4} ($\pt>140\gev$)   & {\alpgen}+{\jimmy} &   14.216  &  1.0  &  11.951 \\
    %% \gjetnj{4} ($\pt>280\gev$)   & {\alpgen}+{\jimmy} &   0.9185  &  1.0  &  48.992 \\
    %% \gjetnj{4} ($\pt>500\gev$)   & {\alpgen}+{\jimmy} &   0.0512  &  1.0  &  156.354 \\

    %% % \gjetnj{5} ($\ptgam>35\gev$) \alpgen+\jimmy  ( 156861 ) &  163.800003  &  1.0  &  0.0458 \\
    %% \gjetnj{5} ($\pt>70\gev$)    & {\alpgen}+{\jimmy} &   7.00  &  1.0  &  18.569 \\
    %% \gjetnj{5} ($\pt>140\gev$)   & {\alpgen}+{\jimmy} &   0.542  &  1.0  &  92.304 \\
    %% \gjetnj{5} ($\pt>280\gev$)   & {\alpgen}+{\jimmy} &   0.0333  &  1.0  &  450.911 \\
    %% \gjetnj{5} ($\pt>500\gev$)   & {\alpgen}+{\jimmy} &   44.334  &  1.0  &  0.970 \\

    \hline
    Multijet JZ1W ($\pt^{\mathrm{jet}} \in [20, 80] \gev$)     & {\pythia} &   $7.285 \cdot 10^{10}$ &  0.000129 & 0.00016 \\
    Multijet JZ2W ($\pt^{\mathrm{jet}} \in [80, 200] \gev$)    & {\pythia} &   $2.634 \cdot 10^{7}$ &  0.003894 & 0.0142 \\
    Multijet JZ3W ($\pt^{\mathrm{jet}} \in [200,500] \gev$)    & {\pythia} &   $5.442 \cdot 10^{5}$ &  0.001219 & 2.26 \\
    Multijet JZ4W ($\pt^{\mathrm{jet}} \in [500,1000] \gev$)   & {\pythia} &   $0.006445$ &  0.000708 & 328 \\
    Multijet JZ5W ($\pt^{\mathrm{jet}} \in [1000,1500] \gev$)  & {\pythia} &   39.74 &  0.002152 & 17400 \\
    Multijet JZ6W ($\pt^{\mathrm{jet}} \in [1500,2000] \gev$)  & {\pythia} &   0.4161 &  0.004684 & $7.68 \cdot 10^{5}$ \\
    Multijet JZ7W ($\pt^{\mathrm{jet}} > 2000 \gev$)           & {\pythia} &   0.04064 &  0.0146 & $2.52\cdot 10^{6}$ \\
    \hline
  \end{tabular}
  \label{tab:bkg_qcd_samples}
\end{table}

\subsection{Diboson}

Los procesos de diboson (WW, WZ, y ZZ) fueron generados utilizando
{\sherpa} y usando la PDF CT10, con la sección eficaz provista por
MCFM\cite{Campbell:2011bn}. Solo los decaimientos leptonicos para
ambos bosones fueron considerados.

\begin{table}[ht!]
  \centering
  \caption{Muestras de Diboson.
    La sección eficaz a LO para cada modo de decaimiento, los factores $k$
    (para la normalización NLO) y las eficiencias del filtro están detalladas,
    así como también la luminosidad integrada correspondiente a la estadística
    total de cada muestra.}

  \begin{tabular}{lccccc}
    \hline
    Proceso & Generador & $\sigma [pb]$ & factor $k$ & Eficiencia & $L [fb^{-1}]$ \\
    \hline
    %% $WW(2l2\nu)$ \sherpa (126892)  & 5.50 & 1.07 & 1 & 458.9 \\
    %% $WZ(3l)$ \sherpa (126893) & 9.75 & 1.06 & 1 & 261.1 \\
    %% $ZZ(4\ell)$ \sherpa (126894)  & 8.74 & 1.11 & 1  &  185.6 \\
    %% $ZZ(2\ell2\nu)$ \sherpa (126895)  & 0.50 & 1.14 & 1 &  1590.8 \\
    $WW (\to \ell\ell\nu\nu)$     & {\sherpa}  & 5.296  & 1.06 & 1.00 & 1400 \\
    $ZZ (\to \ell\ell\nu\nu)$     & {\sherpa}  & 0.494  & 1.05 & 1.00 & 1700 \\
    $WZ (\to \ell\ell\ell\nu)$    & {\sherpa}  & 9.745  & 1.05 & 1.00 & 260 \\
    $WZ (\to \ell\nu\nu\nu)$      & {\sherpa}  & 1.406  & 1.05 & 1.00 & 270 \\
    $ZW (\to eeqq)$               & {\sherpa}  & 1.465  & 1.05 & 1.00 & 110 \\
    $ZZ (\to eeqq)$               & {\sherpa}  & 0.247  & 1.00 & 1.00 & 120 \\
    $ZW (\to \mu\mu qq)$          & {\sherpa}  & 1.463  & 1.05 & 1.00 & 110 \\
    $ZZ (\to \mu\mu qq)$          & {\sherpa}  & 0.248  & 1.00 & 1.00 & 120 \\
    $ZW (\to \tau\tau qq)$        & {\sherpa}  & 1.452  & 1.05 & 1.00 & 120 \\
    $ZZ (\to \tau\tau qq)$        & {\sherpa}  & 0.242  & 1.00 & 1.00 & 120 \\
    $ZW (\to \nu\nu qq)$          & {\sherpa}  & 2.697  & 1.05 & 1.00 & 64 \\
    $ZZ (\to \nu\nu qq)$          & {\sherpa}  & 1.744  & 1.00 & 1.00 & 69 \\
    $WW (\to e\nu qq)$            & {\sherpa}  & 7.285  & 1.06 & 1.00 & 100 \\
    $WZ (\to e\nu qq)$            & {\sherpa}  & 1.904  & 1.05 & 1.00 & 110 \\
    $WW (\to \mu\nu qq)$          & {\sherpa}  & 7.297  & 1.06 & 1.00 & 100 \\
    $WZ (\to \mu\nu qq)$          & {\sherpa}  & 1.906  & 1.05 & 1.00 & 100 \\
    $WW (\to \tau\nu qq)$         & {\sherpa}  & 7.274  & 1.06 & 1.00 & 100 \\
    $WZ (\to \tau\nu qq)$         & {\sherpa}  & 1.915  & 1.05 & 1.00 & 100 \\
    \hline
  \end{tabular}
  \label{tab:bkg_diboson_samples}
 \end{table}



%% Referencias
\nocite{Seymour:2013ega}
