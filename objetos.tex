\chapter{Reconstrucción de objetos físicos} %\section{Selección de objetos}
\label{sec:obj_selection}

%% In this section a description of the reconstruction and calibration is given for
%% the main physics objects this analysis has to deal with (photons, \MET, jets, muons and electrons),
%% following the standard ATLAS procedure \footnote{Objects are defined following the latest SUSY group
%% recommendations which are implemented in the SUSY working group package SUSYTools-00-03-23-02.}.
%% The use of these objects in the different signal and control regions is described in \Sec \ref{sec:signal_regions}
%% and \ref{sec:CRs}, respectively.

En esta sección se describe la reconstrucción y calibración de los
principales objectos utilizados en el presente an\'alisis:
fotones, jets, muones, electrones y \met.

%% ), siguiendo el procedimeinto
%% estandar de ATLAS \footnote{Objects are defined following the latest SUSY
%%   group recommendations which are implemented in the SUSY working group
%%   package SUSYTools-00-03-23-02.}.


\section{Fotones}
\label{sec:obj_photons}

Los fotones son reconstruidos de los clusters en el calorímetro electromagnético
medidos en torres de $3\times5$ celdas en el plano $\eta\times\phi$ de la
segunda capa del calorímetro. Los fotones son clasificados como
\emph{no-convertidos} si no tienen trazas de un vértice de conversión asociadas
al cluster, y sino como \emph{convertidos}. Un vértice de conversión es formado
cuando dos trazas que pasan el TRT high-threshold requirement forman un vértice
consistente como que vienen de una partícula no masiva. Para aumentar la
eficiencia de reconstrucción de fotones convertidos, los candidatos a conversión
donde solo una de las dos trazas es reconstruida (y no tiene ningún hit en la
capa mas interior del detector de pixeles) también son retendidas

Los fotones pre-seleccionados son entonces lo que tienen un {\pt} mayor a 20
{\gev} con $|\eta| < 2.37$, removiendo los que se encuentran en la región del
crack $1.37 < |\eta| < 1.52$, donde $\eta$ se mide en el cluster de la segunda
capa del calorímetro.

Cuando se calcula el {\pt} del fotón, la energía es tomada siempre del cluster
del calorímetro, apropiadamente calibrada \cite{Banfi:1259219}.
Para fotones no-convertidos, el $\eta$ es calculado usando las dos primeras
capas del calorímetro electromagnético. Para fotones convertidos, donde la traza
o las trazas que provienen del vértice de la conversión contiene mas de tres
silicon hits, la dirección $\eta$ se determina extrapolando del cluster del
calorímetro al vértice de la conversión.
Para fotones convertidos que tienen trazas solo en el TRT, $\eta$ es calculada
del \hl{calorimeter pointing}, como en el caso de los no-convertidos. Además
la escala de energía del fotón es corregida para datos y smeareada para MC,
siguiendo \cite{EGScaleTwiki}.

Algunos cortes de limpieza son aplicados a los candidatos a fotones para
identificar clusters con mala calidad o falsos que vienen de problemas
instrumentales \footnote{\url{https://twiki.cern.ch/twiki/bin/view/AtlasProtected/LArCleaningAndObjectQuality\#Photon_Cleaning}}.
Tambien se aplica una limpieza de evntos basada en fotones. Un mal foton
es definido como aquel que tiene un tiempo de cluster $|t|>(10+2/|E_\text{clus}|) \text{ns}$,
donde $E_\text{clus}$ es la correccion de energia al tiempo del cluster,
o si el valor:
\begin{equation}
  \frac{\sum_\text{cluster} E_\text{cell}(Q>4000)}{\sum_\text{cluster} E_\text{cell} } > 0.8\%
\end{equation}
%
y las variables $R_\phi > 1.0$ o $R_\eta > 0.98$, definidas
en \cite{PhotonCleaning}. El factor Q mide la diferencia entre la forma
del pulso medido y la forma esperada que es usada para reconstruir
la energia de la celda.

Otros cortes de identificacion son ademas impuestos para separar los
candidatos a foton de la contaminacion que proviene de {\pizero} o
algun otro hadron neutro decayendo en dos fotones. La identificacion
de fotones se basa en el perfil de la energia depositada en la primer
y segunda capa del calorimetro electromagnetico \cite{ATL-PHYS-PUB-2011-007}.
Los criterios de seleccion en las variables que describen la forma de
la lluvia son independientes de la energia transversa del candidato a
foton, pero varian segun la pseudorapidez reconstruida del foton, para
tener en cuenta las variaciones en la cantidad de material y la geometria
del calorimetro. Estos estan optimizados independientemente para fotones
convertido sy no-convertidos para tener en cuneta las diferencias en las
lluvias en cada caso. Este analisis hace uso de fotones que pasa los
criterios de seleccion \emph{loose} y \emph{tight}, como se definen aca
\cite{ATL-PHYS-PUB-2011-007}.

En areas donde un modulo de la b-layer esta muerto, los electrones son
comunmente reconstruido como fotones convertidos. Esta ambiguedad
es resulta teneinedo en cuenta el mapa de los pixeles muertos como
funcion del tiempo que dura la recolleccion de datos. Las conversiones
con una sola traza falsas (y algunas de dos trazas) son de esta forma
reducidas, con solo un minima reduccion de la eficiencia.

Para la seleccion final, se aplica ademas un corte en la energia
transversa de aislamiento {\etiso}. La anergia de aislamiento
es calculada como la suma de la energia transversa de los clusters
topologicos (calibrada a la escala electromagnetica) dentro de un
cono en el plano $\eta-\phi$ de radio $\Delta R = 0.20$ alrededor
del baricentro del cluster \footnote{internally referred as
  \texttt{topoEtCone20} in ATLAS.}.
Solo los opoclusters con energia positiva son usados. Los topoclusters
incluyen celdas del calorimetro EM y del calorimetro hadronico, pero
las celdas del TileGap3 son explicitamente removidas.
La energia del centro del cono en el calorimetro EM ($5\times7$ celdas alrededor el
baricentro del objeto) es sustraida de la suma. Luego se aplican correciones
debido a la energia ambiente por la actividad del pileup calculadas de acuerdo
a \cite{Hance:1379530}, mejorando la performance de la variable de aislamiento
para alto pileup \cite{Laplace:1444890}. La resultante energia de aislamiento
depues de aplicarle las correciones tiene que ser menor que 5 {\gev} para
pasar la seleccion final.

\subsection{Correciones para fotones}

Algunas correcciones son aplicadas para mejorar el acuerdo entre datos y
simulaciones MC del modelado de las lluvias que dejan los fotones al
pasar por el detector.

\subsubsection{Correccion de identificacion}

Las distribuciones de las variables del calorimetro que son usadas
para discriminar entre fotones y jets difieren entre datos y MC.
Las diferencias en cada variable discriminatoria (DV$^i$) pueden
aproximarse por un \emph{fudge-factor} ($\mu^i$), calculado de
los valores promedio en datos y MC \cite{ATLAS-CONF-2012-123}.
\footnote{\url{https://twiki.cern.ch/twiki/bin/viewauth/AtlasProtected/PhotonFudgeFactors}}:

\begin{equation}
  \Delta \mu^i = \avg{DV^i_\text{DATA}} - \avg{DV^i_\mathrm{MC}}
\end{equation}

Estos factores son utilizados para corregir las variables de las
muestras simuladas, y los cortes de identificacion son nuevamente
aplicados a las variables corregidas. La eficiencia de identificacion
de fotones es por lo tanto ajustadas para igualar las medidas en datos.

Estas modificaciones son calculadas comparando todas las formas de lluvia
de todos los fotones \emph{tight} aislados observados en los datos del 2012
y las muestras MC12a JF17-140, y provistos de forma central como parte del
paquete \texttt{egammaAnalysisUtils (v-00-04-52)}.
La diferencia en las eficiencias de identificacion por los diferentes
criterios de aislamiento utilizados para derivar los dusge factoes es
evaluada en \cref{sec:syst_photonid}.

Tambien son aplicados facotres de escala a todos los fotones identificados
en el MC para tener en cuenta las diferencias observadas en la eficiencia
respecto a los datos. Estos factores son derivados de forma separada para
fotones convertidos y no-covnertidos y tambien son provistos de forma central
por el grupo \emph{Egamma} como parte del paquete
\texttt{PhotonEfficiencyCorrectionTool (tag 00-00-05)}.

\subsubsection{Correccion al aislamiento}

A difference between the photon energy leakage into the isolation cone is
observed between data and Monte Carlo for photons. Therefore, the
corrections applied as described in \cite{Hance:1379530} are slightly
different for data and MC. Evenmore, a remanent dependency of the isolation
energy with the photon \pt\ is observed in data, for what an extra correction factor
has been derived as explained in \cref{sec:opt_ph_iso}.

\section{Electrones}
\label{sec:elec_obj}

Los electrones son reconstruidos de forma similar a los fotones, con
requerimientos adicionales en el detector de trazas \cite{Aad:2011mk}.
Similares criterios de calidad a los descriptos en la sección anterior
se aplican a todos los candidatos a electrones  para identificar fakes
debidos a problemas instrumentales.
La energía de los electrones es reconstruida de clusters en el calorímetro
electromagnético sin tener en cuenta su masa. Mientras que la información
del detector de trazas es usada para reconstruir su dirección. \note{La escala
de energía en datos es corregida para igualar la observada en datos} \cite{EGScaleTwiki}.
%% \hl{An extra spread is introduced on the MC electron energy in order to reproduce the
%%   resolution measured on data for benchmark channels.}
A los electrones preseleccionados se les pide que $\pt>10\gev$ y $|\eta|<2.47$.

Se pide ademas un criterio de identificacion \emph{medium} \cite{ATL-PHYS-PUB-2011-006},
que esta basado en las caracteristicas del desarrollo de la lluvia
electromagentica, la calidad de las trazas reconstruidas y la cercania del match
entre la traza y el deposito en el calorimetro. Este conjunto de requerimientos
provee una alta y uniforma eficiencia de identificacion con un abjo fondo.

%To keep this analysis statistically independent to other searches in ATLAS \cite{leptonphoton}, %making easier the combination of the results,
%% To keep this analysis complementary to other searches in ATLAS \cite{leptonphoton7,Zplusmet7}, %making easier the combination of the results,
%% events containing (at least) a good identified lepton have to be rejected. For this selection, an additional isolation requirement is applied requiring that the sum of the tracks $p_{\rm T}$ in a cone of $\Delta R = 0.30$ around the electron, excluding its own track and any conversion vertex tracks, must be at most 16\% of the electron \pt \footnote{This criteria is referred as LooseIso within the SUSYTools package.}. The tracks have to have 7 silicon hits, at least one of which must be a b-layer hit, and the impact parameter along the beam line (z$_0$), must be less than 1 mm for pileup tolerance. Additionally,  $|z_0 \times \mathrm{sin} \theta | < 0.4$ is required for the electron track.

Al igual que en el caso de los fotones, se aplican factores de escala
a los electrones del MC para corregir las diferencias observadas en la
eficiencia entre datos y MC.

\section{Muones}
\label{sec:muon_obj}

Los candidatos a muon son reconstruidos usando el algoritmo de reconstruccion
\texttt{STACO} \footnote{\url{https://twiki.cern.ch/twiki/bin/viewauth/AtlasProtected/STACODocumentation}},
que combina la informacion del detector de trazas del espectrometro
de muones y el detector interno. Todos los cortes de identificacion
provienen de las recomedndaciones propuestas por el grupo de muones \cite{MCPTwiki}.
Se pide que los muones sean \textit{Combined}, donde el muon es reconstruido
independientemente en el espectrometro de muones y en el detector interno, o
\textit{Segment-tagged}, donde el MS es usada para taggear trazas del ID
como muones, sin requerir una reconstruccion total de la traza en el MS.
El momento transverso reconstruido en ambos detectores (ID y MS)
es esmeareado en MC para reproducir la resolucion del momento en datos.
Todos los muones reconstruidos con $\pt>6 \gev$ (despues del smearing) y $|\eta|<2.5$
se consideran candidatos.
Ademas, se pide que los candidatos pasen un criterio de calidad \textit{Loose}
\footnote{\url{https://twiki.cern.ch/twiki/bin/viewauth/AtlasProtected/QualityDefinitionStaco}},
mas algunas criterios de calidad en el detector interno de trazas como se describe a continuacion:
\begin{itemize}\itemsep0.1cm
\item[-] La traza en el ID debe tener un hit en la b-layer, a menos que el modulo de la b-layer este muerto.
\item[-] La suma del numero de pixel hits y crossed dead pixel sensors must be greater than one.
\item[-] La suma del numero of SCT hits and crossed dead SCT sensors must be at least six.
\item[-] La suma del numero of pixel and SCT holes must be less than three.
\item[-] Si la traza en el ID estra dentro de la aceptancia del TRT, se pide:
  \begin{itemize}\itemsep0.1cm
  \item[-] $n = n_{TRT}^{hits} + n_{TRT}^{outliers}$
  \item[-] Caso 1: $|\eta| < 1.9$. Require $n > 5$ and $n_{TRT}^{outliers} < 0.9n$.
  \item[-] Caso 2: $|\eta| \geq 1.9$. If $n > 5$, then require $n_{TRT}^{outliers} < 0.9n$.
  \end{itemize}
\end{itemize}

Como para electrones, la seleccion final en este analisis veta la presencia
de cualquier mal muon en los eventos.
Se requiere que estos que adicionalmente pasen un criterio de aislamiento de la
traza: que la suma del {\pt} de las trazas, excluyendo la traza del muon, en un
cono de $\Delta R < 0.3$ que sea menor que \unit[12]{\%} del {\pt} del muon.
Las trazas tiene que tener 4 silicon hits y el parametro de impacto a lo largo de
la linea del haz ($z_{0}$), debe tener menos de $\unit[10]{mm}$ de tolerancia de pileup.
El corte en el {\pt} de muones para el veto se sube a $25 \gev$. Ademas se aplican
factores de correcion para llevar la eficiencia en el MC a la medida en datos a partit
del decaimiento del $Z$, provistos por el grupo de performance de muones.

\section{Jets}
\label{sec:jet_obj}

Los jets son reconstruidos usando el algoritmo anti-kt \cite{Cacciari:2008gp} con
un parametro de distancia de $R = 0.4$  (en el espacio $\eta - \phi$) a partir
de clusters calorimetricos topologicos \cite{Lampl:1099735}.
Estos son calibrados usando el metodo de calibracion por pesado por clusters locales (LCW)
que consiste en pesar de forma diferenciada los depositos de energia que vieen de lluvias en
el calorimetro electromagnetico o hadronico.
Estas correcciones en la energia aplicadas a los clusters topologicos son derivadas de
simulaciones Monte Carlo. La calibracion final en la energia del jet tambien incluye la escala
de energia (JES) que corrije la respuesta del calorimetro a la energia verdadera del jet.
Esto corresponde a la calibracion LCW+JES.
Excepto para el calculo de {\met} y la limpieza de eventos, donde ningun corte en $\eta$ es
aplicado, los jets solo se consideran y estan en la region central del detector ($|\eta|<2.8$)
y con un $\pt > 20\gev$.

Para reducir la contaminacion de los jets \hl{falsos} que no provengan de colisiones y mejorar
la resolucion de {\met}, los eventos con al menos un jet que falla alguno de los siguientes
criterios  de calidad son removidos \cite{JetCleaning}.

\begin{itemize}\itemsep0.1cm
\item[-] Si la fraccion de energia en la endcap del calorimetro hadronico
  es mayor a 0.5 $\left(\mathrm{HEC}f > 0.5\right)$, the measured absolute
  value of quality of the jet is greater than 0.5 $\left(|\mathrm{HEC}Q| > 0.5\right)$,
  and the normalized jet quality computed as the energy squared cells mean
  quality is larger than 0.8 $LArQmean > 0.8$.
\item[-] If absolute value of the total energy in cells with a negative value
  is greater than $\unit[60]{GeV}$ the jet is considered bad
  $\left(|\mathrm{neg.}\,E| > \unit[60]{GeV}\right)$.
  For this and the previous item, the signal is consistent with sporadic
  noise in the hadronic endcap calorimeters.
\item[-] If the electromagnetic energy fraction is larger than 0.95
  $\left(\mathrm{EM}f > 0.95\right)$, the absolute jet quality value is greater than
  0.8 $\left(|\mathrm{LAr}Q| > 0.8\right)$, and the normalized jet quality is larger than 0.8
 $LArQmean > 0.8$ for jets with $|\eta| < 2.8$.
\item[-] If the electromagnetic energy fraction is less than 0.05
  $\left(\mathrm{EM}f < 0.05\right)$ and the ratio of the sum \pt\ of the
  tracks associated to the jets divided by the calibrated jet \pt\ is less than 0.05
  $\left(\mathrm{ch}f < 0.05\right)$ for jets with $|\eta| < 2$.
  In the case where the jet has $|\eta| \ge 2$ the jet is considered bad if
  the electromagnetic energy fraction is less than 0.05 with no requirement on
  the jet charge fraction.
\item[-] Si un jet tiene mas del $99\%$ de su energia contenida en una sola capa
  del calorimetro $(\mathrm{F}max > 0.99)$ y tiene $|\eta| < 2$, es consistente
  con una senal de rayos cosmicos o \hl{beam halo muons}.
\end{itemize}

Excepto durante calculo de {\met}, dos cortes de aceptancia extra son requeridos
para los jets: $\pt > 40 \gev$ y $|\eta| < 2.8$.
Todos los jets pasando esta selección \emph{loose} son considerados cuando se aplica
la identificación de objetos descripta en \cref{sec:overlap_romoval_event_veto}.
El corte en {\pt} fue elegido para asegurar una selección robusta contra el pile-up.
Como se muestra en la \cref{fig:NjetvsPV}, el numero medio de jets seleccionados
es de esta forma una distribución \hl{flat} como función del numero de vértices
primarios. Luego, en la selección final, cortes mas alto en el {\pt} de los jets
son utilizados como se describe en \cref{sec:opt_njet}.

\begin{figure}[ht!]
  \centering
  \includegraphics[width=0.70\textwidth]{figures/data_npv_njets}
  \caption{Valor medio del numero de jets vs. el número de vértices primarios
    observados en datos para distintas selecciones de $\pt^{jet}$.}
    \label{fig:NjetvsPV}
\end{figure}


%% \subsection{b-jets}
%% \label{sec:bjet_obj}

%% %% Aunque los $b$-jets no son explicitamente utilizados en la selecc
%% %% Although b-jets are not explicitly used for the analysis selection, they are useful in the definition of control regions
%% %% from which the \wgamma\ %and \ttbargam
%% MC normalization is extracted as described in \Sec \ref{sec:CRs}. b-jets are identified using
%% the MV1 jet tagger \cite{ATLAS-CONF-2012-043} at the 70\% efficiency operating point, corresponding
%% to the requirement $w > 0.7892$ where $w$ is a weight computed from the different discriminating
%% variables forming the jet tagger. The b-tagging efficiencies have been determined by the flavour
%% tagging working group using the \pt\ of muons relative to the axis of the jet ($p_{T}^{rel}$) \cite{ATLAS-CONF-2012-043} and the
%% derived default scale factors (without JVF cut) are applied to Monte Carlo following the official recommendations \cite{bjetsCalib}.
%% In this analysis, only b-tagged jets with $\pt >$~40 GeV and $|\eta| < 2.5$ will be identified as b-jets.
%% An uncertainty on the jet weight and the event weight is calculated propagating the estimated uncertainties
%% on the scale factors. Scale factor uncertainties depend on the kinematics of the jet and also on the jet flavor.

%See https://indico.cern.ch/getFile.py/access?contribId=37&resId=0&materialId=slides&confId=194205
%or https://twiki.cern.ch/twiki/bin/viewauth/AtlasProtected/JetEtmissDataAnalysisRecommendationSummer2010#Recommendation_for_MET_reconstru
\section{Energía faltante transversa}
\label{sec:met_obj}

La energía faltante transversa es calculada con un algoritmo basado en objetos.
%% Missing transverse momentum is calculated with an object-based algorithm at the AOD level
%% \\ (\texttt{MET\_Egamma10NoTau\_RefFinal}). As a consequence, the computation of \met\ uses reconstructed and calibrated physics objects. Calorimeter energy
%% deposits (TopoClusters) are associated to high-pT objects in the following order: electrons, photons, jets and muons. Deposits not associated
%% with any such objects are included in the SoftTerm. The \met\ is calculated as the sum of the following terms:

El algoritmo utilizan para el calculo de la energía faltante, los objetos físicos
reconstruidos y calibrados descriptos en las secciones anteriores. Los depósitos
de energía en el calorímetro (topoclusters) son asociados a los objetos de alto
{\pt} en el siguiente orden: electrones, fotones, jets y muones. Los depósitos
que no están asociados a ningún objeto son incluidos en el termino \emph{soft}:

\begin{equation}
  (\etmiss)_{x(y)} = (\etmiss)^e_{x(y)} + (\etmiss)^\gamma_{x(y)}+(\etmiss)^\text{jet}_{x(y)}+(\etmiss)^\text{muon}_{x(y)}+(\etmiss)^\text{soft}_{x(y)}
\end{equation}
%
donde cada termino es calculada de la suma de los objetos reconstruidos y
calibrados mas el termino soft.

%% where each term is computed from the negative sum of calibrated reconstructed objects and from the SoftTerm.
%% Contribution from electrons are included in $(\etmiss)^\text{RefEle}_{x(y)}$ using electrons passing medium purity criteria
%% with $\pt>10\gev$ and before overlap removal. Contribution from photons are included in $(\etmiss)^\text{RefGamma}_{x(y)}$
%% using photons passing tight purity criteria with $\pt>20\gev$. Contribution from jets are included at the jet energy scale
%% in $(\etmiss)^\text{RefJet}_{x(y)}$ for calibrated jets with $\pt>20\gev$, independently of $\eta$. Contributions from muons
%% are included in $(\etmiss)^\text{Muon}_{x(y)}$, using the muons passing the criteria described above (including $\pt>10\gev$)
%% except the isolation requirement and before overlap removal. $(\etmiss)^\text{SoftTerm}_{x(y)}$ is computed from locally
%% calibrated TopoClusters and tracks unmatched to any reconstructed object, using an energy-flow algorithm. The main differences in \etmiss algorithm
%% used in this analysis with respect to the standard \texttt{MET\_RefFinal} algorithm are: the absence of hadronically decaying
%% $\tau$-leptons term, and a slightly redefined muon term which contain only muons passing selection defined in this analysis.
%% The lack of specific $(\etmiss)^\text{RefTau}_{x(y)}$ term means that hadronic taus are included either in $(\etmiss)^\text{RefJet}_{x(y)}$ term
%% or in $(\etmiss)^\text{SoftTerm}_{x(y)}$ term depending in the \pt\ of the associated jet.

%% The MissingETUtility algorithm is used to correct \etmiss\ for small differences between object definitions used in the standard \texttt{MET\_Egamma10NoTau\_RefFinal} and the baseline %SUSY group standard
%% definitions outlined above (e.g. the smearing of the lepton \pt\ in MC). The tool also allows to propagate the individual objects uncertainties through the \etmiss\ calculation.

% Uncertainties affecting the individual objects are propagated to \etmiss via the MissingETUtility algorithm. %In addition, specific systematic uncertainties on the scale and resolution of the soft term
% have been evaluated in  two different in-situ methods using Z$\to\mu\mu$ events \cite{}.

%Due to conservation of transverse momentum, if all particles produced
%in the primary collision are detectable then there should be no \MET
%in the event unless it arises from detector effects e.g. resolution,
%material effects, or
%non-instrumented regions of the detector.
%Events in which undetectable particles are
%produced, such as the gravitino, can be expected to have large \MET.
%
%
%The missing transverse momentum $\ETM$ is calculated from the energy
%deposited in calorimeter cells up to $|\eta|<4.9$, and from muons.
%Each cell is associated with some object, which is then used to
%define a calibration for the signal observed in the cell. Since this
%analysis will make use of loose photons to form control samples
%for its data-driven background estimates,
%the possibility of using loose rather than tight
%selection criteria for photon-like objects was explored.
%The objects are:
%\begin{itemize}
%\item Medium electrons with $\pt >$ \unit[10]{GeV}, using the default calibration from the egamma group and $\pt >$ \unit[10]{GeV};
%\item Photons (loose or tight depending on the \MET definition in play) with $\pt >$ \unit[10]{GeV}, calibrated at the EM scale;
%\item Tight taus with LCW calibration;
%\item LC topo anti-kT R=0.4 jets (with $\pt >$ \unit[20]{GeV}), using the default JES calibration of the Jet/Etmiss group;
%\item LC topo anti-kT R=0.4 jets (with \unit[10]{GeV}$< \pt <$ \unit[20]{GeV}, calibrated with LCW (not applying the JES) ;
%\item Out-of-cell (CellOut) energy, as determined by the track-cluster matching algorithm, calibrated with LCW;
%\item Muons, as define in https://twiki.cern.ch/twiki/bin/viewauth/AtlasProtected/EtMissMu.
%\end{itemize}
%These directional object energies are combined vectorially, with the resultant being the negative of the missing-tranverse-momentum vector;
%the magnitide of this vector is the missing energy $\MET$.
%
%In developing the event selection for the analysis, several different definitions of the missing transverse energy
%observable were explored. The \unit[5]{fb$^{-1}$} analysis made use of the LocHadTopo $\MET$ observable, calculated from the energy
%deposited in calorimeter cells associated to a topocluster,
%with the calorimeter cell energy is calibrated by applying weights from Local
%Hadron Calibration~\cite{REF_LH} of Topoclusters~\cite{TopoCluster},
%optimized from single pions. For object-based $\MET$ observables, three \MET variables were
%considered: EGamma10NoTauLoosePhotonRef (for which loose photons were calibrated at the EM scale),
%EGamma10NoTauPhotonRef, and the standard MetRefFinal variable used by groups exploring non-photonic signatures.
%
%A number of studies were done with both data and MC to explore the performance of these four \MET variables
%in the p1328 reconstruction. Since the data was blinded above $\MET = 100$ GeV, we report here on the result of
%a study making use of a MC sample of SM $\gamma \gamma$ events. To ensure reasonable statistics for large values of
%\MET, a generator-level filter of $\ET > 95$ GeV was applied to the two leading tight, isolated photons in the sample, and an
%offline cut of $\ET > 100$ GeV was applied. Figure~\ref{fig:met_comp_hist} shows a comparison of the resulting
%\MET distributions for the four candidate \MET varilables. For both intermediate and large values of \MET,
%the MetRefFinal and EGamma10NoTauPhotonRef variables are seen to provide the best performance, while LocHadTopo
%and EGamma10NoTauLoosePhotonRef are observed to be somewhat worse. For reasons that will be motivated in the section
%on QCD backgrounds, the MetRefFinal is chosen as the varialbe to be used in the analysis, with LocHadTopo retained as a
%cross-check.
%
%
%\begin{figure}
%  \centering
%  \includegraphics[width=0.7\textwidth]{figures/dummy.eps}
%%Figure is p3 of https://indico.cern.ch/getFile.py/access?contribId=1&resId=1&materialId=slides&confId=254874 (Osamu)
%  \caption{$\MET$ distributions for the four candidate $\MET$ definitions for a sample of diphoton MC events. MetRefFinal
%and EGamma10NoTauPhotonRef are seen to provide the best performance for intermediate and high values of \MET.
%}   \label{fig:met_comp_hist}
%\end{figure}
%

%%%MOVED TO OPTIMIZATION SECTION!!
%\subsection{Total Transverse Energy (\HT)}
%\label{sec:ht_obj}
%Given the high-mass gluinos produced in the GGM model-space explored in this analysis, the total visible transverse energy
%is expected to be large. Thus, the observable \HT is defined as the scalar sum of the transverse energy of all individual visible
%objects in the final state. After the lepton veto described in \Sec \ref{}, it is effectively defined as:
%
%\begin{eqnarray} \label{eq:htaddition}
%\HT  &\equiv& \pt^{\gamma} + \sum_{i=1}^{N_{jets}} \pt^{jet}
%\end{eqnarray}
%
%
%\subsection{Jet-\MET $\phi$ Separation}
%\label{sec:dphi_obj}
%If significant \MET arises due to mis-measurement of jet energies,
%backgrounds might be expected to accumulate for which there is only
%a smallazimuthal separation between the \ETM vector and the
%direction of a leading or subleading jet. The minimum
%azimuthal angle between \ETM and the direction of
%the leading or subleading jet is defined as
%\begin{eqnarray} \label{eq:dphi}
%\cos \Delta \phi_{min}^{jet}  &\equiv& min [\frac{\vec{\MET} \cdot \vec{E}_{T}^{jet,i}}{\MET |\vec{E}_{T}^{jet,i}|}]; \;\; i = 1,2
%\end{eqnarray}
%where the $x$ and $y$ indicate the projections of the \ETM and jet-energy vectors onto the $x$ and $y$ axes.

\section{Eliminación de objetos superpuestos} %% y veto de eventos\note{?}}
\label{sec:overlap_romoval_event_veto}

De acuerdo a las definiciones de objetos anteriores, un objeto puede estar en mas de una
categoría, contándose dos veces. Por este motivo se realiza un procedimiento para remover
este solapamiento, aplicándose sobre los objetos preseleccionados antes que los criterios
de aislamiento sean impuestos. El orden del overlap removal es el siguiente:

\begin{itemize}\itemsep0.1cm
\item Si los clusters de un fotón o electrón se encuentran dentro de $\Delta R < 0.01$,
  el objeto es considerado como un electrón, removiéndose el fotón correspondiente. Esta
  elección reduce la taza de electrones mal reconstruidos como fotones.
\item Los jets que estén cerca ($\Delta R<0.2$) de un electrón o fotón preseleccionado
  se remueve.
\item Fotones y electrones preseleccionados son removidos si su distancia al jet mas
  cercano es de $\Delta R < 0.4$.
\item Muones preseleccionados son removidos si su distancia al jet mas cercano es $\Delta R < 0.4$.
\end{itemize}

Para poder estar seguro que {\met} esta bien medida, los eventos que satisfacen
alguna de las condiciones siguientes son descartados.
\begin{itemize}\itemsep0.1cm
\item Si el evento (después del overlap removal) contiene jets y al menos uno de ellos
  falla los cortes de limpieza de jets que se definen en \cref{sec:jet_obj}.
\item El evento tiene al menos un muon con  $|z_{0}| >   \unit[1]{mm}$ ó
  $|d_0| > \unit[0.2]{mm}$, donde estos valores son calculados con respecto al vértice
  primario.
\end{itemize}


%% \itodo{Add MET reconstruction, jets, btagging}
