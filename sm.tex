\chapter{\SM}

El {\SM} de la física de altas energías (SM) describe correctamente a los
constituyentes de la materia y sus interacciones. Esta teoría fue formulada en
varios trabajos a partir de la segunda mitad del siglo XX y prácticamente todos
los datos obtenidos con los experimentos de altas energías pueden explicarse
dentro de este marco. Sin embargo este modelo tiene algunas debilidades tanto
teóricas como experimentales y por lo tanto no puede ser considerado como una
teoría fundamental.

En este capítulo se describe brevemente las características mas relevantes de
esta teoría y al final se detallan algunas de las debilidades que dan lugar a
los modelos que intentan explicar la física mas allá del SM.


\section{Las partículas fundamentales y sus interacciones}

De acuerdo al SM toda la materia esta compuesta por un n\'umero peque\~no de
part\'iculas fundamentales de spin $1/2$, conocidas como fermiones, que se
dividen en dos tipos: quarks y leptones (ver \cref{tab:fermions}). Hasta el
momento, ningún experimento ha podido encontrar evidencia de que estos fermiones
tengan una subestructura interna.

\begin{table}[!ht]
  \centering
  \begin{tabular}{c|ccc|ccc|ccc}
    %%\hline
    & \multicolumn{3}{c}{Particula} & \multicolumn{3}{|c|}{Masa} & \multicolumn{3}{|c|}{Carga Electrica} \\

    \hline
    \multirow{2}{*}{Leptones}
    & e & $\mu$ &  $\tau$ & 0.511 \mev & 105.7 \mev & 1768 \mev & -1 & -1 & -1 \\
    & $\nu_e$ & $\nu_\mu$ & $\nu_\tau$ & $<2.2 \eV$ & $< 0.17 \mev$ & $<15.5 \mev$ & 0 & 0 & 0 \\
    \hline
    \multirow{2}{*}{Quarks}
    & $u$ & $c$ & $t$ & 2.4 \mev & 1.27 \gev & 171.2 \gev & 2/3  & 2/3 & 2/3 \\
    & $d$ & $s$ & $b$ & 4.8 \mev & 104 \mev & 4.2 \gev & -1/3 & -1/3 & -1/3 \\
    %\hline
  \end{tabular}
  \caption{Partículas fundamentales...}\label{tab:fermions}.
\end{table}

La distintas interacciones entre fermiones son descriptas en términos del
intercambio de partículas entre estos. Estas partículas de intercambio son
denominadas \emph{bosones}, y a diferencia de los fermiones tienen spin entero.
Los bosones mediadores se muestran en la \cref{tab:bosons}.

Existen cuatro tipos de interacciones fundamentales. La interacción fuerte es la
responsable de mantener los quarks formando los protones y neutrones, y es
mediada por partículas no masivas llamadas \emph{gluones}. Las interacciones
electromagnéticas son las responsables de todos los fenómenos extra nucleares,
como por ejemplo las fuerzas intermoleculares en líquidos y solidos. Estas
interacciones están mediadas por el intercambio de fotones no masivos. La
interacci\'on débil es la responsable de los procesos de decaimiento $\beta$, y
sus mediadores son los bosones $Z^0$ y $W^\pm$, con masas del orden de 100 veces
la masa del protón. Por ultimo existe una interacción que actúa entre todo tipo
de partículas, la interacción gravitatoria. Actualmente no existe ninguna teoria
cuantica completa que explique esta interaccion fundamental, aunque hay muchas
teorias propuestas que postulan la existencia de una particula de spin 2 que
media la gravedad, denominada \emph{graviton}. En la escala de los experimentos
de partículas es la interacción mas débil de todas las interacciones
fundamentales, aunque es la dominante en la escala del universo.

Los leptones interactúan de forma débil y electromagnética en el caso de ser
cargados, o solo débilmente si son neutros. En contraste, los quarks (que son
los constituyentes fermiónicos de los hadrones, y por lo tanto del núcleo
atómico) interactúan además de débil y electromagnéticamente, por medio de la
interacción fuerte. Esta es la distinción fundamental entre quarks y leptones.


\begin{table}[!ht]
  \centering
  \begin{tabular}{ccc}
    \hline Débil & Electromagnética & Fuerte \\
    \hline quarks y leptones & quarks y leptones & quarks \\
    \hline $W^+$, $W^-$ y $Z^0$ & $\gamma$ & gluones \\
    \hline
  \end{tabular}
  \caption{Interacciones y los bosones de gauge mediadores de las mismas.}
  \label{tab:bosons}
\end{table}


%--------------------
% El Modelo Estandar
%--------------------
\section{El \SM}

Formalmente, el {\SM} es una teoría de campos renormalizable que provee una
descripción unificada de las interacciones fuerte, débil y electromagnética.
Estas interacciones surgen del requerimiento de que la teoría sea invariante
bajo transformaciones de gauge locales del grupo de simetría:

%% Formalmente, el {\SM} es una teoria de gauge no abeliana, construida imponiendo invarianza de gauge
%% local sobre los campos cuentificados que describen las particulas fundamentales, dando lugar a los
%% campos de gauge que describen las interacciones. Su grupo de simetria es,

\begin{equation}
  \text{SU}(3)_C \times \text{SU}(2)_L \times \text{U}(1)_Y
\end{equation}
%
donde $Y$, la hipercarga, $L$, la helicidad izquierda, y $C$, la carga de color,
representan kas cantidades conservadas del grupo de simetría. El subgrupo
$\text{SU}(2)_L \times \text{U}(1)_Y$ representa el sector electrodébil, es
decir, la electrodinámica cuántica (QED) mas las interacciones débiles.
%% La electrodinamica cuantica (QED) es una descripcion precisa
%% de las interacciones electromagneticas. Esta teoria fue una de
%% los logros mas importantes del siglo XX.

%% Es una teoria cuantica
%% de campos que conecta el formalismo moderno de la mecanica cuantica
%% con los principios clasicos del electromagnetismo. Uno de los
%% logros mas notables es el calculo preciso del momento magnetico
%% del electron, que acuerda con las medidas experimentales hasta
%% al menos los diez primeras cifras decimales. En QEDm la fuerza
%% entro dos particulas cargadas esta carazterizada por el itercambio
%% de un campo cuantico: el foton. En 1954 Yang and Robert Mills
%% formularon un principio general de invarianza de gauge}

Y la adición del grupo $\text{SU}(3)_C$ incluye la cromodinamica cuántica, que
es la teoria de campos de gauge que describe las interacciones fuertes de los
quarks y gluones que poseen carga de color.


La masa de las partículas en el {\SM} puede ser introducida mediante el llamado
mecanismo de Higgs\cite{PhysRevLett.13.321, PhysRevLett.13.508}, vía la ruptura
espontánea de la simetría electrodébil.

\begin{equation}
  \text{SU}(3)_C \times \text{SU}(2)_L \times \text{U}(1)_Y \to \text{SU}(3)_C \times \text{U}(1)_Q
\end{equation}
%
que resulta en la generación de los bosones de gauge masivos $W^\pm$ y $Z$. Como
consecunecia de esto, ademas, un nuevo campo escalar debe ser agregado al
Lagrangiano, dando lugar a la aparici\'on de un nuevo boson masivo, $H$, de spin
0, al que se lo llam\'o \emph{bos\'on de Higgs}.

Weinberg y Salam fueron los primeros en aplicar el mecanismo de Higgs al
rompimiento de la simetria electrodebil [6,7] y mostraron como este mecanismo
podia ser incorporado a la teoria electroweak de Glashow [8], dando inicios a lo
que hoy conocemos como {\SM} de la fisica de particulas.

La relacion entre las masas de los bosones $W^\pm$ y $Z$ predicha por el SM esta
dada por $\frac{m_W}{m_Z} = \cos \theta_W$, donde $\theta_W$ es el angulo de
mezcla de Weinberg, relaciona las constantes de acoplamiento debil ($g$) con la
electromagnetica ($g'$) como $\tan\theta_W = g'/g$. Los bosones $W^\pm$ y $Z$
fueron descubiertos en 1982 por las colaboraciones UA1 y UA2 del experimento
SppS del CERN.

No solo los bosones de gauge adquieren masa debido al mecanismo de Higgs,
tambien lo hacen los fermiones que forman la materia. El descubrimiento del
quark top en 1995 por las colaboraciones D0 y CDF, con una masa de $\sim 173
\GeV$, termino por cerrar las particulas que conforman la materias


Desde el punto de vista teórico la masa del bosón de Higgs es un parámetro libre
dentro del {\SM} y por lo tanto ninguna predicción puede ser hecha. La búsqueda
del boson de higgs, la unica particula del {\SM} que no habia sido descrubierta
aun, fue uno de los grandes objetivos por los cuales se dise\~no y construyó el
Gran Colisionar de Hadrones (LHC). En el a\~no 2012 el CERN anuncio el
descubrimiento de una particula consistente con el boson de Higgs por parte de
los dos grandes experimentos del LHC, ATLAS y CMS
\cite{Aad:2012tfa,Chatrchyan:2012ufa}. La medicion combinada entre ATLAS y CMS
de la masa del Higgs es $125.09 \pm 0.21 \text{(stat.)} \pm 0.11 \text{(syst.)}
\gev$.

\begin{figure}[!htbp]
  \centering
  \includegraphics[width=0.8\textwidth]{figures/higgs_atlas_cms_mass}
  \caption{Resumen de las mediciones de la masa del bosón de Higgs de los distintos
    analisis de ATLAS y CMS y el analsis combinado. Se indican las incertezas
    sistematicas (bandas de color magenta) y estadistica (bandas de color amarillo),
    y total (bandas negras). La linea roja vertical y la correspondiente sobra gris
    indican el valor central y la incerteza total de la medida combinada, respectivamente
    \cite{HiggsMass_ATLAS_CMS}.}
  \label{fig:higgs_cms_atlas}
\end{figure}



Todas las observaciones experimentales son compatibles con el Modelo Estándar a
un nivel de muy alta precisión. La \cref{fig:sm_atlas_xs} muestra el buen
acuerdo entre la sección eficaz de algunos procesos del SM medidas por
ATLAS\todo{Agregar refencia del plot} y las predicciones teóricas.



\begin{figure}[!htbp]
  \centering
  \includegraphics[width=1\textwidth]{figures/ATLAS_a_SMSummary_TotalXsect.pdf}
  \caption{Resumen de las distintas medidas de secci\'on eficaz de producci\'on de muchos
    procesos del {\SM}, comparadas con sus valores teoricos esperados.
    Los valores teoricos esperados fueron calculados como minimo a NLO.
  }\label{fig:sm_atlas_xs}
    %% Summary of several Standard Model total production cross section measurements,
    %% corrected for leptonic branching fractions, compared to the corresponding theoretical
    %% expectations. All theoretical expectations were calculated at NLO or higher. The W and Z
    %% vector-boson inclusive cross sections were measured with 35 pb-1 of integrated luminosity
    %% from the 2010 dataset. All other measurements were performed using the 2011 dataset or the
    %% 2012 dataset. The luminosity used for each measurement is indicated close to the data point.
    %% Uncertainties for the theoretical predictions are quoted from the original ATLAS papers.
    %% They were not always evaluated using the same prescriptions for PDFs and scales. }
\end{figure}


\section{Física más allá del SM}

Con todo lo dicho anteriormente se sabe que el SM provee una descripción
extremadamente exitosa de todos los fenómenos de partículas accesibles con los
experimentos de altas energías del presente. A pesar de esto, el SM sufre de
algunas debilidades, tanto desde el punto de vista teórico, como experimental.

El hecho de que existan cuatro tipos de interacciones distintos e independientes
es un poco insatisfactorio y desde Einstein se ha especulado que estas
diferentes interacciones sean distintas manifestaciones de un único campo
unificado. En los a\~nos 1970, los experimentos mostraron que la interacciones
débil y electromagnética podían ser unificadas. La interacción fuerte, aunque
esta incluida dentro del SM no esta unificada con las demás, y la interacción
gravitatoria ni siquiera forma parte del SM.

Es claro que el SM es una teoría efectiva a bajas energías muy precisa hasta
escalas de energía de 100 GeV. Sin embargo, los teóricos creen que el éxito del
SM no va a durar a energías mayores. Esta creencia surge de los intentos de
incorporar el SM en una teoría mas fundamental. Incluso ante la ausencia de la
gran unificación de las fuerzas electrodébil y fuerte a una escala muy alta de
energía, el SM debe ser modificado para incorporar los efectos de la gravedad a
la escala de Planck.

%% Theo
En este contexto también resulta un misterio porque el cociente $m_W/m_P \sim 10^{-17}$
es tan chico. Esto es llamado problema de \emph{jerarquía}. Además, en el SM, la
escala de las interacciones electrodebiles se derivan de un campo escalar
elemental que adquiere un valor de expectación de vacío de $v = 2 m_W /g = 246
\gev$. Sin embargo, si uno acopla una teoría de partículas escalares a nueva
física a alguna escala arbitraria $\Lambda$, las correcciones radiativas al
cuadrado de la masa escalar son del orden de $\Lambda^2$, debido a las
divergencias cuadráticas en la auto-energía, lo cual indica la sensibilidad
cuadrática a la mayor escala de energía de la teoría. Por esto, la masa
``natural'' de cualquier partícula escalar es $\Lambda$. Y para tener una teoría
electrodébil exitosa, la masa del Higgs debe ser del orden de la escala
electrodébil. Este hecho que la masa del bisoña de Higgs no puede ser igual a su
valor natural de $M_P$ es llamado del problema de \emph{naturalidad}.

Los físicos teóricos han trabajado durante las ultimas décadas intentando
solucionar estos problemas. Las soluciones propuestas involucran remover las
divergencias cuadráticas de la teoría que son la causa de los problemas de
jerarquía y naturalidad. Se han propuesto dos clases de soluciones. En una, los
escalares elementales son removidos, y por lo tanto uno debe agregar nuevos
fermiones y fuerzas fundamentales. Ejemplos de esto son los modelos de
\emph{tecnicolor} o los modelos \emph{composite}. La segunda clase de modelos
son aquellos en los que se introducen nuevas partículas al SM de forma de
cancelar exactamente las divergencias cuadráticas. Esta cancelación solo puede
ser el resultado de una nueva simetría.

%% Exp
Desde el punto de vista experimental, también existen mediciones que no pueden
acomodarse dentro del SM. El SM considera a los neutrinos como partículas no
masivas, pero distintos
experimentos\cite{PhysRevLett.101.111301,PhysRevD.78.032002} llevaron al
descubrimiento de las oscilaciones de sabor de estos, y como consecuencia a su
masa no nula. Los neutrinos tienen masas muy peque\~nas comparadas con los demás
fermiones. Solo se tienen limites superiores para estas (ver
\cref{tab:fermions}).

El termino de masa para fermiones puede escribirse usando el doblete de Higgs si
existen los fermiones de helicidad izquierda y derecha para un dado sabor. La
masa obtenida por esta interacción es llamada masa de Dirac. Para que este
mecanismo puede ser utilizado en los neutrinos, deberían existe los neutrinos
de helicidad derecha.
%% Aunque todavia no entendemos porque existen los fermiones de helicidad
%% derecha pero no para los nuetrinos.

El SM tampoco provee un candidato para explicar la naturaleza de la materia
oscura. La existencia de la materia oscura fue inferida por primera vez como
resultado de las inconsistencias observadas entre la masa estimada de las curvas
de rotación galácticas y de su luminosidad\cite{DM1} Las mediciones de
distancias de las supernovas de tipo 1a combinadas con las mediciones de las
microondas de fondo cósmico. Solo el 4\% del universo consiste en la materia que
conocemos\cite{DM2}. Cerca del 73\% consiste en energía oscura, y el restante
23\% es materia oscura. Como esta materia oscura no interactúa por medio de
interacciones fuerte y electromagnética, y la interacción débil es despreciable
en largas distancias, la materia oscura solo interactúa vía gravedad. La única
partícula del SM que podría ser un candidato viable de materia oscura es el
neutrino. Pero como la masa del neutrino es muy chica para poder explicar los
fenómenos de materia oscura, puede descartarse.


\note{Como terminar??}

%% Grand unified theories
%% The standard model has three gauge symmetries; the colour SU(3), the weak
%% isospin SU(2), and the hypercharge U(1) symmetry, corresponding to the three
%% fundamental forces. Due to renormalization the coupling constants of each of
%% these symmetries vary with the energy at which they are measured. Around 1016
%% GeV these couplings become approximately equal. This has led to speculation that
%% above this energy the three gauge symmetries of the standard model are unified
%% in one single gauge symmetry with a simple group gauge group, and just one
%% coupling constant. Below this energy the symmetry is spontaneously broken to the
%% standard model symmetries.[11] Popular choices for the unifying group are the
%% special unitary group in five dimensions SU(5) and the special orthogonal group
%% in ten dimensions SO(10).[12] Theories that unify the standard model symmetries
%% in this way are called Grand Unified Theories (or GUTs), and the energy scale at
%% which the unified symmetry is broken is called the GUT scale. Generically, grand
%% unified theories predict the creation of magnetic monopoles in the early
%% universe,[13] and instability of the proton.[14] Neither of these have been
%% observed, and this absence of observation puts limits on the possible GUTs.
