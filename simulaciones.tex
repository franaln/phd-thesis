\chapter{Simulaciones Monte Carlo}


Mirar ``Perspective in SUSY II'' de Kane,
Capitulo 11

\section{Generadores de eventos}

El eslabon entre la teoria (y fenomenologia) de SUSY u cualquier
otra teoria de neuva fisica, y las observaciones experimentales
en los detectores de los colisionadores son los <<generadores
de eventos>>. Dada una teoria de nueva fisica, que en general
predice la existencia de nuevas particulas y/o interaccions,
el generado de eventos permite calcular como esa teoria se
manifiesta en el experimento.

El primero paso para conectar SUSY con el experimento es comenzar
con un modelo particular de SUSY, y calcular el espectro de masas
y acoplamientos de las sparticulas. Existen varios programas que
permiten calcula el espectro de masas: Isasugra, Suspect, SoftSusy
y Spheno.
A partir de este modelo tambien es posible calcular los anchos de
decaimiento y tazas de decaimiento (BR) de todas las sparticulas y
los bosones de Higgs tanto $1 \to 2$ cuerpos y $1 \to 3$ cuerpos.

En este trabajo utilizamos el programa {\susyhit}\note{ref} que
combina {\suspect} junto {\sdecay} y {\hdecay} para calcular
los BRs y anchos de decaimiento. Algunos de estos modos de
decaimientos son calculados a NLO en QCD.

Una vez que el espectro de masas y los decaimientos estan calculados
estos pueden usarse como entrada en los generadores de eventos.

% SLHA
Para facilitar el intercambio de informacion entre distintos programas
de SUSY se propuso una seria de convenciones: SUSY Les Houches Accord (SLHA)\cite{SLHA}.
Este acuerdo permite usar la salida de los codigos que calculan
el espectro de masa y decaimientos como entrada en los generadores de
eventos de una forma consistente y sin ambiguedades.

%% %% En general estos modelos pueden
%% %% ser teorias efectivas
%% La teoria efectiva queda especificada adoptando la simetria de
%% gauge, the (super)field content y el Lagrangiano.
%% {\suspect} runs the 2-loop MSSM RGEs to determine weak scale
%% SUSY parameters in the mSUGRA, GMSB and AMSB models,
%% and in the pMSSM (a more general MSSM model). One-loop
%% sparticle mass corrections are included.
%% Some two loop corrections to Higgs masses are included.

\hl{agregar xs calculators}

\vsp

Los programas descriptos anteriormente permiten pasar de un modelo
de SUSY a las predicciones en la produccion de sparticulas y sus
anchos de decaimientos en estados finales de quarks, leptones, fotones
y gluones (y LSP en modelos donde se conserva paridad-R). Sin embargo
los quarks y gluones no pueden medirse directamente en los detectores.
Los detectores pueden medir trazas de particulas (cuasi)estables
cargadas y su momento en campos magnetics. Tambien pueden medir
depositos de enregia en los calorimetros .
Por lo tanto todavia hay un gap entre estos modelos y las senales
detectadas en los detectores. Los generadores de eventos permiten unir
estas dos cosas. Estos pueden producir una simulacion completa de los
eventos de dispersion esperados. El estado final de cualquier evento
de dispersion simulado esta compuesto por una lista de electrones,
muones, fotones y hadrones de vida media alta y su momento asociado
que puede ser medido en un experimento de colisiones.

Para un dado tipo de colisionador ($e^+e^-, pp, p\bar{p}, ...$) y una
dada energia de centro de masa, y un modelo, el generador de eventos
va a generar un conjunto eventos de pares de sparticulas de acuerdo
a su seccion eficaz. Estas sparticulas van a decaer (posiblemente
en en una cascada de varios pasos) en un estado final partonico,
de acuerdo a los BR fijados por el modelo. Este estado final partonico
es convertido es convertido en uno con particulas que puede ser detectadas
en el detector. Generando un gran numero de eventos de SUSY, se puede
simular los posibles estados finales que se esperada que un cierto
modelo produzca.

Existen varios generadoes que incorporan SUSY: Isajet, Pythia, Herwig, Sherpa, ....

La simulacion de eventos de dispersion en colisionadores hadronicos
puede descomponerse en varios pasos como se muestra en la figura [].

\begin{itemize}\itemsep0.2cm\parskip0.2cm
\item el calculo perturbativo del proceso de dispersion dura en el
  modelo partonico, y la convolucion con las funciones de distribucion
  partonica (PDFs)
\item la inclusion de los decaimientos en cascada de las sparticulas
\item implementacion de las lluvias de particulas pertubativamente
  para los estados de particulas con color inicial y final, y otras
  particulas con color que puedan ser producidas como decaimiento
  de otros objetos mas pesados,
\item implementacion del modelo de hadronizacion que describe la
  formacion de mesones y bariones a partir de los quarks y gluones.
  Tambien las particulas inetables deben decaer a hijas (cuasi)
  estables que son detectadas en el detector, con tasas y distribuciones
  que esten de acuerdo con los valores medidos o predichos.
\item Y finalmente, los remanentes de los haces inicilaes tienen que
  ser modelados para obtener una descripcion valida de la fisica
  en las regiones forward del detector.
\end{itemize}









\section{Herramientas para el calculo de SUSY}

\subsection{SUSY Spectrum Calculator}
\subsection{Sparticle Production and decay modes: production cross section, dacay widths and BF}
\subsection{Generadores de eventos: hard scattering, parton showers, cascade decays, hadronization}




















Las muestras de se\~nal de SUSY y de fondo del {\SM} fueron simuladas utilizando generadores MC
a $\sqrt{s}=8\tev$.
Todas las muestras fueron passed through either a GEANT4-based full simulation \cite{Geant4,AtlasSim}
or ATLFAST-II fast simulation \cite{Richter-Was:683751} of the ATLAS detector \cite{Geant4,AtlasSim},
and reconstructed with the same algorithms used for the data. An event-by-event reweighting is
applied to all MC samples to model the realistic machine conditions of the data sample
under study, by matching the simulated distribution of the number of interactions per bunch
crossing (pile-up) to the one observed in data.
The simulation details are summarized in {\Sec} \ref{sec:sig_samples} and \ref{sec:bkg_samples}.
Most of the MC backgrounds are, however, only used for comparison and cross checks.
The expected backgrounds in the selected sample is estimated, whenever possible,
from the data themselves. %(on average 5.6 interactions per bunch crossing).

\section{Se\~nal}\label{sec:sig_samples}

The present analysis is motivated by the bino-higgsino admixture neutralino decay signatures predicted by General Gauge Mediation (GGM) models,
namely a final-state signature that consists of a photon, jets, and high \MET. The event selection described in {\Sec} \ref{sec:event_selection},
has been designed to maximize the sensitivity to a small signal with this general topology. Any imposition of model-tailored selection cuts has been
avoided, trying to keep the analysis as model independent as possible. However, an interpretation in the framework of a specific model is unavoidable.
A grid of GGM signal points is simulated with a specific set of benchmark parameter values that covers the region in which signal can be established.
The sensitivity of the analysis is evaluated using this grid of points.

In this particular region of the GGM model space, the lightest neutralino is a mixture of bino and higgsino. The neutral wino is much heavier so it
does not contribute. Due to the Weinberg mixing angle in the Standard Model, the bino component of the lightest neutralino couples to both the photon and the Z.
The gluino is regarded as the only relevant coloured sparticle in order to set a conservative limit on the gluino mass. All squark soft masses are set to 2.5 TeV.
The other model parameters are set to M$_2$=2.5 \tev, $\tan\beta$=1.5 and $c\tau_{\mathrm{NLSP}} < 0.1$ mm. The latter assures the neutralino is decaying promptly
and is achieved by making the gravitino sufficiently light ($m_{\tilde{G}}=10^{-9}$ \gev). All trilinear coupling terms are set to zero and masses of sleptons
are set to 2.5 \tev. The Higgs boson is in the decoupling regime with $m_{A}$ = 2 \tev~ and $m_{h}$ = 126 \gev. The last follows the recently measured value
for the SM Higgs boson at the LHC \cite{ATLAS-CONF-2013-014,CMS-PAS-HIG-14-009}. In gauge mediated SUSY scenarios several mechanisms exist \cite{Craig:2011yk,Auzzi:2011eu,Csaki:2012fh,Larsen:2012rq,Craig:2012hc}
to generate a Higgs boson mass as high as this observed value, without changing the phenomenology of the models here considered. No significant effect on the mass spectrum
has been indeed observed when varying this value within a $\pm 10~$ \gev\ range.

\M{1} and $\mu$ determine the lightest neutralino mass, and are related in such a way that the branching ratios of the \ninoone\ are approximately constant,
resulting in ${\rm BR}(\ninoone \to \gam + \gravino) \approx 50\%$, ${\rm BR}(\ninoone \to Z + \gravino) \approx 49\%$ and ${\rm BR}(\ninoone \to h + \gravino) \approx 1\%$,
numbers which vary by $\pm 1\%$ throughout most of the grid ({\fig} \ref{fig:br_n1_x_grav}). For light neutralinos ($<200\gev$) the Higgs production is highly
suppressed and ${\rm BR}(\ninoone \to \gam + \gravino)$ starts falling up to 40\%. The value of $\mu$ must also be positive in order to disfavor the branching ratio
to the Higgs boson, which would lead to a signature already covered by a dedicated analysis in ATLAS \cite{Aad:2012jva}. Similarly, the branching ratio for
($\ninoone \to \gam + \gravino$) is such that maximizes the single photon final state. At larger values the diphoton topology starts to be favoured, which has been
extensively searched for in the past \cite{Aad2012519,Aad:2011kz}. This leaves $M_3$ and $\mu$ as the only two free parameters of the model, spanning the space
within $150\gev < m_{\ninoone} < 1250 \gev$ and $800\gev < m_{\gluino} < 1300 \gev$, with $m_{\ninoone} < m_{\gluino}$. The granularity of the simulation in each
dimension is shown in {\tab}\ \ref{tab:signal_pars}, with the resulting value for the gluino and neutralino masses.

The full mass spectrum, the gluino and neutralino branching ratios and decay widths are calculated from
these set of parameters using SUSPECT v2.41 \cite{Djouadi2007426}, SDECAY v1.3b \cite{Muhlleitner:2004mka}
and HDECAY v3.4 \cite{Djouadi:1997yw}, run as part of the SUSYHIT package v1.3 \cite{Djouadi:2006bz}.
An example of the mass spectrum in the configuration is shown in {\fig} \ref{fig:mass_spectra}, for one of
the signal grid points. The total decay branching ratios for \gluino-initiated \ninoone\ production are
shown in {\fig} \ref{fig:br_gl_n1}, for 2-body\footnote{only effectively, the gluino decays through a virtual
quark-squark loop in this case.} and 3-body gluino decays.
Simulated events were generated with HERWIG++ v2.5.2 \cite{Bahr:2008pv} for the 124 signal points in the
grid (5K per each), using the CTEQ6L1 \cite{Nadolsky:2008zw} parton density distributions. A generator level
filter requiring a photon with $p_{T}>100~$ \gev\ was applied to get higher statistics, specially at low neutralino
mass. The filter efficiency for all simulated points are shown in {\tab} \ref{tab:signal_filter_eff}.

Signal production cross sections and uncertainties were calculated using the SUSYSignalUncertainties package \cite{SUSYsigunc}. Cross sections
for the signal processes involving the production of gluino pairs are calculated to next-to-leading order (NLO)
in the strong coupling constant, adding the resummation of soft gluon emission at next-to-leading-logarithmic
accuracy (NLO+NLL) \cite{Beenakker:1996ch,Kulesza:2008jb,Kulesza:2009kq,Beenakker:2009ha,Beenakker:2011fu}.
The EWK $\tilde{\chi}\tilde{\chi}$ production cross sections are calculated to next-to-leading order in the strong coupling constant (NLO) using PROSPINO v2.1~\cite{Beenakker:1999xh}.
%are computed at NLO with Prospino v2.1 \cite{Beenakker:1996ed}.
The cross sections and uncertainties are tabulated in {\tab} \ref{tab:signal_xs_strong} and \ref{tab:signal_xs_ewk}, and shown in {\fig} \ref{fig:signal_xs_strong} and \ref{fig:signal_xs_ewk}, for the strong and EWK production respectively.
The total nominal cross section and the relative contribution of EWK produced processes are also shown in {\fig} \ref{fig:signal_xs_total}.
The total uncertainty is taken from an envelope of cross section predictions using different PDF
sets and factorisation and renormalisation scales, as described in \cite{Kramer:2012bx}.
More details of the uncertainties treatment are given in {\Sec} \ref{sec:syst_signal}. %The dependence of the cross section with $M_3$ is also shown in Fig. \ref{fig:signal_xs}.

All signal samples have been simulated with the faster, parametrized detector simulation ATLFAST-II \cite{Richter-Was:683751}. Derived datasets in the {\sc NTUP\_SUSY}
format are used throughout this analysis, with the configuration tag p1328.


\begin{figure}[ht!]
   \centering
   \includegraphics[width=0.4\textwidth]{figures/n1_gravgam}
   \includegraphics[width=0.4\textwidth]{figures/n1_gravz}\\
   \includegraphics[width=0.4\textwidth]{figures/n1_gravh}
   \caption{Branching ratios of the lightest neutralino (N1) decay. As expected in GGM models,
     the gravitino (as LSP) is always produced, in association with either a photon, a Z boson
     or a light Higgs boson, depending on the neutralino's nature. The grid parameters in this
     analysis have been chosen in order to keep the BR($\ninoone\to\gam\gravino)\sim50$\% across the whole phase space, maximizing the probability of the single photon final state. See text for more details.}
   \label{fig:br_n1_x_grav}
\end{figure}

\begin{table}[ht!]
  \centering
  \caption{ Model parameters of the GGM signal grid. $M_3$ and $\mu$ are regarded as the free parameters of the model. }
  \begin{tabular}{ccc || cccc}
    \hline
    \hline
    $M_3$ [\gev]& $m_{\gluino}$ [\gev]& &  & $\mu$ [\gev]& $M_1$ [\gev]& $m_{\ninoone}$ [\gev] \\
    \hline
    800 & 885.5   & & & 150  &  300  & 147.0  \tabularnewline
    850 & 931.7   & & & 175  &  270  & 168.3  \tabularnewline
    900 & 977.6   & & & 200  &  267  & 190.3  \tabularnewline
    950 & 1023.1  & & & 250  &  288  & 235.8  \tabularnewline
    1000 & 1068.3 & & & 350  &  365  & 332.4  \tabularnewline
    1050 & 1113.3 & & & 450  &  456  & 433.2  \tabularnewline
    1100 & 1157.9 & & & 550  &  551  & 535.6  \tabularnewline
    1150 & 1202.3 & & & 650  &  647  & 638.3  \tabularnewline
    1200 & 1246.4 & & & 750  &  745  & 742.0  \tabularnewline
    1250 & 1290.3 & & & 838  &  837  & 836.4  \tabularnewline
    1300 & 1333.9 & & & 850  &  845  & 846.7  \tabularnewline
    1350 & 1377.3 & & & 883  &  882  & 883.7  \tabularnewline
    1400 & 1420.5 & & & 928  &  926  & 930.2  \tabularnewline
    1450 & 1463.4 & & & 950  &  942  & 949.6  \tabularnewline
             &        & & & 973  &  970  & 976.6  \tabularnewline
             &        & & & 1017 &  1015 & 1023.4 \tabularnewline
             &        & & & 1050 &  1040 & 1053.0 \tabularnewline
             &        & & & 1062 &  1058 & 1068.9 \tabularnewline
             &        & & & 1106 &  1102 & 1114.8 \tabularnewline
             &        & & & 1149 &  1145 & 1160.0 \tabularnewline
             &        & & & 1150 &  1140 & 1157.5 \tabularnewline
             &        & & & 1250 &  1238 & 1260.6 \tabularnewline
    \hline
    \hline
  \end{tabular}
  \label{tab:signal_pars}
\end{table}

\begin{table}[ht]
  \centering
  \caption{Sección eficaz a  NLO+NLL para la producción fuerte de
    la grid de señal. La ultima columna es la incerteza teórica.
    Mas detalles se encuentran en la {\Sec} \ref{sec:syst_signal}.}
  \begin{tabular}{c|c|c}
    \hline
    \hline
    $M_3$ [\gev] & $\sigma$(NLO+NLL) [pb] & Incerteza Total Teorica [$\%$]\tabularnewline
    \hline
        800  &  0.06905 & 22.5  \\
        850  &  0.04492 & 23.8  \\
        900  &  0.02973 & 25.2  \\
        950  &  0.01983 & 26.5  \\
        1000 &  0.01341 & 27.7  \\
        1050 &  0.00910 & 29.0  \\
        1100 &  0.00628 & 30.4  \\
        1150 &  0.00432 & 32.0  \\
        1200 &  0.00301 & 33.7  \\
        1250 &  0.00210 & 35.2  \\
        1300 &  0.00148 & 36.7  \\
        1350 &  0.00105 & 38.2  \\
        1400 &  0.00074 & 39.8  \\
        1450 &  0.00053 & 41.5  \\
    \hline
    \hline
  \end{tabular}
  \label{tab:signal_xs_strong}
\end{table}

\begin{table}[ht]
  \centering
  \caption{Sección eficaz total a NLO para la producción EWK
    de la grid de señal. La ultima columna  es la incerteza
    teórica. Mas detalles se encuentran en la {\Sec} \ref{sec:syst_signal}.}
  \begin{tabular}{c|c|c}
    \hline
    \hline
    $\mu$ [\gev] & $\sigma$(NLO) [pb] & Total Theo. Uncertainty [$\%$]\tabularnewline
    \hline
    150  & 2.68 & 6.3  \\
    175  & 1.42 & 6.7  \\
    200  & 0.84 & 6.9   \\
    250  & 0.28 & 6.4     \\
    350  & 0.050 & 7.0    \\
    450  & 0.013 & 7.6    \\
    550  & 4.1e-03 & 8.0  \\
    650  & 1.4e-03 & 8.5   \\
    750  & 5.3e-04 & 8.9  \\
    850  & 2.1e-04 & 9.3  \\
    950  & 8.57e-05 & 10.3  \\
    1050  & 3.56e-05 & 11.0  \\
    1150  & 1.55e-05 & 13.3   \\
    1250  & 6.67e-06 & 16.9   \\
    \hline
    \hline
  \end{tabular}
  \label{tab:signal_xs_ewk}
\end{table}


\begin{table}[ht]
  \centering
  %\small
  \footnotesize
  \caption{Eficiencia del filtro a nivel generador [\%]
    para los puntos de señal simulados.}
  \begin{tabular}{c|ccccccccccccccccc}
    \hline
    \hline
%       &    \multicolumn{11}{c}{Filter efficiency [\%]} \\
%	\hline
     &   \multicolumn{14}{c}{ $M_3$ [\gev] } \\
    $\mu$ [\gev] &  800 & 850 & 900 & 950 & 1000 & 1050 & 1100 &1150 & 1200  & 1250 & 1300 & 1350 & 1400 & 1450 \\
    \hline
    150  &   39.54 &   39.8 &  41.9    &   42.6   &   43.0   &   44.2   &   45.5    &   46.3    &    47.1  &   48.3  &   49.1 &      &      &       \\
    175  &   44.55 &   44.6 &  46.5    &   47.5   &   47.8   &   48.9   &   50.3    &   51.4    &    51.5  &   52.4  &   53.3 &      &      &       \\
    200  &   47.66 &   48.4 &  50.1    &   50.6   &   52.0   &   52.8   &   53.8    &   55.0    &    55.0  &   56.2  &   56.0 &      &      &       \\
    250  &   55.09 &   56.0 &  56.1    &   57.1   &   56.7   &   58.0   &   58.9    &   59.5    &    60.5  &   60.7  &   61.2 & 62.1 & 62.2 &  63.9 \\
    350  &   65.82 &   66.1 &  64.6    &   65.8   &   66.4   &   66.7   &   66.5    &   67.7    &    67.4  &   67.6  &   67.8 & 68.0 & 69.2 &  68.4 \\
    450  &   71.29 &   71.4 &  71.4    &   71.8   &   72.1   &   72.5   &   72.0    &   72.6    &    72.1  &   72.5  &   73.4 & 72.9 & 72.2 &  73.4 \\
    550  &   73.08 &   72.5 &  73.3    &   73.7   &   74.6   &   74.3   &   75.0    &   74.8    &    74.2  &   74.7  &   75.6 & 75.7 & 74.9 &  75.3 \\
    650  &   69.78 &   72.0 &  73.2    &   74.1   &   75.6   &   76.2   &   75.9    &   75.8    &    77.0  &   77.0  &   76.2 & 76.7 & 77.1 &  76.6 \\
    750  &   47.69 &   60.2 &  66.9    &   70.4   &   73.4   &   74.0   &   75.9    &   76.5    &    77.1  &   77.3  &   76.3 & 77.9 & 77.2 &  77.0 \\
    838  &    9.0  &        &          &          &          &          &           &           &          &         &        &      &      &       \\
    883  &         &   9.1  &          &          &          &          &           &           &          &         &        &      &      &       \\
    850  &         &        &  34.2    &   50.7   &   60.1   &   66.4   &   70.4    &   73.5    &    73.6  &   75.4  &   76.6 & 77.1 & 78.2 &  76.8 \\
    928  &         &        &   9.4    &          &          &          &           &           &          &         &        &      &      &       \\
    950  &         &        &          &          &   25.4   &   41.4   &   54.0    &   62.2    &    66.8  &   70.9  &   75.2 & 75.9 & 78.1 &  77.2 \\
    973  &         &        &          &   10.0   &          &          &           &           &          &         &        &      &      &       \\
    1017 &         &        &          &          &   10.3   &          &           &           &          &         &        &      &      &       \\
    1050 &         &        &          &          &          &          &   19.2    &   31.6    &    45.2  &   56.8  &   62.7 & 68.9 & 73.1 &  75.6 \\
    1062 &         &        &          &          &          &   11.1   &           &           &          &         &        &      &      &       \\
    1106 &         &        &          &          &          &          &   11.7    &           &          &         &        &      &      &       \\
    1149 &         &        &          &          &          &          &           &   12.6    &          &         &        &      &      &       \\
    1150 &         &        &          &          &          &          &           &           &    16.1  &   24.9  &   36.9 & 49.0 &      &       \\
    1250 &         &        &          &          &          &          &           &           &          &         &   16.1 &      &      &       \\
    \hline
         \hline
  \end{tabular}
  \label{tab:signal_filter_eff}
\end{table}


%\begin{table}[ht]
%  \centering
%  \small
%  \begin{tabular}{|c|c|c|c|c|c|}
%    \hline
%    \hline
%	M3 [\gev] & $\sigma$(NLO+NLL) [\gev] & Uncertainty [$\%$] & k-factor & $m_{\neut}$ [\gev] & Filter efficiency [$\%$] \tabularnewline
%    \hline
%	\multirow{9}{*}{800} & \multirow{9}{*}{5.96 E-2} & \multirow{9}{*}{26.1} & \multirow{9}{*}{2.84} &  150 &  39.54 \\
%    \tiny
%	& & & & 175 & 44.55 \\
%	& & & & 200 & 47.66 \\
%	& & & & 250 & 55.09 \\
%	& & & & 350 & 65.82 \\
%	& & & & 450 & 71.29 \\
%	& & & & 550 & 73.08 \\
%	& & & & 650 & 69.78 \\
%	& & & & 750 & 47.69 \\
%	\hline
%	850 & 3.81 E-2 & 27.8 & 2.95 &  150 & 39.8 \tabularnewline
%	& & & & 175 & 44.6 \\
%	& & & & 200 &  48.4 \\
%	& & & & 250 &  56.0 \\
%	& & & & 350 &  66.1 \\
%	& & & & 450 &  71.4 \\
%	& & & & 550 &  72.5 \\
%	& & & & 650 &  72.0 \\
%	& & & & 750 &  60.2 \\
%	\hline
%	900 & 2.47 E-2 & 29.5 & 3.07 & 150 & 41.9 \tabularnewline
%	& & & & 175 &  46.5 \\
%	& & & & 200 &  50.1 \\
%	& & & & 250 &  56.1 \\
%	& & & & 350 &  64.6 \\
%	& & & & 450 &  71.4 \\
%	& & & & 550 &  73.3 \\
%	& & & & 650 &  73.2 \\
%	& & & & 750 &  66.9 \\
%	& & & & 850 &  34.2 \\
%	\hline
%	950 & 1.62 E-2 & 31.4 & 3.19 & 150 & 42.6 \tabularnewline
%	& & & & 175 &   47.5 \\
%	& & & & 200 &   50.6 \\
%	& & & & 250 &   57.1 \\
%	& & & & 350 &   65.8 \\
%	& & & & 450 &   71.8 \\
%	& & & & 550 &   73.7 \\
%	& & & & 650 &   74.1 \\
%	& & & & 750 &   70.4 \\
%	& & & & 850 &   50.7 \\
%	\hline
%	1000 & 1.08 E-2 & 33.6 & 3.33 & 150 & 43.0 \tabularnewline
%	& & & & 175 &   47.8 \\
%	& & & & 200 &   52.0 \\
%	& & & & 250 &    56.7 \\
%	& & & & 350 &    66.4 \\
%	& & & & 450 &    72.1 \\
%	& & & & 550 &    74.6 \\
%	& & & & 650 &    75.6 \\
%	& & & & 750 &    73.4 \\
%	& & & & 850 &    60.1 \\
%	& & & & 950 &    25.4 \\
%	\hline
%	1050 & 7.19 E-3 & 35.8 & 3.49 & 150 & 44.2 \tabularnewline
%	& & & & 175 &   48.9 \\
%	& & & & 200 &   52.8 \\
%	& & & & 250 &   58.0 \\
%	& & & & 350 &   66.7 \\
%	& & & & 450 &   72.5  \\
%	& & & & 550 &    74.3 \\
%	& & & & 650 &    76.2 \\
%	& & & & 750 &    74.0 \\
%	& & & & 850 &    66.4 \\
%	& & & & 950 &    41.4 \\
%	\hline
%	1100 & 4.87 E-3 & 37.8 & 3.66 & 150 & 45.5 \tabularnewline
%	& & & & 175 &   50.3 \\
%	& & & & 200 &   53.8 \\
%	& & & & 250 &   58.9 \\
%	& & & & 350 &   66.5 \\
%	& & & & 450 &   72.0 \\
%	& & & & 550 &   75.0 \\
%	& & & & 650 &   75.9 \\
%	& & & & 750 &   75.9 \\
%	& & & & 850 &   70.4 \\
%	& & & & 950 &   54.0 \\
%	& & & & 1050 &  19.2 \\
%	\hline
%	1150 & 3.29 E-3 & 40.0 & 3.85 &  150 & 46.3 \tabularnewline
%	& & & & 175 &   51.4 \\
%	& & & & 200 &   55.0 \\
%	& & & & 250 &   59.5 \\
%	& & & & 350 &   67.7 \\
%	& & & & 450 &   72.6 \\
%	& & & & 550 &   74.8 \\
%	& & & & 650 &   75.8 \\
%	& & & & 750 &   76.5 \\
%	& & & & 850 &   73.5 \\
%	& & & & 950 &   62.2 \\
%	& & & & 1050 &  31.6 \\
%	\hline
%	1200 & 2.24 E-3 & 42.3 & 4.05 &  150 & 47.1 \tabularnewline
%	& & & & 175 &   51.5 \\
%	& & & & 200 &   55.0 \\
%	& & & & 250 &   60.5 \\
%	& & & & 350 &   67.4 \\
%	& & & & 450 &   72.1 \\
%	& & & & 550 &   74.2 \\
%	& & & & 650 &   77.0 \\
%	& & & & 750 &   77.1 \\
%	& & & & 850 &   73.6 \\
%	& & & & 950 &   66.8 \\
%	& & & & 1050 &  45.2 \\
%	& & & & 1150 &  16.1 \\
%	\hline
%	1250 & 1.55 E-3 & 44.6 & 4.31 & 150 & 48.3 \tabularnewline
%	& & & & 175 &   52.4 \\
%	& & & & 200 &   56.2 \\
%	& & & & 250 &   60.7 \\
%	& & & & 350 &   67.6 \\
%	& & & & 450 &   72.5 \\
%	& & & & 550 &   74.7 \\
%	& & & & 650 &   77.0 \\
%	& & & & 750 &   77.3 \\
%	& & & & 850 &   75.4 \\
%	& & & & 950 &   70.9 \\
%	& & & & 1050 & 56.8 \\
%	& & & & 1150 & 24.9 \\
%	\hline
%	1300 & 1.07 E-3 & 46.9 & 4.56 & 150 & 49.1 \tabularnewline
%	& & & & 175 &   53.3 \\
%	& & & & 200 &   56.0 \\
%	& & & & 250 &   61.2 \\
%	& & & & 350 &   67.8 \\
%	& & & & 450 &   73.4 \\
%	& & & & 550 &   75.6 \\
%	& & & & 650 &   76.2 \\
%	& & & & 750 &   76.3 \\
%	& & & & 850 &   76.6 \\
%	& & & & 950 &   75.2 \\
%	& & & & 1050 & 62.7 \\
%	& & & & 1150 & 36.9 \\
%	& & & & 1250 & 16.1 \\
%    \hline
%    \hline
%  \end{tabular}
%  \caption{ The total NLO+NLL cross sections with uncertainties and K factors for GGM  signal points. The last column shows the efficiency of the generator level filter applied, to be considered for the final signal normalization.}
%  \label{tab:signal_xs}
%\end{table}

\begin{figure}[ht!]
   \centering
   \includegraphics[width=0.7\textwidth]{figures/figura} %%mass_spectrum_GGM.eps}
   \caption{Espectro de masas de un punto el grid de senal.
     Solo $M_3$ y $\mu$ son los parametros libres, en este caso
     $(M_3, \mu) = (800~\gev, 250~\gev)$. }
   \label{fig:mass_spectra}
\end{figure}

\begin{figure}[ht!] %  figure placement: here, top, bottom, or page
   \centering
   \includegraphics[width=0.32\textwidth]{figures/gl_n1g_full}
   \includegraphics[width=0.32\textwidth]{figures/gl_n1qq_full}
   \includegraphics[width=0.32\textwidth]{figures/gl_n1_full} \\
   \includegraphics[width=0.32\textwidth]{figures/gl_n2g_full}
   \includegraphics[width=0.32\textwidth]{figures/gl_n2qq_full}
   \includegraphics[width=0.32\textwidth]{figures/gl_n2_full} \\
   \includegraphics[width=0.32\textwidth]{figures/gl_n3g_full}
   \includegraphics[width=0.32\textwidth]{figures/gl_n3qq_full}
   \includegraphics[width=0.32\textwidth]{figures/gl_n3_full} \\
   \includegraphics[width=0.32\textwidth]{figures/gl_c1qq_full} \\

   \caption{Tasas de decaimiento para $\gluino \to \ninoone$,
     para todas las posibles cadenas de decaimiento permitidas en la grid
     de produccion fuerte. La mayoria de los graficos son la suma
     de deciamientos de 2 cuerpos (izquierda) y 3 cuerpos (centro).
     Para el decaimiento $\gluino \to \chinopm$), solo el decaimiento de
     tres cuerpos es posible.}
   \label{fig:br_gl_n1}
\end{figure}

\clearpage

%\begin{figure}[ht!]
%  \centering
%  \includegraphics[width=0.5\textwidth]{figures/GGM_xs_vs_M3}
%  \caption{Signal cross section as function of M3, calculated at NLO+NLL.}% See text for more details. }
%  \includegraphics[width=0.5\textwidth, angle=90]{figures/GGM_xs_vs_mgl}
%  \caption{Signal cross section as function of $m_{\gluino}$, calculated at NLO+NLL. \tosolve{UPDATE}}% See text for more details. }
%  \caption{Signal cross section for as function of $m_{\gluino}$, calculated at NLO+NLL. \tosolve{UPDATE}}% See text for more details. }
%  \label{fig:signal_xs_vs_M3}
%\end{figure}

%% \begin{figure}[htbp]
%% \centering
%% \subfloat[NLO+NLL cross section $pp \to \gluino\gluino$]{
%%   \includegraphics[width=0.49\textwidth]{figures/figura} %SigXsec_strong}
%%   \label{fig:signal_xs_strong}
%% }
%% \subfloat[NLO cross section $pp\to\susy{\chi}\susy{\chi}$]{
%%   \includegraphics[width= 0.49\textwidth]{figures/figura} %SigXsec_ewk}
%%   \label{fig:signal_xs_ewk}
%% }
%% \caption{Signal cross sections for the strong (\ref{fig:signal_xs_strong})
%%   and EWK (\ref{fig:signal_xs_ewk}) production grids.}
%% \label{fig:signal_xs}
%% \end{figure}

\begin{figure}[ht!]
  \centering
  \includegraphics[width=0.49\textwidth]{figures/SigXsec_total}
  \includegraphics[width=0.49\textwidth]{figures/SigXsec_ewkFrac}
  \caption{Sección eficaz total (izquierda) y fracción relativa
    de producción EWK (derecha).}
  \label{fig:signal_xs_total}
\end{figure}

\section{Fondos del {\SM}}\label{sec:bkg_samples}

Existen muchos procesos del {\SM} que pueden aparentar
una señal de SUSY con fotones, jets y energía faltante.
Estos pueden dividirse en varias categorías:

\begin{itemize}
\item {\MET} real (EWK)
  \begin{itemize}
  \item $\gamma$ real
    \begin{itemize}
    \item $Z(\to\nu\nu)+\gamma$
    \item $W(\to\tau\nu)+\gamma$, hadronic $\tau$-decay
    \item $W(\to e\nu)+\gamma$, the e is not reconstructed
    \item $W(\to\mu\nu)+\gamma$ and $W(\to\tau\nu)+\gamma$, the $\mu / \tau$ is not reconstructed
    \item $t\bar{t}+\gamma$,  the e$/ \mu$ (when produced) is not reconstructed
    \end{itemize}

  \item Electron$/$Jet faking photon
    \begin{itemize}
    \item $W(\to l\nu)$+jets
    \item $Z(\to \nu\nu)+$jets
    \item $t\bar{t}$
    \item dibosons
    \end{itemize}
  \end{itemize}

\item {\MET} instrumental
  \begin{itemize}
  \item $\gamma$ real ($\gamma+$jets)
  \item $\gamma$ falso (multijet, $Z(\to ll)+$jets)
  \end{itemize}
\end{itemize}

Las muestras simuladas con MC utilizadas en este analisis se describen
a continuacion. Como se discutira en la {\Sec} \ref{sec:background_estimation},
la contaminacion de fotones mal dentificados provenientes de jets y electrones
es estimado con metodos basados en datos. Sin embargo, las muestras MC tambien
han sido consideradas en estos casos para los estudios de optimizacion y la
evaluacion de las incertezas sistematicas.

\subsection{W/Z$+\gamma$}

Se espera que la produccion de {\wgam} y {\zgam} sea un fondo importante
para esta busqueda. Ambas muestras fueron generadas usando el generador
de eventos {\sherpa} v1.4.1 \cite{SherpaGen}, con hasta 3 partones en el
ME+PS y usando las funciones de densidad partonica CT10.
La combinacion de los elementos de matriz con las lluvias partonicas
es realizada de acuerdo a un procedimiento mejorado CKKW \cite{Catani:2001cc,Krauss:2002up}.
Un filtro a nivel generados es aplicado requiriendo al menos un foton
con $\pt > 80(70) \gev$ en el estado final de las muestras de {\wgam} (\zgam).
%option1
%Only leptonic decays of the Z were considered, including its invisible decay Z$\to\nu\nu$. Additional samples of \wgamma events with some variation of the simulation settings were used to compute systematic uncertainties on the expected yields for this background. All \Vgamma samples are listed in {\tab} \ref{tab:bkg_wzgamma_samples}.
%option2
Todos los decaimientos leptonicos del boson $Z$ fueron considerados,
incluyendo el decaimiento invisible $Z\to\nu\nu$.
Tambien se tuvo en cuenta un muestra de $V(\to qq)+\gamma  (V=W,Z/\gamma*)$
debido a que cierta energia faltante real puede  ser producida en el caso
de heavy flavour decays.
%Their contribution had been found negligible in all regions here considered. \tosolve{CHECK!}
%

\begin{table}[ht!]
  \centering
  \caption{Muestras de W/Z$+\gamma$ utilizadas en este analisis.
    La seccion eficas a LO se especifica para cada modo de decaimiento,
    al igual que los factores $k$, y las eficiencias del filtro.
    La luminosidad integrada correspondiente a la estadistica total
    de cada muestra esta tambien presente.}
  %\includegraphics[width=1\textwidth]{figures/tabla}
  \begin{tabular}{ l | c | c | c | c }
    \hline
    \hline
    Proceso (ID) & $\sigma~[pb]$ & $k$ & Eficiencia & $L [fb^{-1}]$ \\
    \hline
    {\wenugam} {\sherpa} (126741) &  0.7193  &  1.0  &  1.0  &  695.16 \\
    {\wmunugam} {\sherpa}  (126744) &  0.7178  &  1.0  &  1.0  &  696.56 \\
    {\wtaunugam} {\sherpa}  (158727) &  0.7199  &  1.0  &  1.0  &  694.57 \\
    {\zeegam} {\sherpa}  (158728) &  0.1861  &  1.0  &  1.0  &  1069.53 \\
    {\zmumugam} {\sherpa}  (158729) &  0.1858  &  1.0  &  1.0  &  1076.71 \\
    {\ztautaugam} {\sherpa}  (158730) &  0.1858  &  1.0  &  1.0  &  1076.19 \\
    {\znunugam} {\sherpa}  (126022) &  0.7625  &  1.0  &  1.0  &  655.74 \\
  %%   % in case we use this in the end 'option2'
    {\vqqgam} {\sherpa}  (164438) &  6.756  &  1.0  &  1.0  &  89.0 \\
  %%   \hline
  %%   \hline
  %%   \multicolumn{5}{l}{Systematics variations} \\
  %%   \hline
  %%    \wenugam (fact. 0.25x) \sherpa (204733) &  0.7193  &  1.0  &  1.0  &  278.05 \\
  %%    \wenugam (fact. 4x) \sherpa (204734) &  0.7193  &  1.0  &  1.0  &  278.05 \\
  %%    \wenugam (renorm. 0.25x) \sherpa (204735) &  0.7193  &  1.0  &  1.0  &  278.05 \\
  %%    \wenugam (renorm. 4x) \sherpa (204736) &  0.7193  &  1.0  &  1.0  &  278.05 \\
  %%    \wenugam (ckkw 15) \sherpa (204731) &  0.7193  &  1.0  &  1.0  &  278.05 \\
  %%    \wenugam (ckkw 30) \sherpa (204732) &  0.7193  &  1.0  &  1.0  &  278.05 \\
  %%    \wmunugam (fact. 0.25x) \sherpa (204739) &  0.7193  &  1.0  &  1.0  &  278.63 \\
  %%    \wmunugam (fact. 4x) \sherpa (204740) &  0.7193  &  1.0  &  1.0  &  278.63 \\
  %%    \wmunugam (renorm. 0.25x) \sherpa (204741) &  0.7193  &  1.0  &  1.0  &  278.63 \\
  %%    \wmunugam (renorm. 4x) \sherpa (204742) &  0.7193  &  1.0  &  1.0  &  278.63 \\
  %%    \wmunugam (ckkw 15) \sherpa (204737) &  0.7193  &  1.0  &  1.0  &  278.63 \\
  %%    \wmunugam (ckkw 30) \sherpa (204738) &  0.7193  &  1.0  &  1.0  &  278.63 \\
  %%    \wtaunugam (fact. 0.25x) \sherpa (204745) &  0.7193  &  1.0  &  1.0  &  277.81 \\
  %%    \wtaunugam (fact. 4x) \sherpa (204746) &  0.7193  &  1.0  &  1.0  &  277.81 \\
  %%    \wtaunugam (renorm. 0.25x) \sherpa (204747) &  0.7193  &  1.0  &  1.0  &  277.81 \\
  %%    \wtaunugam (renorm. 4x) \sherpa (204748) &  0.7193  &  1.0  &  1.0  &  277.81 \\
  %%    \wtaunugam (ckkw 15) \sherpa (204743) &  0.7193  &  1.0  &  1.0  &  277.81 \\
  %%    \wtaunugam (ckkw 30) \sherpa (204744) &  0.7193  &  1.0  &  1.0  &  277.81 \\
    \hline
    \hline
  \end{tabular}
  \label{tab:bkg_wzgamma_samples}
\end{table}

\subsubsection{W/Z+jets} \label{mc_wzjets}

Se espera que la produccion de W$^{\pm}$ y bosones $Z$ en asociacion con jets
contribuya a esta busqueda, con los fotones provenientes de electrones y jets
mal identificados. Especialemente para los segundos, esta contaminacion no esta
bien descripta por el MC. Por esta razon se utilizan metodos basados en datos
para estimar su contirbucion en las diferentes regiones de senal y control, como
se describe en el Capítulo \ref{cap:fondos}. De igual maner varias muestras de MC
fueron consideradas para validar los metodos.

Como se describe en el Capítulo \ref{cap:seleccion} la seleccion de senal involucra
muchos jets en el estado final, es importante modelar los estados final multipartonicos
de forma adecuada. Con esto en mente, el generador de events {alpgen} (version 2.14)
fue utilizado, incluyendo los efectos EWK y QCD a LO para los procesos de interaccion
fuerte multipartonicos. La produccion de jets fue generada for up to five-parton
matrix elements. Este generados fue interfaceado con {\herwig} version 6.5.2
para la simulacion de las lluvias y los procesos de fragmentacion y con {\jimmy}
para la simulacion de los eventos subyacentes. Las funciones de densidad partonica
utilizadas fueron las CTEQ6L1. La normalizacion a la luminosidad integrada acumulada
fue hecha escalenado la seccion eficas mostrada en la {\tab} \ref{tab:bkg_wzjets_samples}
usando calculos QCD a NNLO de el programa FEWZ \cite{Anastasiou:2003ds}.
En cada caso los mismos factores de normalizacion fueron aplicados a los elementos
de matriz de {\alpgen}.
%%applied for all \alpgen matrix element parton multiplicities.
Finalmente, se realiza la remocion de eventos para evitar el conteo doble de eventos
que ya fueron tenidos en cuenta por las muestras de Z$\gamma$ y W$\gamma$.
Para esto, los eventos de $W(Z)+\text{jets}$ con fotones con $\pt > 80(70)\gev$
y $\Delta{\rm R}(e/\mu/\tau/$light-quarks$, \gamma) > 0.1$ fueron removidos
de las muestras.

\begin{table}[ht!]
  \centering
  \caption{Muestras de $W/Z+\text{jets}$ utilizadas en este analisis.
    La seccion eficas a LO para cada modo de decaimiento, los factores $k$
    (para la normalizacion NLO) y las eficiencias del filtro son reportadas,
    asi como tabine la luminosidad integrada correspondiente a la estadistica
    total de cada muestra.}
  \begin{tabular}{ l | c | c | c | c }
    \hline
    \hline
    Proceso (ID) & $\sigma~[pb]$ & k-factor & filter eff. & $\int{\mathcal{L}dt}~[fb^{-1}]$ \\
    \hline
    \zeenj{0}  \alpgen+\pythia (117650) & 718.89 & 1.18 & 1 & 7.80 \\
%    \zeenj{1}  \alpgen+\pythia (117651) & 175.60 & 1.18 & 1 & 6.42 \\
%    \zeenj{2}  \alpgen+\pythia (117652) & 58.846 & 1.18 & 1 & 5.83 \\
%    \zeenj{3}  \alpgen+\pythia (117653) & 15.56   & 1.18 & 1 & 5.99 \\
%    \zeenj{4}  \alpgen+\pythia (117654) & 3.9322 & 1.18 & 1 & 6.47 \\
%    \zeenj{5}  \alpgen+\pythia (117655) & 1.1994 & 1.18 & 1 & 7.07 \\
%    \zmmnj{0}  \alpgen+\pythia (117660) & 718.89 & 1.18 & 1 & 7.79 \\
%    \zmmnj{1}  \alpgen+\pythia (117661) & 175.81 & 1.18 & 1 & 6.43 \\
%    \zmmnj{2}  \alpgen+\pythia (117662) & 58.805 & 1.18 & 1 & 5.84 \\
%    \zmmnj{3}  \alpgen+\pythia (117663) & 15.589 & 1.18 & 1 & 5.98 \\
%    \zmmnj{4}  \alpgen+\pythia (117664) & 3.9072 & 1.18 & 1 & 6.51 \\
%    \zmmnj{5}  \alpgen+\pythia (117665) & 1.1933 & 1.18 & 1 & 7.10 \\
%    \zttnj{0} \alpgen+\pythia (117670) & 718.85 & 1.18 & 1 & 7.80 \\
%    \zttnj{1} \alpgen+\pythia (117671) & 175.83 & 1.18 & 1 & 6.43 \\
%    \zttnj{2} \alpgen+\pythia (117672) & 58.630 & 1.18 & 1 & 5.85 \\
%    \zttnj{3} \alpgen+\pythia (117673) & 15.508 & 1.18 & 1 & 5.96 \\
%    \zttnj{4} \alpgen+\pythia (117674) & 3.9526 & 1.18 & 1 & 6.43 \\
%    \zttnj{5} \alpgen+\pythia (117675) & 1.1805 & 1.18 & 1 & 7.18 \\
    \zeenj{0}  \alpgen+\jimmy (107650) & 711.77 & 1.23 & 1 & 7.548 \\
    \zeenj{1}  \alpgen+\jimmy (107651) & 155.17 & 1.23 & 1 & 6.994 \\
    \zeenj{2}  \alpgen+\jimmy (107652) & 48.745 & 1.23 & 1 & 6.746 \\
    \zeenj{3}  \alpgen+\jimmy (107653) & 14.225 & 1.23 & 1 & 6.286 \\
    \zeenj{4}  \alpgen+\jimmy (107654) & 3.7595 & 1.23 & 1 & 6.487 \\
    \zeenj{5}  \alpgen+\jimmy (107655) & 1.0945 & 1.23 & 1 & 7.428 \\
    \zmmnj{0}  \alpgen+\jimmy (107660) & 712.11 & 1.23 & 1 & 7.557 \\
    \zmmnj{1}  \alpgen+\jimmy (107661) & 154.77 & 1.23 & 1 & 7.011 \\
    \zmmnj{2}  \alpgen+\jimmy (107662) & 48.912 & 1.23 & 1 & 6.731 \\
    \zmmnj{3}  \alpgen+\jimmy (107663) & 14.226 & 1.23 & 1 & 6.286 \\
    \zmmnj{4}  \alpgen+\jimmy (107664) & 3.7838 & 1.23 & 1 & 6.445 \\
    \zmmnj{5}  \alpgen+\jimmy (107665) & 1.1148 & 1.23 & 1 & 7.292 \\
    \zttnj{0} \alpgen+\jimmy (107670) & 711.81 & 1.23 & 1 &  7.560 \\
    \zttnj{1} \alpgen+\jimmy (107671) & 155.13 & 1.23 & 1 &  6.996 \\
    \zttnj{2} \alpgen+\jimmy (107672) & 48.804 & 1.23 & 1 &  6.746 \\
    \zttnj{3} \alpgen+\jimmy (107673) & 14.160 & 1.23 & 1 &  6.315 \\
    \zttnj{4} \alpgen+\jimmy (107674) & 3.7744 & 1.23 & 1 &  6.462 \\
    \zttnj{5} \alpgen+\jimmy (107675) & 1.1163 & 1.23 & 1 &  7.283 \\
    \hline
    \hline
    \wenunj{0}  \alpgen+\jimmy (107680) &8037.10   & 1.186 & 1 & 0.362 \\
    \wenunj{1}  \alpgen+\jimmy (107681) &1579.20   & 1.186 & 1 & 1.334 \\
    \wenunj{2}  \alpgen+\jimmy (107682) &477.20     & 1.186 & 1 & 6.661 \\
    \wenunj{3}  \alpgen+\jimmy (107683) &133.93     & 1.186 & 1 & 6.358 \\
    \wenunj{4}  \alpgen+\jimmy (107684) &35.62       & 1.186 & 1 &  5.917\\
    \wenunj{5}  \alpgen+\jimmy (107685) &10.55       & 1.186 & 1 &  5.592\\
    \wmnunj{0}  \alpgen+\jimmy (107690) &8040.00 & 1.186 & 1 &  0.363\\
    \wmnunj{1}  \alpgen+\jimmy (107691) &1580.30 & 1.186 & 1 &  1.333\\
    \wmnunj{2}  \alpgen+\jimmy (107692) &477.50   & 1.186 & 1 &  6.656\\
    \wmnunj{3}  \alpgen+\jimmy (107693) &133.94   & 1.186 & 1 &  6.357\\
    \wmnunj{4}  \alpgen+\jimmy (107694) &35.64      & 1.186 & 1 &  6.033\\
    \wmnunj{5}  \alpgen+\jimmy (107695) &10.57      & 1.186 & 1 &  1.595\\
    \wtnunj{0} \alpgen+\jimmy (107700)    &8035.80  & 1.186 & 1 & 0.353 \\
    \wtnunj{1} \alpgen+\jimmy (107701)    &1579.80  & 1.186 & 1 & 1.307 \\
    \wtnunj{2} \alpgen+\jimmy (107702)    &477.55    & 1.186 & 1 &  6.567\\
    \wtnunj{3} \alpgen+\jimmy (107703)    &133.79    & 1.186 & 1 &  6.365\\
    \wtnunj{4} \alpgen+\jimmy (107704)    &35.58      & 1.186 & 1 &  5.921\\
    \wtnunj{5} \alpgen+\jimmy (107705)    &10.54      & 1.186 & 1 &  5.199\\
    \hline
    \hline
  \end{tabular}
  \label{tab:bkg_wzjets_samples}
\end{table}


\subsubsection{Top pair ($+\gam$) production}\label{sec:mcttbargam}

Another important background to this analysis is \ttgam. This MC sample was generated using the
{\madgraph} \cite{Alwall:2007st} MC generator and the CTEQ6L1 PDF. %No fully hadronic \ttbar\ was included.
  {\pythiasix} \cite{pythia} was used for parton showering, fragmentation and underlying event
simulation. Additional photon radiation was added with
{\photos} \cite{photos}, and tau leptons were decayed with
{\tauola} \cite{tauola}. The truth-level photons were required to have
$\pt > 80 \gev$. To avoid kinematic effects introduced by the filter, the photon \pt\ cut on the reconstructed sample was raised to 95 \gev.
A $k$-factor of $1.9 \pm 0.4$ is used \cite{Melnikov:2011ta, tth}. %This was calculated by
%the theorists with the particular generator-level cuts used to                                                                                                                     %generate this MC sample.

Simulation details are given in {\tab} \ref{tab:bkg_ttbar_samples}. The MC samples with IDs 202332--202337 are truth-only systematic variation samples, used to estimate systematic
uncertainties as explained in {\Sec} \ref{sec:syst_ttbargamma}.


% NOT ANYMORE , but explained later why
%As described in {\sec} \ref{}, the normalization of this sample was extracted together with that for %W$+\gamma$ events by fitting
%simultaneously the two MC to data in specifically designed control regions.

%More information can be found in
%Ref.~\cite{ttbargammaSupport}.

La produccion de {\ttbar}, donde los electrones o los jets
son mal identificados como fotones es una fuente de fondo que
vale la pena considerar.

%% Although both fakes contamination were estimated from the data, simulated events were used at the  optimization stage
%% and for cross checks of the data-driven methods. The MC sample was generated using the
%% {\powheg} \cite{Nason:2004rx,Frixione:2007vw,Alioli:2010xd} generator, with the parton showering and fragmentation done in \pythia.
%% The new Perugia 2011C tune \footnote{see some details at \url{https://twiki.cern.ch/twiki/bin/viewauth/AtlasProtected/P2011C}.} was used for the underlying event,
%%  with the CTEQ6L1 LO* set of PDFs. Additional photon radiation was added with {\photos} \cite{photos}.
%% Overlap removal is performed to prevent double-counting the phase-space
%% covered by the {\ttgam} MC sample. Events with truth prompt photons with
%% $\pt > 95 \gev$ and $\Delta{\rm R}(e/\mu/\tau/g/$light-quarks$, \gamma) > 0.1$ are removed from the \ttbar\ sample.


%% ttbar normalization??

%The nominal normalization of this sample is based on a cross section
%calculated at approximate NNLO in QCD using Hathor
%1.2~\cite{Aliev:2010zk} using the MSTW2008 \unit[90]{\%} NNLO PDF
%sets~\cite{MSTW2008} and incorporating PDF+$\alpha_S$ uncertainties
%according to the MSTW prescription~\cite{Martin:2009bu}. The value is
%cross checked with the NLO+NNLL calculation of Cacciari et
%al.~\cite{Cacciari:2011hy} as implemented in Top++
%1.0~\cite{Czakon:2011xx}. An uncertainty of \unit[11.0]{\%} is assigned
%to the NNLO cross section (which incorporates the mass uncertainty).
%


\begin{table}[ht!]
  \centering
  \caption{\ttgam\ samples used for the analysis. The LO cross-section for specified decay mode, k-factors (for NLO normalisation) and filter efficiencies are reported. The integrated luminosities corresponding to the total statistics in each sample are also given. The bottom group of samples was used to study systematic uncertainties.}
  \includegraphics[width=1\textwidth]{figures/tabla}
  %% \begin{tabular}{ l | c | c | c | c }
  %%   \hline
  %%   \hline
  %%   Process (ID) & $\sigma~[pb]$ & k-factor & filter eff. & $\int{\mathcal{L}dt}~[fb^{-1}]$ \\
  %%   \hline
  %%   \hline
  %%   \ttbar\ \powheg+\pythia (117050)  & 253.00 & 1 & 0.543 & 580 \\
  %%   \hline
  %%   %    \ttbargam noAllHad \madgraph (164439) & 0.092363 & 1.9 & 1 & 1139.7 \\
  %%   \ttbargam\ noAllHad \madgraph (177998) & 0.09873 & 1.9 & 1 & 1066.2 \\
  %%   \ttbargam\ AllHad \madgraph (174382) & 0.068599 & 1.9 & 1 & 1534.5 \\
  %%   \hline
  %%   \hline
  %%   \multicolumn{5}{l}{Systematics variations} \\
  %%   \hline
  %%   \ttbargam\ noAllHad (scaleUP) \madgraph (202332) & 0.09873 & 1.9 & 1 & 1066.2 \\
  %%   \ttbargam\ noAllHad (scaleDN) \madgraph (202333) & 0.09873 & 1.9 & 1 & 1066.2 \\
  %%   \ttbargam\ noAllHad (alpsUP) \madgraph (202334) & 0.09873 & 1.9 & 1 & 1066.2 \\
  %%   \ttbargam\ noAllHad (alpsDN) \madgraph (202335) & 0.09873 & 1.9 & 1 & 1066.2 \\
  %%   \ttbargam\ noAllHad (moreFSR) \madgraph (202336) & 0.09873 & 1.9 & 1 & 1066.2 \\
  %%   \ttbargam\ noAllHad (lessFSR) \madgraph (202337) & 0.09873 & 1.9 & 1 & 1066.2 \\
  %%   \hline
  %%   \hline
  %% \end{tabular}
  \label{tab:bkg_ttbar_samples}
\end{table}

\subsubsection{Single top (+ $\gamma$) production}

Single top with an associated photon samples were produced using
Whizard 2.1.1 \cite{whizard, whizard2}, with 4-flavor /5-flavor
matching provided using Hoppet~\cite{hoppet}.\footnote{Thanks to
Fabian Bach for providing an early version of the matching, which
will be standard in Whizard 2.2.0.} The extra photon could be in
either the single top production or the subsequent decays. Production
and decay, however, was treated separately, so interference effects
are ignored. {\pythia} \cite{pythia} was used for parton showering and
fragmentation. Additional photon radiation was added with
{\photos} \cite{photos}, and tau leptons were decayed with
{\tauola} \cite{tauola}.

%We used sample IDs 202621 and 202622 for \tchangamma production, and
%sample IDs 202623--202631 for \tWgamma production.

\begin{table}[ht!]
  \centering
  \caption{Single top and \tgam\ samples used for the analysis. The NNLO cross-section, filter efficiencies and the integrated luminosities corresponding to the total statistics in each sample are also given.}
  \includegraphics[width=1\textwidth]{figures/tabla}
  %% \begin{tabular}{ l | c | c | c }
  %%   \hline
  %%   \hline
  %%   Process (ID) & $\sigma~[pb]$ & filter eff. & $\int{\mathcal{L}dt}~[fb^{-1}]$ \\
  %%   \hline
  %%   t-channel \acermc (110101)  & 28.4 & 1 & 271 \\
  %%   Wt        \powheg (110140)  & 22.4 & 1 & 892 \\
  %%   s-channel \powheg (110119)  & 1.82 & 1 & 3299 \\
  %%   \hline
  %%   \hline
  %%   \tgam (t-channel) \wizhard+\pythia (202621)  & 0.187298 & 0.121980 & 4810 \\
  %%   \tgam (t-channel) \wizhard+\pythia (202622)  & 0.313866 & 0.012927 & 4930 \\
  %%   \hline
  %%   \twgam (dilep.) \wizhard+\pythia (202623)  & 0.012915  & 0.164370 & 4710 \\
  %%   \twgam (dilep. tDec) \wizhard+\pythia (202624)  & 0.014538  & 0.028748 & 12000 \\
  %%   \twgam (dilep. WDec) \wizhard+\pythia (202625)  & 0.010405  & 0.075489 & 6370 \\
  %%   \twgam (tlepWhad) \wizhard+\pythia (202626)  & 0.025825 & 0.162440 & 4770 \\
  %%   \twgam (tlepWhad tDec) \wizhard+\pythia (202627)  & 0.029084  & 0.027609 & 6230 \\
  %%   \twgam (tlepWhad WDec) \wizhard+\pythia (202628)  & 0.011594  & 0.064709 & 6660 \\
  %%   \twgam (thadWlep) \wizhard+\pythia (202629)  & 0.025817 & 0.161780 & 4790 \\
  %%   \twgam (thadWlep tDec) \wizhard+\pythia (202630)  & 0.020127  & 0.041977 & 5920 \\
  %%   \twgam (thadWlep WDec) \wizhard+\pythia (202631)  & 0.020788  & 0.075740 & 3180 \\
  %%   \hline
  %%   \hline
  %% \end{tabular}
  \label{tab:bkg_st_samples}
\end{table}

The single top process is a small background for this analysis, mostly
important for the control and validation regions. $Wt$ production (ID 110140) was generated using the \powheg, including full next-to-leading order QCD
corrections. Parton showering and fragmentation were simulated by
\pythia with the P2011C tune. The CT10 next-to-leading-order parton
set is used for the matrix element, the parton shower and the
underlying event. The samples were scaled to the cross section
calculated in \cite{Kidonakis:2010ux}. For $t$-channel
production, the MC samples with sample ID 110101 were used, with the
$W$ boson decaying leptonically. These were generated with
\acermc \cite{acer}, with parton showering and fragmentation performed
by {\pythia} with the P2011C tune and CTEQ6L1 PDF set.  The samples were
scaled to the cross section calculated by \cite{Kidonakis:2011wy}.
Single top produced by $s$-channel was not used because it was found
to be negligible.

Overlap between the single top and single top $\gamma$ samples has been removed.

\subsubsection{Direct $\gamma+$jets and QCD multijet}

%The QCD background is one of the main source of background in this analysis, particularly prompt photon events. The QCD contamination is in all cases a result of pathological events (jet faking a photon, badly reconstructed jet or photon making high \MET) for which the simulation is not reliable.
%For this reason this background is computed with a data driven approach, as explained in {\sec} \ref{sec:jetsmearing}. Several MC simulations are used anyways for consistency checks and to assess systematic uncertainties.

The QCD contamination is in all cases a result of pathological events
(jet faking a photon, badly reconstructed jet or photon making high \MET).
However, it is not expected to be a dominant source of background in the
phase space explored in this analysis. The contribution from events with
jets faking a photon is estimated with the data driven described in
{\Sec} \ref{sec:jetfakes}. The QCD multijet samples listed in {\tab}
\ref{tab:bkg_qcd_samples} were used for optimisation and preliminar
sensitivity studies. Prompt photon production was simulated with
with {\sherpa} v1.4.1 \cite{SherpaGen}, with up to four partons in the ME+PS
and using the CT10 set of parton density functions. The inclusive spectrum
is sliced in ranges of photon \pt\ to optimise the event generation.
Alternative samples were used to asses systematics uncertainties and
cross checks, generated with {\pythiaeight} (using CTEQ6L1) and {\alpgen}
v2.14 (with same config as V$+$jets events described in {\Sec} \ref{mc_wzjets}).
Further details are given in {\tab} \ref{tab:bkg_qcd_samples}.

\begin{table}[ht!]
  \centering
  \caption{Muestras de QCD {\gjet} y multijet utilizadas en este analisis.
    La seccion eficas a LO para cada modo de decaimiento,
    y las eficiencias del filtro son reportadas,
    asi como tabine la luminosidad integrada correspondiente a la estadistica
    total de cada muestra.}

  %% \includegraphics[width=1\textwidth]{figures/tabla}

   \begin{tabular}{ l | c | c | c }
    \hline
    \hline
    Process (ID) & $\sigma~[pb]$ & filter eff. & $\int{\mathcal{L}dt}~[fb^{-1}]$ \\
    \hline
    {\gjet} ($\pt>70\gev$) {\sherpa}  ( 113715 ) &  2153.0  &  1.0  &  1.160 \\
    {\gjet} ($\pt>140\gev$) {\sherpa} ( 113716 ) &  137.85  &  1.0  &  10.881 \\
    {\gjet} ($\pt>280\gev$) {\sherpa}  ( 113717 ) &  5.963  &  1.0  &  167.657 \\
    {\gjet} ($\pt>500\gev$) {\sherpa}  ( 126731 ) &  0.276  &  1.0  &  3617.291 \\
    {\gjet} ($\pt>800\gev$) {\sherpa}  ( 126955 ) &  0.0133  &  1.0  &  7492.807 \\
    {\gjet} ($\pt>1000\gev$) {\sherpa}  ( 126956 ) &  0.00238  &  1.0  &  41980.269 \\
    \hline

    \hline
  \gjet\ ($\pt>70\gev$)   \pythiaeight ( 129172 ) &  3425000  &  $5.7 \times 10^{-4}$  &  1535.4  \\
  \gjet\ ($\pt>140\gev$) \pythiaeight ( 129173 ) &  122170  &  $9.7 \times 10^{-4}$  &  8449.2 \\
  \gjet\ ($\pt>280\gev$) \pythiaeight ( 129174 ) &  3348.7  &  $1.45 \times 10^{-3}$  &  206559.7 \\
  \gjet\ ($\pt>500\gev$) \pythiaeight ( 129175 ) &  115.63  &  $1.8 \times 10^{-3}$  &  4789097.0\\
    \hline

% \gjetnj{1} ($\ptgam>35\gev) \alpgen+\jimmy  ( 156841 ) &  9554.200195  &  1.0  &  0.00890 \\
 \gjetnj{1} ($\pt>70\gev$) \alpgen+\jimmy  ( 156843 ) &  577.480  &  1.0  &  0.147 \\
 \gjetnj{1} ($\pt>140\gev$) \alpgen+\jimmy  ( 156839 ) &  26.198  &  1.0  &  3.626 \\
 \gjetnj{1} ($\pt>280\gev$) \alpgen+\jimmy  ( 156840 ) &  0.83119  &  1.0  &  30.077 \\
 \gjetnj{1} ($\pt>500\gev$) \alpgen+\jimmy  ( 156842 ) &  0.029141  &  1.0  &  343.159 \\

% \gjetnj{2} ($\ptgam>35\gev$) \alpgen+\jimmy  ( 156846 ) &  4515.0  &  1.0  &  0.00886 \\
 \gjetnj{2} ($\pt>70\gev$) \alpgen+\jimmy  ( 156848 ) &  571.870  &  1.0  &  0.175 \\
 \gjetnj{2} ($\pt>140\gev$) \alpgen+\jimmy  ( 156844 ) &  38.671001  &  1.0  &  3.879 \\
 \gjetnj{2} ($\pt>280\gev$) \alpgen+\jimmy  ( 156845 ) &  1.6811  &  1.0  &  29.741 \\
 \gjetnj{2} ($\pt>500\gev$) \alpgen+\jimmy  ( 156847 ) &  0.075517  &  1.0  &  264.841 \\

% \gjetnj{3} ($\ptgam>35\gev$) \alpgen+\jimmy  ( 156851 ) &  1717.0  &  1.0  &  0.00874 \\
 \gjetnj{3} ($\pt>70\gev$) \alpgen+\jimmy  ( 156853 ) &  306.10  &  1.0  &  0.049\\
 \gjetnj{3} ($\pt>140\gev$) \alpgen+\jimmy  ( 156849 ) &  28.57  &  1.0  &  5.250 \\
 \gjetnj{3} ($\pt>280\gev$) \alpgen+\jimmy  ( 156850 ) &  1.538  &  1.0  &  32.503 \\
 \gjetnj{3} ($\pt>500\gev$) \alpgen+\jimmy  ( 156852 ) &  0.077073  &  1.0  &  77.822 \\

% \gjetnj{4} ($\ptgam>35\gev$) \alpgen+\jimmy  ( 156856 ) &  513.940002  &  1.0  &  0.00778 \\
 \gjetnj{4} ($\pt>70\gev$) \alpgen+\jimmy  ( 156858 ) &  115.850  &  1.0  &  0.216 \\
 \gjetnj{4} ($\pt>140\gev$) \alpgen+\jimmy  ( 156854 ) &  14.216  &  1.0  &  11.951 \\
 \gjetnj{4} ($\pt>280\gev$) \alpgen+\jimmy  ( 156855 ) &  0.9185  &  1.0  &  48.992 \\
 \gjetnj{4} ($\pt>500\gev$) \alpgen+\jimmy  ( 156857 ) &  0.0512  &  1.0  &  156.354 \\

% \gjetnj{5} ($\ptgam>35\gev$) \alpgen+\jimmy  ( 156861 ) &  163.800003  &  1.0  &  0.0458 \\
 \gjetnj{5} ($\pt>70\gev$) \alpgen+\jimmy  ( 156859 ) &  7.00  &  1.0  &  18.569 \\
 \gjetnj{5} ($\pt>140\gev$) \alpgen+\jimmy  ( 156860 ) &  0.542  &  1.0  &  92.304 \\
 \gjetnj{5} ($\pt>280\gev$) \alpgen+\jimmy  ( 156862 ) &  0.0333  &  1.0  &  450.911 \\
 \gjetnj{5} ($\pt>500\gev$) \alpgen+\jimmy  ( 156863 ) &  44.334  &  1.0  &  0.970 \\
    \hline
    \hline
    JZ1W ($20 \gev < p^{\rm leading jet}_{\rm T} < 80 \gev$) \pythia (147911) &  $7.285 \times 10^{10}$ &  0.000129 & 0.00016 \\
    JZ2W ($80 \gev < p^{\rm leading jet}_{\rm T} < 200 \gev$) \pythia (147912) &  $2.634 \times 10^{7}$ &  0.003894 & 0.0142 \\
    JZ3W ($200 \gev < p^{\rm leading jet}_{\rm T} < 500 \gev$) \pythia (147913) &  $5.442 \times 10^{5}$ &  0.001219 & 2.26 \\
    JZ4W ($500 \gev < p^{\rm leading jet}_{\rm T} < 1000 \gev$) \pythia (147914) &  $6.445 \times 10^{3}$ &  0.000708 & 328 \\
    JZ5W ($1000 \gev < p^{\rm leading jet}_{\rm T} < 1500 \gev$) \pythia (147915) &  39.74 &  0.002152 & 17400 \\
    JZ6W ($1500 \gev < p^{\rm leading jet}_{\rm T} < 2000 \gev$) \pythia (147916) &  0.4161 &  0.004684 & $7.68 \times 10^{5}$ \\
    JZ7W ($p^{\rm leading jet}_{\rm T} > 2000 \gev$) \pythia (147917) &  0.04064 &  0.0146 & $2.52\times 10^{6}$ \\
    \hline
  \end{tabular}
  \label{tab:bkg_qcd_samples}
\end{table}

\subsubsection{Dibosons}

Diboson (WW, WZ, and ZZ) events were generated with SHERPA using the CT10 PDF, with cross sections provided by MCFM \cite{Campbell:2011bn}. Only leptonic decays for both W and Z bosons were considered.

\begin{table}[ht!]
  \centering
  \caption{Muestras de Diboson utilizadas en este analisis.
    La seccion eficas a LO para cada modo de decaimiento, los factores $k$
    (para la normalizacion NLO) y las eficiencias del filtro son reportadas,
    asi como tabine la luminosidad integrada correspondiente a la estadistica
    total de cada muestra.}

  \includegraphics[width=1\textwidth]{figures/tabla}
  \begin{tabular}{ l | c | c | c | c }
    \hline
    \hline
    Proceso (ID) & $\sigma~[pb]$ & factor $k$ & Eficiencia & $\int{\mathcal{L}dt}~[fb^{-1}]$ \\
    \hline
    %% $WW(2l2\nu)$ \sherpa (126892)  & 5.50 & 1.07 & 1 & 458.9 \\
    %% $WZ(3l)$ \sherpa (126893) & 9.75 & 1.06 & 1 & 261.1 \\
    %% $ZZ(4l)$ \sherpa (126894)  & 8.74 & 1.11 & 1  &  185.6 \\
    %% $ZZ(2l2\nu)$ \sherpa (126895)  & 0.50 & 1.14 & 1 &  1590.8 \\
    $WW(ll\nu\nu)$ \sherpa    (177997)  & 5.2963  & 1.06 & 1 & 1400 \\
    $ZZ(ll\nu\nu)$ \sherpa    (177999)  & 0.49434 & 1.05 & 1 & 1700 \\
    $WZ(lll\nu)$ \sherpa      (179974)  & 9.74456 & 1.05 & 1 & 260 \\
    $WZ(l\nu\nu\nu)$ \sherpa  (179975)  & 1.4047  & 1.05 & 1 & 270 \\
    $ZW(eeqq)$ \sherpa        (183585)  & 1.4648  & 1.05 & 1 & 110 \\
    $ZZ(eeqq)$ \sherpa        (183586)  & 0.24672 & 1    & 1 & 120 \\
    $ZW(\mu\mu qq)$ \sherpa   (183587)  & 1.4634  & 1.05 & 1 & 110 \\
    $ZZ(\mu\mu qq)$ \sherpa   (183588)  & 0.24757 & 1    & 1 & 120 \\
    $ZW(\tau\tau qq)$ \sherpa (183589)  & 1.4523  & 1.05 & 1 & 120 \\
    $ZZ(\tau\tau qq)$ \sherpa (183590)  & 0.24167 & 1    & 1 & 120 \\
    $ZW(\nu\nu qq)$ \sherpa   (183591)  & 2.6972  & 1.05 & 1 & 64 \\
    $ZZ(\nu\nu qq)$ \sherpa   (183592)  & 1.7440  & 1    & 1 & 69 \\
    $WW(e\nu qq)$ \sherpa     (183734)  & 7.2854  & 1.06 & 1 & 100 \\
    $WZ(e\nu qq)$ \sherpa     (183735)  & 1.9036  & 1.05 & 1 & 110 \\
    $WW(\mu\nu qq)$ \sherpa   (183736)  & 7.2974  & 1.06 & 1 & 100 \\
    $WZ(\mu\nu qq)$ \sherpa   (183737)  & 1.9057  & 1.05 & 1 & 100 \\
    $WW(\tau\nu qq)$ \sherpa  (183738)  & 7.2741  & 1.06 & 1 & 100 \\
    $WZ(\tau\nu qq)$ \sherpa  (183739)  & 1.9152  & 1.05 & 1 & 100 \\
    \hline
    \hline
  \end{tabular}
  \label{tab:bkg_diboson_samples}
\end{table}

%
