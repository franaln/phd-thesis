\thispagestyle{plain}
\begin{center}
  \Large
  \textbf{Search for Supersymmetry in events with a photon, jets and
    missing energy with the ATLAS detector}

  \vspace{0.4cm}
  \large
  %% Thesis Subtitle

  \vspace{0.4cm}
  Francisco Alonso

  \vspace{2cm}
  \textbf{Abstract}
\end{center}

Supersymmetry (SUSY) is one of the most motivated theories for physics beyond
the Standard Model, giving a framework to the unification of the particle
physics and gravity, which is governed by the Plank energy scale. As none of
the predicted supersymmetric particles has been discovered, SUSY must be a
broken symmetry in nature. SUSY phenomenology is hardly determined by the SUSY
breaking mechanism. GGM models in which the SUSY breaking is mediated by the
usual gauge fields of the Standard Model provide a suitable scenario for SUSY
searches at the LHC, with very characteristic mass spectrum and decays. This
thesis presents the first search of new physics with one energetic photon, jets
and high missing energy in the final state in proton-proton collisions at a
centre-of mass energy of 8 {\tev}. The analysis was realized with all the data
collected by the ATLAS detector during 2012, corresponding to a total
integrated luminosity of 20.3 \ifb. No excess over the Standard Model
predictions was observed, so an upper limit in the number of new physics events
was established at 95 \% CL for this final state. Additionally, the results were
also interpreted in the context of a GGM SUSY model considering a bino-higgsino
neutralino NLSP admixture, excluding the production of gluinos with masses up to
1.25 \tev, giving the most strict limits at the moment.
